\chapter{Blow-up}
\section{基本概念和例子学习}
\subsection{例子一}
\subsubsection{Some notations}

\begin{definition}[][transgressive]
    An element $\alpha\in H^{p,q}_{\bdd}(F)$  is called transgressive if there is a representative $\alpha\in \mA^{p,q}_{\bdd}(F)$ which extends to a form $\widetilde{\alpha}\in \mA^{p,q}_{\bdd}(F)$ such that $\bd\widetilde{\alpha}=\pi^*\beta$ for some $\bd$-closed form $\beta\in \mA^{p,q+1}_{\bdd}(B)$.
\end{definition}

The following are all the settings in this paper.
\begin{enumerate}
    \item $i^*\colon \mA^{p,q}(M)\to\mA^{p,q}(N)$ is surjective, which induces a short exact sequence for the pair $(M,N)$ of complexes 
    \[
        0\to\mA^{p,\bullet}(M,N)\to \mA^{p,\bullet}(M)\stackrel{i^*}{\to}\mA^{p,\bullet}(N)\to 0
    \]
     and a long exact sequence
     \[
        \cdots\to H^{p,\bullet}_{\bdd} (M,N)\to H^{p,\bullet}_{\bdd} (M) \to H^{p,\bullet}_{\bdd} (N)\to \cdots.
     \]
     !Note that here $i\colon N\hookrightarrow M$ is an inclusion, which is naturally injective. Associating with the above condition, the exactness at $\mA^{p,\bullet}(M)$ is clear.
    \item 
\end{enumerate}

\subsection{Blow-up with the center at a point}

\begin{definition}[][Blow-up]
    Let $X$ be a complex manifolds and let $Y\subset X$ be a closed submanifold. The \textit{Blow-up of $X$ along $Y$} which is a complex manifold $\widehat{X}:=\operatorname{Bl}_Y (X)$ together with a proper holomorphic map $\sigma\colon \widehat{X}\to X$.
\end{definition}


\begin{example}[][Blow-up with the center at a point]
    Consider the situation of line bundle $\mO(-1)$ over projective manifold $\bP^n$, which is given as the \textbf{incidence variety} 
    \[
        \begin{tikzcd}
\bP^n \ar[r,-Stealth,hook,"\pi^*"]\ar[d,-Stealth]&\mO(-1)\ar[d,-Stealth,"\sigma"]\\ 
\{0\}\ar[r,-Stealth,hook,"\operatorname*{id}"]&\bC^{n+1}
        \end{tikzcd}
    \]
    Recall that the set $\mO(-1)\subset \bP^n\times \bC^{n+1}$ that consists of all pairs $(\ell,z)\in\bP^n\times \bC^{n+1}$ with $z\in\ell$ forms in a natural way a holomorphic line bundle over $\bP^n$ by Huybrechets,Prop 2.2.6.  First, we have the projective $\pi\colon \mO(-1)\to\bP^n$, and the fibres of $\mO(-1)$ is $\pi^{-1}(\ell) , \ell\in\bP^n$. Without loss generality, let $\bP^n=\bigcup_{i=0}^n U_i$ be the standard open covering, which induces the fact that $\ell$ must belong to one of $U_i$. Here we just set $\ell\in U_i$ for conviniance. Then we have the \textit{trivialization of $\mO(-1)$ over $U_i$}, which is given by 
    \[
        \begin{aligned}
                   \psi_i\colon \pi^{-1}(U_i)&\cong U_i\times \bC\\ 
                    (\ell,z)&\mapsto (\ell,z_i).
        \end{aligned}
    \] 
The corresponding transition maps are
\begin{align*}
    \psi_{ij}(\ell=(z_0:\cdots:z_n))\colon \bC&\to\bC\\ 
w&\mapsto \frac{z_j}{z_i}w.
\end{align*}
In other words, the transition maps can also be written as 
\begin{align*}
    \psi_{ij}\colon U_i\times\bC&\to U_j\times \bC\\ 
(\ell,w)=(\ell,z_i) &\mapsto (\ell,\frac{z_j}{z_i}w)=(\ell,z_j)\\ 
\psi_i(\ell) &\to \psi_j(\ell)
\end{align*}
Now we can easily see that the fibre of $\mO(-1)$ over $\ell\in\bP^n$ is naturally isomorphic to $\ell$ itself. Let us consider the other canonical projection $\sigma\colon \mO(-1)\to \bC^{n+1}$. Theirs' relations are the following diagram.
\[
    \begin{tikzcd}
        &\mO(-1)\subset \bP^n\times \bC^{n+1}\dlar[-Stealth,"\pi"]\drar[-Stealth,"\sigma"]&\\
        \bP^n=\bigcup_{i=0}^n U_i && \bC^{n+1}
    \end{tikzcd}
\]
By the definition of the projective bundle, we obtain that all line $\ell\in \bP^n$ pass through the origin point, which means the pre-image of $\sigma$ at $0$ is just the whole projective space $\bP^n$, i.e., $\sigma^{-1}(0)=\bP^n$. 

Thus we construct a \textit{Blow-up} $\sigma\colon \operatorname{Bl}_0 (\bC^{n+1})\to\bC^{n+1}$ of $\bC^{n+1}$ along the zero dimensional submanifold $\{0\}$, which by definition is a holomorphic line bundle $\mO(-1)$ with the center at $0$ . Note that $E:=\operatorname{Bl}_0 (\bC^{n+1})=\bP^n$ is the exceptional divisor.
\end{example}

\begin{example}[][Blow-up along a linear subspace][exam:blow-up at subspace]
    Let $\bC^m\subset \bC^n$ be the linear subspace satisfying $z_{m+1}=\cdots=z_n=0$ and denote by $(x_{m+1}:\cdots:x_n)$ the homogeneous coordinates of $\bP^{n-m-1}$.

    Define 
    \[
        \operatorname{Bl}_{\bC^m}(\bC^n):=\{(x,z)\mid z_i\cdot x_j=z_j\cdot x_i, i,j=m+1,\cdots,n\}\subset \bP^{n-m-1}\times \bC^{n}.
    \]
\end{example}

\subsection{例子二}
好的,让我们一步一步地来理解这个Blow-up的构造。

首先,考虑一个简单的情况:Blow-up原点于$\mathbb{C}^2$。原点的Blow-up是通过替换原点为所有通过原点的直线的集合来完成的。在代数几何中,这些直线可以通过它们的斜率来参数化,这些斜率可以在射影空间$P^1$中表示。因此,Blow-up的结果是一个更大的空间,其中包括了原始$\mathbb{C}^2$中除原点外的所有点,以及所有通过原点的直线的额外信息。

现在,让我们回到你的问题。我们想要Blow-up $\mathbb{C}^n$ 中的子空间$\mathbb{C}^m$。这里的子空间由坐标$x_{m+1} = x_{m+2} = \ldots = x_n = 0$定义。\textit{Blow-up的目的是在每个点$(x_1, \ldots, x_m, 0, \ldots, 0)$附近引入新的坐标,这些坐标描述了通过这个点的所有可能的“方向”。}

为了实现这一点,我们在$\mathbb{C}^n$中的每一个点$(x_1, \ldots, x_n)$引入射影空间$P^{n-m-1}$的坐标$[z_{m+1} : \ldots : z_n]$。这些$z$坐标代表了$\mathbb{C}^n$中的点在从$\mathbb{C}^m$子空间“吹开”成$P^{n-m-1}$时的方向。为了确保这些方向是良好定义的,我们需要$z$坐标满足一定的等价关系,即任意两个$z_i$和$z_j$之间必须有相同的比例因子。这就是为什么我们要求$z_i \cdot x_j = z_j \cdot x_i$,对于所有$i,j = m+1, \ldots, n$。

这组方程确保了如果我们固定了$x$坐标中的一个非零分量(比如$x_k \neq 0$),那么所有的$z_i$可以通过$x_i/x_k$来唯一确定。这样,$z$坐标就代表了通过点$(x_1, \ldots, x_n)$的直线在$\mathbb{C}^n$中的方向。

总结来说,这个构造通过在每个点附近引入射影空间的坐标来“吹开”子空间$\mathbb{C}^m$,从而得到一个新的空间$\operatorname{Bl}_{\mathbb{C}^m}(\mathbb{C}^n)$。这个空间在$\mathbb{C}^m$之外与原来的$\mathbb{C}^n$相同,在$\mathbb{C}^m$上则有一个非平凡的纤维结构,这个纤维结构是由$P^{n-m-1}$给出的。这种Blow-up构造在处理几何问题时非常有用,特别是当需要解决奇点或者复杂的交叉结构时。

当然,让我们更详细地探讨这个Blow-up构造的严谨性。

首先,我们需要确认的是,通过这组方程$z_i \cdot x_j = z_j \cdot x_i$定义的空间$\operatorname{Bl}_{\mathbb{C}^m}(\mathbb{C}^n)$确实是良好定义的。这意味着我们需要验证这些方程确实描述了一个几何对象,即一个代数簇,它在代数几何的语境中是有意义的。

这些方程定义了一个在$P^{n-m-1} \times \mathbb{C}^n$中的闭子集,因为它们是齐次的。这个闭子集被称为incidence variety,它包含了所有满足给定方程的$(x,z)$对。在这里,$z$坐标代表了射影空间中的点,而$x$坐标代表了$\mathbb{C}^n$中的点。

为了确保这个构造是良好定义的,我们需要检查两件事情:

1. 等价关系:$z$坐标必须满足射影空间中的等价关系。这意味着如果我们有两个坐标$z_i$和$z_j$,它们必须满足$z_i = \lambda z_j$,对于某个非零复数$\lambda$。这是射影空间的基本性质,即点$[z_{m+1} : \ldots : z_n]$与点$[\lambda z_{m+1} : \ldots : \lambda z_n]$是相同的。这个性质由我们的方程组自然满足,因为如果$z_i \cdot x_j = z_j \cdot x_i$对于所有的$i,j$都成立,那么对于任何非零常数$\lambda$,我们也有$(\lambda z_i) \cdot x_j = (\lambda z_j) \cdot x_i$。

2. 局部坐标系统:在$\mathbb{C}^m$的每一点附近,我们需要能够用$z$坐标来描述通过该点的所有可能方向。这是通过确保$z$坐标与$x$坐标之间的关系来实现的。如果$x_k \neq 0$,那么我们可以通过$x_i/x_k$来唯一确定$z_i$,因为方程$z_i \cdot x_k = z_k \cdot x_i$意味着$z_i = z_k \cdot (x_i/x_k)$。这表明,即使$z_k$是射影坐标中的一个“自由”选择,其他的$z_i$都是由$x$坐标唯一确定的,这样就在每个点附近定义了一个局部坐标系统。

通过这些检查,我们可以确认这个Blow-up构造是良好定义的。它在$\mathbb{C}^m$的每一点附近引入了额外的结构,而在$\mathbb{C}^m$之外则保持了原空间$\mathbb{C}^n$的结构。这种细化使得我们可以在原空间中处理奇点或复杂结构时,有更多的灵活性和控制力。

\begin{example}[][Example 2.5.2 Blow-up along a linear subspace.]
    Let $\mathbb{C}^m \subset \mathbb{C}^n$ be the linear subspace satisfying $z_{m+1}=\ldots=z_n=0$ and denote by $\left(x_{m+1}: \ldots: x_n\right)$ the homogeneous coordinates of $\mathbb{P}^{n-m-1}$. We define
$$
\mathrm{Bl}_{\mathbb{C}^m}\left(\mathbb{C}^n\right):=\left\{(x, z) \mid z_i \cdot x_j=z_j \cdot x_i, i, j=m+1, \ldots, n\right\} \subset \mathbb{P}^{n-m-1} \times \mathbb{C}^n .
$$

In other words, \textcolor{purple}{$\operatorname{Bl}_{\mathbb{C}^m}\left(\mathbb{C}^n\right)$ is the incidence variety $\left\{(\ell, z) \mid z \in\left\langle\mathbb{C}^m, \ell\right\rangle\right\}$, where $\ell \in \mathbb{P}^{n-m-1}$ is a line in the complement $\mathbb{C}^{n-m}$ of $\mathbb{C}^m \subset \mathbb{C}^n$ and $\left\langle\mathbb{C}^m, \ell\right\rangle$ is the span of $\mathbb{C}^m$ and the line $\ell$.}

Using the projection $\pi: \mathrm{Bl}_{\mathbb{C}^m}\left(\mathbb{C}^n\right) \rightarrow \mathbb{P}^{n-m-1}$ one realizes $\mathrm{Bl}_{\mathbb{C}^m}\left(\mathbb{C}^n\right)$ as a $\mathbb{C}^{m+1}$-bundle over $\mathbb{P}^{n-m-1}$. The fibre over $\ell \in \mathbb{P}^{n-m-1}$ is just $\pi^{-1}(\ell)=\langle \mathbb{C}^m,\ell\rangle$. Thus, $\operatorname{Bl}_{\mathbb{C}^m}(\mathbb{C}^n)$ is a complex manifold.

Here is the corresponding blow-up diagram
\[
    \begin{tikzcd}
        \mathbb{P}^{n-m-1}\dar[-Stealth]\rar[-Stealth,hook,"\pi^{-1}"] & \mathrm{Bl}_{\mathbb{C}^m}\left(\mathbb{C}^n\right) \subset \mathbb{P}^{n-m-1}\times \mathbb{C}^n\dar[-Stealth,"\sigma"]\\ 
        \mathbb{C}^m \rar[-Stealth,hook] & \mathbb{C}^n
    \end{tikzcd}
\]  
, where $\mathbb{C}^m$, $\mathbb{P}^{n-m-1}$, $\mathrm{Bl}_{\mathbb{C}^m}\left(\mathbb{C}^n\right)$, $\mathbb{C}^n$ are the center of the blow-up, the exceptional divisor,  the \textbf{incidence variety} and the whole space, respectively.

Moreover, the projection $\sigma: \mathrm{Bl}_{\mathbb{C}^m}\left(\mathbb{C}^n\right) \rightarrow \mathbb{C}^n$ is an isomorphism over $\mathbb{C}^n \backslash$ $\mathbb{C}^m$ , that is the original space structure in $\mathbb{C}^n\backslash\mathbb{C}^m$ is invariable under the projection and $\sigma^{-1}\left(\mathbb{C}^m\right)$ is canonically isomorphic to $\mathbb{P}\left(\mathcal{N}_{\mathbb{C}^m / \mathbb{C}^n}\right)$, where the normal bundle $\mathcal{N}_{\mathbb{C}^m / \mathbb{C}^n}$ is canonically isomorphic to the trivial bundle $\mathbb{C}^m \times \mathbb{C}^{n-m}$ over $\mathbb{C}^m$.
\end{example}

\begin{remark}
首先,$\sigma: \mathrm{Bl}_{\mathbb{C}^m}\left(\mathbb{C}^n\right) \rightarrow \mathbb{C}^n$ 是Blow-up构造的投影映射,它将Blow-up空间映射回原始的$\mathbb{C}^n$空间。

$\sigma^{-1}\left(\mathbb{C}^m\right)$ 是$\mathbb{C}^m$在Blow-up空间中的原像,也就是Blow-up过程中被“吹开”的部分。这部分空间是由所有通过$\mathbb{C}^m$中点的直线构成的,这些直线在射影空间$P^{n-m-1}$中可以被参数化。

$\mathbb{P}\left(\mathcal{N}_{\mathbb{C}^m / \mathbb{C}^n}\right)$ 是$\mathbb{C}^m$在$\mathbb{C}^n$中的法向量束的射影化。法向量束$\mathcal{N}_{\mathbb{C}^m / \mathbb{C}^n}$是在$\mathbb{C}^m$上的向量束,它的纤维是$\mathbb{C}^m$在$\mathbb{C}^n$中的法向量空间。射影化的过程是将每个纤维视为一个射影空间。

$\mathcal{N}_{\mathbb{C}^m / \mathbb{C}^n}$被认为是平凡的向量束$\mathbb{C}^m \times \mathbb{C}^{n-m}$,这是因为$\mathbb{C}^m$是$\mathbb{C}^n$的线性子空间,所以它的法向量空间在每一点都是相同的,即$\mathbb{C}^{n-m}$。

所以,这个陈述的含义是:Blow-up过程中被“吹开”的部分,即$\sigma^{-1}\left(\mathbb{C}^m\right)$,在几何上等同于$\mathbb{C}^m$的法向量束的射影化。这个等同是通过将每个点的“吹开”方向与其法向量空间中的方向相对应来实现的。这种对应关系是自然的,因为在Blow-up过程中,我们正是在每个点引入了描述通过该点的所有可能方向的新坐标。
\end{remark}


\subsection{构建一般复流形上的Blow-ups}

Let us now construct the blow-up of an arbitrary complex manifold $X$ of dimension $n$ along an arbitrary submanifold $Y \subset X$ of dimension $m$. Here we have a natural inclusion map : $\iota\colon Y\to X$.

In order to do so, we choose an atlas $X=\bigcup U_i, \varphi_i: U_i \cong \varphi_i\left(U_i\right) \subset \mathbb{C}^n$ such that $\varphi_i\left(U_i \cap Y\right)=\varphi\left(U_i\right) \cap \mathbb{C}^m$.

Let $\sigma: \mathrm{Bl}_{\mathbb{C}^m}\left(\mathbb{C}^n\right) \rightarrow \mathbb{C}^n$ be the blow-up of $\mathbb{C}^n$ along $\mathbb{C}^m$ as constructed in the example above and denote by $\sigma_i: Z_i \rightarrow \varphi_i\left(U_i\right)$ its restriction to the open subset $\varphi_i\left(U_i\right) \subset \mathbb{C}^n$, i.e. $Z_i=\sigma^{-1}\left(\varphi_i\left(U_i\right)\right)$ and $\sigma_i=\left.\sigma\right|_{Z_i}$. We shall prove that all the blow-ups on the various charts $\varphi_i\left(U_i\right)$ naturally glue.

\[
    \begin{tikzcd}
        \operatorname{Bl}_{\mathbb{C}^m}(\mathbb{C}^n)\rar[-Stealth,"\sigma"] & \mathbb{C}^n  & X\lar[-Stealth,"\varphi"]& \lar[-Stealth,"\sigma"] \operatorname{Bl}_{Y}(X)\\ 
        \mathbb{P}^{n-m-1}\uar[-Stealth,hook] \rar[-Stealth] &\mathbb{C}^m \uar[-Stealth,hook]&Y\lar[-Stealth,"\varphi_i"] \uar[-Stealth,hook]&\mathbb{P}^{n-m-1}\uar[-Stealth,hook]\lar[-Stealth]
    \end{tikzcd}
\]

Consider arbitrary open subsets $U, V \subset \mathbb{C}^n$ and a biholomorphic map $\phi: U \cong V$ with the property that $\phi\left(U \cap \mathbb{C}^m\right)=V \cap \mathbb{C}^m$. Write $\phi=\left(\phi^1, \ldots, \phi^n\right)$. Then for $k>m$ one has $\phi^k=\sum_{j=m+1}^n z_j \phi_{k, j}$ (see the proof of Proposition 2.4.7). Let us define the biholomorphic map $\hat{\phi}: \sigma^{-1}(U) \cong \sigma^{-1}(V)$ as
$$
\hat{\phi}(x, z):=\left(\left(\phi_{k, j}(z)\right)_{k, j=m+1, \ldots, n} \cdot x, \phi(z)\right) .
$$

It is straightforward to check that $\hat{\phi}(x, z)$ is indeed contained in the incidence variety. In order to obtain the global blow-up $\sigma: \mathrm{Bl}_{\mathbb{C}^m}\left(\mathbb{C}^n\right) \rightarrow X$ we have to ensure that these gluings are compatible. This is clear over $X \backslash Y$. Over $Y$ the matrices we obtain for every $\left.\phi_{i j}\right|_{\mathbb{C}^m}$ are by definition the cocycle of the normal bundle $\mathcal{N}_{Y / X}$. Thus, they do satisfy the cocycle condition. At the same time this proves that $\sigma^{-1}(Y) \cong \mathbb{P}\left(\mathcal{N}_{Y / X}\right)$.


\subsection{Dolbeault blow-up formula}

\begin{lemma}
    There exists an open neighborhood $\mU$ of $Y$ in $X$ such that $i^*\colon \mA^{p,q}(\mU)\to\mA^{p,q}(Y)$ is surjective, where $i:=\iota|_{\mU}\colon Y\hookrightarrow\mU $  is the inclusion and 
    \[
        \mA^{p,\bullet}(X,Y):=\{\alpha\in\mA^{p,\bullet}(X)\mid i^*\alpha=0\}=\ker (i^*).
    \]
\end{lemma}

\begin{lemma}[][][lem:sur]

    The pullback $\iota^*\colon \mA^{p,q}(X)\to\mA^{p,q}(Y)$ is surjective. Here $\iota\colon Y\hookrightarrow X$  is an inclusion.
\end{lemma}
Thus $j$ is an inclusion, which is naturally injective and $\iota^*$ is surjective by lemma   \ref{lem:sur}. Now we then obtains a short exact sequence for the pair $(X,Y)$ of complexes
 \[
    \begin{tikzcd}
        0\rar[-Stealth]  & \mA^{p,\bullet}(X,Y) \rar[-Stealth,"j"] & \mA^{p,\bullet}(X)\rar[-Stealth,"\iota^*"] &  \mA^{p,\bullet}(Y) \rar[-Stealth]  &0.
    \end{tikzcd}
 \]

The blow-up $\widehat{X}$ of $x$ with the center $Z$ is a projective morphism $\pi\colon \widehat{X}\to X$ such that 
\[
    \pi\colon \widehat{X}- E\stackrel{\cong}{\tikz\draw[thick,-Stealth] (0,0)--+(1cm,0);} X-Z
\]
is a biholomorphism by \cite[Proposition 2.5.3]{Huybrechts2004ComplexGA}. Here 
\[E:= \pi^{-1}(Z)\cong \bP(\mN_{Z/X})\]
is the \textit{exceptional divisor} of the blow-up. Then one has the following blow-up diagram
\[
    \begin{tikzcd}
        E\dar["\pi_E",swap,-Stealth]\rar[-Stealth,hook,"\tilde{\iota}"] &\widehat{X}\dar[-Stealth,"\pi"]\\ 
        Z \rar[-Stealth,"\iota",hook]& X.
    \end{tikzcd}
\]
, which then induces a natural commutative diagram for the short exact sequences of complexes 
\[
    \begin{tikzcd}
        0\rar[-Stealth]  & \mA^{p,\bullet}(X,Z) \dar[-Stealth,"\pi^*"]\rar[-Stealth] & \mA^{p,\bullet}(X)\dar[-Stealth,"\pi^*"]\rar[-Stealth] &  \mA^{p,\bullet}(Z) \rar[-Stealth] \dar[-Stealth,"\pi_E^*"] &0\\
        0\rar[-Stealth]  & \mA^{p,\bullet}(\widehat{X},E) \rar[-Stealth] & \mA^{p,\bullet}(\widehat{X})\rar[-Stealth] &  \mA^{p,\bullet}(E) \rar[-Stealth]  &0
    \end{tikzcd}
\]
and a commutative diagram for the long exact sequences
\[
    \begin{tikzcd}
        \cdots\rar[-Stealth]  & H_{\bdd}^{p,q}(X,Z) \dar[-Stealth,"\pi^*"]\rar[-Stealth] & H_{\bdd}^{p,q}(X)\dar[-Stealth,"\pi^*"]\rar[-Stealth] &  H_{\bdd}^{p,q}(Z) \rar[-Stealth,"\delta"] \dar[-Stealth,"\pi_E^*"] & H_{\bdd}^{p,q+1}(X,Z) \dar[-Stealth,"\pi^*"]\rar[-Stealth] &\cdots\\
        \cdots\rar[-Stealth]  & H_{\bdd}^{p,q}(\widehat{X},E) \rar[-Stealth] & H_{\bdd}^{p,q}(\widehat{X})\rar[-Stealth] &  H_{\bdd}^{p,q}(E) \rar[-Stealth,"\overline{\delta}"]  & H_{\bdd}^{p,q+1}(\widehat{X},E) \rar[-Stealth] &\cdots
    \end{tikzcd}
\]
where $\delta,\overline{\delta}$ are the corresponding coboundary operators. 













