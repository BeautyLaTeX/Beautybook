\documentclass[12pt,twoside]{book}
\overfullrule=1pt
\usepackage[fontsize=13.5pt]{fontsize}
\usepackage[a4paper,left=2.2cm, right=2.2cm, top=2cm, bottom=2cm]{geometry}
\usepackage{tikz}
\usepackage{fontawesome5}
\usepackage[colorlinks,linkcolor = blue,anchorcolor = blue,urlcolor = blue,citecolor = red]{hyperref}
\usepackage{times}
\usepackage{graphicx} % Required for inserting images
\title{Constructing families of abelian varieties  of $\text{GL}_2$-type over 4 punctured complex projective line  via $p$-adic Hodge theorem and Langlands correspondence and application to algebraic solutions of Painleve VI equation}
\makeatletter
\usepackage{fancyhdr}
\pagestyle{fancy}
\fancyhf{}
\fancyhead[EL]{\thepage}
\fancyhead[ER]{Jinbang Yang, Kang zuo}
\fancyhead[OL]{Constructing families of abelian varieties...}
\fancyhead[OR]{\thepage}
\fancyfoot[EL]{\includegraphics[scale=0.4]{brand.jpg}}
\fancyfoot[OR]{\includegraphics[scale=0.4]{brand.jpg}}
\setlength\headwidth{\textwidth}
\renewcommand{\headrulewidth}{0.5mm} %页眉线宽,设为0可以去页眉线
% \renewcommand{\footrulewidth}{0.1mm} %页脚线宽,设为0可以去页眉线
\makeatother
\usepackage{varwidth}
\usepackage{lipsum}
\setlength{\headheight}{14.49998pt}
\setlength{\footskip}{64.57396pt}
\usepackage{amsmath,amsthm,amsfonts}
\usepackage{mtpro2}
\newcommand\R{\mathbb{R}}
\usepackage{physics}
\newenvironment{solution}{\begin{proof}[Solution]}{\end{proof}}
\usepackage[titles]{tocloft}
\setlength{\cftsecindent}{0em}
\setlength{\cftsecnumwidth}{1.5em}
\setlength{\cftsubsecindent}{1em}
\setlength{\cftsubsecnumwidth}{1.8em}
\makeatletter
\renewcommand{\@dotsep}{1.8}
\renewcommand{\tableofcontents}{%
\section*{Contents}
\vspace*{-5pt}%
\@starttoc{toc}}
\makeatother

\linespread{1.05}
\usepackage{aliascnt}
\usepackage{amscd}
\usepackage{appendix}
\usepackage{bm}
\usepackage{color}
\usepackage[shortlabels]{enumitem}
\usepackage{float}
%%%%%%H-N %%%%%%
\usepackage{indentfirst}
\usepackage{latexsym}
\usepackage{mathrsfs}
\usepackage{mathtools}
\usepackage{multirow}
%%%%%%O-T %%%%%%
\usepackage{pstricks}
\usepackage{rotating}
\usepackage{changepage}
%%%%%%U-Z %%%%%%
%\usepackage{ulem} %下划线标重点,而非斜体%
\usepackage{url}
\usepackage{xcolor}
\usepackage[all,poly,knot]{xy}
%-------------------- end usepackage -----------------------
%%%%%%%%%%%%%%%%%%%%%%%%%%%%%%定理格式 %%%%%%%%%%%%%%%%%%%%%
\theoremstyle{plain}
%--------------------------------------
\newtheorem{theorem}{Theorem}[section]
\newtheorem*{mainthm}{Main Theorem}
%--------------------------------------
\newtheorem*{acknowledgement*}{\protect \acknowledgementname}
\providecommand{\acknowledgementname}{Acknowledgement}
%--------------------------------------
\newaliascnt{setup}{theorem}
\newtheorem{setup}[setup]{Setup}
\aliascntresetthe{setup}
\providecommand*{\setupautorefname}{Setup}
%--------------------------------------
\newaliascnt{question}{theorem}
\newtheorem{question}[question]{Question}
\aliascntresetthe{question}
\providecommand*{\questionautorefname}{Question}
%--------------------------------------
\newaliascnt{lemma}{theorem}
\newtheorem{lemma}[lemma]{Lemma}
\aliascntresetthe{lemma}
\providecommand*{\lemmaautorefname}{Lemma}
%--------------------------------------
\newaliascnt{assumption}{theorem}
\newtheorem{assumption}[assumption]{Assumption}
\aliascntresetthe{assumption}
\providecommand*{\lemmaautorefname}{Assumption}
%--------------------------------------
\newaliascnt{conjecture}{theorem}
\newtheorem{conjecture}[conjecture]{Conjecture}
\aliascntresetthe{conjecture}
\providecommand*{\conjectureautorefname}{Conjecture}
%--------------------------------------
\newaliascnt{proposition}{theorem}
\newtheorem{proposition}[proposition]{Proposition}
\aliascntresetthe{proposition}
\providecommand*{\propositionautorefname}{Proposition}
%--------------------------------------
\newaliascnt{corollary}{theorem}
\newtheorem{corollary}[corollary]{Corollary}
\aliascntresetthe{corollary}
\providecommand*{\corollaryautorefname}{Corollary}
%--------------------------------------
\newaliascnt{problem}{theorem}
\newtheorem{problem}[problem]{Problem}
\aliascntresetthe{problem}
\providecommand*{\problemautorefname}{Problem}
%--------------------------------------
\newaliascnt{claim}{theorem}
\newtheorem{claim}[claim]{Claim}
\aliascntresetthe{claim}
\providecommand*{\claimautorefname}{Claim}
%%%%%%%%%%%%%%%%$ Text roman 定义格式 %%%%%%%%%%%%%%%%%%%%%%%
\theoremstyle{definition}
%--------------------------------------
\newaliascnt{definition}{theorem}
\newtheorem{definition}[definition]{Definition}
\aliascntresetthe{definition}
\providecommand*{\definitionautorefname}{Definition}
%--------------------------------------
\newaliascnt{example}{theorem}
\newtheorem{example}[example]{Example}
\aliascntresetthe{example}
\providecommand*{\exampleautorefname}{Example}
%%%%%%%%%%%%%%%%%%%%Text roman 注记格式 %%%%%%%%%%%%%%%%%%%%
\theoremstyle{remark}
%--------------------------------------
\newaliascnt{remark}{theorem}
\newtheorem{remark}[remark]{Remark}
\aliascntresetthe{remark}
\providecommand*{\remarkautorefname}{Remark}
%--------------------------------------
\newaliascnt{remarks}{theorem}
\newtheorem{remarks}[remarks]{Remarks}
\aliascntresetthe{remarks}
\providecommand*{\remarksautorefname}{Remarks}
%-------------------- 定理格式结束 --------------------------
%%%%%%%%%%%%%%%%%%%%%%%%%%begin list of symbols %%%%%%%%%%%%
%---------------------------------
\newcommand{\fraka}{{\mathfrak a}}
\newcommand{\frakb}{{\mathfrak b}}
\newcommand{\frakc}{{\mathfrak c}}
\newcommand{\frakd}{{\mathfrak d}}
\newcommand{\frake}{{\mathfrak e}}
\newcommand{\frakf}{{\mathfrak f}}
\newcommand{\frakg}{{\mathfrak g}}
\newcommand{\frakh}{{\mathfrak h}}
\newcommand{\fraki}{{\mathfrak i}}
\newcommand{\frakj}{{\mathfrak j}}
\newcommand{\frakk}{{\mathfrak k}}
\newcommand{\frakl}{{\mathfrak l}}
\newcommand{\frakm}{{\mathfrak m}}
\newcommand{\frakn}{{\mathfrak n}}
\newcommand{\frako}{{\mathfrak o}}
\newcommand{\frakp}{{\mathfrak p}}
\newcommand{\frakq}{{\mathfrak q}}
\newcommand{\frakr}{{\mathfrak r}}
\newcommand{\fraks}{{\mathfrak s}}
\newcommand{\frakt}{{\mathfrak t}}
\newcommand{\fraku}{{\mathfrak u}}
\newcommand{\frakv}{{\mathfrak v}}
\newcommand{\frakw}{{\mathfrak w}}
\newcommand{\frakx}{{\mathfrak x}}
\newcommand{\fraky}{{\mathfrak y}}
\newcommand{\frakz}{{\mathfrak z}}
%---------------------------------
%---------------------------------
\newcommand{\frakA}{{\mathfrak A}}
\newcommand{\frakB}{{\mathfrak B}}
\newcommand{\frakC}{{\mathfrak C}}
\newcommand{\frakD}{{\mathfrak D}}
\newcommand{\frakE}{{\mathfrak E}}
\newcommand{\frakF}{{\mathfrak F}}
\newcommand{\frakG}{{\mathfrak G}}
\newcommand{\frakH}{{\mathfrak H}}
\newcommand{\frakI}{{\mathfrak I}}
\newcommand{\frakJ}{{\mathfrak J}}
\newcommand{\frakK}{{\mathfrak K}}
\newcommand{\frakL}{{\mathfrak L}}
\newcommand{\frakM}{{\mathfrak V}}
\newcommand{\frakN}{{\mathfrak N}}
\newcommand{\frakO}{{\mathfrak O}}
\newcommand{\frakP}{{\mathfrak P}}
\newcommand{\frakQ}{{\mathfrak Q}}
\newcommand{\frakR}{{\mathfrak R}}
\newcommand{\frakS}{{\mathfrak S}}
\newcommand{\frakT}{{\mathfrak T}}
\newcommand{\frakU}{{\mathfrak U}}
\newcommand{\frakV}{{\mathfrak V}}
\newcommand{\frakW}{{\mathfrak W}}
\newcommand{\frakX}{{\mathfrak X}}
\newcommand{\frakY}{{\mathfrak Y}}
\newcommand{\frakZ}{{\mathfrak Z}}
%---------------------------------
%---------------------------------
\newcommand{\bA}{{\mathbb A}}
\newcommand{\bB}{{\mathbb B}}
\newcommand{\bC}{{\mathbb C}}
\newcommand{\bD}{{\mathbb D}}
\newcommand{\bE}{{\mathbb E}}
\newcommand{\bF}{{\mathbb F}}
\newcommand{\bG}{{\mathbb G}}
\newcommand{\bH}{{\mathbb H}}
\newcommand{\bI}{{\mathbb I}}
\newcommand{\bJ}{{\mathbb J}}
\newcommand{\bK}{{\mathbb K}}
\newcommand{\bL}{{\mathbb L}}
\newcommand{\bM}{{\mathbb M}}
\newcommand{\bN}{{\mathbb N}}
\newcommand{\bO}{{\mathbb O}}
\newcommand{\bP}{{\mathbb P}}
\newcommand{\bQ}{{\mathbb Q}}
\newcommand{\bR}{{\mathbb R}}
\newcommand{\bS}{{\mathbb S}}
\newcommand{\bT}{{\mathbb T}}
\newcommand{\bU}{{\mathbb U}}
\newcommand{\bV}{{\mathbb V}}
\newcommand{\bW}{{\mathbb W}}
\newcommand{\bX}{{\mathbb X}}
\newcommand{\bY}{{\mathbb Y}}
\newcommand{\bZ}{{\mathbb Z}}
%---------------------------------
%---------------------------------
\newcommand{\mA}{{\mathcal A}}
\newcommand{\mB}{{\mathcal B}}
\newcommand{\mC}{{\mathcal C}}
\newcommand{\mD}{{\mathcal D}}
\newcommand{\mE}{{\mathcal E}}
\newcommand{\mF}{{\mathcal F}}
\newcommand{\mG}{{\mathcal G}}
\newcommand{\mH}{{\mathcal H}}
\newcommand{\mI}{{\mathcal I}}
\newcommand{\mJ}{{\mathcal J}}
\newcommand{\mK}{{\mathcal K}}
\newcommand{\mL}{{\mathcal L}}
\newcommand{\mM}{{\mathcal V}}
\newcommand{\mN}{{\mathcal N}}
\newcommand{\mO}{{\mathcal O}}
\newcommand{\mP}{{\mathcal P}}
\newcommand{\mQ}{{\mathcal Q}}
\newcommand{\mR}{{\mathcal R}}
\newcommand{\mS}{{\mathcal S}}
\newcommand{\mT}{{\mathcal T}}
\newcommand{\mU}{{\mathcal U}}
\newcommand{\mV}{{\mathcal V}}
\newcommand{\mW}{{\mathcal W}}
\newcommand{\mX}{{\mathcal X}}
\newcommand{\mY}{{\mathcal Y}}
\newcommand{\mZ}{{\mathcal Z}}
%---------------------------------
%---------------------------------
\newcommand{\sA}{{\mathscr A}}
\newcommand{\sB}{{\mathscr B}}
\newcommand{\sC}{{\mathscr C}}
\newcommand{\sD}{{\mathscr D}}
\newcommand{\sE}{{\mathscr E}}
\newcommand{\sF}{{\mathscr F}}
\newcommand{\sG}{{\mathscr G}}
\newcommand{\sH}{{\mathscr H}}
\newcommand{\sI}{{\mathscr I}}
\newcommand{\sJ}{{\mathscr J}}
\newcommand{\sK}{{\mathscr K}}
\newcommand{\sL}{{\mathscr L}}
\newcommand{\sM}{{\mathscr V}}
\newcommand{\sN}{{\mathscr N}}
\newcommand{\sO}{{\mathscr O}}
\newcommand{\sP}{{\mathscr P}}
\newcommand{\sQ}{{\mathscr Q}}
\newcommand{\sR}{{\mathscr R}}
\newcommand{\sS}{{\mathscr S}}
\newcommand{\sT}{{\mathscr T}}
\newcommand{\sU}{{\mathscr U}}
\newcommand{\sV}{{\mathscr V}}
\newcommand{\sW}{{\mathscr W}}
\newcommand{\sX}{{\mathscr X}}
\newcommand{\sY}{{\mathscr Y}}
\newcommand{\sZ}{{\mathscr Z}}
%---------------------------------
%------------------------- end list of symbols --------------
%%%%%%%%%%%%%%%%%%%begin math operator %%%%%%%%%%%%%%%%%%%%%%
%---------------------------------
%\DeclareMathOperator ---> amsopn ---> amsmath
\DeclareMathOperator\Aut{Aut}
\DeclareMathOperator\Div{Div}
\DeclareMathOperator\cech{\check Cech}
\DeclareMathOperator\Char{char}
\DeclareMathOperator\rmd{d}
\DeclareMathOperator\End{End}
\DeclareMathOperator\mEnd{\mathcal End}
\DeclareMathOperator\Fil{Fil}
\DeclareMathOperator\Frac{Frac}
\DeclareMathOperator\Gal{Gal}
\DeclareMathOperator\GL{GL}
%\DeclareMathOperator\gcd{gcd}
\DeclareMathOperator\gl{mathscr{gl}}
\DeclareMathOperator\Gr{Gr}
\DeclareMathOperator\Hom{Hom}
\DeclareMathOperator\mHom{\mathcal Hom}
\DeclareMathOperator\id{id}
\DeclareMathOperator\lcm{lcm}
\DeclareMathOperator\Pic{Pic}
\DeclareMathOperator\PGL{PGL}
\DeclareMathOperator\res{res}
%\DeclareMathOperator\sl{\mathscr{sl}}
\DeclareMathOperator\SL{SL}
\DeclareMathOperator\Spec{Spec}
\DeclareMathOperator\Spf{Spf}
\DeclareMathOperator\Sp{Sp}
\DeclareMathOperator\Spa{Spa}
%---------------------------------
\newcommand{\dR}{\mathrm{dR}}
\newcommand{\crys}{\mathrm{crys}}
\newcommand{\et}{\mathrm{et}}
\newcommand{\MF}{\mathcal{MF}}
\newcommand{\MCF}{\mathrm{MCF}}
\newcommand{\MIC}{\mathrm{MIC}}
\newcommand{\HDF}{\mathrm{HDF}}
\newcommand{\HIG}{\mathrm{HIG}}
\DeclareMathOperator{\FIsoc}{F-Isoc}
\newcommand{\FIsocd}{\FIsoc^\dagger}
\DeclareMathOperator{\Crys}{Crys}
\DeclareMathOperator{\FCrys}{F-Crys}
\DeclareMathOperator{\logCrys}{\log-Crys}
\DeclareMathOperator{\logFCrys}{\log-F-Crys}
%------------------------- end math operator ----------------
%%%%%%%%%%%begin notion in number theory %%%%%%%%%%%%%%%%%%%%
\newcommand{\Zbar}{{\overline{\mathbb Z}}}
\newcommand{\Zp}{{\mathbb{Z}_p}}
\newcommand{\Zq}{{\mathbb{Z}_q}}
\newcommand{\Zpbar}{{\overline{\mathbb Q}_p}}
\newcommand{\Zl}{{\mathbb{Z}_\ell}}
%-----------------------------
\newcommand{\Fp}{{\mathbb{F}_p}}
\newcommand{\Fq}{{\mathbb{F}_q}}
\newcommand{\Fpbar}{{\overline{\mathbb F}_p}}
%-----------------------------
\newcommand{\Qbar}{{\overline{\mathbb Q}}}
\newcommand{\Qp}{{\mathbb{Q}_p}}
\newcommand{\Qq}{{\mathbb{Q}_q}}
\newcommand{\Qpbar}{{\overline{\mathbb Q}_p}}
%-----------------------------
\newcommand{\Ql}{{\mathbb{Q}_{\ell}}}
\newcommand{\Qlbar}{{\overline{\mathbb{Q}}_{\ell}}}

\numberwithin{equation}{section}
%%%%%%%%%%%%%%%%%%%%本笔记中的专属记号 %%%%%%%%%%%%%%%h=half %%%%%%%%%%%%%%
\def\FIsoch{{\FIsoc^{\dagger{1\over 2}}_{\lambda} }}
\def\FIsochf{{\FIsoc^{\dagger{1\over 2}}_{\lambda,f} }}
\def\HDFh{\mathrm{HDF}^{{1\over 2}}_{\lambda}}
\def\High{{\mathrm{HIG}^{{\rm gr}\,{1\over 2}}_{\lambda}}}
\def\Loch{{\mathrm{LOC}^{\ell,{1\over 2}}_{\lambda}}}
\def\Lochf{{\mathrm{LOC}^{\ell,{1\over 2}}_{\lambda,f}}}
\def\MdRh{{M^{{1\over 2}}_{{\rm dR}\,\lambda}}}

\def\MFh{\MF^{{1\over 2}}_{\lambda}}
\def\MFhf{\MF^{{1\over 2}}_{\lambda,f}}

\def\PHDFh{\mathrm{PHDF}^{{1\over 2}}_{\lambda}}
\def\PHDFhf{\mathrm{PHDF}^{{1\over 2}}_{\lambda,f}}
\def\PHigh{{\mathrm{PHIG}^{{\rm gr}\,{1\over 2}}_{\lambda}}}
\def\PHighf{{\mathrm{PHIG}^{{\rm gr}\,{1\over 2}}_{\lambda,f}}}
%-----------------------------------------
\def\Flow{\mathrm{Flow}}
\def\Frob{\mathrm{Frob}}
\def\EK{{\mathbb E}}
\def\QP{{\mathrm{QP}}}
\def\P{{\mathrm{P}}}
\def\DR{{\mathrm{DR}}}
%------------------------------------------
\def\BBone{{(\mathbb P^1_k,D_k)/k}}
\def\BBtwo{{(\mathbb P^1_{W_2(k)}, D_{W_2(k)})/W_2(k)}}
\def\BBn{{(\mathbb P^1_{W_n(k)}, D_{W_n(k)})/W_n(k)}}
\def\bark{{\overline k}_0}
%-----------------------------------------
\def\barFp{{\overline{\mathbb F}_p}}
\def\barQp{{\overline{\mathbb Q}_p}}
\setcounter{secnumdepth}{3} 
\begin{document}
\begin{titlepage}
\begin{tikzpicture}[remember picture,overlay]
%\node[] at ([shift={(5.8cm,1.4cm)}]current page.south west) {published online: 20 March 2023};
\node[] at ([shift={(-3cm,1.5cm)}]current page.south east){\includegraphics[scale=0.4]{brand.jpg}};
\end{tikzpicture}
\vspace*{-5em}
\begin{flushleft}
\begin{minipage}[b]{.88\linewidth}
Invent. math \\ 
\href{https://doi.org/10.1007/s00222-023-01182-9}{https://doi.org/10.1007/s00222-023-01182-9}
\end{minipage}
\hfill
\begin{minipage}[b]{.1\linewidth}
\raisebox{-0.1cm}{\includegraphics[width=1.3cm]{removed-background.png}}
\end{minipage}
\\[-0.5em]
\tikz[overlay]\draw[line width=1.2pt,black] (0,0) --++(\linewidth,0);\\[3em]
\makeatletter
{\Large\bfseries
\@title\\[1em]
}{\large\bfseries
Jinbang Yang$^1$  
\raisebox{.25em}{\tikz\draw[fill=black] (0,0) circle(2pt);}
Kang Zuo$^2$
}\\
\vspace{4em}
\makeatother
{\footnotesize Received: 16 November 2020 \/ Accepted: 31 January 2023\\
© The Author(s), under exclusive licence to Springer-Verlag GmbH Germany, part of\\
Springer Nature 2023}\\[1.5em]

\begin{varwidth}{\linewidth}
\textbf{Abstract}\hspace{0.2em}
This paper fills in all details in the announcement \cite{YaZu23a} on our results:
we construct infinitely many non-isotrivial families of abelian varieties of $\text{GL}_2$-type over  four punctured projective lines with bad reduction of type-$(1/2)_\infty$ via $p$-adic Hodge theory and Langlands correspondence. They lead to algebraic solutions of Painleve VI equation. Recently Lin-Sheng-Wang  proved  the conjecture on the torsioness  of zeros of Kodaira-Spencer maps of those type families.  Based on their theorem we show the set of those type families of  abelian varieties is
{\sl exactly } parameterized by torsion sections of the universal family of elliptic curves   modulo the involution.
\end{varwidth}

\vspace{5em}
\vfill
%\tikz[overlay]\draw[thin] (0,0)--++(.3\linewidth,0);\\
%The authors were supported by the ANR Grant DynGeo ANR-11-BS01-013. They
%acknowledge support from U.S. National Science Foundation Grants DMS 1107452, 1107263,
%1107367 “RNMS: Geometric structures And Representation varieties” (the GEAR Network).
%F. L. appreciates the support of the Mathematical Sciences Research Institute during the fall of
%2019 (NSF DMS-1440140) as well the support of the Institut Universitaire de France.\\[-0.8em]
%\tikz[overlay]\draw[thin] (0,0)--++(.3\linewidth,0);\\[0.5em]

\faEnvelope[regular]\hspace{0.2em} \href{mailto:yjb@mail.ustc.edu.cn}{yjb@mail.ustc.edu.cn} \\[0.5em]
1\hspace{0.8em} School of Mathematical Sciences, University of Science and Technology of China, Hefei, Anhui 230026, PR China\\
\faEnvelope[regular]\hspace{0.2em}  \href{mailto:zuok@uni-mainz.de}{zuok@uni-mainz.de} \\[0.5em]
2\hspace{0.8em} School of Mathematics and Statistics, Wuhan University, Luojiashan, Wuchang, Wuhan, Hubei, 430072, P.R. China.; Institut f\"ur Mathematik, Universit\"at Mainz, Mainz 55099, Germany
\end{flushleft}
\end{titlepage}
\let\cleardoublepage\clearpage
\setcounter{tocdepth}{2}
\tableofcontents\let\cleardoublepage\clearpage
\thispagestyle{fancy}
\newpage
\renewcommand\thesection{\arabic {section}}
\section{\bf Holomorphic functions of several Variables}
\subsection{Hartogs' Phenomenon and Hartogs' Theorem}
\begin{example}
Let 
\begin{align*}
G =&\{(z,w)\in \mathbb{C}^2 : |z|<1,\beta<1,\beta <|w|<1\}\\
&\bigcup \{(z,w)\in\mathbb{C}^2 : |z|<\alpha<1,|w|<1\}.
\end{align*}
then every holomorphic function on $G$ can be expanded to double cyclindrical domain $\{(z,w)\in \mathbb{C}^2 \mid  |z|<1,|w|<1\}$.
\end{example}
\begin{solution}
Now let we see the Figure 1 and let $S$ denote the shadow part of $\R^2$. Define map $\varphi$ by
\[\varphi\colon \mathbb{C}^2 \to \R^2 ; (z,w)\mapsto (|z|,|w|),\]
then $G=\varphi^{-1}(S)$. Next we will show that the conclusion above is true.
\paragraph{Setp I. Taking Laurent expansion of $f(z,w)$.}

For every fixed $|z|<1$, $f(z,w)$ can be expressed as Laurent series  
\[f(z,w)=\sum_{v=-\infty}^{+\infty}a_v (z)w^v.\]
,where $a_v (z)$ is holomorphic on $D_z(0,1)$. For fixed $|z|<1$, $\varphi^{-1}$ transfers into an single variable complex function $\varphi^{-1}_{|z|<1}(|w|)$, $ |w|<1$, then $f$ is holomorphic  about $w$
and posses Laurent series expansion on $D_w (0,1)$.
Owing to $a_v (z)$ being holomorphic on $D_z (0,1)$, when $|z|<\alpha$, the Laurent series has no term with negative power, in other words, $a_v (z)=0, \forall v<0$.

Because $a_v (z)$ is holomorphic on $D_z(0,1)$ and $a_v (z)=0,\forall v<0$, then according to Identity theorem, we yield $a_v (z)\equiv 0$ on $D_z(0,1)$. It is clear that $f(z,w)$ is holomorphic on $\{(z,w)\in \mathbb{C}^2\mid |z|<1,|w|<1\}$.
\paragraph{Step II. The expression of $f(z,w)$ analytic continuation is obtained by using Cauchy Integral theorem. }

On step I, we have shown that $f(z,w)$ is holomorphic on $\{(z,w)\in \mathbb{C}^2\mid |z|<1,|w|<1\}$, then on step II, our purpose is to ascertain the expression of the expanded function. 

Let $\beta^\prime<\beta<1$. Define a function by using Cauchy Integral theorem, we gain 
\begin{equation}
    \widetilde{f}(z,w)=\frac{1}{2\pi i}\int_{|\xi|=\beta^\prime} \frac{f(z,\xi)}{\xi-w}\dd \xi. 
\end{equation}
Where $\widetilde{f}(z,w)$ is holomorphic function on $\{(z,w)\in \mathbb{C}^2\mid |z|<1,|w|<\beta^\prime\}$. In particular, $\widetilde{f}(z,w)=f(z,w)$ on $\{(z,w)\in \mathbb{C}^2\mid |z|<1,|w|<1\}$. So $\widetilde{f}(z,w)$ dose the expanded function what we find. 
\end{solution}

\begin{lemma}\label{lem:1.1}
    Let $U\subset \mathbb{C}^n$ be an open subset and let $V\subset \mathbb{C} $ be an open neighbourhood of the boundary of $B_\varepsilon(0)\subset \mathbb{C}$. Assume that $f\colon V\times U\to \mathbb{C}$ is a holomorphic function. Then 
    \[g(z):=g(z_1,\cdots,z_n);=\int_{|\xi|=\varepsilon}f(\xi,z_1,\cdots,z_n)\dd \xi\]
    is holomorphic function on $U$.
\end{lemma}

\begin{theorem}[Hartogs' Theorem]
    Suppose $\varepsilon=(\varepsilon_1,\cdots,\varepsilon_n) $ and $\varepsilon^\prime=(\varepsilon_1^\prime,\cdots,\varepsilon_n^\prime)$ are given such that for$\forall i$ one has $\varepsilon_i^\prime <\varepsilon$. If $n>1$ , then any holomorphic map $f\colon  B_\varepsilon (0)\backslash \overline{B_{\varepsilon^\prime}(0)}$ $\to \mathbb{C}$ can be uniformly extended to a holomorphic map $\widetilde{f}: B_\varepsilon(0)\to \mathbb{C}$.
\end{theorem}
\begin{proof}
    Let $\varepsilon=(1,\cdots,1) $ and $\exists\delta>0$ such that 
    \[V=\{z\in \mathbb{C}^n\mid 1-\delta<|z_1|<1,|z_{i\neq 1|<1}\}\bigcup \{z\in \mathbb{C}^n\mid 1-\delta<|z_2|<1,|z_{i\neq 2|<1}\}.\]
    is contained in $B_\varepsilon(0)\backslash\overline{B_{\varepsilon^\prime}(0)}$. So $f$ is holomorphic on $V$. Thus, for any $w := (z_2,\cdots,z_n)$ with $|z_j|<1,j=2,\cdots,n$, there exists a holomorphic function $$f_w(z_1):=f(z_1;z_2,\cdots,z_n)$$ on annulus $1-\delta<|z_1|<1$. 
    \begin{remark}
        For the Lemma \ref{lem:1.1}, $V\subset \mathbb{C}^n$ is open subset and let $\{z\in\mathbb{C}^n\mid 1-\delta<|z_1|<1,|z_{i\neq 1}|<1\}\subset \mathbb{C}$ be an neighbourhood of the boundary of $B_1(0)\subset \mathbb{C}$. Because $f$ is holomorphic on $\{z\in\mathbb{C}^n\mid 1-\delta<|z_1|<1,|z_{i\neq 1}|<1\}\times V$, so $g(z_1;z_2,\cdots,z_n):=f(z_1,\cdots,z_n):=f_w(z_1)$ is holomorphic on $\{z\in\mathbb{C}^n\mid 1-\delta<|z_1|<1,|z_{i\neq 1}|<1\}$. 
    \end{remark}
    Now due to $f_w(z_1)$ is holomorphic on $\{z\in\mathbb{C}^n\mid 1-\delta<|z_1|<1,|z_{i\neq 1}|<1\}\subset \mathbb{C}$, then $f_w(z_1)$ can be expanded to Laurent series by $f_w(z_1)=\sum_{n=-\infty}^{+\infty}a_n (w) z_1 ^n$ with the coefficient $a_n (w)=\frac{1}{2\pi i}\int_{|\xi|=1-\delta/2}\frac{f(\xi)}{\xi^{n+1}}\dd \xi$. 

    By Lemma \ref{lem:1.1} ,$a_n (w)$ is holomorphic for $w$ in the unit polydisc of $\mathbb{C}^{n-1}$.

    On the other hand, the function $f_w\colon z_1\mapsto f_w(z_1)$ is holomorphic on the unit disc for fixed $w$ such that $1-\delta<|z_1|<1$. 
    \begin{remark}
        On above description, we have shown that $f_w\colon z_1\mapsto f_w(z_1)$ is holomorphic on $\{z\in\mathbb{C}^n\mid 1-\delta<|z_1|<1,|z_{i\neq 1}|<1\}$, so it's obviously that $f_w$ is also holomorphic on 
        \begin{align*}
        &\{z\in\mathbb{C}^n\mid 1-\delta<|z_1|<1,1-\delta<|z_2|<1,|z_{i\neq 1,2}|<1\}\\
        &\subset \{z\in\mathbb{C}^n\mid 1-\delta<|z_1|<1,|z_{i\neq 1}|<1\}. \end{align*}
    \end{remark}
    Thus, $a_n (w)=0$, for $n<0$ and $1-\delta<|z_2|<1$. 
    \begin{remark}
        $a_n(w)$ is holomorphic for $w$ on the unit disc of $\mathbb{C}^n$, so $a_n (w)=0, \forall n<0$. 
    \end{remark}
    By the Identity theorem , we show that $a_n(w)\equiv 0$ for $n<0$. But then we define the holomorphic extension $\widetilde{f}$ of $f$ by the power series $\sum_{n=0}^{\infty}a_n (w)z_1^n$. (without terms of negative power)
    \begin{remark}
        Also, we could use the Cauchy Integral theorem, which is equivalent to that.
    \end{remark}
    This series converges uniformly ,as $a_n (w)$ are holomorphic and attain maximum at the boundary. 
    \begin{remark}
        For $\widetilde{f}=\sum_{n=0}^{\infty}a_n(w)z_1^n, 1-\delta<|z_1|<1$, if we want the series converges uniformly, it just need $a_n(w)$ are bounded on $\{z\in\mathbb{C}^n\mid 1-\delta<|z_2|<1,|z_{i\neq 2}|<1\}$. By $a_n(w)$ be holomorphic on $\{z\in\mathbb{C}^n\mid 1-\delta<|z_2|<1,|z_{i\neq 2}|<1\}$, then use Maximum princple, $a_n(w)$ attain their maximum at the boundary.
        \end{remark}
        So the convergence of the Laurent series on the annulus yields the uniformly convergence everywhere. Clearly, the holomorphic function given by the series (the power series) glues with $f$ to give the disired holomorphic function.
\end{proof}
 
\section{\bf Introduction}

Let $R$ be a commutative ring with identity and let $X$ be a scheme over $R$. A \emph{family of varieties} over $X$ of dimension $g$ is a flat morphism $\pi\colon Y\rightarrow X$ of finite type with geometric fibers that are pure, $g$-dimensional, connected, and reduced. For any $x\in X$, the fiber of $\pi$ over $x$ is denoted by $Y_x\coloneqq\pi^{-1}(x)$. An \emph{abelian scheme} (or a \emph{smooth family of abelian varieties}) over $X$ is a smooth, projective family of varieties $f\colon A\rightarrow X$ along with a section $s\colon X\rightarrow A$ such that the fiber $(A_x,s_x)$ forms an abelian variety for each $x\in X$. A \emph{family of abelian varieties} is a projective family of varieties $f\colon A\rightarrow X$ along with a section $s\colon X\rightarrow A$ such that, for some nonempty open dense subset $U\subseteq X$, the restricted family $f\mid_U\colon A\times_XU\rightarrow U$ together with the restricted section $s\mid_U$ form an abelian scheme.

An abelian scheme $A$ over a scheme $X$, together with a polarization $\mu$, is said to be \emph{of $\GL_2$-type} if there exists a number field $\EK$ of degree $\dim_X A$ such that the ring of integers $\mO_\EK$ can be embedded into the endomorphism ring $\End_{\mu}(A/X)$. If we want to emphasize the role of $\EK$, we call $A$ \emph{of $\GL_2(\EK)$-type}. Similarly, a family of abelian varieties over $X$ is said to be \emph{of $\GL_2$-type} if its restriction to the smooth locus is of $\GL_2$-type.

Consider a $\GL_2(\EK)$-type family of abelian varieties $f\colon A\rightarrow X$. Let $D$ denote the discriminant locus and $X^0$ denote the smooth locus, which is the complement of $D$ in $X$. We define $\Delta$ as the inverse image of $D$ under the structure morphism $f$ and $A^0$ as the complement of $\Delta$ in $A$. Then we obtain an abelian scheme $f^0\colon A^0\rightarrow X^0$.

This gives rise to the following commutative diagram:
\begin{equation}
\xymatrix{
A^0\ar[r] \ar[d]^{f^0} & A \ar[d]^f & \Delta \ar[d]\ar[l] \\
X^0\ar[r] & X & D \ar[l] \\
}
\end{equation}

For $R= \bC$ we consider the Betti-local system
\[\bV= R^1_\mathrm{B}f^0_* \bZ_{A^0}\]
attached to $f^0$, which is a $\bZ$-local system over the base $X^0$. Since $f$ is of $\GL_2(\EK)$-type, the action of $\mO_\EK$ on $f$ induces an action of $\EK$ on the $\Qbar$-local system $\bV\otimes\Qbar$. Taking the $\EK$-eigen sheaves decomposition
\[\bV\otimes\Qbar= \bigoplus_{i=1}^g \bL_i.\]
Then these $\bL_i$'s are $\Qbar$-local systems of rank-$2$ over $X^0$ and defined over the ring of integers of some number field. On the other hand, consider the logarithmic de Rham bundle attached to the family of abelian varieties $f$ and denote
\[(V,\nabla)= R^1_\dR f_*\Big(\Omega^*_{A/X}(\log\Delta),\rmd\Big).\]
On this de Rham bundle, there is a canonical filtration satisfying Griffiths transversality given by relative differential $1$-forms
\[E^{1,0}\coloneqq R^0f_* \Omega^1_{A/X}(\log\Delta))\subset V.\]
Taking the grading with respect to this filtration, one gets a logarithmic graded Higgs bundle, which is so-called Kodaira-Spencer map attached to $f$
\begin{equation} \label{equ:KS}
(E,\theta)\coloneqq(E^{1,0} \oplus E^{0,1},\theta)\coloneqq \Gr_{E^{1,0}} (V,\nabla) = \Big(R^0f_*\Omega^1_{A/X}(\log\Delta)\oplus R^1f_*\mO_A,\Gr(\nabla)\Big).
\end{equation}
Since $f$ is of $\GL_2(\EK)$-type, one also gets a $\EK$-eigen decomposition of the Higgs bundle
\begin{equation} \label{equ:decomp_Higgs}
(E,\theta)=\bigoplus_{i=1}^{g} (E,\theta)_i.
\end{equation}
Under Hitchin-Simpson's non-abelian Hodge theory, these eigensheaves $\{(E,\theta)_i\}_{i=1,\cdots,g}$ are just those Higgs bundles correspond to the local systems $\{\bL_i\}_{i=1,\cdots,g}$.

Those  local systems and Higgs bundles above are examples of    motivic local systems and and motivic Higgs bundles.
 Sometimes they are also called \emph{coming from geometry origin}. Simpson had found a characterization for a rank-2 local system being motivic.
\begin{theorem}[Simpson\cite{Sim92}] \label{thm_Simpson}
A rank-2 local system $\mathbb L$ over a smooth complex quasi-projective curve $U$ is an eigen sheaf of
an abelian scheme of $\GL_2$-type over $U$ if and only if the following two conditions hold:
\begin{enumerate}
\item $\mathbb L$ is defined over the ring of integers of some number field, and
\item for each element $\sigma \in \mathrm{Gal}(\overline{\mathbb Q}/\mathbb Q)$
the Higgs bundle corresponding to the Galois conjugation $\mathbb L^\sigma$ is again graded.
\end{enumerate}
\end{theorem}
\begin{conjecture}[Simpson] \label{conj_Simpson}
A rigid local system is motivic.
\end{conjecture}
Simpson and Corlette  \cite{KeSi08}  proved \autoref{conj_Simpson} holds true for rank-$2$ case. They actually showed that a rank-2 rigid local system does satisfy these two properties required in Theorem 1.1.  Another  crucial point is the construction of the polarization from the harmonic metric on the local system.  
Simpson's conjecture for rank-3 case has be proven by Langer-Simpson \cite{LaSi18} for cohomologically rigid local systems. The conjecture predicts that any rigid local system $\bL$ shall enjoy all properties of motivic local systems. For example,
\begin{itemize}
\item its corresponding filtered de Rham bundle is isomorphic to the underlying filtered de Rham bundle of some Fontaine-Faltings modules at almost all places, and
\item if $\bL$ is in addition cohomologically rigid, then it is defined over the ring of integers of some number field.

\end{itemize}
Those two properties have been verified by Esnault-Groechenig recently \cite{EsGr18,EsGr20}.

We propose a program  on searching for loci of motivic Higgs bundles in moduli spaces of semistable  Higgs bundles   with trivial Chern classes on a  given smooth  complex quasi-projective variety $X-D$, though the dimensions of moduli spaces could be positive.

In this note, we make   the first step towards to this program  by taking $X$ as the complex projective line $\bP^1_{\bC}$ and $D$ as the $4$ punctures $\{0,1,\infty,\lambda\}$. In this case the moduli space  of rank-2 semistable Higgs bundles on $\mathbb P^1$ with prescribed parabolic structure on 4-punctures always has positive dimension.
Our goal is looking for the locus of rank-$2$  motivic graded Higgs bundles over $(\bP^1_{\bC},\{0,1,\infty,\lambda\})$
with prescribed parabolic structure at four punctures $\{0, 1,\lambda, \infty\}.$\\[.2cm]
Beauville \cite{Bea82} has shown that there exist exactly 6 non-isotrivial families of elliptic curves over $\bP^1_{\bC}$ with semistable reductions over $\{0,1,\infty,\lambda_i\}$ for ${1\leq i\leq 6}$. All of them are modular curves of  certain mixed level structures. The same statement has also shown by Viehweg-Zuo for families of higher dimension ablelian varieties  on $\mathbb P^1$ with semistable reduction on 4-punctures.
So except Beauville's example any non-isotrivial smooth families of abelian varieties over $\bP^1\setminus\{0,1,\infty,\lambda\}$ of $\GL_2$-type must have non-semistable reduction at some point in $\{0,1,\infty,\lambda\}$. In this case, the some eigenvalues of the local monodromies of motivic local system must be roots of unity other than $1$.

Consider the Legendre family of elliptic curves over $\mathbb P^1$ defined by the equation
\[ y^2=x(x-1)(x-\lambda),\quad \lambda \in \mathbb P^1-\{0,\,1,\infty\}.\]
The family has semistable reduction over $\{0,\, 1\}$ and potentially semistable reduction
over $\{\infty\}$ with local monodromy around $\infty$ of eigenvalues $\{e^{2i\pi \over 2},\, e^{2i\pi \over 2}\}$.
The family  is called having bad reduction at discriminant locus of type $(1/2)_\infty$. Motivated by this example
we are looking for  more families of elliptic curves/abelian varieties over $\mathbb P^1$ with bad reduction on 4-punctures of type $(1/2)_\infty$.

In the paper \cite{SYZ22} we studied rank-2 $p$-adic graded Higgs bundles on 4-punctured $\mathbb P^1$ with parabolic structure on punctures of the prescribed type $(1/2)_\infty$. Motivated by Simpson's theorem on rank-2 motivic Higgs over complex number field we were looking for motivic Higgs bundles   from  those Higgs bundles which are graded Higgs bundles from Fontaine-Faltings module.

By Fontaine-Faltings' work on crystalline local systems and the work by Lan-Sheng-Zuo on Higgs-de Rham flow \cite{LSZ19}, a motivic Higgs bundles must be periodic points of the self map of Higgs-de Rham flow.  We shall point out, the notion of Higgs-de Rham flow has been already introduced in a unpublished paper \cite{ShZu12} by M. Sheng and K. Zuo  for the category of sub Higgs bundles in graded Higgs bundles arising from Fontaine-Faltings modules. Though the main object is the category of sub Higgs bundles in a given graded Higgs bundle from Fontaine-Faltings module, the lifting of inverse Cartier transform on the category of the category of sub Higgs bundles over $W_{n}(k)$ which are periodic (modulo $p^{n-1}$) has been originally constructed in this paper).

In \cite{SYZ22}  one has found the explicit expression of the self map. By  identifying the moduli space  $\text{HIG}^{\text{gr} {1\over 2}}_\lambda $ of Higgs bundles on $\mathbb P^1_k$ with parabolic structure on the punctures $\{0,1,\lambda, \infty\}$ of type-$(1/2)_\infty$ with $\mathbb P^1_k$ then the self map on $\High$ is a polynomial map of degree $p^2$ composed with the Frobenius map, See \cite[appendix A]{SYZ22}.

More mysterious things happen, we define the elliptic curve $C_\lambda$ associated to a 4-punctured $(\mathbb P^1,\{0,1,\lambda, \infty\})$ as the double cover
$$ \pi: C_\lambda\to \mathbb P^1$$ ramified on $\{0,1,\lambda, \infty\}$
and choosing $\infty$ as the origin for group law, we have examined the formula for the self-map for primes $2< p< 50$ and found that the self map on  $\text{HIG}^{\text{gr} {1\over 2}}_\lambda =\mathbb P^1$  coincides with the multiplication by $p$ map on the elliptic curve $C_\lambda$ via $\pi$.  Consequently  if $(E,\theta)$ is period, (i.e. it is the grading of a $p$-torsion Fontaine-Faltings module) if and only if the zero of the Higgs field $(\theta)_0\in \mathbb P^1(\mathbb F_q)$ is the image of a torsion point in $C_\lambda$. See the more detailed discussions after \autoref{thm_main_painleve} and \autoref{conj:SYZ}.

Given an abelian scheme $A$ over $\mathcal O[1/N]$, where $\mO$ is the ring of integers of some number field $K$ and $N$ is a positive integer. Pink's theorem \cite{Pin04} implies that a point $z\in A(K)$ is a torsion point if and only the order of the modulo $\frakp$ reduction $z\pmod{\frakp} \in A(k_{\frakp})$ is bounded above by some number which is independent of the choice of the finite place $\frakp$. It motivates us to make the following conjecture (in a talk held in Lyon by the second named author in April 2018.
\begin{conjecture} [J. Yang and K. Zuo \cite{Lyon-Talk}] \label{Conj_2}
A complex semistable parabolic graded Higgs bundle  of degree $0$ on the projective line with 4-punctures $\{0,1,\lambda,\infty\}$
of parabolic type $(1/2)_\infty$ is motivic if and only if the zero of the Higgs field $(\theta)_0$ is a torsion point in $C_\lambda$.
\end{conjecture}
In particular, \autoref{Conj_2} implies that there exist infinitely many rank-2 motivic Higgs bundles on any complex 4-punctured $\mathbb P^1$ of parabolic type $(1/2)_\infty$.

J. Lu, X. Lv and J.B. Yang have found 26 families of complex elliptic curves on $\mathbb P^1$ with bad reductions on $\{0,1, \lambda, \infty\}$ such that the zero of Kodaira-Spencer maps are torsion of order 1, 2, 3, 4 and 6. \cite{LLY} by applying Voisin's result on Jacobian ring and computer program.

In a recent preprint \cite{LSW22} Lin-Sheng-Wang have solved one direction of Conjecture 0.3.
\begin{theorem}[Lin-Sheng-Wang] \label{thm_LSW_torsion}
If $(E,\theta)$ is a rank-2 motivic Higgs bundle on a 4-punctured $\mathbb P^1$ over $\mathbb C$ with parabolic structure on the 4-punctures of type $(1/2)_\infty$ then the zero the Higgs field $(\theta)_0$ is the image of a torsion point in $C_\lambda$.
\end{theorem}
Actually they have solved \autoref{conj:SYZ} on the property of  the torsioness of zeros of Higgs fields of graded Higgs bundles come  from $p$-torsion Fontaine-Faltings modules. Combining this characteristic $p$ result with the Pink's theorem mentioned above they have obtained \autoref{thm_LSW_torsion}.

Our first result in this paper shows the existence part claimed in \autoref{Conj_2}.
\begin{theorem}\label{thm_main}
A complex semistable parabolic graded Higgs bundle  of degree $0$ on the projective line with 4-punctures $\{0,1,\lambda,\infty\}$
of parabolic type $(1/2)_\infty$ is motivic if the zero of the Higgs field $(\theta)_0$ is a torsion point in $C_\lambda$.
\end{theorem}

\begin{remark} For given 4-punctured complex projective line $(\mathbb P^1, \{ 0,\, 1,\, \lambda,\, \infty\})$, \autoref{thm_main} implies that there exists infinitely many non-isotrivial $GL_2$-type families of abelian varieties over $\bP^1$ with the discriminant locus contained in $\{0,1,\infty,\lambda\}$ and whose associated rank-$2$ eigen local systems are of type-$(1/2)_\infty$.
\end{remark}

Let $M_{0,n}$ denote the moduli space of isomorphism classes of $n$-marked projective line (i.e. projective line with $n$-ordered distinct marked points). Let $S_{0,n}$ denote the total space of the universal family of $n$-marked projective line with structure morphism
\[p_n\colon S_{0,n}\rightarrow M_{0,n}.\]

Then $\bP^1\setminus\{0,1,\infty\}$ is naturally isomorphic to $M_{0,4}$ by sending $\lambda$ to isomorphic class of the projective line with $4$-marked points $\{0,1,\infty,\lambda\}$. Once we identify $M_{0,4}$ with $\bP^1\setminus\{0,1,\infty\}$
\[M_{0,4} = \bP^1\setminus\{0,1,\infty\},\]
then $S_{0,4}= \bP^1\setminus\{0,1,\infty\} \times \bP^1$ is an algebraic surface, the structure morphism $p_4$ is given by $p_4(\lambda,z)=\lambda$ and $4$ structure sections are $\sigma_0(\lambda)=(\lambda,0)$, $\sigma_0(\lambda)=(\lambda,1)$, $\sigma_0(\lambda)=(\lambda,\infty)$, $\sigma_0(\lambda)=(\lambda,\lambda)$.


The following result is the heart part in our paper.
\begin{theorem} \label{thm_mainII}
Let $\lambda_0 \in M_{0,4}$ is an algebraic number, assume $\bar \lambda_0\in M_{0,4}(\mathbb F_q)$ for some prime $p$ is supersingular, i.e. modulo $p$ reduction of $C_\lambda$ is supersingular. Then any
$(\bar E, \bar \theta)\in M^{1/2}_{Hig\, \bar \lambda_0}(\bar {\mathbb F}_p)$ lifts uniquely to a motivic Higgs bundle
$(E,\theta)\in M^{1/2}_{Hig\, \lambda_0}(\bar {\mathbb Q})$. To be more precisely, there exists an abelian scheme of $\text{GL}_2$-type
$$f: A\to \tilde S_{0,4}$$
over the fiber product of a finite \'etale base change
\begin{equation*}
\xymatrix@C=2cm{
\widetilde{S}_{0,4}:=\widetilde {M}_{0,4} \times_{M_{0,4}} S_{0,4} \ar[r] \ar[d] & {S}_{0,4} \ar[d]^{p_4}\\
\widetilde {M}_{0,4}^{} \ar[r]^{\text{finite \'etale}} & M_{0,4}\\
}
\end{equation*}
and such that the local monodromies of $f$ around $\{0,1,\lambda\}$ are unipotent and around $\{\infty\}$ is quasi-unipotent with all eigenvalues being $-1$,  and such that
$(E,\theta)$ is an eigen sheaf attached to the abelian scheme $f_{\lambda_0}$.
\end{theorem}




It is clear that each rank-2 eigen sheaf $\tilde{\mathbb L}$ associated to $f$ is a local system on $\tilde S_{0,4}$ arising from isomonodromy deformation of an eigen sheaf $\mathbb L_{\lambda_0}$ associated to the family of abelian varieties restricted to the fiber over $\lambda_0$
$$f_{\lambda_0} : A_{\lambda_0}\to \mathbb P^1$$
with bad reduction on $\{0,1,\lambda,\infty\}$ of type-$(1/2)_\infty$.
\begin{corollary} \label{thm_main_painleve}
 Let $f\colon A\to\widetilde S_{0,4}$ be a family given in \autoref{thm_mainII}. Then all rank-$2$ eigen local systems associated to the family $f$ are algebraic solutions of Painleve VI equation of the type-$(1/2)_\infty$.
\end{corollary}

For given $\lambda\in\bP^1\setminus\{0,1,\infty\}$, any family $f_\lambda\colon A_\lambda\rightarrow\bP^1$ in \autoref{thm_main} has semistable reduction over $\{0,1,\lambda\}$ and potentially semistable reduction over $\infty$. Thus the eigen Higgs bundles $(E,\theta)_i$ (constructed in \eqref{equ:decomp_Higgs}) associated to this family  have the following form
\begin{equation} \label{equ:Higgs_form}
E_i=\mO\oplus\mO(-1),\qquad\theta_i\colon\mO\xrightarrow{\neq0} \mO(-1) \otimes\Omega^1_{\bP^1}(\log\{0,1,\infty,\lambda\})
\end{equation}
and are endowed with natural parabolic structures on the punctures $\{0,1,\infty,\lambda\}$ of type-$(1/2)_\infty$.
Here type-$(1/2)_\infty$ parabolic structures means that the parabolic structures at $0$, $1$ and $\lambda$ are trivial and the parabolic filtration at $\infty$ is
\[\left(E_{i}\mid_\infty\right)_\alpha=\left\{\begin{array}{cc}
E_{i}\mid_\infty & 0\leq\alpha\leq1/2,\\
0 & 1/2 <\alpha < 1.\\
\end{array}\right.\]
Let $\High$ denote the moduli space of rank-2 semi-stable graded Higgs bundles over $\bP^1$ with the parabolic structure on $\{0,1,\infty,\lambda\}$ of type-$(1/2)_\infty$ and with parabolic degree $0$. Then any Higgs bundle $(E,\theta)\in\High$ is parabolic stable and has the form as in \eqref{equ:Higgs_form}.

In view of $p$-adic Hodge theory, a Higgs bundle $(E,\theta)$ over the Witt ring $W(\Fq)$ realized by a family of abelian varieties over $W(\Fq)$ of $\GL_2(\EK)$-type has to be the grading of an $\EK$-eigen sheaf of the Fontaine-Faltings module
attached to the family of abelian varieties. Hence, by Lan-Sheng-Zuo functor, the graded Higgs bundle $(E,\theta)$ is \emph{periodic} on $\High$ over $W(\Fq)$ under the map induced by Higgs-de Rham flow.

One identifies the moduli space $\High$ with the projective line $\bP^1$ by sending $(E,\theta)$ to the zero locus of the Higgs map $(\theta)_0\in\bP^1$
\[\High= \bP^1.\]
Let $C_{\lambda}$ be the elliptic curve defined by the Weierstrass function $y^2=z(z-1)(z- \lambda)$, which is just the double cover of the projective line ramified on $\{0,1,\infty,\lambda\}$
\[\pi\colon C_\lambda\to\bP^1.\]
\begin{conjecture} [Sun-Yang-Zuo \cite{SYZ22}] \label{conj:SYZ}
The self-map $\phi$ induced by Higgs-de Rham flow on ${\High}\otimes {\Fq}$ comes from multiplication-by-$p$ map on the elliptic curve $C_\lambda\otimes\Fq$. In other words, the following diagram commutes
\[\xymatrix{
& C_\lambda\otimes {\Fq} \ar[d]_{\pi} \ar[r]^{[p]} & C_\lambda\otimes {\Fq} \ar[d]^{\pi} & \\
M_{Hig\lambda}^{gr {1\over 2}} \otimes {\Fq}\ar@/_12pt/[rrr]_{\phi} \ar@{=}[r] & \bP^1_{\Fq} \ar[r] & \bP^1_{\Fq} \ar@{=}[r] & \High\otimes {\Fq} \\
}\]
\end{conjecture}

The conjecture implies two things:
\begin{enumerate}
\item a Higgs bundle $(E,\theta)$ is $f$-periodic under the map $\phi$ if and only if the two points in $\pi^{-1}(\theta)_0$ are both torsion in $C_\lambda$ and of order $p^f\pm1$.
\item for a prime $p>2$ and assume $C_\lambda$ is supersingular then $\phi_\lambda(z)=z^{p^2}$. Hence, any Higgs bundle $(E,\theta)\in {\High}(\overline{\bF}_{q})$ is periodic.
\end{enumerate}

The \autoref{conj:SYZ} has been checked by Sun-Yang-Zuo for $p<50$. Very recently it has been proved by
Lin-Sheng-Wang and becomes a theorem.
\begin{theorem} [Lin-Sheng-Wang \cite{LSW22}] \label{thm_LSW}
\autoref{conj:SYZ} holds true.
\end{theorem}
\autoref{thm_mainII} combined with \autoref{thm_LSW} lead us to prove \autoref{thm_main} the part of the existence of
rank-2 motivic Higgs bundles in terms of torsioness of zeros of Higgs fields claimed in \autoref{Conj_2}.
\begin{remark}
\autoref{thm_main} and \autoref{thm_mainII} implies that. $\widetilde M_{0,4}$ is a moduli curve. It looks very interesting to such kind of properties on modularity appeared already as modular forms in the work by C. S. Lin and C. L. Wang on Painleve VI and Lame equations \cite{LiWa10}
\end{remark}

Given a semi-stable Higgs bundle $(E,\theta)$ with trivial Chern classes on a smooth scheme $\mathcal X$ over the ring of integers of some number field. Then for almost all finite places $\frakp$ the reduction $(E, \theta)\pmod{\frakp}$ is semistable. Thus, the Higgs bundle $(E, \theta)\pmod{\frakp}$ is preperiodic under the Higgs-de Rham flow. We take the length $\ell_{(E, \theta)\pmod{\frakp}}$ of the periodicity of $(E, \theta)\pmod{\frakp}$ at $p$.
\begin{corollary}
A Higgs bundle $(E,\theta)\in \text{HIG}^{\text{gr} {1\over 2}}_\lambda (\bar{\mathbb Q})$ is motivic if and only if
the set of preperiodic lengths $\{ \ell_{(E, \theta)\pmod{\frakp}} \}_{\frakp}$ is bounded above.
\end{corollary}
\begin{conjecture} A semistable Higgs bundle with trivial Chern classes on a smooth scheme $\mathcal X$ over the ring of integers of some number field is motivic if and only if the set of preperiodic lengths is bounded above.
\end{conjecture}
Consider a $n$-punctured projective line $(\mathbb P^1, \{x_1,\cdots, x_n\}=:D),\, n\geq 4$.. Then the moduli space
$\text{HIG}_D^{\text{gr}, {1\over 2}}$ of rank-2 semistable graded Higgs bundles on $\mathbb P^1$ of $\deg=0$ with parabolic structure of type $(1/2)_{x_n}$ contains a component  isomorphic to $\mathbb P^{n-3}$ of the maximal dimension. 
In \cite{SYZ22} we showed that $\mathbb P^{n-3}(\bar {\mathbb F}_p)$ contains a dense set $\text{HIG}_D^{\text{per}, {1\over 2}}( \bar {\mathbb F}_p)$ of periodic Higgs bundles.
\begin{question}
\begin{enumerate}
\item Does the self-map of Higgs-de Rham on $\text{HIG}_D^{\text{gr}, {1\over 2}}$  come from a multiplication by $p$ map on an abelian variety associated the $n$-punctured projective line?
\item Can we find motivic Higgs bundles in $\text{HIG}_D^{\text{per}, {1\over 2}}( \bar {\mathbb F}_p)$ ?
Can they be characterized by torsion points on the possible existing abelian variety?
\end{enumerate}
\end{question}

\begin{acknowledgement*} Our program on searching  for loci of motivic Higgs bundles in  moduli spaces is highly inspired by Simpson's conjecture and question on motivic local systems. We thank him for explaining on his theorem on rank-2 motivic local systems.

We thank Ariyan Javanpeykar for discussing on the relation between the selfmap and the multiplication by $p$ map on supersingular elliptic curves.

After a lecture by the first named author held  at University Mainz in November 2017, Duco Van Straten  introduced Kontsevich's proposal on a characterization of   rank-2 $\ell$-adic local systems  on 4-puctured $\mathbb P^1$  as fixed points of possibly existing additive maps on moduli spaces. It made us seriously to consider a possible connection between $p$-adic theory and $\ell$-adic theory. We thank him for very useful conversations.

 We thank warmly Raju Krishnamoorthy for  his constant support to our work. We have learned from him about Deligne's $p$-to$\ell$ companion, techniques of $F$-isocrystals  and Kato's work.

We  thank Tomoyuki Abe for answering our question on the compatibility between local and global Langlands correspondence in several email exchanges.

We thank Mao Sheng for discussing on the solution by Lin-Sheng-Wang of  \autoref{conj:SYZ} on torsioness of Kodaira-Spencer map, Certainly their  theorem plays a very important role in our paper.

We thank Hongjie Yu for showing us his beautiful solution of Deligne's conjecture on counting numbers $\ell$-adic local systems on punctured projective lines in terms of numbers of parabolic graded Higgs bundles over finite fields. His theorem is  very crucial in our paper.

During the preparation of this paper we also got a lot of benefits of sharing ideas and working knowledge from the following people.  The authors warmly thank Chung Pan Mok,  Ruiran Sun, Chin-Lung Wang , Jun Yi Xie, 
Shing-Tung Yau.
\end{acknowledgement*}



\section*{\bf Structure of the paper}

\subsection{The idea}

The underlying principle behind the proof is very simple, the so-called isomonodromy deformation for motivic local systems over mixed characteristic.
Let's first look at the situation over complex numbers. We assume, there exists a family of abelian varieties
$f_{\lambda_0}\colon A_{\lambda_0}\to \bP^1_\bC$ of $\GL_2(\EK)$-type over complex projective line with bad reduction on $\{0,1,\lambda_0,\infty\}$ of type-(1/2).
Then the filtered logarithmic de Rham bundle decomposes as $\EK$-eigen sheaves
\[(V,\nabla,E^{1,0})=:( R^1_\dR f_* \Omega^*_{A_{\lambda_0}/\bP^1}(\log \Delta),d),R^0f_* \Omega^1_{A_{\lambda_0}/\bP^1}(\log \Delta))=\bigoplus_{i=1}^g(V,\nabla,E^{1,0})_i,\]
where each eigen sheaf has the form
\[(V,\nabla,E^{1,0})_i\simeq (\mO\oplus \mO(-1),\nabla_i,\mO). \]
Consider the universal family of 4-punctured lines
\[S_{0,4}|_{\hat U_{\lambda_0}}\to \hat U_{\lambda_0}\]
over a formal neighborhood $\hat U_{\lambda_0}\subset M_{0\,4}$ of $\lambda_0$.
Then by forgetting the Hodge filtration the de Rham bundle extends to a de Rham bundle
$(V,\nabla)_{S_{0,4}|_{\hat U_{\lambda_0}}}$
over $S_{0,4}|_{\hat U_{\lambda_0}}$. It is known the family of abelian varieties extends over $S_{0,4}|_{{\hat U_{\lambda_0}}}$
if and only if the Hodge filtration $E^{1,0}$ extends to a sub bundle in the de Rham bundle $(V,\nabla)_{S_{0,4}|_{\hat U_{\lambda_0}}}$. Using the $\EK$-eigen sheave decomposition we see that the obstruction for extending the Hodge filtration $E^{1,0}=\bigoplus_{i=1}^g \mO$
lies in
$\bigoplus_{i=1}^g H^1(\bP^1,\mO(-1))=0$. Hence, the family of abelian varieties $f_{\lambda_0}$
extends over the base $S_{0,4}|_{\hat U_{\lambda_0}}$. A standard argument on the moduli space of period mappings from curves with fixed genus shows that this formal extension leads an algebraic extension on $S_{0,4}$.

Back to the situation over mixed characteristic, along the diagram below. We like to construct a family of abelian varieties on $\mathbb P^1_{\mathbb F_q}$ with bad reduction on 4-punctures of type-$(1/2)_\infty$ and such that the Hodge filtration can be lifted as a sub bundle in the Dieudann\'e module attached to this family over characteristic-zero and then a type of Grothendieck-Messing-Kato logarithmic deformation theorem for log classifying mapping applies.
\[\tiny \xymatrix@C=2.5cm @ R=2cm{
\left\{\text{Crystalline}\atop \text{representations}\right\}
\ar@{<->}[r]^-{\text{Fontaine-Faltings}}_-{p\text{\rm-adic RH}}
& \left\{\text{Fontaine-Faltings}\atop \text{modules}\right\}
\ar@{^(->}[r]^-{\text{forgetting Hodge}}_-{\text{filtration,}\otimes \bQ_p}
\ar@{<->}[d]_{\text{Higgs-de Rham flow}}^{\text{by Lan-Sheng-Zuo}}
& \left\{\text{Overconvergent}\atop \text{$F$-isocrystal}\right\}
\ar@{<->}[r]^{\text{Deligne's } p-\ell}_{\text{companion by Abe}}
& \left\{\text{$\ell$-adic}\atop \text{representations}\right\} \\
& \left\{\text{periodic}\atop \text{Higgs bundles}\right\} \ar@{^(->}[r]
& \left\{\text{Higgs}\atop \text{bundles}\right\}
& \\
}\]

\subsection{Technique steps}

\subsubsection{\bf A bijection from the set of parabolic graded semi stable Higgs bundles over $\mathbb F_q$ to the set of $\ell$-adic local systems over $\mathbb F_q$ via Abe's theorem on Deligne's $p$-to-$\ell$ companion.}

\begin{itemize}
\item In \autoref{sec_main_para}, we recall the notion of parabolic de Rham bundle and parabolic Higgs bundle. In \autoref{sec_para_classification_lower_rank}, we classify rank-$2$ de Rham bundles and Higgs bundles on $\mathbb P^1$ with parabolic structures on 4-punctures of type-$(1/2)_\infty$. The main result in this section is \autoref{thm_ClassfyR2PdE} and \autoref{thm_ClassfyR2PHiggs}.

\item In \autoref{sec_main_FF_and_Higgs_de_Rham}, we introduce the notion of parabolic Fontaine-Faltings modules and parabolic Higgs-de Rham flows. For a $\lambda\in W(k)$ such that $\lambda\pmod{p}$ is supersingular, every Higgs bundle on $\mathbb P^1_k$ with parabolic structure on $\{0,1,\bar\lambda,\infty\}$ of type-$(1/2)_\infty$ is periodic and lifts uniquely as a periodic Higgs bundle on $\mathbb P^1_{W(k)}$. In other words, there is an natural injection 
\[\High(k) \hookrightarrow [\MFh(W(k))]\]
from the set of Higgs bundles on $\mathbb P^1_k$ with parabolic structure on $\{0,1,\bar\lambda,\infty\}$ of type-$(1/2)_\infty$ 
to the set of Fontaine-Faltings modules on $\mathbb P^1_W(k)$ with parabolic structure on $\{0,1, \lambda,\infty\}
$ of type-$(1/2)_\infty$ modulo an equivalent relation. The main result in this section is \autoref{mthm_PHIGk2PHIGW}.

\item In \autoref{sec_main_F_Isoc}, we consider the set of rank-2 overconvergent $F$-isocrystals over 4-punctured projective line $(\mathbb P^1, \{0,1,\lambda,\infty\}$ over $k$ with given exponents. In \autoref{mthm_FF2Isoc_classes} and \autoref{thm_PHIG_k_to_F_Isoc_k}, we show an injective map from
$\text{HIG}^{\text{gr} {1\over 2}}_{\bar \lambda}(k) $ to a set of overconvergent $F$-isocrystals with given exponent and of trivial determinants.
\item In \autoref{sec_main_p_to_l_bijections}, we make use of Abe's theorem on Deligne's $p$-to-$\ell$ companion to the set of overconvergent $F$-isocrystals with given exponent. Composed with the injective maps obtained in the previous sections, in \autoref{mthm_HIGk2Lock}, the $p$-to-$\ell$ companion induces an injective map
\[ \High(k) \hookrightarrow \Loch(\bark)^{\Frob_x},\]
the latter consists of rank-$2$ geometric local systems on $(\mathbb P^1\setminus \{0,1, \lambda, \infty\})_{\bar k}$ of local monodromy around the punctures of type-$(1/2)_\infty$ and stabilized by $\Frob_k$.
\item By Yu's formula\autoref{thm_Yu} for numeric Simpson correspondence, the above injective map is actually bijective, \autoref{thm_genuineSimCorr}
\begin{equation} \label{eq_bij_Hig_k_to_loc_bark}
\High(k) \xrightarrow{\ \simeq\ } \Loch(\bark)^{\Frob_x}.
\end{equation}
As a consequence, the trace field of any local system in $\Loch(\bark)^{\Frob_k}$ is unramified at $p$.
\end{itemize}

\subsubsection{\bf Constructing families of abelian varieties over $\Fq$ by Drinfeld's work on Langlands correspondence over characteristic $p$ and lifting Hodge filtrations characteristic zero.}

Given a local system $\bL\in \Loch(k_2')$ fixed by the Frobenius $\text{Frob}_k$ with cyclotomic determinant. By applying Drinfeld's \autoref{thm_Drinfeld_GL2}, in \autoref{thm_main_loc_to_family_char_p}, we find a family of abelian varieties of $\GL_2(\EK)$-type 
\[f'\colon A' \to (\mathbb P^1-\{0,1,\lambda,\infty\})_{k'}=:U_{k_2'}\]
such that $\bL$ appears as an eigen $\ell$-adic local system and all other eigen local systems are located in $\Loch(k_2')$ and fixed by the Frobenius $\text{Frob}_k$ with cyclotomic determinant.

Consider the Dieudonn\'e crystal\footnote{By Kato's \autoref{thm_Kato_equivalent}, we identify the underlying crystal with its realization over the formal completion of $U_{W(k)}$.}
\[(V,\nabla,\Phi,\mV)\]
attached to $f'$, which is automatically overconvergent. After extending the coefficient from $\Qp$ to $\bQ_{p^f}$, the $\GL_2(\EK)$-structure induces an $\EK$-eigen sheaves decomposition of overconvergent $F$-isocrystals with $\bQ_{p^f}$-coefficients
\[(V,\nabla,\Phi)_{\bQ_p}=\bigoplus _{i=1}^g(V,\nabla,\Phi)_{i}\]
where $(V,\nabla,\Phi)_{i}$ has cyclotomic determinant.  By the bijection \eqref{eq_bij_Hig_k_to_loc_bark}, in \autoref{sec_subsub_family_to_rank2_cystal}, we show that each $(V,\nabla,\Phi)_i$ has an integral extension, which underlies a Fontaine-Faltings module with a $\bZ_{p^f}$-endomorphism structure. As a consequence, there exists an isogeny of the Dieudonn\'e crystal of $f'$, which  carries a Hodge filtration $\Fil$. According to equivalence of the Dieudonn\'e functor, from this isogeny Dieudonn\'e crystal, one gets a $p$-isogeny 
\begin{equation} \label{eq_family_char_p}
f\colon A\to \bP^1_{k_2'}
\end{equation}
 of the original family of abelian varieties such that the isogeny Dieudonn\'e crystal is isomorphic to that attached to $f$, see  in \autoref{thm_main_loc_to_family_char_p}. In this case, the Hodge filtration $E^{1,0}_{f}$ attached to $f$ coincides with $\Fil\otimes k$. Thus $E^{1,0}_{f}$ lifts to characteristic zero.


\subsubsection{\bf Lifting families of abelian varieties from characteristic $p$ to characteristic zero by Grothendieck-Messing-Kato logarithmic deformation theorem}

The goal in \autoref{sec_main_lifting}  is to lift the family $f$ in \autoref{eq_family_char_p} from characteristic $p$ to characteristic $0$. Our idea is to lift the ``classifying mapping'' attached to this family. Because once a family is obtained by pulling back of some universal family along a classifying mapping, then to lift the given family is equivalent to lift the classifying mapping. 

To get a classifying mapping attached to our family $f$, we need to choose a good moduli spaces, add a level structure and a principal polarization structure on the family. 

For the moduli space, we take the fine arithmetic moduli space $\mA_{8g, 1, 3}=: \mX^0$ of principle polarized abelian varieties with level-3 exists over $\bZ[e^{{2i \pi \over 3}}, 1/3]$. The advantage is that the moduli space and universal family has good compactifications by a theorem due to Faltings-Chai \cite{FaCh90}.


The strategy for adding level structure is pulling back along some finite covering mapping \autoref{sec_subsub_level}, and that for adding principal polarization structure is to utilize the Zarkin's trick \autoref{sec_subsub_Zarhin}. After these proceeding, one gets new family
\[f^{(4,4)}\colon A^{4,4} \rightarrow C_k\setminus D_k\]
which carries a principle polarization and a full $3$-level structure. By the universal property of moduli space $\mA_{8g,1,3}$, one obtains a classifying mapping \autoref{thm_classifying_mapping_k}
\[\overline{\varphi}_k \colon C_k \rightarrow \overline{\mA}_{8g,1,3}.\]



Moreover, the Hodge filtration attached to this family can be lifted to characteristic $0$ by the discuss in \autoref{sec_main_family_over_k}. In \autoref{sec_sub_polarization_and_Fil}, we show that the polarization is compatible with the lifting Hodge filtration \autoref{thm_polarization_and_filtration}. Then the classifying mapping $\overline{\varphi}_k$ lifts to a mapping
\[\overline{\psi}_{W(k)} \colon C_{W(k)} \rightarrow \overline{\mA}_{8g,1,3}\]
by applying the main result \autoref{thm_main_comparing_obstruction} in \autoref{sec_main_compare_obstructions}, which identifies the obstruction of lifting the Hodge filtration with that of lifting the classifying mapping via Faltings-Chai universal Kodaira-Spencer map \cite{FaCh90}. 

By the rigidity of $\overline{\varphi}_{W(k)}$ (see \cite[Section 4]{KYZ22}), the family is actually defined over some number field. By using Weil restriction and Simpson's \autoref{thm_Simpson}, we show $\overline{\varphi}_{W(k)}$ splits out a family of abelian varieties of $\GL_2$-type over the projective line such that the given Higgs bundle $(E,\theta)$ appears as an eigen sheaf \autoref{thm_main_Higg_mod_p_to_largest_family}.



\section*{Conventions, Notation, and Terminology}
For convenience, we explicitly state conventions and notations. These are in full force unless otherwise stated.
\begin{itemize}
\item Let $k_0$ be a finite field with cardinality.
\item Let $\lambda\in W(k_0)$ be an element satisfying $\lambda\neq 0,1\pmod{p}$. By abusing notation, we sometimes using $\lambda$ to stand for $\lambda\pmod{p} \in k_0$.
\item For any finite extension $k$ of $k_0$, let $k_n$ denote the field extension of $k$ of degree $n$ for any $n\geq1$.
\item For any finite extension $k$ of $k_0$, assume it has cardinality $p^h$, the $h$-iteration of the absolute Frobenius on $\mathbb P^1_{k}$ is a morphism of $k$-schemes preserves the divisor $\{0,1,\lambda,\infty\}$. We denote it by
\[\Frob_k\colon (\mathbb P_k^1,\{0,1,\lambda,\infty\}) \rightarrow (\mathbb P_k^1,\{0,1,\lambda,\infty\}).\]
By abusing notation, we also use $\Frob_k$ to stand for its base change to $k_n$ or $\bark$
\[ \Frob_k\colon (\mathbb P_{k_n}^1,\{0,1,\lambda,\infty\}) \rightarrow (\mathbb P_{k_n}^1,\{0,1,\lambda,\infty\}),\]
\[ \Frob_k\colon (\mathbb P_{\bark}^1,\{0,1,\lambda,\infty\}) \rightarrow (\mathbb P_{\bark}^1,\{0,1,\lambda,\infty\}).\]
\end{itemize}

\newpage

\section{\bf Parabolic structures} \label{sec_main_para}




Throughout this section, we will use the terminology and notation from \cite{IySi07} and \cite{KrSh20} to facilitate understanding and usage. We will give several definitions of parabolic objects in this section, without claiming originality for any of them.

\subsection{Introduction to Parabolic vector bundles}

Our intention is not to introduce parabolic objects on arbitrary spaces. Instead, for the purposes of our application, it is sufficient to focus on parabolic objects in the following special $(Y,D_Y)/S$ spaces.

Let $p$ be an odd prime number and let $S$ be our base space. We assume it is one of the following spaces.
\begin{enumerate}
\item[(i)] $S=\Spec(K)$, where $K$ is a field of characteristic $0$;
\item[(ii)] $S=\Spec(W_n(k))$, where $W_n(k)$ is a ring of truncated Witt vectors with coefficients in a finite field $k$;
\item[(iii)] $S=\Spec(\mO_K)$, where $K$ is an unramified $p$-adic number field.
\item[(iv)] $S=\Spf(\mO_K)$, where $K$ is an unramified $p$-adic number field.
\end{enumerate}
For a smooth curve $Y$ over $S$ (or a smooth formal curve over $S$ if $S$ is a formal scheme), we define the reduced divisor $D_Y$ by $n$ $S$-sections $x_i\colon S\rightarrow Y$, $i=1,\cdots,n$, that do not intersect with each other. We denote by $U_Y := Y - D_Y$ and by $j_Y$ the open immersion $j_Y\colon U_Y\rightarrow Y$. The irreducible components of $D_Y$ are denoted by $D_{Y,i}$, $i=1,2,\cdots,n$, and we have $D_Y = \bigcup_{i=1}^n D_{Y,i}$.

We set $\Omega_{Y/S}^1$ to be the sheaf of relative $1$-forms and $\Omega_{Y/S}^1(\log D_Y)$ to be the sheaf of relative $1$-forms with logarithmic poles along $D_Y$. By the smoothness of $Y$ over $S$, both of these sheaves are line bundles over $Y$.


\subsubsection{Parabolic vector bundles}

In this section, we introduce the concept of parabolic vector bundles, which is based on \cite{IySi07}.

\begin{definition}
A \emph{parabolic sheaf} on $(Y,D_Y)/S$ is a collection of torsion-free coherent sheaves $V={V_\alpha}$ on $Y$, which are flat over $S$, indexed by multi-indices $\alpha=(\alpha_1,\cdots,\alpha_n)\in \bQ^n$. The sheaves $\{V_\alpha\}$ are subject to the following conditions:
\begin{itemize}
\item (inclusion) $V_\alpha \hookrightarrow V_\beta$ with cokernel flat over $S$, where $\alpha\leq \beta$ (i.e. where $\alpha_i\leq \beta_i$ for all $i$).
\item (normalization) $V_{\alpha+\delta^i} = V_\alpha(D_{Y,i})$ where $\delta^i=(0,\cdots,1,\cdots,0)\in\bQ^n$.
\item (semicontinuity) for any given $\alpha$ there exists a constant $c=c(\alpha)>0$ such that $V_{\alpha+\epsilon} = V_\alpha$ for $\epsilon=(\epsilon_1,\cdots,\epsilon_n)\in\bQ^n$ with $0\leq\epsilon_i\leq c$.
\end{itemize}
A \emph{morphism between two parabolic sheaves} from $F$ to $F'$ is a collection of compatible morphisms of sheaves $f_\alpha\colon F_\alpha\rightarrow F'_\alpha$.
\end{definition}

\begin{remark}
\begin{enumerate}
\item[1).] The second condition implies that the quotient sheaves $F_\alpha/F_\beta$ for $\beta\leq \alpha$ are supported at $D_Y$.
\item[2).] The third condition means that the structure is determined by the sheaves $F_\alpha$ for a finite collections of indices $\alpha$ with $0\leq\alpha_i<1$.
\item[3).] The extension $j_{Y*}\left( j_Y^* F_\beta\right)$ does not depend on the choice of $\beta\in\bQ^n$, denote by $F_\infty$. Note that the $F_\alpha$ may all be consider as subsheaves of $F_\infty$.
\end{enumerate}
\end{remark}

\begin{definition}
Let $F$ and $F'$ be two parabolic sheaves. Denote
\[(F\otimes F')_{\alpha}:= \sum_{\beta+\gamma = \alpha} F_\beta \otimes F_{\gamma}\subset F_\infty \otimes F'_\infty.\]
Then $F\otimes F':=\{(F\otimes F')_{\alpha}\}\}$ forms a parabolic sheaf, which satisfies the universal property of tensor product of $F$ and $F'$ in the category of parabolic sheaves. Similarly, one can define the wedge product, symmetric product, and determinant for parabolic vector bundles as usual.
\end{definition}



For any $\alpha=(\alpha_1,\cdots,\alpha_n)\in\bQ^n$, denote
\[\blue{\alpha D_Y} = \alpha_1D_{Y,1} + \cdots + \alpha_n D_{Y,n}\]
which is a rational divisor supported on $D_Y$. Of course, all rational divisor supported on $D_Y$ are of this form. Denote
\[\blue{\lfloor\alpha\rfloor} \coloneqq (\lfloor\alpha_1\rfloor,\lfloor\alpha_2\rfloor,\cdots,\lfloor\alpha_n\rfloor)\]
where $\lfloor\alpha_i\rfloor$ is the maximal integer smaller than or equal to $\alpha_i$. In particular $\lfloor\alpha\rfloor D_Y$ is an integral divisor supported on $D_Y$.

\begin{example} For any torsion-free sheaf $E$ on $Y$ then it may be considered as a parabolic sheaf (we say ``with trivial parabolic structure'') by setting
\[E_\alpha = E(\lfloor\alpha\rfloor D).\]

\end{example}

\begin{example}\label{exa:paraLineBund} Let $L$ be a line bundle and let $\gamma=(\gamma_1,\cdots,\gamma_n)\in\bQ^n$ be a rational multi-indices. Then there is a parabolic sheaf denoted
\[\mL=L(\gamma D_Y) \]
by setting
\[\mL_{\alpha} \coloneqq L(\lfloor\alpha+\gamma\rfloor D_Y).\]
Clearly, for any two line bundles $L,L'$ and two multi-indices $\gamma,\gamma'\in\bQ^n$, one has
\[L(\gamma D)\otimes L'(\gamma' D') = (L\otimes L')\Big((\gamma +\gamma') D\Big).\]
Then the set of all isomorphic classes of parabolic line bundles forms an abelian group under the tensor product, which contains the Picard group of $Y$ as a subgroup in the natural way.

The fractional part of the real number $\gamma_i$ is just the parabolic weight of $F$ along $D_{Y,i}$. We denote
\[\deg(\mL) = \deg(L) + \sum_{i=1}^n \gamma_i.\]
\end{example}

\begin{definition}
\begin{itemize}
\item[1).] The parabolic sheaves appeared in \autoref{exa:paraLineBund} are called \emph{parabolic line bundles}.
\item[2).] A \emph{parabolic vector bundle} is a parabolic sheaf which is locally isomorphic to a direct sum of parabolic line bundles. (Simpson called this a \emph{locally abelian parabolic vector bundle}.)
A \emph{morphism between two parabolic vector bundles} is a morphism between their underlying parabolic sheaves.
\end{itemize}

\end{definition}

\begin{example}
Let $V$ be a vector bundle and let $\gamma \in \bQ^n$. Then
\[V(\gamma D):= V\otimes \mO(\gamma D)\]
is a parabolic vector bundle.
\end{example}


\subsubsection{Quasi-parabolic structures, parabolic weights}
Historically, the parabolic vector bundles was defined by using some filtrations. In this subsection, we show the equivalency of the new and old definitions for parabolic vector bundles.

\begin{definition}
Let $V$ be a vector bundle over $(Y,D_Y)$. An \emph{quasi-parabolic structure on $V$} is a decreasing filtration of direct summands of $V\mid_{D_{Y,i}}$.
\[V\mid_{D_{Y,i}} = \QP^1(V\mid_{D_{Y,i}}) \supsetneqq \QP^2(V\mid_{D_{Y,i}}) \supsetneqq \cdots \supsetneqq \QP^{n_i}(V\mid_{D_{Y,i}}) \supsetneqq 0.\]
A set of \emph{parabolic weights} attached to the quasi-parabolic structure is a set of rational numbers $\alpha_i^1,\alpha_i^2,\cdots,\alpha_i^{n_i}$ satisfying
\[0\leq\alpha_i^1<\alpha_i^2<\cdots<\alpha_i^{n_i}<1.\]
A \emph{previous parabolic vector bundle} is triple $(V,\QP,\alpha)$, which consists a vector bundle, a quasi-parabolic structure, and a set of parabolic weights. A \emph{morphism between two previous parabolic vector bundles} from $(V,\QP,\alpha)$ to $(W,\QP,\beta)$, is a morphism of vector bundle $f\colon V\rightarrow W$ such that
\[f\mid_{D_{Y,i}} (\QP^j(V\mid_{D_{Y,i}})) \subseteq \QP^k(W\mid_{D_{Y,i}})\]
for any triple of indices $(i,j,k)$ satisfying $\alpha_i^j\geq \beta_i^k$.
\end{definition}

\begin{remark} \label{rmk:freeness}Since our base space $S$ is connected, for any direct summands $W$ of $V\mid_{D_{Y,i}}$, it is actually a sub vector bundle of $V\mid_{D_{Y,i}}$ over $D_{Y,i}$. Moreover, if one consider the pullback of $W$ along the surjective morphism $V\rightarrow V\mid_{D_{Y,i}}$, one gets a subsheaf $F_W$ of $V$, which is also a vector bundle over $Y$. In fact, the subsheaf $F_W$ fits in the following morphism of short exact sequences
\begin{equation*}\xymatrix{
0 \ar[r] & V(-D_{Y,i}) \ar@{=}[d] \ar[r] & F_W \ar@{^(->}[d] \ar[r] & W \ar[r] \ar@{^(->}[d] & 0\\
0 \ar[r] & V(-D_{Y,i}) \ar[r] & V \ar[r] & V\mid_{D_{Y,i}} \ar[r] & 0.\\
}\end{equation*}
Since both $W$ and $V(-D_{Y,i})$ are flat over $S$, the quasi-coherent sheaf $F_W$ is also flat over $S$. According \cite[\href{https://stacks.math.columbia.edu/tag/080Q}{Tag 080Q}]{stacks-project}, the local freeness of $F_W$ can be checked fiberwisely.
\end{remark}

\begin{example} \label{exa:AnotherDefiPara}
Let $F=\{F_\alpha\}$ be a parabolic vector bundle over $(Y,D_Y)/S$. There is a natural previous parabolic vector bundle with underlying vector bundle $V\coloneqq F_0$.

For all $\epsilon\in[0,1)$, denote
\[\P^{\epsilon}(F_0\mid_{D_{Y,i}})\coloneqq F_{-\epsilon\delta^i}/F_{-\delta^i}\subseteq F_0/F_{-\delta^i}= F_0\mid_{D_{Y,i}}\]
which form a left continue descending filtration of direct summands index by rational number in $[0,1)$. To show that $\P^{\epsilon}(F_0\mid_{D_{Y,i}})$ is a direct summand, we may reduce to the case $F$ being a parabolic vector bundle and check that directly. Denote by $\{\alpha_i^j\}_{j=1}^{n_i}\subset [0,1)$ the finite set of all jumping locus of the filtration $\P^{\epsilon}(F_0\mid_{D_{Y,i}})$. By reordering the index, we may assume $0\leq \alpha_i^1< \alpha_i^2< \cdots < \alpha^{n_i}_i<1$. Then the filtration is uniquely determined by the following sub filtration
\begin{equation} \label{equ:QuasiPara}
F_0\mid_{D_{Y,i}} = \P^{\alpha_i^1}(F_0\mid_{D_{Y,i}}) \supsetneqq \P^{\alpha_i^2}(F_0\mid_{D_{Y,i}}) \supsetneqq \cdots \supsetneqq\P^{\alpha_i^{n_i}}(F_0\mid_{D_{Y,i}}) \supsetneqq 0.
\end{equation}
Clearly, this forms a quasi-parabolic structure of $F_0$ along $D_{Y,i}$, which we call \emph{the quasi-parabolic structure of $F$ along $D_{Y,i}$}. And the numbers in the set $\{\alpha_i^j\}_{j=1}^{n_i}$ forms a set of parabolic weights, which we call \emph{the parabolic weights of $F$ along $D_{Y,i}$}, or \emph{the parabolic weight of the quasi-parabolic structure of $F$ along $D_{Y,i}$}.
\end{example}

\begin{remark}
By the definition of our parabolic vector bundle, the parabolic weights are all rational.
\end{remark}

\begin{lemma}
The construction in \autoref{exa:AnotherDefiPara} induces an equivalent functor from the category of all parabolic vector bundles to the category of all previous parabolic vector bundles.
\end{lemma}

\begin{proof}
Let $V$ be a vector bundle over $(Y,D_Y)$ with a quasi-parabolic structures
\[V\mid_{D_{Y,i}} = \QP^1(V\mid_{D_{Y,i}}) \supsetneqq \QP^2(V\mid_{D_{Y,i}}) \supsetneqq \cdots \supsetneqq \QP^{n_i}(V\mid_{D_{Y,i}}) \supsetneqq 0\]
and parabolic weights
\[0\leq\alpha_i^1<\alpha_i^2<\cdots<\alpha_i^{n_i}<1\]
for $i=1,\cdots,n$.
We need to show that it associates a unique parabolic vector bundle $F=\{F_\alpha\}$ in the sense under the construction in \autoref{exa:AnotherDefiPara}.

\textbf{Existence:} For any $\epsilon\in [0,1)\cap\bQ$, denote
\[\P^\epsilon(V\mid_{D_{Y,i}}) \coloneqq \left\{
\begin{array}{ll}
\QP^1(V\mid_{D_{Y,i}}) & \text{if} \epsilon \leq \alpha_i^1 \\
\QP^j(V\mid_{D_{Y,i}}) & \text{if} \alpha_{i}^{j-1} < \epsilon \leq \alpha_i^j \text{for some} j=2,\cdots,n_i\\
0 & \text{if} \alpha_{i}^{n_i} < \epsilon \\
\end{array}\right.\]
Denote by $F_{-\epsilon \delta^i}$ the pull back of $\P^\epsilon(V\mid_{D_{Y,i}})\subset V\mid_{D_{Y,i}}$ under the surjective morphisms $V\rightarrow V\mid_{D_{Y,i}}$. By \autoref{rmk:freeness}, the $F_{-\epsilon \delta^i}$ is also a vector bundle. For all $\alpha=(\alpha_1,\cdots,\alpha_n)\in \Big((-1,0]\cap\bQ\Big)^n$, denote
\[F_\alpha = \bigcap_{i=1}^n F_{\alpha_i \delta^i}.\]
By normalization in the definition of the parabolic vector bundles, we defines $F_{\alpha}$ for any $\alpha\in\bQ^n$.
By direct computation, one checks that $\{F_\alpha\}$ forms a parabolic vector bundle associated to the given previous parabolic vector bundle.

\textbf{Uniqueness:} Let $\{F_\alpha\}$ and $\{F'_\alpha\}$ be two parabolic vector bundles associated to the given previous parabolic vector bundle. Then the isomorphism $F_0\cong V\cong F'_0$ can be restricted to isomorphisms $F_{-\epsilon \delta^i} \cong F'_{-\epsilon \delta^i}$, because of $P^\epsilon(F_0\mid_{D_{Y,i}}) \cong P^\epsilon(F'_0\mid_{D_{Y,i}})$ for all $\epsilon\in[0,1)$. By normalization, one gets an natural isomorphism between $\{F_\alpha\}$ and $\{F'_\alpha\}$.
\end{proof}


\subsubsection{Degrees and semistability}
In this subsection, we assume $Y$ is projective over $S$ to define the degree of a parabolic vector bundle and the semistability of a parabolic vector bundle.

Let $F$ be a coherent sheaf on $Y$ which is flat over $S$. By the locally constancy of the Chern classes, the first Chern class of the coherent sheaf $F_s$ over the curve $Y_s$ for any points $s\in S$ does not depend on the choice of the point $s$. It is well-defined to set
\[\deg(F)\coloneqq c_1(F_s).\]
We define the degree of a parabolic vector bundle as follows.
\begin{definition} Let $F$ be a parabolic sheaf over $(Y,D_Y)/S$.
\begin{itemize}
\item The \emph{degree of $F$} is defined as
\[\deg(F) \coloneqq \deg(F_0) + \sum_{D_{Y,i}\in D_Y}\sum_{j=1}^{n_i} \alpha_i^j \cdot \rank( \P^{\alpha_i^j}(F_0\mid_{D_{Y,i}})/\P^{\alpha_i^{j+1}}(F_0\mid_{D_{Y,i}})) \]
\item The parabolic vector bundle $F$ is called \emph{semistable} (resp. \emph{stable}) if for any proper sub parabolic sheaves $F'\subsetneq F$, one has
\[\frac{\deg(F')}{\rank(F')} \leq \frac{\deg(F)}{\rank(F)} \qquad \text{\Big(resp.} \frac{\deg(F')}{\rank(F')} < \frac{\deg(F)}{\rank(F)} \text{\Big)}.\]
\end{itemize}
\end{definition}

\subsubsection{The pullback of parabolic vector bundles}

Let $f\colon (Y',D_{Y'}) \rightarrow (Y,D)$ be a morphism of smooth curves with relative normal crossings divisors over $S$ such that $f^{-1}(D)\subset D'$. We recall a definition proposed by Simpson in \cite[Section 2.2]{IySi07} for the pullback $f^*F$ of a parabolic vector bundle $F$.

\begin{definition} \label{def:pullbackbundle}
\begin{enumerate}
\item[(1)] If $F$ is a parabolic line bundle, then there exists a line bundle $L$ and a rational divisor $B$ which is supported on $D$ such that $F=L(B)$. We define
\[f^*F \coloneqq (f^*L)(f^*B).\]
\item[(2)] In general case, by localization, we reduce to the case of parabolic line bundles.
\end{enumerate}
\end{definition}
In the following, we will give an easy-to-use format description of pullback and then extend the definition for parabolic de Rham bundles.

Let $V=\{V_\alpha\}$ be a parabolic vector bundle over $(Y,D_Y)/S$. For any $\gamma\in\bQ^n$, we identify the two parabolic vector bundles
\[V = V(-\gamma D)\otimes\mO_Y(\gamma D).\]
Clearly, once we identify the usual vector bundle with its associated parabolic vector bundle, then $V_0$ can be view as a parabolic subsheaf of $V$
\[V_0 \subseteq V.\]
In particular, for the parabolic bundle $V(-\gamma D)$, we have a parabolic subsheaf
\[V(-\gamma D)_0\hookrightarrow V(-\gamma D).\]
By pullback along $f$ and tensoring $f^*\Big(\mO_Y(\gamma D)\Big)$, we gets a parabolic subsheaf
\[f^*_{\gamma}(V)\coloneqq f^*\Big(V(-\gamma D)_0\Big)\otimes f^*\Big(\mO_Y(\gamma D)\Big) \subseteq f^*\Big(V(-\gamma D)\Big)\otimes f^*\Big(\mO_Y(\gamma D)\Big) = f^*(V).\]

\begin{proposition} \label{prop:PullbackParabolicBundles}
$f^*V = \sum\limits_{\gamma\in\bQ^n} f^*_{\gamma}(V)$.
\end{proposition}

\begin{proof}
By localization, we reduce it to parabolic line bundle case, which follows by taking $\mL$ to be the parabolic line bundle of its self.
\end{proof}

\begin{remark}
By replace the parabolic line bundle $\mO_Y(\gamma D)$ with a general parabolic line bundle $\mL$, one can define a parabolic subsheaf
\[f_{\mL}^*(V):= f^*\Big((V\otimes\mL^{-1})_0\Big) \otimes f^*\mL \subset f^*V.\]
\end{remark}

\subsection{Parabolic de Rham bundles}

\subsubsection{logarithmic de Rham bundle}
A \emph{(logarithmic) connection} on a sheaf $V$ of $\mO_Y$-modules over $(Y,D_Y)/S$ is an $\mO_S$-linear map $\nabla\colon V\rightarrow V\otimes_{\mO_Y} \Omega_{Y/S}^1(\log D_Y)$ satisfying the Leibniz rule $\nabla(rv) = v\otimes \rmd r + r\nabla(v)$ for any local section $r\in \mO_Y$ and $v\in V$. Give a (logarithmic) connection, there are canonical maps
\[\nabla\colon V\otimes\Omega_{Y/S}^i(\log D_Y) \rightarrow V\otimes\Omega_{Y/S}^{i+1}(\log D_Y)\]
given by $s\otimes \omega \mapsto\nabla(s)\wedge \omega +s \otimes \rmd\omega$.
The curvature $\nabla\circ\nabla$, the composition of the first two canonical maps, is $\mO_Y$-linear and contained in $\mEnd(V)\otimes_{\mO_Y} \Omega^2_{Y/S}(\log D_Y)$. The connection is called \emph{integrable} and $(V,\nabla)$ is called a \emph{(logarithmic) de Rham sheaf} if the curvature vanishes. For a de Rham sheaf, one has a natural \emph{de Rham complex}:
\[\DR(V,\nabla):\qquad 0\rightarrow V
\xrightarrow{\nabla} V\otimes\Omega^1_{Y/S}(\log D_Y)
\xrightarrow{\nabla} V\otimes\Omega^2_{Y/S}(\log D_Y)
\xrightarrow{\nabla} V\otimes\Omega^3_{Y/S}(\log D_Y) \rightarrow \cdots \]
A logarithmic de Rham sheaf is called a \emph{logarithmic de Rham bundle}, if the underlying sheaf is a vector bundle over $Y$. Denote by \emph{$\MIC(Y,D_Y)$} the category of all logarithmic de Rham bundle over $(Y,D_Y)$.

In the same way, we define a \emph{(logarithmic) $p$-connection along $D_Y$} by replacing the Leibniz rule in the definition of connection with
\[\nabla(rv) = pv\otimes \rmd r + r\nabla(v).\]
Denote by \emph{$\widetilde{\MIC}(Y,D_Y)$} the category of vector bundles over $Y$ with integrable logarithmic $p$-connections along $D_Y$.

\subsubsection{residue of a logarithmic de Rham bundle} Let $(V,\nabla)$ be a logarithmic de Rham bundle. On any given sufficient small open subset $U$, we choose a coordinate functor $f_i$ for each irreducible component $D_{Y,i}$. Let $e=(e_1,\cdots,e_r)$ be the local frame of $V$, and $\omega$ the relative connection matrix of $D$ with respect to $e$ for $V$ on $U$. Then for each $i$, the matrix $\omega$ can be written as
\[\omega = \sum_{i=1}^{r} R_i \cdot \frac{\rmd f_i}{f_i} + S\]
where $R_i$ is an $r\times r$ matrix with entries in $\mO_Y(U)$ and $S$ is a $r\times r$ matrix with entries in $(\Omega_{Y/S}^1)(U)$. Denote
\[\res_{Y/S}(\omega,D_{Y,i}) := R_i\mid_{U\cap D_{Y,i}}\]
which is $r\times r$ matrix whose entries are contained in $\mO_{D_{Y,i}}(U\cap D_{Y,i})$, and it is independent of choice of $f_i$. When $U$ run through a fine enough covering of $Y$, these matrices are glued into a global section
\[\emph{$\res_{Y/S}(\nabla,D_{Y,i})$} \in \mathrm{H}^0(D_{Y,i},\mEnd_{\mathcal O_Y}(V)\mid_{D_{Y,i}}),\]
which is called the \emph{residue of the connection $\nabla$ along $D_{Y,i}$}. The residue map of the connection along $D_{Y,i}$ is also represented as an $\mO_{D_{Y,i}}$-endomorphism of $V\mid_{D_{Y,i}}$
\[\res_{D_{Y,i}}(\nabla)\colon V\mid_{D_{Y,i}}\longrightarrow V\mid_{D_{Y,i}}.\]
We assume $\res_{D_{Y,i}}(\nabla)$ is quasi-nilpotent and with rational eigenvalues in $[0,1)$.

\subsubsection{Hodge filtration and filtered logarithmic de Rham bundle}
In this paper, a filtration $\Fil^\bullet$ on a logarithmic de Rham bundle $(V,\nabla)$ over $(Y,D_Y)/S$ will be called a \emph{Hodge filtration of level in $[a,b]$} if the following conditions hold:
\begin{itemize}
\item[-] $\Fil^i V$'s are subbundles of $V$, with
\[V=\Fil^aV\supset \Fil^{a+1}V \supset\cdots \supset \Fil^bV\supset \Fil^{b+1}V=0.\]
\item[-] $\Fil$ satisfies Griffiths transversality with respect to the logarithmic connection $\nabla$.
\end{itemize}
In this case, the triple $(V,\nabla,\Fil)$ is called a \emph{filtered logarithmic de Rham bundle}. We denote by \emph{$\MCF(Y,D_Y)$} the category of filtered logarithmic de Rham bundles over $(Y,D_Y)$ and by \emph{$\MCF_a(Y,D_Y)$} those with level in $[0,a]$.


\subsubsection{The parabolic vector bundles associated to logarithmic de Rham bundles (over $\bC$)}

We first recall the following well know result.
\begin{lemma}\label{thm_NatExtLogConn} Let $(V,\nabla)$ be a logarithmic de Rham bundle over $(Y,D_Y)/S$. Denote by $i$ the open immersion of $U=Y\setminus D_Y$ into $Y$. Then
\begin{enumerate}
\item[(1)] the connection $\nabla$ on $V$ can be uniquely extended onto $V_\infty := j_{Y*}j^*_Y(V)$ under the natural injection $V\hookrightarrow V_\infty$, and
\item[(2)] the extension connection can be restricted onto $V(D')$ for any integral divisor $D'$ supported on $D$. In particular, if $D'$ is also positive, then the logarithmic de Rham bundle $(V,\nabla)$ extends to another one $(V(D'),\nabla)$ with natural injection
\[(V,\nabla) \hookrightarrow (V(D'),\nabla).\]
\item[(3)] Let $V'$ be a vector bundle over $Y$ contained in $V_\infty$ as a subsheaf. Then there is at most one connection $\nabla'$ onto $V'$ such that $j_Y^*\nabla = j_Y^*\nabla'$.
\end{enumerate}
\end{lemma}

Denote by $\bQ^S$ the maximal subring of $\bQ$ such that the natural ring homomorphism $\bZ\rightarrow \mO_S(S)$ can be extended to $\bQ^S$
\[ \iota\colon \bQ^S \rightarrow \mO_S(S)\]
Then
\begin{equation*}
\bQ^S = \left\{\begin{array}{ll}
\bQ,& S \text{is in case (i)};\\
\bQ_{(p)} & \text{otherwise.}
\end{array}\right.
\end{equation*}
We say that a logarithmic de Rham bundle $(V,\nabla)$ over $(Y,D_Y)/S$ \emph{has rational eigenvalues}, if the eigenvalues of the residues are all contained in the image of $\iota$.

In the rest of this subsection, we set $S=\Spec(\bC)$.

\begin{proposition}[Iyer-Simpson 2006] \label{IvSi06}
Let $(V,\nabla)$ be a logarithmic de Rham bundle over $(Y,D_Y)$. Assume the residues have rational eigenvalues. Then
\begin{enumerate}
\item[$1).$] 
there exists a unique (locally abelian) parabolic vector bundle $F$ associated to $(V,\nabla)$. I.e. a parabolic vector bundle $F=\{F_\alpha\}$ together with isomorphisms
\[F_\alpha\mid_{Y^0} \cong V\mid_{Y^0}\]
such that for each $\alpha$, the $\nabla$ extends to a logarithmic connection $\nabla_\alpha$ on $F_\alpha$ with the residue on the piece $\Gr_\alpha(F)\coloneqq F_\alpha/F_{\alpha-\varepsilon\delta^i}$ being an operator with eigenvalue $-\alpha_i$.
\item[$2).$]
The eigenvalues of the residue of $\nabla_\alpha$ along $D_{Y,i}$ are contained in the interval $[-\alpha_i,1-\alpha_i)$.
\item[$3).$] 
The construction preserves short exact sequences and restrictions.
\end{enumerate}
\end{proposition}

\begin{corollary} If the eigenvalues $\{\eta_i^j\}$ of the residues are all located in $[0,1)$, then under suitable reordering of these eigenvalues, for each indices pair $(i,j)$, one has
\[\eta_i^j = \left\{\begin{array}{ll}
0,& \text{if} \alpha_i^j=0;\\
1-\alpha_i^j,& \text{if} \alpha_i^j\neq 0\\
\end{array}\right.\]
\end{corollary}

\begin{remark} Let $(V,\nabla)$ be a logarithmic de Rham bundle over $(X,D_Y)$.
\begin{itemize}
\item From the construction of the associated parabolic vector bundle,
\begin{itemize}
\item if the eigenvalues of the residues of $\nabla$ are located in $[0,1)$, then
\[(V,\nabla)=(F_0,\nabla_0)\]
\item if the eigenvalues of the residues of $\nabla$ are located in $[-1,0)$, then
\[(V,\nabla)=(F_{(1,\cdots,1)},\nabla_{(1,\cdots,1)})\]
\item if the eigenvalues of the residues of $\nabla$ are located in $(-1,0]$, then
\[(V,\nabla)=\bigcup_{\varepsilon<1} (F_{(\varepsilon,\cdots,\varepsilon)},\nabla_{(\varepsilon,\cdots,\varepsilon)}).\]
\end{itemize}
\item $F$ is also the parabolic vector bundle associated to logarithmic de Rham bundle $(F_\alpha,\nabla_\alpha)$.
\end{itemize}
\end{remark}


\subsubsection{Parabolic de Rham bundles}

Inspired by \autoref{IvSi06}, one can give a definition for parabolic de Rham bundles and parabolic Higgs bundles.
\begin{definition}
A \emph{parabolic de Rham bundle} $(V,\nabla)=\{(V_\alpha,\nabla_\alpha)\}$ over $(Y,D_Y)/S$ is parabolic vector bundle $V=\{V_\alpha\}$ together with integrable connections $\nabla_\alpha$ having logarithmic pole along $D_Y$ such that the inclusions $V_\alpha\hookrightarrow V_\beta$ preserves the connections.
We call $\nabla:=\{\nabla_\alpha\}$ a \emph{parabolic connection} on the parabolic vector bundle $V$.

A parabolic de Rham bundle $(V,\nabla)=\{(V_\alpha,\nabla_\alpha)\}$
is called \emph{standard} if all parabolic weights are contained in $\bQ^{S}$, and for any $\alpha\in \Big(\bQ^{S}\Big)^n$, the residue on the piece $V_\alpha/V_{\alpha-\varepsilon\delta^i}$ of the logarithmic connection $\nabla_\alpha$ is an operator with eigenvalue $\iota(-\alpha_i)$.
\end{definition}

\begin{remark}
\begin{enumerate}
\item[(1).] The $(V_\alpha,\nabla_\alpha)$ may be considered as de Rham subsheaves of
\[(V_\infty,\nabla_\infty) := \varinjlim_\beta (V_\beta,\nabla_\beta) = j_{Y*}j^*_Y (V_\alpha,\nabla_\alpha).\]
\item[(2).] Tensor product of two parabolic de Rham bundles can be naturally defined: for any two parabolic de Rham bundle $(V,\nabla)$ and $(V',\nabla')$, the underlying parabolic bundle of they tensor product is just $V\otimes V'$ and the parabolic connection is defined as the restrictions of the connection $\Big(\nabla_\infty\otimes \id + \id\otimes\nabla'_\infty\Big)$ on $V_\infty\otimes V'_\infty$.
\end{enumerate}
\end{remark}

\begin{example} Let $\gamma\in\bQ^n$ be a multiple indices and let $(V,\nabla)$ be a logarithmic de Rham bundle over $(Y,D_Y)/S$. There is a natural parabolic connection, denote by $\nabla(\gamma D)$, on the parabolic vector bundle $V(\gamma D)$ given by
\[\nabla(\gamma D)_\alpha:=\nabla_\infty \mid_{V(\gamma D)_{\alpha}}.\]
The logarithmic de Rham bundle $(V,\nabla)$ may be considered as a parabolic de Rham bundle (we say ``with trivial parabolic structure'') by identifying it with $(V(0D),\nabla(0D))$.
Assume $(V,\nabla)=(\mO_Y,\rmd)$. We call parabolic de Rham line bundle of form $(\mO_Y(\gamma D),\rmd(\gamma D))$ \emph{shifting parabolic de Rham line bundles}. A shifting parabolic de Rham line bundle $(\mO_Y(\gamma D),\rmd(\gamma D))$ is of standard if and only if $\gamma_i\in \ker\iota$ for all $i=1,\cdots,n$.
\end{example}

\begin{proposition} If $S$ has characteristic zero, then any standard parabolic de Rham bundle over $(X,D_Y)$ is semistable and of degree zero.
\end{proposition}

Recall that a logarithmic $p$-connection on a vector bundle $V$ over $(Y,D_Y)/S$ is an $\mO_S$-linear mapping
\[\nabla \colon V \rightarrow V\otimes \Omega_{Y/S}(\log D_Y)\]
satisfying, for any local section $s \in \mO_Y$ and any local section $v\in V$
\[\nabla(sv) = p v\otimes \rmd s + s\nabla(v)\]
We note that the multiplication of a connection with $p$ is always a $p$-connection and if $p$ is invert in $\mO_S(S)$, then all $p$-connections are coming from this way.
Similarly, one can defines parabolic $p$-connections on parabolic vector bundles, and their tensor products. We left the routine definitions to the readers.


\subsubsection{Parabolic de Rham line bundles}

By tensor some shifting parabolic de Rham line bundle, a parabolic de Rham bundle can be modified to be with trivial parabolic structure. Thus we have following result.
\begin{lemma} Let $(\mL,\nabla)$ be a parabolic de Rham line bundle over $(Y,D_Y)/S$. Denote by $w_i\in [0,1)\cap \bQ$ the parabolic weight of $\mL$ along $D_{Y,i}$ and denote $w=(w_1,\cdots,w_n)$. Then
\begin{enumerate}
\item[$(1)$] $(\mL,\nabla)$ can be decomposed as a tensor product of usual logarithmic de Rham bundle and a shifting parabolic de Rham line bundle.
\[(\mL,\nabla) = (L,\nabla)\otimes (\mO_Y(\gamma D),\rmd(\gamma D)).\]
\item[$(2)$] Moreover, if we require $\gamma \in \Big([0,1)\cap \bQ\Big)^n$, then $\gamma=w$ and $(L,\nabla)=(L_0,\nabla_0)$. In other words, the decomposition is unique of form
\[(\mL,\nabla) = (\mL_0,\nabla_0) \otimes (\mO_Y(w D),\rmd(w D)).\]
\end{enumerate}
\end{lemma}


\begin{remark} By the same method, one has similar decompositions for parabolic vector bundle with parabolic $p$-connection.
\end{remark}


\subsubsection{The pullback of a parabolic de Rham vector bundles}

Now, we consider the pullback for parabolic de Rham bundle case. Let $(V,\nabla)$ be a parabolic de Rham bundle over $(Y,D_Y)/S$. For any $\gamma \in \bQ^n$, then $f^*(\gamma D)$ is also a rational divisor over $Y'$ supported on $D'$. We simply set
\[f^*\Big(\mO_Y(\gamma D),\rmd_Y(\gamma D)\Big) := \Big(\mO_{Y'}\big(f^*(\gamma D)\big),\rmd_{Y'}\big(f^*(\gamma D)\big)\Big).\]

Similarly as the parabolic vector bundle case, for each $\gamma \in \bQ^n$, we set
\[f^*_{\gamma}(V,\nabla)
\coloneqq f^*\Big(\big(V(-\gamma D),\nabla(-\gamma D)\big)_0\Big) \otimes f^*\Big(\mO_Y(\gamma D),\rmd_Y(\gamma D)\Big).\]

Denote $U_Y=Y-D_Y$ and $U_{Y'}=Y'-D_{Y'}$. If we restrict $f^*_{\gamma}(V,\nabla)$ onto the open subset $U_{Y'}$, then by the construction, one gets
\[\Big(f^*_{\gamma}(V,\nabla)\Big)\mid_{U_{Y'}}=f^*\Big((V,\nabla)\mid_{U_Y}\Big).\]
By \autoref{thm_NatExtLogConn}, the connections on $f^*_{\gamma}(V,\nabla)$ for difference choice of $\gamma$ are coincide with each others over the maximal common subsheaf.
\begin{definition} \label{def:pullbackdR}
Let $(V,\nabla)$ be a parabolic de Rham bundle over $(Y,D_Y)/S$. We define the pullback of $(V,\nabla)$ along $f$ as
\[f^*(V,\nabla)\coloneqq \bigcup_{\gamma\in\bQ^n} f^*_{\gamma}(V,\nabla).\]
\end{definition}

\subsection{Parabolic Higgs bundles}

\subsubsection{Graded vector bundles}
Let $\{\Gr^\ell E\}_{\ell\in \bZ}$ be subbundles of $V$. The pair $(E,\Gr)$ is called \emph{graded vector bundle} over $X$ if the natural map $ \oplus_{\ell\in \bZ} \Gr^\ell E\cong E$ is an isomorphism.

\subsubsection{logarithmic Higgs bundles}
Let $E$ be vector bundle over $X$. Let $\theta\colon E\rightarrow E\otimes_{\mO_X} \Omega^1_{X/S}(\log D)$ be an $\mO_X$-linear morphism. The pair $(E,\theta)$ is called \emph{(logarithmic) Higgs bundle} over $(X,D)/S$ if $\theta$ is integrable. i.e. $\theta\wedge\theta = 0$. For a Higgs bundle, one has a natural \emph{Higgs complex}
\[\DR(E,\theta): 0\rightarrow E
\xrightarrow{\theta} E\otimes\Omega^1_{X/S}(\log D)
\xrightarrow{\theta} E\otimes\Omega^2_{X/S}(\log D)
\xrightarrow{\theta} E\otimes\Omega^3_{X/S}(\log D) \rightarrow \cdots \]

\subsubsection{graded logarithmic Higgs bundles} A \emph{graded (logarithmic) Higgs bundle} over $(X,D)/S$ is a Higgs bundle $(E,\theta)$ together with a grading structure $\Gr$ on $E$ satisfying
\[\theta(\Gr^\ell E) \subset \Gr^{\ell -1}E\otimes_{\mO_X} \Omega^1_{X/S}(\log D).\]
Thus we have subcomplexes of $\DR(E,\theta)$
\[\Gr^\ell \DR(E,\theta):
 0
\rightarrow \Gr^\ell E
\xrightarrow{\theta} \Gr^{\ell-1} E \otimes \Omega^1_{X/S}(\log D)
\xrightarrow{\theta} \Gr^{\ell-2} E \otimes \Omega^2_{X/S}(\log D)
\xrightarrow{\theta} \cdots.\]

The following is the main example we will be concerned with.
\begin{example}
Let $(V,\nabla,\Fil)$ be a filtered de Rham bundle over $V$. Denote $E =\bigoplus_{\ell\in \mathbb Z} \Fil^\ell$ $ V/\Fil^{\ell+1}V$ and $\Gr^\ell E = \Fil^\ell V/\Fil^{\ell+1}V$. By Griffith's transversality, the connection induces an $\mO_X$-linear map $\theta \colon \Gr^\ell E \rightarrow \Gr^{\ell-1} E \otimes_{\mO_X} \Omega^1_{X/S}(\log D)$ for each $\ell\in \bZ$. Then $(E,\theta,\Gr)$ is a graded Higgs bundle. Moreover we have
\[\Gr^{\ell} \DR(E,\theta) = \Fil^\ell \DR(V,\nabla) / \Fil^{\ell+1} \DR(V,\nabla).\]
\end{example}

\subsubsection{Parabolic Higgs bundles}

\begin{definition}
A \emph{parabolic Higgs bundle} $(E,\theta)=\{(E_\alpha,\theta_\alpha)\}$ over $(Y,D_Y)$ is
\begin{itemize}
\item a parabolic vector bundle $E=\{E_\alpha\}$, together with
\item integrable Higgs fields $\theta_\alpha$ having logarithmic pole along $D_Y$
\end{itemize}
such that the inclusions $E_\alpha\hookrightarrow E_\beta$ preserves the Higgs fields.

A parabolic Higgs bundle $(E,\theta)$ is called \emph{graded}, if there is a grading structure $Gr$ on $E$ satisfying decomposition of the underlying parabolic vector bundle $E$
\[\theta(\Gr^\ell E) \subset \Gr^{\ell -1}E\otimes_{\mO_X} \Omega^1_{X/S}(\log D).\]
\end{definition}

\begin{remark}
\begin{enumerate}
\item[(1).] The $(E_\alpha,\theta_\alpha)$ may be considered as Higgs subsheaves of
\[(E_\infty,\theta_\infty) := \varinjlim_\beta (E_\beta,\theta_\beta) = j_{Y*}j^*_Y (E_\alpha,\theta_\alpha).\]
\item[(2).] Tensor product of two parabolic Higgs bundle bundles can be naturally defined: for any two parabolic Higgs bundle $(E,\theta)$ and $(E',\theta')$, the underlying parabolic bundle of they tensor product is just $E\otimes E'$ and the parabolic connection is defined as the restrictions of the connection $\Big(\theta_\infty\otimes \id + \id\otimes \theta'_\infty\Big)$ on $E_\infty\otimes E'_\infty$.
\end{enumerate}
\end{remark}


\subsection{Parabolic de Rham bundles and parabolic Higgs bundles of lower rank over projective lines} \label{sec_para_classification_lower_rank}

In this section, we take $X=\bP^1_S$ as the projective line over $S$ and take $D=D_S\subset \bP_S^1$ as the divisor given by $4$ $S$-points $\{0,1,\infty,\lambda\}$. Denote by $D_i$ the reduce and irreducible divisor given by the point $x$ for any $x\in\{0,1,\infty,\lambda\}$.

\subsubsection{Classification of parabolic de Rham line bundles}

We first classifies all logarithmic de Rham line bundle over $(\bP^1_S,D_S)/S$.
\begin{lemma} \label{thm_ClassifyLogdRLineBundle}
\begin{enumerate}
\item[$(1)$] Let $e=(e_0,e_1,e_\lambda,e_\infty)\in \Big(\mO_S(S)\Big)^4$ and $d\in \bZ$ satisfy
\[e_0+e_1+e_\lambda+e_\infty + d = 0 \in \mO_S(S).\]
Then up to an isomorphism there is a unique logarithmic de Rham line bundle $(L^{(e,d)},\nabla^{(e,d)})$ over $(\bP^1_S,D_S)/S$ such that
\begin{itemize}
\item the degree of $L^{(e,d)}$ is of degree $d$, and
\item $e_x$ is the eigenvalue of the residue of $\nabla^{(e,d)}$ along $D_x$ for all $x=\{0,1,\lambda,\infty\}$.
\end{itemize}
\item[$(2)$] Any logarithmic de Rham line bundle is of this form.
\end{enumerate}
\end{lemma}
\begin{proof}
(1). \textbf{Existence:} We first take $L^{(e,d)}=\mO_{\bP^1_S}(-d(\infty))$ and set
\[\nabla^{(e,d)}(1) = 1\otimes \Big(e_0 \rmd\log t + e_1\rmd\log(t-1) + e_\lambda \log(t-\lambda)\Big)\]
where $t$ is the parameter of the projective line. clearly, the residues of $\nabla^{(e,d)}$ at $0,1,\lambda$ are $e_0,e_1,e_\lambda$ respectively. We only need to show $\nabla^{(e,d)}$ can extends over $\infty$ and the residue is equal to $e_\infty$. This follows the following explicit computation:
\begin{equation*}
\begin{split}
\nabla^{(e)}\left(t^d\right) & = 1\otimes \rmd\left(t^d\right) + t^d\cdot\nabla^{(e)}\left(1\right)\\
& = t^d \otimes \Big(-d\cdot\rmd\log\frac1t + e_0 \rmd\log t + e_1\rmd\log(t-1) + e_\lambda \log(t-\lambda)\Big) \\
& = t^d \otimes \Big(-d-e_0 -e_1\cdot\frac{t}{t-1} - e_\lambda\cdot\frac{t}{t-\lambda}\Big)\cdot \rmd\log\frac1t\\
\end{split}
\end{equation*}

\textbf{Uniqueness:} Suppose there are two logarithm de Rham line bundles $(L,\nabla)$ and $(L',\nabla')$ satisfy the conditions. Since our base space is projective line, $L$ and $L'$ has the some degree, they must isomorphic to each other. So we may identify them $L=L'$. Now, on this line bundle there are two logarithmic connections with the same residues. They must coincide with each other, since
\[\nabla-\nabla'\in \Hom_{\mO_{\bP^1_S}}(L,L \otimes \Omega^1_{\bP^1_S}) \cong H^0(\bP^1_S,\Omega^1_{\bP^1_S})=0.\]

(2). Suppose $(L,\nabla)$ be a logarithmic de Rham line bundle with rational eigenvalues $(e_0,e_1,e_\infty,e_\lambda)$ and degree $d$. Denote $e_\infty'\coloneqq -(e_0+e_1+e_\lambda+\iota(d))$ and $e'=(e_0,e_1,e_\lambda,e_\infty')$. By (1), one has logarithmic de Rham bundle $(L^{(e',d)},\nabla^{(e',d)})$. By similar proof as in (1), we may identify $L^{(e',d)}$ with $L$, and gets
\[\nabla-\nabla^{(e',d)} \in \Hom_{\mO_{\bP^1_S}}(L,L \otimes \Omega^1_{\bP^1_S}(\log D_\infty)) \cong H^0(\bP^1_S,\Omega^1_{\bP^1_S}(\log D_\infty))=0.\]
Thus $e'_\infty=e_\infty$ and $(L,\nabla)\cong (L^{(e',d)},\nabla^{(e',d)})$.
\end{proof}

\begin{lemma} Let $e=(e_0,e_1,e_\lambda,e_\infty)\in \Big(\mO_S(S)\Big)^4$ and $d\in \bZ$ satisfy
\[e_0+e_1+e_\lambda+e_\infty + d = 0 \in \mO_S(S).\]
For any $m=(m_0,m_1,n_\lambda,m_\infty)\in \bN^4$, denote
\[e' = (e_0-m_0,e_1-m_1,e_\lambda-m_\lambda,e_\infty-m_\infty) \quad \text{and} \quad d'=d+m_0+m_1+m_\lambda+m_\infty.\]
Then the natural extension $(L^{(e,d)}(D'),\nabla^{(e,d)})$ (\autoref{thm_NatExtLogConn}) with respect to $D'=m_0D_0 + m_1D_1 + m_\lambda D_\lambda + m_\infty D_\infty$ is isomorphic to $(L^{(e',d')},\nabla^{(e',d')})$.
\end{lemma}

\begin{proof}
By the uniqueness in \autoref{thm_ClassifyLogdRLineBundle}, we only show that $e'$ is coincide with the residues of $\nabla^{(e,d)}$ on $L^{(e,d)}(D')$. Let $f$ be a local generator of $L^{(e,d)}$ around $t=0$ and suppose $\nabla^{(e,d)}(f)=f\otimes \omega$. Then $x^{-m_0}f$ is the local generator of $L^{(e,d)}(D')$ and
\[\nabla^{(e,d)}(x^{-m_0}f) = x^{-m_0}f \otimes (-m_0 + \omega).\]
Thus $e_0-m_0$ is the residue of $\nabla^{(e,d)}$ on $L^{(e,d)}(D')$ along $D_0$. Similarly, one checks that around the other three points.
\end{proof}

We classifies all parabolic de Rham line bundle over $(\bP^1_S,D_S)/S$.
\begin{lemma} \label{thm_ClassfyParabolicdRLineBundle}
\begin{enumerate}
\item[(1)] Let $w=(w_0,w_1,w_\lambda,w_\infty)\in \Big([0,1)\cap \bQ\Big)^4$, $e=(e_0,e_1,e_\lambda,e_\infty)\in \Big(\mO_S(S)\Big)^4$, and $d\in \bZ$ satisfy
\[e_0+e_1+e_\lambda+e_\infty + d = 0 \in \mO_S(S).\]
Then there exists a unique parabolic de Rham line bundle $\{(L^{(e,d,w)}_\alpha,\nabla^{(e,d,w)}_\alpha)\}_{\alpha}$ such that
\begin{itemize}
\item $d=\deg(L^{(e,d,w)}_0)$;
\item $e_x$ is the eigenvalue of residue of $\nabla^{(e,d,w)}_0$ along $D_x$ for $x\in\{0,1,\lambda,\infty\}$, and
\item $w_x$ is the parabolic weight of the parabolic line bundle $\{L^{(w,d,e)}_\alpha\}_\alpha$ along $D_x$ for $x\in\{0,1,\lambda,\infty\}$;
\end{itemize}
\item[(2)] The underlying parabolic line bundle of $\{(L^{(e,d,w)}_\alpha,\nabla^{(e,d,w)}_\alpha)\}_{\alpha}$ is isomorphic to
\[\mO_{\bP^1_S}(d)(w_0D_0 + w_1D_1 + w_\lambda D_\lambda +w_\infty D_\infty).\]
The line bundle $L^{(e,d,w)}_\alpha$ is of degree
\[d'\coloneqq d+[\alpha_0+w_0]+[\alpha_1+w_1]+[\alpha_\lambda+w_\lambda]+[\alpha_\infty+w_\infty]\]
and the eigenvalue of the residue of $\nabla^{(e,d,w)}_\alpha$ along $D_x$ is $e'_x\coloneqq e_x-[\alpha_x+w_x]$. In other words,
\[(L^{(e,d,w)}_\alpha,\nabla^{(e,d,w)}_\alpha) \cong (L^{(e',d')},\nabla^{(e',d')}).\]
\item[(3)] Any parabolic de Rham line bundle over $(\bP^1_S,D_S)/S$ is of the form given in (1).
\end{enumerate}
\end{lemma}

\begin{proof}
(1) \textbf{Existence:} For any $\alpha=(\alpha_0,\alpha_1,\alpha_\lambda,\alpha_\infty)\in\bQ^4$, denote $D_\alpha = \alpha_0D_0 + \alpha_1D_1 + \alpha_\lambda D_ \lambda+ \alpha_\infty D_\infty$. We set
\[(L^{(e,d,w)}_\alpha,\nabla^{(e,d,w)}_\alpha) \coloneqq (L^{(e,d)}(D_{[\alpha+w]}),\nabla^{(e,d)})\]
where $(L^{(e,d)},\nabla^{(e,d)})$ is the logarithmic de Rham line bundle given in \autoref{thm_ClassifyLogdRLineBundle} and the extension $(L^{(e,d)}(D_{[\alpha+w]}),\nabla^{(e,d)})$ is defined as in \autoref{thm_NatExtLogConn}. The natural injections introduced in \autoref{thm_NatExtLogConn} makes the collection $\{(L^{(e,d,w)}_\alpha,\nabla^{(e,d,w)}_\alpha)\}$ forming a parabolic de Rham bundle, which satisfies our requirement clearly.

\textbf{Uniqueness:} Let $(L_\alpha,\nabla_\alpha)$ and $(L'_\alpha,\nabla'_\alpha)$ be two parabolic de Rham line bundles satisfy our requirement. Then we may identify $(L_0,\nabla_0)$ and $(L'_0,\nabla'_0)$ according \autoref{thm_ClassifyLogdRLineBundle}. Thus
\[(L_\alpha,\nabla_\alpha) = (L_0(D_{[\alpha+w]}),\nabla_0) = (L'_0(D_{[\alpha+w]}),\nabla'_0) =(L'_\alpha,\nabla'_\alpha),\]
where the first and third equalities follow from the normalization and the definition of the weights of parabolic de Rham bundles.

(2) This follows the construction in (1).

(3) For any parabolic de Rham line bundle $\{(L_\alpha,\nabla_\alpha)\}_{\alpha}$, we have associated datum $(w,e,d)$ consisting of weights, eigenvalues and degree. By the uniqueness in (1), the only thing we need to check is \[e_0+e_1+e_\lambda+e_\infty + d = 0 \in \mO_S(S).\]
And this follows \autoref{thm_ClassifyLogdRLineBundle}.
\end{proof}

We classifies all standard parabolic de Rham line bundle over $(\bP^1_S,D_S)/S$.
\begin{lemma} Let $w=(w_0,w_1,w_\infty,w_\lambda)$ be an element in $\Big([0,1)\cap \bQ^S\Big)^4$. Denote
\[e_{w,x}\coloneqq\iota(w_x) \text{\quad and \quad} e_w\coloneqq(e_{w,0},e_{w,1},e_{w,\lambda},e_{w,\infty})\in \Big(O_S(S)\Big)^4.\]
For any integer $d$ with
\[e_{w,0}+e_{w,1}+e_{w,\lambda}+e_{w,\infty}+d=0\in \mO_S(S)\]
the parabolic de Rham line bundle $\{(L^{(e_w,d,w)}_\alpha,\nabla^{(e_w,d,w)}_\alpha)\}_{\alpha}$ is standard. And any standard parabolic de Rham line bundle is of this form.
\end{lemma}

\begin{proof}
Fix a parabolic de Rham bundle $\{(L^{(e,d,w)}_\alpha,\nabla^{(e,d,w)}_\alpha)\}_{\alpha}$.
Recall that it is standard if and only if the eigenvalue of the residue of the connection $\nabla^{(e,d,w)}_\alpha$ on the grading piece $V_\alpha/V_{\alpha-\varepsilon\delta_x}$ is $\iota(-\alpha_x)$ for any $\alpha$. By the definition of the parabolic weights, for any $\alpha$, the grading piece does not vanish if and only if $\alpha_x+w_x$ is an integer. On the other hand, according (2) in \autoref{thm_ClassfyParabolicdRLineBundle}, $(L^{(e,d,w)}_\alpha,\nabla^{(e,d,w)}_\alpha)$ along $D_x$ is $e_{x}-[\alpha_x+w_x]$. Thus $\{(L^{(e,d,w)}_\alpha,\nabla^{(e,d,w)}_\alpha)\}_{\alpha}$ is standard if and only if for any $\alpha$ with $\alpha_x+w_x$ being an integer one has $e_{x}-[\alpha_x+w_x]=\iota(-\alpha_x) \in \mO_S(S)$.

If $\{(L^{(e,d,w)}_\alpha,\nabla^{(e,d,w)}_\alpha)\}_{\alpha}$ is standard, then taking $\alpha$ such that $\alpha_x = -w_x$ one gets
\[e_x = \iota(w_x)=e_{w,x} \in \mO_S(S).\]
In other words, it is of the form given in the Lemma.

Since $e_{w,x}=\iota(w_x)$, for any $\alpha$ with $\alpha_x+w_x$, one has $e_{w,x}-[\alpha_x+w_x]= e_{w,x} - \iota(\alpha_x) - \iota(w_x) =\iota(-\alpha_x) \mO_S(S)$. Thus $\{(L^{(e_w,d,w)}_\alpha,\nabla^{(e_w,d,w)}_\alpha)\}_{\alpha}$ is indeed standard.
\end{proof}

\begin{corollary}
\begin{enumerate}
\item[$(1)$] A standard parabolic de Rham line bundle is uniquely determined by its weights and degree.
\item[$(2)$] Suppose $\Char(\mO_S(S))=0$. Then a standard parabolic de Rham line bundle is uniquely determined by its weights. In this case the degree of the underlying parabolic line bundle is zero.
\end{enumerate}
\end{corollary}

\subsubsection{Some parabolic de Rham bundles of rank $2$}

Denote by \emph{$\MdRh(S)$} the set of all isomorphic classes of rank-$2$ stable standard parabolic de Rham bundles $(V,\nabla)$ of degree zero on $(\bP^1_S,D_S)/S$ with all parabolic weights being zero at $\{0,1,\lambda\}$ and with all parabolic weights being $1/2$ at $\infty$.

\begin{proposition} \label{thm_ClassfyR2PdE} Let $(V,\nabla)$ be a parabolic de Rham bundle in $\MdRh(S)$. Then
\begin{enumerate}
\item[$(1)$]
the parabolic de Rham bundle $(V,\nabla)$ has the form
\[(\mL\oplus \mL^{-1},\nabla).\]
where $\mL=\mO(\frac12(\infty))$.
\item[$(2)$]
if we take the \emph{parabolic Hodge line bundle} as $\mL$, then the associated graded parabolic Higgs field is nonzero and is of form
\[\theta\colon \mL\rightarrow \mL^{-1}\otimes \Omega^1_{X/S}(\log D).\]
In particular, the graded parabolic Higgs bundle $(\mL\oplus \mL^{-1},\theta)$ is stable and is of degree zero.
\end{enumerate}
\end{proposition}

\begin{proof}
By tensoring with $\mO_{\bP^1_S}(-\frac12(\infty))$, one gets a parabolic vector bundle $V(-\frac12(\infty))$ of degree $-1$ and with trivial parabolic weights. Thus it is usual vector bundle of degree $-1$. We decompose $V(-\frac12(\infty))$ into direct sum of $\mO_{\bP^1_S}(d)$ and $\mO_{\bP^1_S}(-d-1)$ for some $d\geq0$. Denote $\mL=O(\frac{2d+1}2(\infty))$. Then one has
\[V = \mL\oplus \mL^{-1}.\]
In the following, we use the stability to show $d=0$. Since $(V,\nabla)$ is stable, one has $\nabla(\mL)\not\subset \mL\otimes \Omega^1_{\bP^1_S}(\log D)$. In other words, the graded parabolic Higgs field, which is defined as the composition of following maps
\[\mL \xrightarrow{\nabla} V\otimes\Omega^1_{\bP^1_S}(\log D) \twoheadrightarrow \mL^{-1}\otimes\Omega^1_{\bP^1_S}(\log D)\]
is nonzero. Compute the degree on both sides of the nonzero map, one gets
\[\frac{2d+1}2 \leq -\frac{2d+1}2 + 2.\]
Thus $d\leq 0$. The lemma follows.
\end{proof}


\subsubsection{Some graded parabolic Higgs bundles of rank $2$}

Denote by \emph{$\High(S)$} the set of all isomorphic classes of rank-$2$ stable graded parabolic Higgs bundles $(E,\theta)$ of degree zero on $(\bP^1_S,D_S)/S$ with all parabolic weights being zero at $\{0,1,\lambda\}$ and with all parabolic weights being $1/2$ at $\infty$.

\begin{proposition}\label{thm_ClassfyR2PHiggs} Let $(E,\theta)$ be a graded parabolic Higgs bundle in $\High(S)$. Then
\[V = \mL\oplus \mL^{-1},\]
where $\mL=\mO(\frac12(\infty))$ and the parabolic Higgs field is nonzero and is of form
\[\theta\colon \mL\rightarrow \mL^{-1}\otimes \Omega^1_{X/S}(\log D).\]
\end{proposition}

\begin{proof}
By tensoring with $\mO_{\bP^1_S}(-\frac12(\infty))$, one gets a parabolic vector bundle $V(-\frac12(\infty))$ of degree $-1$ and with trivial parabolic weights. Thus it is usual vector bundle of degree $-1$. We decompose $V(-\frac12(\infty))$ into direct sum of $\mO_{\bP^1_S}(d)$ and $\mO_{\bP^1_S}(-d-1)$ for some $d\geq0$. Denote $\mL=O(\frac{2d+1}2(\infty))$. Then one has
\[V = \mL\oplus \mL^{-1}.\]
By the stability, $\theta(\mL)\not\subset \mL\otimes \Omega^1_{\bP^1_S}(\log D)$. Thus graded parabolic Higgs field nonzero and is of form
\[\theta\colon \mL \rightarrow \mL^{-1}\otimes\Omega^1_{\bP^1_S}(\log D).\]
\end{proof}

\begin{corollary}\label{thm_Higgs0ProjLine} Any parabolic Higgs bundle $(E,\theta)\in \High(S)$ is uniquely determined by $(\theta)_0 \in \bP^1_S(S)$, the zero of the Higgs field $\theta$. One has a natural bijection induced by taking zeros
\[\High(S) \xrightarrow[(E,\theta)\mapsto (\theta)_0]{1:1} \bP^1_S(S).\]
\end{corollary}

\newpage

\section {\bf Parabolic Fontaine-Faltings Modules and parabolic Higgs-de Rham Flows} \label{sec_main_FF_and_Higgs_de_Rham}

In order to facilitate use, we introduce the definition of Parabolic Fontaine-Faltings Modules and parabolic Higgs-de Rham Flows. One can also see \cite{KrSh20}, for original definitions and more detailed studies.

\subsection{Introduction to Fontaine-Faltings modules}

In this subsection, we set $S=\mathrm{Spf}(W)$. Then $\mY$ is a proper smooth formal scheme over $S$ and $D_\mY$ be a reduced relative simple normal crossing $S$-divisor in $\mY$.

\subsubsection{Faltings' tilde functor   } 
We recall a functor
\[\widetilde{(\cdot)}\colon \MCF(\mY,D_\mY) \rightarrow \widetilde{\MIC}(\mY,D_\mY)\]
which was introduced by Faltings in \cite{Fal89}. We call it \emph{Faltings' tilde functor} and denote it by \emph{$\widetilde{(\cdot)}$}. For an object $(V,\nabla,\Fil)$ in $\MCF_{a}(\mY,D_\mY)$, denote by $\widetilde{V}$ the quotient $\bigoplus\limits_{i=0}^{a}\Fil^iV/\sim$ with $x\sim py$ for any local section $x\in\Fil^iV$ and $y$ the image of $x$ under the natural inclusion $\Fil^iV\hookrightarrow\Fil^{i-1}V$. Then connection $\nabla$ naturally induces a $p$-connection on $\widetilde{\nabla}$ on $\widetilde{V}$. We use $\widetilde{(V,\nabla,\Fil)}$ stands for the pair $(\widetilde{V},\widetilde{\nabla})$. The morphisms under the functor are defined in the obvious way.

If $V$ is $p$-torsion free, then the tilde functor can be easily described as follows:
\[(\widetilde{V},\widetilde{\nabla}):= \left(\sum_{i\in\bZ} \frac1{p^\ell}\Fil^iV,p\nabla\right)\subset (V,p\nabla)\otimes_{\Zp}\Qp.\]

\subsubsection{The Frobenius pullback functor $\mF$} Faltings also constructed a functor \cite{Fal89}
\[\mF\colon \widetilde{\MIC}_a(\mY,D_\mY) \rightarrow \MIC(\mY,D_\mY)\]
where $\widetilde{\MIC}_a(\mY,D_\mY)$ is the full subcategory of $\widetilde{\MIC}(\mY,D_\mY)$ consisting of the essential image of $\MCF_a(\mY,D_\mY)$ under Faltings' tilde functor for $a\leq p-2$.\footnote{The condition here is essential, which ensure the functor is globally well-defined.} We recall the definition as follows.

For small affine open subset $\mU$ of $\mY$, there exists endomorphism $F_\mU$ on $\mU$ which lifts the absolute Frobenius on $\mU_k$ and is compatible with the Frobenius map $F_S$ on $S=\Spec(W(k))$. Let $(\widetilde{V},\widetilde{\nabla})$ be an object in $\widetilde{MIC}_a(\mY,D_\mY)$. Locally on $\mU$, applying the functor $F_{\mU}^*$, we get a de Rham bundle over $\mU$
\[F_{\mU}^*(\widetilde{V}\mid_{\mU},\widetilde{\nabla}\mid_{\mU})\]
where the underlying bundle is just pullback of $\widetilde{V}\mid_{\mU}$ along $F_{\mU}^*$ and the connection is the pullback of $\widetilde{\nabla}\mid_{\mU}$ along $F_{\mU}^*$ dividing $p$. By Taylor formula, up to a canonical isomorphism, it does not depend on the choice of $F_\mU$ in case $a\leq p-2$. In particular, on the overlap of two small affine open subsets, there is an canonical isomorphism of two logarithmic de Rham bundles. By gluing via those isomorphisms, one gets a logarithmic de Rham bundle over $(\mY,D_\mY)$, we denote it by
\[\mF(\widetilde{V},\widetilde{\nabla}).\]
The morphisms under the functor are defined in the obvious way.

\subsubsection{A logarithmic Fontaine-Faltings module}
A logarithmic Fontaine-Faltings module\footnote{We note that the definition is slightly different with the original \cite{Fal89} in textual representation, but they are essentially the same, this can be get from \cite[Section 2.3]{SYZ22}.} over $(\mY,D_\mY)$ is a quadruple $(V,\nabla,\Fil,\Phi)$ where $(V,\nabla,\Fil)$ is a logarithmic de Rham bundle over $(\mY,D_\mY)$ and
\[\Phi\colon \mF\widetilde{(V,\nabla,\Fil)} \cong (V,\nabla)\]
is an isomorphism of logarithmic de Rham bundles over $(\mY,D_\mY)$. We call $\Phi$ the \emph{Frobenius structure} in the Fontaine-Faltings module.


\subsubsection{Local logarithmic Fontaine-Faltings modules}
Let $Y=\Spec(R)$ be an affine $W$-scheme with an \'etale map $$W[T_1,T_2,\cdots,T_{d}]\rightarrow R,$$ over $W$,let $D$ be the divisor in $Y$ defined by $T_1\cdots T_d=0$,and let $U$ be the complement of $D$ in $Y$. Therefore, $U$ is a small affine scheme. In this context, we say that $(Y,D)$ is \emph{log small}. Denote by $\mY$ the $p$-adic formal completion of $Y$ along the special fiber $Y_1$ and by $\mY_K$ the rigid-analytic space associated to $\mY$,which is an open subset of $Y_K^{\rm an}$. We construct spaces $D_K$,$\mD$,$\mD_K$,$U_K$, $\mU$ and $\mU_K$ exactly analogously to those for $Y$. Denote $\mY^\circ_K:=\mY_K-\mD_K$. Denote by $\widehat{R}$ the $p$-adic completion of $R$, so $\mY=\text{Spf}(\widehat{R})$.

Choose $\Phi:\widehat{R}\rightarrow\widehat{R}$ a lifting of the absolute Frobenius on $R/pR$ such that $\Phi(T_i)=unit\cdot T_i^p$. Recall a \emph{logarithmic Fontaine-Faltings module} over the $p$-adic formal completion $(\mY,\mD_\mY)$ of $(Y,D)$ with Hodge-Tate weights in $[a,b]$ is a quadruple $(V,\nabla,\Fil,\varphi)$, where
\begin{itemize}
\item[-] $(V,\nabla)$ is a finitely generated locally free\footnote{For our application, we only consider the locally free case at here. We note that the general definition does not require such kind condition.} de Rham $\widehat{R}$-module with logarithmic poles along $T_1\dots T_d=0$;
\item[-] $\Fil$ is a Hodge filtration on $(V,\nabla)$ of level in $[a,b]$;
\item[-] $\widetilde{V}$ is the quotient $\bigoplus\limits_{i=a}^b\Fil^i/\sim$ with $x\sim py$ for $x\in\Fil^iV$ with $y$ being the image of $x$ under the natural inclusion $\Fil^iV\hookrightarrow\Fil^{i-1}V$;
\item[-] $\varphi$ is an $\widehat{R}$-linear isomorphism \[\varphi:\widetilde{V}\otimes_{\Phi}\widehat{R} \longrightarrow V,\]
\item[-] The relative Frobenius $\varphi$ is horizontal with respect to the connections.
\end{itemize}
In particular, a logarithmic Fontaine-Faltings module may be considered as a filtered logarithmic $F$-crystal in finite, locally free modules.
Denote by $\MF_{[a,b]}^{\nabla,\Phi}((\mY,\mD_\mY)/W)$ the category of logarithmic Fontaine-Faltings modules over $(\mY,\mD_\mY)$ with Hodge-Tate weights in $[a,b]$. For the rest of what follows, we assume that $b-a\leq p-2$. (It will follow that the resulting category is independent of the choice of $\Phi$, exactly as in the non-logarithmic case.)


\subsubsection{Decomposition of the ring $\bZ_{p^f}\otimes_{\bZ_p} \bZ_{p^f}$}

Take an generator $\zeta\in \bZ_{p^f}$ over $\bZ_p$ (in other words, $\bZ_{p^f}=\bZ_{p}[\zeta]$) and denote
\begin{equation}
e_i:= \frac{\prod\limits_{j=0\atop j\neq i}^{f-1}\Big(1\otimes\zeta-\zeta^{\sigma^j}\otimes 1\Big)}{\prod\limits_{j=0\atop j\neq i}^{f-1}\Big(\zeta^{\sigma^i}\otimes 1-\zeta^{\sigma^j}\otimes 1\Big)} \in \bZ_{p^f}\otimes_{\bZ_p} \bZ_{p^f}.
\end{equation}

\begin{lemma}
\begin{enumerate}
\item[$(1)$] the decomposition $1=e_0+e_1+\cdots+e_{f-1}$ is an idempotent one,
\item[$(2)$] $(\sigma\otimes\id)(e_i) = e_{i+1}$,
\item[$(3)$] $(1\otimes\zeta)\cdot e_i = (\zeta^{\sigma^i}\otimes1)\cdot e_i$.
\end{enumerate}
\end{lemma}

\subsubsection{Endomorphism structure on logarithmic Fontaine-Faltings module.}

Let $(V,\nabla,\Fil,\varphi)$ be a logarithmic Fontaine-Faltings module. Recall that a
\emph{$\bZ_{p^f}$-endomorphism structure} is a ring homomorphism
\[\tau\colon \bZ_{p^f} \rightarrow \End\big((V,\nabla,\Fil,\varphi)\big).\]
In this case, $V$ are endow with both $\widehat{R}$-module structure and $\bZ_{p^f}$-module structure. Suppose $\bF_{p^f}\subseteq k$. Then one has canonical embedding $\bZ_{p^f}\subset W(k) \subset \widehat{R}$. Thus one gets two $\bZ_{p^f}$-module structures on $V$. It is natural to consider the sub-$\widehat{R}$-modules, for all $i\geq0$
\[V_i:=V^{\tau=\sigma^i} := \{v\in V\mid \tau(a)(v) = \sigma^{i}(a)v,\text{for all} a\in \bZ_{p^f}\},\]
where $\sigma\colon \bZ_{p^f}\rightarrow \bZ_{p^f}$ is the lifting of the absolute Frobenius map on $\bF_{p^f}$. Clearly $V_{i}=V_{i+f}$ for any $i\geq0$ because of $\sigma^f=\id$.

\begin{lemma} Let $(V,\nabla,\Fil,\varphi)$ be a logarithmic Fontaine-Faltings module over $(\mY,\mZ)$ endowed with a $\bZ_{p^f}$-endomorphism structure
\[\tau\colon \bZ_{p^f} \rightarrow \End\big((V,\nabla,\Fil,\varphi)\big)\]
Assume $\bF_{p^f}\subseteq k$. Then
\begin{itemize}
\item[$(1)$] the connection and filtration can be restricted on $V_i$ for all $i\geq0$, and there is a decomposition of filtered logarithmic de Rham module
\[(V,\nabla,\Fil) = (V_0,\nabla_0,\Fil_0) \oplus (V_1,\nabla_1,\Fil_1) \oplus \cdots \oplus (V_{f-1},\nabla_{f-1},\Fil_{f-1}).\]
\item[$(2)$] The restriction $\varphi_i$ of $\varphi$ on $\widehat{R}\otimes_{\Phi}\widetilde{V}_i$ gives an isomorphism of de Rham modules
\[\varphi_i\colon \widehat{R} \otimes_{\Phi} \widetilde{V}_i \overset{\simeq}{\longrightarrow} V_{i+1}.\]
\end{itemize}
\end{lemma}

\begin{proof} (1) Since $\nabla$ is $\bZ_{p^f}$-linear and $\Fil$ consists of sub-$\bZ_{p^f}$-modules, they both can be restricted on $V_i$. We only need to show
\[V=V_0\oplus V_1 \oplus \cdots \oplus V_{f-1}.\]
Take an generator $\zeta\in \bZ_{p^f}$ over $\bZ_p$ (in other words, $\bZ_{p^f}=\bZ_{p}[\zeta]$) and denote
\[e_i:= \frac{\prod\limits_{j=0\atop j\neq i}^{f-1}\Big(\tau(\zeta)-\zeta^{\sigma^j}\Big)}{\prod\limits_{j=0\atop j\neq i}^{f-1}\Big(\zeta^{\sigma^i}-\zeta^{\sigma^j}\Big)}\in \End_{\widehat{R}}(V).\]
It is easy to check that $\id=e_0+e_1+\cdots+e_{f-1}$ is an idempotent decomposition, and that
\[\tau(a) e_i=a^{\sigma^i} e_i,\qquad \text{for any} a\in \bZ_{p^f}.\]
In particular, it induces a decomposition
\[V = e_0V \oplus e_1V \oplus \cdots \oplus e_{f-1}V.\]
and $V_i = e_iV$ for all $i=0,\cdots,f-1$.

(2). Since the endomorphism structure preserves the Frobenius structure $\varphi$, we have $\tau(a)\circ\varphi = \varphi\circ(\id\otimes\tau(a))$ for any $a\in\bZ_{p^f}$. Thus, for any $v_i\in \Fil^\ell V_i$ and any $a\in\bZ_{p^f}$, we have
\begin{equation*}
\begin{split}
\tau(a)\big(\varphi(1\otimes_\Phi [v_i])\big)= \varphi\circ(\id\otimes_\Phi \tau(a)) (1\otimes[v_i]) = \varphi(1\otimes_\Phi a^{\sigma^i}[v_i])= a^{\sigma^{i+1}} \cdot \varphi(1\otimes[v_i]).
\end{split}
\end{equation*}
where $[v_i]$ is the image of $v_i$ under the natural morphism $\Fil^\ell V_i \rightarrow \widetilde{V}_i$.
In other words, $\varphi(1\otimes_{\Phi}[v_i])\in V_{i+1}$. Then (2) follows.
\end{proof}

\subsubsection{Constant Fontaine-Faltings modules}

\begin{definition}
an \emph{admissible filtered $\varphi$-module of rank $r$ over $W$} is a triple $(V,\Fil,\varphi)$, where
\begin{itemize}
\item $V$ is a finite generated free $W$-module of rank $r$,
\item $\Fil$ is a filtration of direct summands of $V$ of form (for some $a,b\in\bZ$)
\[V = \Fil^aV \supseteq \Fil^{a+1}V \supseteq \cdots \supseteq \Fil^{b}V\supseteq \Fil^{b+1}V=0\]
\item $\varphi\colon \widetilde{V} \rightarrow V$ is a $\sigma$-semilinear isomorphism (sometimes, we also use $\varphi$ to stand for the induced $W$-linear isomorphism $W\otimes_{\sigma} \widetilde{V}\rightarrow V$) where
\[\widetilde{V}=\sum\limits_{\ell=a}^b\frac1{p^\ell} \Fil^\ell V \subset V_K:=V\otimes_WK.\]
\end{itemize}
The jumping indices of the filtration are called the \emph{levels} of $(V,\Fil,\varphi)$.
Denote by $\MF^{\varphi}(W)$ the category of all admissible filtered $\varphi$-module over $W$. Denote by $\MF_{[a,b]}^{\varphi}(W)$ the full subcategory of all admissible filtered $\varphi$-module over $W$ with levels contained in $[a,b]$. An object in $\MF_{[0,p-2]}^{\varphi}(W)$ is called a \emph{constant Fontaine-Faltings module}.
\end{definition}

\begin{remark}
\begin{enumerate}
\item[$(1)$] The third term is equivalent to give a $\sigma$-semilinear isomorphism $\varphi_K\colon V_K\rightarrow V_K$ such that it is admissible (or, compatible with the filtration) in the sense that its restriction induces isomorphism from $\widetilde{V}$ to $V$. The admissible condition here is also called strong $p$-divisibility condition in some other literature.
\item[$(2)$] The pair $(V_K,\varphi_K)$ forms an $F$-isocrystal over $k$. One has natural functor
\[\MF^\varphi(W) \rightarrow \FIsoc(k).\]
\item[$(3)$] Let $\Fil_K$ be the induced filtration on $V_K$ from $\Fil$. Then the triple $(V_K,\Fil_K,\varphi_K)$ is an admissible filtered $\varphi$-module over $K$ in the sense \cite{FoOu}.
\item[$(4)$] If the index $a$ appeared in the filtration is zero. Then $V$ is contained in $\widetilde{V}$. The restriction of $\varphi$ gives a $\sigma$-semilinear injection $\varphi_V\colon V\rightarrow V$. In other words, the pair $(V,\varphi_V)$ forms an $F$-crystal over $k$.
\end{enumerate}
\end{remark}

\begin{example} For each $w \in K^\times$, it can be write uniquely as $w=p^ru$, for some $r\in\bZ$ and $u\in W^\times$. We construct an admissible filtered $\varphi$-module $M_w=(V,\Fil,\varphi)$ of rank $1$ over $W$ as following:
Let $V$ be a free $W$-module of rank $1$ with a generator $v$. The filtration $\Fil$ is given by
\[V = \Fil^rV\supset \Fil^{r+1}V=0,\]
and the Frobenius structure is given by $\varphi(1\otimes \frac{v}{p^r})=uv$. Actually, all admissible filtered $\varphi$-modules of rank $1$ over $W$ are of this form. For any two $w,w'\in K^\times$, $M_w\simeq M_{w'}$ if and only if there exists some $\xi\in W^\times$ such that $w/w'=\sigma(\xi)/\xi$. In particular, the $p$-adic values of $w$ and $w'$ are of the same.
\end{example}

\begin{definition}
An \emph{admissible filtered $\varphi$-module with $\bZ_{p^f}$-endomorphism structure of rank $r$ over $W$} is a tuple $(V,\Fil,\varphi,\tau)$, where $(V,\Fil,\varphi)\in \MF^\varphi(W)$ is of rank $rf$ and $\tau$ is a ring homomorphism ( which is called a $\bZ_{p^f}$-endomorphism structure on $(V,\Fil,\varphi)$)
\[\tau\colon \bZ_{p^f} \rightarrow \End((V,\Fil,\varphi)).\]
Denote by \emph{$\MF^\varphi(W)_{\bZ_{p^f}}$} the category of all admissible filtered $\varphi$-modules with $\bZ_{p^f}$-endomorphism structures over $W$. Similarly we define the full subcategory \emph{$\MF_{[a,b]}^\varphi(W)_{\bZ_{p^f}}$}. The objects in
$\MF_{[0,p-2]}^\varphi(W)_{\bZ_{p^f}}$ are called \emph{constant Fontaine-Faltings modules with $\bZ_{p^f}$-endomorphism structures}.
\end{definition}

\begin{definition} The triple $(V_K,\varphi_K,\tau)$ forms an $F$-isocrystal over $k$ with coefficient $\bQ_{p^f}$.
Denote
\[P((V,\Fil,\varphi,\tau),t):=P((V_K,\varphi_K,\tau),t),\quad and \quad \tr(V,\Fil,\varphi,\tau) := \tr(V_K,\varphi_K,\tau).\]
Call them the \emph{characteristic polynomial and trace of $(V,\Fil,\varphi,\tau)$}.
\end{definition}

\begin{remark} Let $(V,\Fil,\varphi,\tau) \in \MF^\varphi(W)_{\bZ_{p^f}}$ be of rank $r$. Then on $V$ there are two $\bZ_{p^f}$-moudle structures, the first is the nature one and the second is given by $\tau$. So we may view $V$ as a $W\otimes_{\bZ_p}\bZ_{p^f}$-module. By the existence of the Frobenius structure, one actually shows that $V$ is locally free over $W\otimes_{\bZ_p}\bZ_{p^f}$ of rank $r$. Since the filtration $\Fil$ is direct summands as $W$-modules and is preserved by the action of $\bZ_{p^f}$ via $\tau$, it consists of direct summands as $W\otimes_{\bZ_p}\bZ_{p^f}$-modules.
\end{remark}

\begin{example} Suppose $\bF_{p^f}\subset k$. For any $w\in K\otimes\bQ_{p^f}$ it can be uniquely written as
\[w = \sum_{i=0}^{f-1} (p^{r_i}u_i\otimes 1)\cdot e_i,\]
where $r_i\in \bZ$ and $u_i \in W^\times$ for all $i=0,1,\cdots,f-1$. We construct a rank $1$ object $M_{w}=(V,\Fil,\varphi,\tau) \in \MF^\varphi(W)_{\bZ_{p^f}}$ as follows: The $V$ is free of rank $1$ over $W\otimes \bZ_{p^f}$ with a basis $v$, denote $v_i=e_iv$ and $V_i=Wv_i$. The action $\tau$ of $\bZ_{p^f}$ given by the second factor. In other words, $\tau(a)(v_i) = (1\otimes a)\cdot v$. On $V$, we set the Filtration $\Fil$ via
\[(V,\Fil) = (V_0,\Fil_0) \oplus (V_1,\Fil_1) \oplus \cdots \oplus (V_{f-1},\Fil_{f-1}),\]
where the filtration $\Fil_i$ is defined by ($r_f:=r_0$)
\[V_i = \Fil_i^{r_{i+1}}V_i\supset \Fil_i^{r_{i+1}+1}V_i=0 \quad
\text{for all} i=0,1,\cdots,f-1.\]
The Frobenius structure is given by ($u_f:=u_0$ and $v_f:=v_0$)
\[\varphi (\frac{v_i}{p^{r_{i+1}}}) = u_{i+1}v_{i+1} \quad
\text{for all} i=0,1,\cdots,f-1.\]
In this case, the associated $F$-isocrystal over $k$ with coefficient in $\bQ_{p^f}$ can be simply represented as $(K\otimes_{\bQ_p}\bQ_{p^f}\cdot v,F)$ with Frobenius given by
\[F(v) = w\cdot v.\]
\end{example}

\begin{definition} The constant Fontaine-Faltings module corresponding to $w=p$ is called \emph{cyclotomic Fontaine-Faltings module}.
\end{definition}

By direct computation, one easily checks the following result.
\begin{lemma} \label{thm_FilPhiModRank1Tensor}
Suppose $\bF_{p^f}\subset k$. Then all objects of rank $1$ in $\MF^\varphi(W)_{\bZ_{p^f}}$ are of the form $M_w$ for some $w\in (K\otimes \bQ_{p^f})^\times$. And one has
\[M_{w} \otimes M_{w'} \cong M_{w\cdot w'}.\]
\end{lemma}


\begin{corollary} \label{thm_sqroot_constant_FFM_level0}
Let $k$ be a finite field and denote by $k'$ the quadratic field extension of $k$. Let $\mE\in \MF^{\varphi}_{[0,0]}(W(k))_{\bZ_{p^f}}$ be of rank $1$. Then there exists a rank $1$ object in $\mE'=\MF_{[0,0]}(W(k'))_{\bZ_{p^f}}$ such that
\[\mE = \mE'\otimes \mE' \in \MF_{[0,0]}(W(k'))_{\bZ_{p^f}}.\]
\end{corollary}
\begin{proof}
Let $\mE$ corresponds to $w=\sum_{i=0}^{f-1} (u_i\otimes1)\cdot e_i$, where $u_i\in W(k)^\times$. Since $k'$ is the quadratic extension of $k$ and $u_i$ is a unit in $W$, one has $\sqrt{u_i}\in W(k')$. Denote by $\mE'$ the object corresponding to $w':=\sum_{i=0}^{f-1} (\sqrt{u_i}\otimes 1)\cdot e_i$. Then the corollary follows \autoref{thm_FilPhiModRank1Tensor}.
\end{proof}

\begin{definition}
The $M_w$ is called of finite order, if there exists some $d$ such that $\underbrace{M_w \otimes M_w \otimes \cdots \otimes M_w}_{d} \cong M_1$. The smallest integer $d$ satisfying this condition is called the order of $M_w$.
\end{definition}

\begin{corollary} Let $\mE\in \MF^{\varphi}_{[0,0]}(W(k))_{\bZ_{p^f}}$ be of rank $1$. Then there exists a rank $1$ object in $\mE'=\MF_{[0,0]}(W(k))_{\bZ_{p^f}}$ such that
\[\mE\otimes \mE'\otimes \mE'\]
is of finite order. The order divides $\#k-1$.
\end{corollary}


\subsubsection{Constant Fontaine-Faltings modules over a base}
Let $Y$ be a smooth scheme over $W$ with geometrically connected generic fiber. The pullback operator along the structure morphism $\mY\rightarrow \Spf(W)$ induces a natural functor
\[\MF_{[0,p-2]}^{\varphi}(W) \rightarrow \MF_{[0,p-2]}(\mY/W).\]
We also call the essential image of this pullback functor \emph{constant Fontaine-Faltings modules}. Similarly for those with endomorphism structures.

\begin{lemma}
If $Y$ is projective over $W$, then a Fontaine-Faltings module is constant if and only if its underlying de Rham bundle is direct sum of copies of the trivial de Rham line bundle $(\mO_{\mY},\rmd)$.
\end{lemma}


\subsubsection{Restriction of Fontaine-Faltings modules on points.}

Let $Y$ be a smooth scheme over $W$ with geometrically connected generic fiber. Let $x$ be a closed point in $Y_k$. Denote by $k_x$ the residue field at $x$, which is a finite extension of $k$. By the smoothness of $Y$, there exists a $W(k_x)$-point $\widehat{x}$ on $Y$, which lifts $x$. By restriction on $\widehat{x}$, one gets a functor
\[\MF_{[0,p-2]}(\mY/W) \rightarrow \MF_{[0,p-2]}^{\varphi}(W(k_x)).\]
In the following, we describe this functor clearly and show that it does not depend on the choice of the lifting $\widehat{x}$.

Let $(V,\nabla,\Fil,\varphi) \in \MF_{[0,p-2]}(\mY/W)$. By shrinking $Y$, we may assume $Y = \Spec(R)$ is small affine and contains $\widehat{x}$ such that there exists a Frobenius lifting $\Phi\colon \widehat{R}\rightarrow \widehat{R}$ preserves the $W$-point $\widehat{x}$. In other words, we have commutative diagram
\begin{equation*}
\xymatrix{
\widehat{R} \ar[r]^\Phi \ar@{->>}[d]_{\widehat{x}} & \widehat{R} \ar@{->>}[d]^{\widehat{x}} \\
W \ar[r]^{\sigma} & W \\
}
\end{equation*}
In this case, $V$ is a finite generated free $R$-module and $\Fil$ is a filtration of direct summands of $V$. Denote
\[V_{\widehat{x}}:= V \otimes_{\widehat{x}} W \quad\text{and}\quad \Fil_{\widehat{x}}^\ell V_{\widehat{x}}:= \Fil^\ell V \otimes_{\widehat{x}} W,\text{for all} \ell.\]
Clearly, one has $\widetilde{V_{\widehat{x}}} = \widetilde{V} \otimes_{\widehat{x}} W$. Taking $\otimes_{\widehat{x}} W$ on the Frobenius structure $\varphi\colon \widehat{R}\otimes_\Phi \widetilde{V} \rightarrow \widetilde{V}$, one gets isomorphism
\[\varphi_{\widehat{x}} \colon W\otimes_\sigma \widetilde{V_{\widehat{x}}} \rightarrow V_{\widehat{x}}.\]
Thus one gets a constant Fontaine-Faltings module over $W(k_x)$
\[\Big(V_{\widehat{x}},\Fil_{\widehat{x}},\varphi_{\widehat{x}}\Big) \in \MF_{[0,p-2]}^{\varphi}(W(k_x)).\]

\begin{remark}
Up to a canonical isomorphism, this constant Fontaine-Faltings module does not depend on the choice of $\Phi$. Because up to a canonical equivalent the category of Fontaine-Faltings module do not. The deep-seated reason is that a Fontaine-Faltings module corresponds to an $F$-crystal after forgetting the Hodge filtration.
\end{remark}

\begin{lemma} Let $\widehat{x}$ and $\widehat{x}'$ be two lifting of $x$. Then there exists a canonical isomorphism
\[\Big(V_{\widehat{x}},\Fil_{\widehat{x}},\varphi_{\widehat{x}}\Big) \simeq \Big(V_{\widehat{x}'},\Fil_{\widehat{x}'},\varphi_{\widehat{x}'}\Big).\]
In other words, the isomorphic class of $\Big(V_{\widehat{x}},\Fil_{\widehat{x}},\varphi_{\widehat{x}}\Big)$ does not depend on the choice of the lifting $\widehat{x}$.
\end{lemma}

\begin{proof}
By the smoothness of $Y$, after shrinking $Y$, we may find an automorphism of $\mY$
\[\widehat{\id} \colon \mY \rightarrow \mY\]
which lifts the identity map on $Y_k$ such that it sends $\widehat{x}$ to $\widehat{x}'$. By using Taylor formula, one gets an canonical equivalent functor
\[\widehat{\id}^*\colon \MF_{[0,p-2]}(\mY/W) \rightarrow \MF_{[0,p-2]}(\mY/W).\]
Choose a Frobenius lifting $\Phi$ that preserves $\widehat{x}$. Denote $\Phi':= \widehat{\id}^{-1} \circ \Phi \circ \widehat{\id}$, which is a Frobenius lifting that preserves $\widehat{x}'$. Now one has the following commutative diagram of functors
\begin{equation*}
\xymatrix{
\MF^{\Phi}_{[0,p-2]}(\mY/W) \ar[r]^{\widehat{\id}^*} \ar[d] & \MF^{\Phi'}_{[0,p-2]}(\mY/W)\ar[d]
\\
\MF^{\varphi}_{[0,p-2]}(W(k_x)) \ar@{=}[r] & \MF^{\varphi}_{[0,p-2]}(W(k_x))
\\
}
\end{equation*}
\end{proof}

\newpage

\subsection{Introduction to Higgs-de Rham flows}

Lan-Sheng-Zuo have introduced the notion of (periodic) Higgs-de Rham flow over a smooth log scheme $(X,D)$ over $W(k)$ and have shown that the category of $f$-periodic Higgs bundles is equivalent to the category of log Fontaine-Faltings modules over $(X,D)$ endowed with an $\bZ_{p^f}$-endomorphism structure.

Lately, Sun-Yang-Zuo introduced projective Higgs-de Rham flow, which generalized the notion of Higgs-de Rham flow. Actually, it is just the parabolic Higgs-de Rham flow with only one parabolic weight along each component of the boundary divisor.

\subsubsection{Functor $\overline{\Gr}$}
Denote by $\mathcal{H}_a(Y,D_Y)$ the category of tuples $(E,\theta,V,\nabla,\Fil,\psi)$, where
\begin{itemize}
\item[-] $(E,\theta)$ is a graded logarithmic Higgs bundle
\footnote{A Higgs bundle $(E,\theta)$ is called graded if $E$ can be written as direct sum of subbundles $E^{i}$ with $\theta(E^i)\subset E^{i-1}\otimes\Omega^1$. Obviously, a graded Higgs bundle is also nilpotent.}
over $Y$;
\item[-] $(V,\nabla,\Fil)_a\in \MCF(Y,D_Y)$;
\item[-] and $\psi: \Gr(V,\nabla,\Fil) \cong (E,\theta)$ is an isomorphism of logarithmic Higgs bundles over $(Y,D_Y)$.
\end{itemize}
There is a natural functor from $\MCF_a(Y,D_Y)$ to $\mathcal{H}_a(Y,D_Y)$ given as follows for any object $(V,\nabla,\Fil)\in \MCF(Y,D_Y)$
\[\overline{\Gr}(V,\nabla,\Fil):=(E,\theta,V,\nabla,\Fil,\psi),\]
where $(E,\theta):=\Gr(V,\nabla,\Fil)$ is the graded bundle with the graded Higgs field induced by the connection, and $\overline{\psi}$ is the identifying map $\Gr(V,\nabla,\Fil)\cong (E,\theta)$.
\begin{remark} There are also truncated version $\overline{\Gr}_n$ of this functor defined in \cite{LSZ19,SYZ22}. For our application, we only use the version defined here, which is the limit functor of $\overline{\Gr}_n$. We note that the functor $\overline{\Gr}$ is actual an equivalence, but the $\overline{\Gr}_n$ is not.
\end{remark}

\subsubsection{The functor $\mathcal{T}$ and inverse Cartier functor $\mathcal{C}^{-1}$}
One defines a natural functor from the category $\mathcal{H}(Y,D_Y)$ to the category $\widetilde{\MIC}(Y,D_Y)$, for any object $(E,\theta,V,\nabla,\Fil,\psi) \in \mathcal{H}(Y,D_Y)$
\[\mathcal{T}(E,\theta,V,\nabla,\Fil,\psi):= \widetilde{(V,\nabla,\Fil)}.\]
The inverse Cartier functor is defined to be the composition
\[\mC^{-1} \colon \mathcal{H}_a(Y,D_Y) \xrightarrow{\mT} \widetilde{\MIC}_a(Y,D_Y) \xrightarrow{\mF} \MIC(Y,D_Y)\]
for $a\leq p-2$.
\begin{remark}
\begin{enumerate}
\item Similar there is also truncated version $\mathcal{T}_n$ in \cite{LSZ19,SYZ22}, whose definition is much more complicated and its limit is just $\mathcal{T}$.
\item Faltings tilde functor $\widetilde{(\cdot)}$ is just the composition of $\mT$ and $\overline{\Gr}$. For $0\leq a\leq p-2$, one has the following commutative diagram
\begin{equation}
\xymatrix@R=2cm@C=2cm{
& \mathcal{H}_a(Y,D_Y)\ar[dr]^-{\mC^{-1}} \ar@/^50pt/[dd]^{\mT} \ar[d]|{\text{forget}}&\\
\MCF_a(Y,D_Y)\ar[dr]_{\widetilde{(\cdot)}}\ar[ur]^{\overline{\Gr}} \ar[r]|{\text{grading}}
& \mathrm{HIG}^{\rm gr}(Y,D_Y) & \MIC(Y,D_Y)\\
&\widetilde{\MIC}_a(Y,D_Y) \ar[ur]_-{\mF} \ar[u]|{\text{grading}}\\
}
\end{equation}
where $\mathrm{HIG}^{\rm gr}(Y,D_Y)$ is the category of graded logarithmic Higgs bundle over $(Y,D_Y)$, and the ``forget'' and ``grading'' stand for the nature functors.
\end{enumerate}
\end{remark}

\subsubsection{Higgs-de Rham flow}
Recall~\cite{LSZ19} that a \emph{Higgs-de Rham flow} over $(Y,D_Y)$ is a sequence consisting of infinitely many alternating terms of filtered logarithmic de Rham bundles and logarithmic Higgs bundles
\[\left\{
(V,\nabla,\Fil)_{-1},
(E,\theta)_{0},
(V,\nabla,\Fil)_{0},
(E,\theta)_{1},
(V,\nabla,\Fil)_{1},
\cdots\right\},\]
which are related to each other by the following diagram inductively
\begin{equation} \label{diag:HDF}
\xymatrix@W=10mm@C=3mm@R=5mm{
(V,\nabla,\Fil)_{-1} \ar[dr]^{\Gr} && (V,\nabla,\Fil)_{0}\ar[dr]^{\Gr}
&& (V,\nabla,\Fil)_{1}\ar[dr]^{\Gr}
&\\
& (E,\theta)_{0} \ar[ur]^{\mC^{-1}}
&& (E,\theta)_{1} \ar[ur]^{\mC^{-1}}
&& \cdots\\
}
\end{equation}
where
\begin{itemize}
\item[-] $(V,\nabla,\Fil)_{-1}$ is a filtered de Rham bundle over $(Y,D_Y)$ of level in $[0,p-2]$;
\item[-] Inductively, for $i\geq1$, $(E,\theta)_i$ is the graded Higgs bundle $\Gr\left((V,\nabla,\Fil)_{i-1}\right)$, \[(V,\nabla)_i:=\mC^{-1} ((E,\theta)_i,(V,\nabla,\Fil)_{i-1} ,\mathrm{id})\]
and $\Fil_i$ is a Hodge filtration on $(V,\nabla)_i$ of level in $[0,p-2]$.
\end{itemize}

\begin{remark}
\begin{enumerate}
\item In \cite{LSZ19}, Higgs-de Rham flow is defined over any truncated level, whose definition is much more complicated.
\item The essential data given in a Higgs-de Rham flow are just
\[V_{-1},\nabla_{-1},\Fil_{-1},\Fil_{0},\Fil_{1},\Fil_{2},\cdots\]
since the other terms can be constructed from these, e.g. $E_0$, $\theta_0$, $V_1$, $\nabla_1$, $\cdots$.
\end{enumerate}
\end{remark}

The Higgs-de Rham flow is called $f$-periodic if there exists an isomorphism
\[\Phi\colon (E_{f},\theta_{f},V_{f-1},\nabla_{f-1},\Fil_{f-1},\mathrm{id}) \cong (E_0,\theta_0,V_{-1},\nabla_{-1},\Fil_{-1},\mathrm{id})\]
such that its induces isomorphisms, for all $i\geq0$,
\[(E,\theta)_{f+i}\cong (E,\theta)_{i} \quad\text{and}\quad (V,\nabla,\Fil)_{f+i}\cong (V,\nabla,\Fil)_{i}.\]
We simply represent this periodic Higgs-de Rham flow with the following diagram
\begin{equation} \label{diag:HDF}
\xymatrix@W=10mm@C=3mm@R=10mm{
& (V,\nabla,\Fil)_{0} \ar[dr]^{\Gr}
&& (V,\nabla,\Fil)_{1} \ar[dr]^{\Gr}
&& (V,\nabla,\Fil)_{f-1} \ar[dr]^{\Gr}
&\\
(E,\theta)_{0} \ar[ur]^{\mC^{-1}}
&& (E,\theta)_{1} \ar[ur]^{\mC^{-1}}
&& \cdots \ar[ur]^{\mC^{-1}}
&& (E,\theta)_{f} \ar@/^30pt/[llllll]^{\Phi}\\}
\end{equation}
\begin{theorem}[Lan-Shen-Zuo\cite{LSZ19}]\label{equiv:FF&HDF}
There exists an equivalence between the category of logarithmic Fontaine-Faltings modules over $(Y,D_Y)$ and the category of $1$-periodic logarithmic Higgs-de Rham flows over $(Y,D_Y)$.
\end{theorem}

\newpage

\subsection{Parabolic Fontaine-Faltings modules and Parabolic Higgs-de Rham flows}

\subsubsection{Parabolic versions of the some categories and functors}

In order to define parabolic version Higgs-de Rham flow, we need the parabolic versions of the categories and functors appeared in a diagram from \cite[Section 1.2.1]{SYZ22}. The most crucial part is to define the inverse Cartier functor (or, equivalently, the Frobenius pullback functor).

\begin{equation}\label{diag:C^{-1}}
\xymatrix{
& \mH(X_n)\ar[drr]^-{C_n^{- ,1}} \ar[dd]^{\mT_n}
&&\\
\MCF_{p-2}(X_n)\ar[dr]_{\widetilde{(\cdot)}}\ar[ur]^{\overline{\Gr}}
&&& \MIC(X_n)\\
&\widetilde{\MIC}(X_n) \ar[urr]_-{F_n=\{F_\mU^*\}_\mU}&&\\
}
\end{equation}

In this section, we set $S=\Spec(W(k))$ and $S_n = \Spec(W_n(k))$. Denote by
\[(Y_n,D_{Y_n})\coloneqq (Y,D_Y)\times_SS_n.\]

\begin{definition}
Denote by $\MIC((Y_n,D_{Y_n})/S_n)$ the category of all parabolic de Rham bundles over $(Y_n,D_{Y_n})/S_n$.
\end{definition}

\begin{definition}
A \emph{filtered parabolic de Rham bundle over $(Y_n,D_{Y_n})/S_n$} is a triple $(V,\nabla,\Fil)$, which consists of a parabolic de Rham bundle $(V,\nabla)$ over $(Y_n,D_{Y_n})/S_n$ and a filtration $\Fil$ of parabolic sub bundles such that $(V_\alpha,\nabla_\alpha,\Fil_\alpha)$ forms a usual filtered de Rham bundle over $(Y_n,D_{Y_n})/S_n$ for each $\alpha\in \bQ^n$. Denote by \emph{$\MCF_{p-2}((Y_n,D_{Y_n})/S_n)$} the category of all filtered parabolic de Rham bundles over $(Y_n,D_{Y_n})/S_n$ with the levels are contained in $[0,a]$.
\end{definition}

\begin{definition}
Denote by \emph{$\mH((Y_n,D_{Y_n})/S_n)$} the category of tuples $(E,\theta,\overline{V},\overline{\nabla},\overline{\Fil},\overline{\psi})$, consisting of
\begin{itemize}
\item a graded parabolic Higgs bundle $(E,\theta)$ over $(Y_n,D_{Y_n})/S_n$ with exponent $\leq p-2$,
\item a filtered parabolic de Rham bundle over $(Y_n,D_{Y_{n-1}})/S_{n-1}$, and
\item an isomorphism of graded parabolic Higgs bundles over $(Y_n,D_{Y_{n-1}})/S_{n-1}$
\[\overline{\psi}\colon \Gr(\overline{V},\overline{\nabla},\overline{\Fil}) \cong (E,\theta)\otimes_{W_n(k)} W_{n-1}(k).\]
\end{itemize}
\end{definition}

\begin{definition}
An \emph{parabolic integrable $p$-connection $\nabla=\{\nabla_\alpha\}$} on a parabolic vector bundle $V$ over $(Y_n,D_{Y_n})/S_n$ is a collection of compatible integrable $p$-connections $\nabla_\alpha$ on the $V_\alpha$. I.e. for any $\alpha\leq\beta$, the following diagram commutes
\begin{equation*}
\xymatrix{
V_\alpha \ar[r]^-{\nabla_\alpha} \ar[d] & V_\alpha \otimes \Omega_{Y_n/S_n}(\log D_{Y_n}) \ar[d] \\
V_\beta \ar[r]^-{\nabla_\beta} & V_\beta \otimes \Omega_{Y_n/S_n}(\log D_{Y_n})\\
}
\end{equation*}
Let \emph{$\widetilde{\MIC}((Y_n,D_{Y_n})/S_n)$} denote the category of pairs $(V,\nabla)$, consisting of a parabolic vector bundle $V$ and a parabolic integrable $p$-connection $\nabla$ on $V$.
\end{definition}

\paragraph{\emph{Functor $\overline{\Gr}$.}} For an object $(V,\nabla,\Fil)$ in $\MCF_{p-2}((Y_n,D_{Y_n})/S_n)$, the functor $\overline{\Gr}$ is given by
\[\overline{\Gr}(V,\nabla,\Fil)=(E,\theta,\overline{V},\overline{\nabla},\overline{\Fil},\overline{\psi}),\]
where $(E,\theta)=\Gr(V,\nabla,\Fil)$ is the graded parabolic Higgs bundle, $(\overline{V},\overline{\nabla},\overline{\Fil})$ is the modulo $p^{n-1}$-reduction of $(V,\nabla,\Fil)$ and $\overline{\psi}$ is the identifying map \[\id\colon \Gr(\overline{V},\overline{\nabla},\overline{\Fil})= (E,\theta)\otimes_{W_n(k)}W_{n-1}(k).\]

\paragraph{\emph{Faltings tilde functor .}} For an object $(V,\nabla,\Fil)$ in $\MCF_{p-2}(X_n)$, the $\widetilde{(V,\nabla,\Fil)}$ will be denoted as the quotient $\bigoplus\limits_{i=0}^{p-2}\Fil^i/\sim$ with $x\sim py$ for any $x\in \Fil^iV$ and $y$ the image of $x$ under the natural inclusion $\Fil^iV\hookrightarrow \Fil^{i-1}V$.

\paragraph{\emph{The construction of functor $\mT_n$.}} Let $(E,\theta,\bar{V},\bar{\nabla},\overline{\Fil},\psi)$ be an object in $\mH((Y_n,D_{Y_n})/S_n)$. For any $\alpha\in\bQ^n$, denote
\[(\widetilde{V}_\alpha,\widetilde{\nabla}_\alpha)\coloneqq\mT_n(E_\alpha,\theta_\alpha,\bar{V}_\alpha,\bar{\nabla}_\alpha,\overline{\Fil}_\alpha,\psi_\alpha),\]
where the functor on the right hand side, still denote by $\mT_n$, is the usual functor defined as in \cite{SYZ22}. Then the collection $(\{\widetilde{V}_\alpha\},\{\widetilde{\nabla}_\alpha\})\}$ forms a pair of parabolic vector bundle and a parabolic $p$-connections. We denote it by
\[\mT_n(E,\theta,\bar{V},\bar{\nabla},\overline{\Fil},\psi)\coloneqq\{((\widetilde{V}_\alpha,\widetilde{\nabla}_\alpha))\}\]

\paragraph{\emph{The Frobenius pullback functor $\Phi^*_n$.}}

Base on the usual Frobenius pullback functor $\Phi^*_n$ for usual bundles with $p$-connections, we define the Frobenius pullback for parabolic vector bundles with parabolic $p$-connections, which will be still denoted $\Phi^*_n$ by abusing notions.

For any $\gamma\in \bQ^n$, we has a canonical parabolic $p$-connection $p\cdot \rmd(\gamma D)$ on the parabolic vector bundle $\mO_Y(\gamma D)$. We simply set
\[\Phi^*_n\Big( \mO_Y(\gamma D),p\cdot \rmd(\gamma D)\Big) \coloneqq \Big( \mO_Y(p\gamma D),\rmd(p\gamma D)\Big).\]

Similarly as the pullback of parabolic de Rham bundle, we can define the Frobenius pullback functor as follows.
\begin{definition}
For any $(V,\nabla)$ in $\widetilde{\MIC}((Y_n,D_{Y_n})/S_n)$, we first denote
\[\Phi^*_{n,\gamma}(V,\nabla)
\coloneqq\Phi_n^*\Big(\big((V,\nabla) \otimes (\mO_Y(\gamma D),p\cdot \rmd(\gamma D))^{-1}\big)_0\Big)\otimes \Phi_n^*\big(\mO_Y(\gamma D),p\cdot \rmd(\gamma D)\big).\]
And then set
\[\Phi^*_n(V,\nabla)\coloneqq \bigcup_{\gamma\in\bQ^n} \Phi^*_{n,\gamma}(V,\nabla).\]
Denote by $C_n^{-1} \coloneqq \Phi_n^*\circ \mT_n$ the composition functor, call it \emph{the inverse Cartier functor}.
\end{definition}


\subsubsection{Parabolic Fontaine-Faltings modules with endomorphism structure}
Recall \cite[Lemma 1.1]{SYZ22} or \cite[Lemma 5.6]{LSZ19}], we extend the definition to parabolic version.
\begin{definition} A \emph{parabolic Fontaine-Faltings module over $(Y_n,D_{Y_n})/S_n$} is a tuple $(V,\nabla,\Fil,\varphi)$, where
\begin{itemize}
\item[-] $(V,\nabla,\Fil)\in \MCF_{p-2}((Y_n,D_{Y_n})/S_n)$ is a filtered parabolic de Rham bundle over $(Y_n,D_{Y_n})/S_n$ of level in $[0,p-2]$;
\item[-] $\varphi\colon C_n^{-1}\circ \overline{\Gr} (V,\nabla,\Fil)\rightarrow (V,\nabla)$ is an isomorphism of parabolic de Rham bundles.
\end{itemize}
We denote by \emph{$\MF_{[0,p-2]}^\nabla((Y_n,D_{Y_n})/S_n)$} the category of all parabolic Fontaine-Faltings modules over $(Y_n,D_{Y_n})/S_n$.
\end{definition}

\begin{definition}
Let $M\in \MF_{[0,p-2]}^\nabla((Y_n,D_{Y_n})/S_n)$. A ring homomorphism
\[\iota\colon \mathbb Z_{p^f}\rightarrow \End(M)\]
is called a \emph{$\Zp$-endomorphism structure}. Denote by \emph{$\MF_{[0,p-2]}^\nabla((Y_n,D_{Y_n})/S_n)_{\bZ_{p^f}}$} the category of all parabolic Fontaine-Faltings modules with $\Zp$-endomorphism structures over $(Y_n,D_{Y_n})/S_n$.
\end{definition}

Let $(V,\nabla,\Fil,\varphi)$ be a parabolic Fontaine-Faltings module with level in $[0,a]$ and $(V,\nabla,\Fil,\varphi)'$ be a parabolic Fontaine-Faltings module with level in $[0,b]$. Then on the tensor parabolic de Rham bundle
\[(V,\nabla)\otimes (V,\nabla)'\]
Set
\[\Fil_{tot}^n(V\otimes V'):= \sum_{i+j=n}\Fil^iV\otimes \Fil^jV\quad \text{for all $n\in\bZ$},\]
it forms a Hodge filtration on the parabolic de Rham $(V,\nabla)\otimes (V,\nabla)'$. Denote
\[(V,\nabla,\Fil)\otimes (V,\nabla,\Fil)':=(V\otimes V',\nabla\otimes\id +\id\otimes\nabla',\Fil_{tot}).\]
By the definition of parabolic version of inverse Cartier, it naturally preserves tensor products, so we can set a Frobenius structure on $(V,\nabla,\Fil)\otimes (V,\nabla,\Fil)'$ given by $\varphi\otimes\varphi'$, and then get the tensor product of
\[(V,\nabla,\Fil,\varphi) \otimes (V,\nabla,\Fil,\varphi)'.\]

For the tensor product of Fontaine-Faltings modules with endomorphism structure, we set the underlying parabolic Fontaine-Faltings module of
\[(V,\nabla,\Fil,\varphi,\iota) \otimes (V,\nabla,\Fil,\varphi,\iota)'\]
to be
\[\Big((V,\nabla,\Fil,\varphi) \otimes (V,\nabla,\Fil,\varphi)'\Big)^{\iota=\iota'}.\]
on which the action of $\iota$ and $\iota'$ are coincide with each other, and we using this common action to define the $\bZ_{p^f}$-endomorphism structure.

We also define the symmetric product, wedge product and determinant for parabolic Fontaine-Faltings modules as usual way.

\subsubsection{Parabolic Higgs-de Rham flows}
\begin{definition} a parabolic \emph{Higgs-de Rham flow} over $(Y_n,D_{Y_n})\subset (Y_{n+1},D_{Y_{n+1}})$ is a sequence consisting of infinitely many alternating terms of filtered parabolic de Rham bundles and graded parabolic Higgs bundles
\[\left\{
(V,\nabla,\Fil)^{(n-1)}_{-1},
(E,\theta)_{0}^{(n)},
(V,\nabla,\Fil)_{0}^{(n)},
(E,\theta)_{1}^{(n)},
(V,\nabla,\Fil)_{1}^{(n)},
\cdots\right\},\]
which are related to each other by the following diagram inductively
\begin{equation*}\tiny
\xymatrix@W=10mm@C=-3mm@R=5mm{
&& (V,\nabla,\Fil)_{0}^{(n)} \ar[dr]^{\Gr}
&& (V,\nabla,\Fil)_{1}^{(n)} \ar[dr]^{\Gr}
&& (V,\nabla,\Fil)_{2}^{(n)} \ar[dr]^{\Gr}
&\\
& (E,\theta)_{0}^{(n)} \ar[ur]^{\mC^{-1}_n} \ar@{..>}[dd]
&& (E,\theta)_{1}^{(n)} \ar[ur]^{\mC^{-1}_n}
&& (E,\theta)_{2}^{(n)} \ar[ur]^{\mC^{-1}_n}
&& \cdots\\
(V,\nabla,\Fil)^{(n-1)}_{-1} \ar[dr]^{\Gr}
&&&&&&&\\
&\Gr\left((V,\nabla,\Fil)^{(n-1)}_{-1}\right)
&&&&\\
}
\end{equation*}
where
\begin{itemize}
\item[-] $(V,\nabla,\Fil)^{(n-1)}_{-1}\in \MCF_{p-2}((Y_n,D_{Y_n})/S_n)$;
\item[-] $(E,\theta)_0^{(n)}$ is a lifting of the graded parabolic Higgs bundle $\Gr\left((V,\nabla,\Fil)^{(n-1)}_{-1}\right)$ over $(Y_n,D_{Y_n})/S_n$, $(V,\nabla)_0^{(n)}\coloneqq C^{-1}_n ((E,\theta)_0^{(n)},(V,\nabla,\Fil)^{(n-1)}_{-1} ,\psi)$ and $\Fil^{(n)}_0$ is a parabolic Hodge filtration on $(V,\nabla)_0^{(n)}$ of level in $[0,p-2]$;
\item[-] Inductively, for $m\geq1$, $(E,\theta)_m^{(n)}\coloneqq\Gr\left((V,\nabla,\Fil)_{m-1}^{(n)}\right)$ and
\[(V,\nabla)_m^{(n)}\coloneqq C^{-1}_n \left(
(E,\theta)_{m}^{(n)},
(V,\nabla,\Fil)^{(n-1)}_{m-1},
\id
\right).\]
Here $(V,\nabla,\Fil)^{(n-1)}_{m-1}$ is the reduction of $(V,\nabla,\Fil)^{(n)}_{m-1}$ on $X_{n-1}$. And $\Fil^{(n)}_m$ is a Hodge filtration on $(V,\nabla)_m^{(n)}$.
\end{itemize}
\end{definition}

\begin{definition}
Let \[\Flow = \left\{
(V,\nabla,\Fil)^{(n-1)}_{-1},
(E,\theta)_{0}^{(n)},
(V,\nabla,\Fil)_{0}^{(n)},
(E,\theta)_{1}^{(n)},
(V,\nabla,\Fil)_{1}^{(n)},
\cdots\right\},\]
and
\[\Flow'=\left\{
(V',\nabla',\Fil')^{(n-1)}_{-1},
(E',\theta')_{0}^{(n)},
(V',\nabla',\Fil')_{0}^{(n)},
(E',\theta')_{1}^{(n)},
(V',\nabla',\Fil')_{1}^{(n)},
\cdots\right\},\]
be two Higgs-de Rham flows over $(Y_n,D_{Y_n})\subset (Y_{n+1},D_{Y_{n+1}})$. A morphism from $\Flow$ to $\Flow'$ is a compatible system of morphisms
\[\{\varphi_{-1}^{(n-1)},\psi_{0}^{(n)},\varphi_{0}^{(n)},\psi_{1}^{(n)},\varphi_{1}^{(n)},\cdots\}\]
between the terms respectively in the following sense that
\begin{itemize}
\item $\Gr(\varphi_{-1}^{(n-1)})=\psi_{0}^{(n)}\pmod{p^{n-1}}$
\item $C_n^{-1}(\psi_m^{(n)},\varphi_{m-1}^{(n-1)}) = \varphi_m^{(n)},$ (if $m\geq1$, then here $\varphi_m^{(n-1)}:=\varphi_m^{(n)}\pmod{p^{n-1}}$), and
\item $\Gr(\varphi_m^{(n)})=\varphi_{m+1}^{(n)}$.
\end{itemize}
\end{definition}

\begin{remark}
The morphism is uniquely determined by the first two terms $\varphi_{-1}^{(n-1)},\psi_{0}^{(n)}$. If $n=1$, then the first morphism is vacuous.
\end{remark}

\begin{definition}
Let
\[\Flow = \left\{
(V,\nabla,\Fil)^{(n-1)}_{-1},
(E,\theta)_{0}^{(n)},
(V,\nabla,\Fil)_{0}^{(n)},
(E,\theta)_{1}^{(n)},
(V,\nabla,\Fil)_{1}^{(n)},
\cdots\right\},\]
be a flow, we call the flow
\[\Flow[f] = \left\{
(V,\nabla,\Fil)^{(n-1)}_{f-1},
(E,\theta)_{f}^{(n)},
(V,\nabla,\Fil)_{f}^{(n)},
(E,\theta)_{f+1}^{(n)},
(V,\nabla,\Fil)_{f+1}^{(n)},
\cdots\right\}\]
the \emph{$f$-th shifting of $\Flow$}, where $(V,\nabla,\Fil)^{(n-1)}_{f-1}:=(V,\nabla,\Fil)^{(n)}_{f-1}\pmod{p^{n-1}}$.
\end{definition}

\begin{definition}
If there is an isomorphism
\[\psi:=\{\varphi_{-1}^{(n-1)},\psi_{0}^{(n)},\varphi_{0}^{(n)},\psi_{1}^{(n)},\varphi_{1}^{(n)},\cdots\}\]
from $\Flow[f]$ to $\Flow$, then we call the pair $(\Flow,\psi)$ is a \emph{periodic Higgs-de Rham flow}. Note that the $\psi$ is part of data in the periodic Higgs-de Rham flow, which we will call a periodic mapping of $\Flow$.
Since $\psi$ is uniquely determined by $\varphi_{-1}^{(n-1)},\psi_{0}^{(n)}$, we sometime use $(\Flow,(\varphi_{-1}^{(n-1)},\psi_{0}^{(n)}))$ to represent $(\Flow,\psi)$ and call . In case $n=1$, the first term in the flow is vacuous, we also use $(\Flow,\psi_{0}^{(1)})$ to represents $(\Flow,\psi)$.
\end{definition}

\begin{lemma} \label{thm_equFunctorHdRF&FFMod} Suppose $\bF_{p^f}$ is contained in $k$. Then there is an equivalent functor from the category of $f$-periodic parabolic Higgs-de Rham flows to the category of parabolic Fontaine-Faltings modules with $\bZ_{p^f}$-endomorphism structure.
\end{lemma}



\newpage

\subsection{Parabolic Higgs-de Rham flows over projective line}

Let $k$ be a finite field with cardinality $q=p^{h}$. Let $\lambda \in W(k)$ such that $\lambda\pmod{p}\neq 0,1\in k$. Denote the formal projective line over $W(k)$ and a divisor on it
\[\mP^1_{W(k)}:=\bP^1_{W(k)}\otimes_{\Spec(W(k))}\Spf(W(k)) \text{and} \mD_{W(k)}=\{0,1,\lambda,\infty\}\subset\mP^1_{W(k)}.\]
By modulo $p^n$, one gets logarithmic pair $(\mP^1_{W_n(k)},\mD_{W_n(k)})/W_n(k)$. In this section, we will study some periodic Higgs-de Rham flows over $(\mP^1_{W_n(k)},\mD_{W_n(k)})/W_n(k)$ and over $(\mP^1_{W(k)},\mD_{W_n(k)})/W(k)$. To reduce the repetition of writing, we sometimes use $W_\infty(k)$ to stand for $W(k)$.

We first recall that, for any $n\in\{\infty,1,2,\cdots\}$, \[\High(W_n(k))\coloneqq\High(\Spf(W_n(k)))\]
is the set of all isomorphic classes of rank-$2$ stable graded parabolic Higgs bundles $(E,\theta)$ of degree zero on $(\mP^1_{W_n(k)},\mD_{W_n(k)})/W_n(k)$ with all parabolic weights being zero at $\{0,1,\lambda\}$ and with all parabolic weights being $1/2$ at $\infty$.

\subsubsection{Parabolic Higg-de Rham flows initialed with given parabolic Higgs bundles in $\High(k)$}

We first construct a parabolic Higgs-de Rham flow initialed with a parabolic Higgs bundle in $\High(k)$.

\begin{lemma} \label{thm_Higgs2HDF_k}
Let $(E,\theta) \in \High(k)$. Then there is a unique (up to an isomorphism) parabolic Higgs-de Rham flow
\[\Flow = \left\{
(E,\theta)_{0},
(V,\nabla,\Fil)_{0},
(E,\theta)_{1},
(V,\nabla,\Fil)_{1},
\cdots\right\},\]
initialed with $(E,\theta)_0=(E,\theta)$, such that Higgs terms $(E,\theta)_{i}$ are contained in $\High(k)$ for all $i\geq 0$.
Moreover $(V,\nabla,\Fil)_{i}\in\MdRh(k)$ for all $i\geq 0$.
\end{lemma}

\begin{proof}
By \autoref{thm_ClassfyR2PHiggs}, $(E,\theta)$ has the form
\[\theta\colon \mL\rightarrow \mL^{-1}\otimes \Omega^1_{\bP^1_k/k}(\log D_k)\]
with a single zero $(\theta)_0\in \bP^1_k(k)$. Then taking the inverse Cartier $(V,\nabla)_0$, one gets a parabolic de Rham bundle, which is stable and of degree $0$. Hence it is contained in $\MdRh(k)$.

To make the graded Higgs bundle contained in $\High(\barFp)$ the Hodge filtration must be given by $\mL$, see \autoref{thm_ClassfyR2PdE}. Taking the grading of $(V,\nabla)_0$ with respect to the Hodge filtration, one gets a graded parabolic Higgs bundle contained in $\High(k)$;

From above, the first filtered de Rham term and the second Higgs term both exist and are uniquely determined by the first Higgs term.

Repeating the above procedure, one then get the unique parabolic Higgs-de Rham flow initialed with $(E,\theta)$:

\begin{equation*}
\xymatrix@C=2mm@R=5mm{
& {\scriptstyle ( V,\nabla,\Fil)_0} \ar[rd]|{\text{Gr}}
&
& {\scriptstyle ( V,\nabla,\Fil)_1} \ar[rd]|{\text{Gr}}
&
& {\scriptstyle ( V,\nabla,\Fil)_2} \ar[rd]|{\text{Gr}}
&
& {\scriptstyle \cdots}
\\
{\scriptstyle ( E,\theta)_0} \ar[ru]|{\mC^{-1}}
&
& {\scriptstyle ( E,\theta)_1} \ar[ru]|{\mC^{-1}}
&
& {\scriptstyle ( E,\theta)_2} \ar[ru]|{\mC^{-1}}
&
& {\scriptstyle ( E,\theta)_3} \ar[ru]|{\mC^{-1}}
&
& {\scriptstyle \quad \cdots}
}
\end{equation*}
\end{proof}
Denote by $\PHighf(k)$ the set of Higgs bundle $(E,\theta)_0$ which is $f$-periodic. I.e. there is an isomorphism between $(E,\theta)_0$ and the $f$-th Higgs term $(E,\theta)_f$. And denote
\[\PHigh(k) = \bigcup_{f} \PHighf(k).\]

\begin{proposition} \label{mthm_PHIGf2PHIG}
If $(\#k+1)!\mid f$, then
\[\PHigh(k) = \PHighf(k)\]
\end{proposition}
\begin{proof}
By \autoref{thm_Higgs0ProjLine},
we know $\#\High(k)$ has cardinality $\#k+1$. Thus the periodicity of any periodic Higgs bundle in $\High(k)$ is smaller than or equal to $\#k+1$.
\end{proof}

\begin{remark}
Although the Higgs-de Rham flow exists and must be preperiodic due to the finiteness of $\High(k)$, there is some freedom in the choice of the position of repeating part and the period mapping. For example, if $(E,\theta)_e\cong (E,\theta)_{e+f}$, then we always have
\[(E,\theta)_i\cong (E,\theta)_{i+kf},\qquad \text{for any $i\geq e$ and any $k>0$}.\]
So the cycle nodes can be chosen at $i,i+kf$, and the period mapping can be chosen to be any isomorphism between $(E,\theta)_i$ and $(E,\theta)_{i+kf}$.
\end{remark}

There is a theoretical way to find periodic Higgs bundles. Under the natural bijection $\High(k) \simeq \bP^1_k(k)$ in \autoref{thm_Higgs0ProjLine}, Sun-Yang-Zuo have shown that the self-map
\[\phi\coloneqq \text{Gr}\circ
\mC^{-1}\colon \High(k)\rightarrow \High(k)\]
is induced by an endomorphism of $\bP_{k_0}^1$ give by a rational function of form $\phi(z)=\psi(z^p)$, where $\psi$ is a rational function of degree $p$. To find periodic Higgs-de Rham flow one only need to find periodic points of the map $\phi$. In particular, we obtain
\begin{proposition}
The number of $f$-periodic Higgs bundles in $\High(\bark)$ is $p^{2f}+1$.
\end{proposition}
We take then the elliptic curve $C_{\lambda}$ over the field $k_0$ defined by the Weierstrass equation $y^2=z(z-1)(z- \lambda)$. Modulo involution on the elliptic curve induces natural double cover
\[\pi\colon C_\lambda\to \bP^1_{k_0}\] ramified on $\{0,1,\infty,\lambda\}$ and $\infty$ as the origin for the group law. Sun-Yang-Zuo have asked the following conjecture.
\begin{conjecture}
The self-map $\phi$ comes from multiplication map by $p$ on the associated elliptic curve $C_\lambda$ over $k_0$. In other words, the following diagram commutes
\[\xymatrix{
& C_\lambda \ar[d]_{\pi} \ar[r]^{[p]} & C_\lambda\ar[d]^\pi & \\
M_{Higg,\lambda}^{gr}\ar@/_12pt/[rrr]_{\phi} \ar@{=}[r] & \bP^1_{k_0} \ar[r]^{\phi} & \bP^1_{k_0} \ar@{=}[r] & M_{Higg,\lambda}^{gr} \\
}\]
\end{conjecture}

The conjecture implies two things:
\begin{enumerate}
\item a Higgs bundle $(E,\theta)$ is $f$-periodic under the map $\phi$ if and only if the two points in $\pi^{-1}(\theta)_0$ are both torsion in $C_\lambda$ and of order $p^f\pm1$.
\item for a prime $p>2$ and assume $C_\lambda$ is supersingular then $\phi_\lambda(z)=z^{p^2}$. Hence, any Higgs bundle $(E,\theta)\in {\High}(\bark)$ is periodic.
\end{enumerate}

The Conjecture has been checked by Sun-Yang-Zuo for $p<50$. Very recently it has been proved by
Lin-Sheng-Wang \cite{LSW22} and becomes a theorem.
\begin{theorem} [Lin-Sheng-Wang]
\autoref{conj:SYZ} holds true.
\end{theorem}

\begin{corollary} \label{mthm_HIG2PHIG}
If $C_\lambda$ is supersingular, then any Higgs bundle $(E,\theta)\in {\High}(\bark)$ is periodic.
\end{corollary}


\subsubsection{Parabolic Higgs-de Rham flows initialed with given parabolic Higgs bundles in $\High(W_n(k))$}

In this subsubsection, we take $n\in\{\infty,1,2,\cdots\}$. We show that the is at most one parabolic Higgs-de Rham flow initialed with a given parabolic Higgs bundle in $\High(W_n(k))$.

\begin{definition} \label{thm_Higgs2HDF_Wn}
Let $(E,\theta) \in \High(W_n(k))$. A parabolic Higgs-de Rham flow
\[\Flow = \left\{
(\overline{V},\overline{\nabla},\overline{\Fil})_{-1},
(E,\theta)_{0},
(V,\nabla,\Fil)_{0},
(E,\theta)_{1},
(V,\nabla,\Fil)_{1},
\cdots\right\},\]
over $\BBn$ is called initialed\footnote{We note that when $n=1$, the $-1$-th term $(V,\nabla,\Fil)_{-1}$ is vacuous and $(E,\theta)_{0}$ is indeed the leading term. This is why we call $0$-th term the initial one for general $n$.} with $(E,\theta)$, if there is an isomorphism between $(E,\theta)$ and $(E,\theta)_0$.
\end{definition}

Due to the uniqueness of the Hodge filtration in \autoref{thm_ClassfyR2PdE}, we may repeat the proof for \autoref{thm_Higgs2HDF_k} and get following result.
\begin{lemma} \label{thm_Higgs2HDF_W}
Let $(\overline{V},\overline{\nabla},\overline{\Fil})_{-1} \in \MdRh(W_{n-1}(k))$ and $(E,\theta)_{0}\in \High(W_n(k))$ with $\Gr(\overline{V},\overline{\nabla},\overline{\Fil}) = (E,\theta)_{0}\pmod{p^{n-1}}$. Then there exists a unique (up to an isomorphism) parabolic Higgs-de Rham flow
\[\Flow = \left\{
(\overline{V},\overline{\nabla},\overline{\Fil})_{-1},
(E,\theta)_{0},
(V,\nabla,\Fil)_{0},
(E,\theta)_{1},
(V,\nabla,\Fil)_{1},
\cdots\right\},\]
initialed with $(E,\theta)_0$, and the $-1$-th de Rham term being $(\overline{V},\overline{\nabla},\overline{\Fil})_{-1}$ such that Higgs terms $(E,\theta)_{i}$ are contained in $\High(W_n(k))$ for all $i\geq 0$.
Moreover $(V,\nabla,\Fil)_{i}\in\MdRh(W_n(k))$ for all $i\geq 0$.
\end{lemma}

\begin{lemma}
Up to an isomorphism, there is at most one periodic parabolic Higgs-de Rham flow initialed with $(E,\theta)\in \High(W_n(k))$.
\end{lemma}

\begin{proof}
Suppose $(\Flow,\psi)$ and $(\Flow',\psi')$ be two $f$-periodic flows initialed with $(E,\theta)$, denote by $(\Flow^{(n)},\psi^{(n)})$ and $(\Flow'^{(n)},\psi'^{(n)})$ their modulo $p^n$ reductions. By the uniqueness in \autoref{thm_Higgs2HDF_k}, we may identify $\Flow^{(1)}$ and $\Flow'^{(1)}$. By shifting the isomorphism on the $f-1$-th de Rham terms via the periodic maps, one gets an isomorphism between the $-1$-th de Rham terms in the flow $\Flow^{(2)}$ and $\Flow'^{(2)}$, By uniqueness in \autoref{thm_Higgs2HDF_W}, we may identify $\Flow^{(2)}$ and $\Flow'^{(2)}$. Inductively, one can identify $\Flow^{(n)}$ and $\Flow'^{(n)}$ for all $n$.
\end{proof}

\subsubsection{An equivalence on the set of isomorphic classes of periodic Higgs-de Rham flows.}

Let $(\Flow,\psi)$ be an $f$-periodic Higgs-de Rham flow with
\[\Flow = \left\{
(\overline{V},\overline{\nabla},\overline{\Fil})_{-1},
(E,\theta)_{0},(V,\nabla,\Fil)_{0},(E,\theta)_{1},(V,\nabla,\Fil)_{1},\cdots\right\},\]
and $\psi\colon \Flow[f]\cong \Flow$. By shifting the index, one gets isomorphisms of flows
\[\psi[k]\colon \Flow[f+k] \rightarrow \Flow[k],\quad \text{for all $k\geq0$}.\]
For any $k\geq1$, there is a natural isomorphism
\[\psi^k:=\psi[(k-1)f]\circ\cdots\circ\psi[f]\circ\psi\colon \Flow[kf]\rightarrow\Flow\]
Thus one gets a periodic flow $(\Flow,\psi^k)$ for any $k\geq1$.

\begin{definition} \label{def_diffByConstant}
Let $(\Flow_1,\psi_1)$ and $(\Flow_2,\psi_2)$ be two $f$-periodic Higgs-de Rham flows over $\BBn$. We call they are \emph{differed by a constant}, if
\begin{itemize}
\item there exists an isomorphism of the underlying flows, and
\item once we identify the flows via the isomorphism, there exists a unit $u\in W_n(k)^\times$ such that
\[\psi_1 = u\cdot \psi_2.\]
\end{itemize}
\end{definition}

Let $\PHDFh(W_n(k))$ be the set of isomorphic classes of periodic Higgs-de Rham flows (Higgs-de Rham flows with periodic mappings) with all Higgs terms are contained in $\High(W_n(k))$ and all de Rham terms are contained in $\MdRh(W_n(k))$.

\begin{lemma}
Two periodic Higgs-de Rham flows in $\PHDFhf(W_n(k))$ are differed by a constant if and only if they have isomorphic initial terms.
\end{lemma}
\begin{proof}
The ``only if'' part is trivial. Now, we consider the ``if'' part and assume the two flow have isomorphic initial terms. By \autoref{thm_Higgs2HDF_k}, there is an isomorphism between the underlying flows. We may identify this two flow. Then there are two periodic mappings on this flow. We need to show this two mappings are differed by a unit in the sense \autoref{def_diffByConstant}. This follows that fact that the modulo $p$ reduction of all Higgs terms appeared in the flow are stable.
\end{proof}

\begin{corollary} \label{mthm_PHDFf2PHIGf}
Differed by a constant is an equivalent relations on the sets $\PHDFhf(W_n(k))$. Taking initial terms induces bijection
\[ [\PHDFhf(W_n(k))] \xrightarrow{1:1} \PHighf(W_n(k)).\]
Taking inverse limits, one gets an bijection
\[ [\PHDFhf(W(k))] \xrightarrow{1:1} \PHighf(W(k)).\]
\end{corollary}

\subsection{Parabolic Fontaine-Faltings modules over projective line}
\subsubsection{Fontaine-Faltings modules associated to periodic flows in $\PHDFh(W(k))$}\label{sec_perHiggs2FFM} \label{sec_PHDF2MF}

In this subsubsection, we construct Lan-Sheng-Zuo equivalent functor in parabolic setting.

Let $(\Flow,\psi)$ be an $f$-periodic flow in $\PHDFhf(W(k))$ with
\[\Flow = \left\{
(\overline{V},\overline{\nabla},\overline{\Fil})_{-1},
(E,\theta)_{0},(V,\nabla,\Fil)_{0},(E,\theta)_{1},(V,\nabla,\Fil)_{1},\cdots\right\}\]
and $\psi=\{\overline{\varphi}_{-1},\psi_{0},\varphi_{0},\psi_{1},\varphi_{1},\cdots\}$. By adding up all filtered de Rham terms appeared in the repeating part, one gets a parabolic de Rham bundle of rank $2f$
\begin{equation}\label{equ_dRtermsSum}
(V,\nabla,\Fil):= ( V,\nabla,\Fil)_0 \oplus ( V,\nabla,\Fil)_1 \oplus \cdots \oplus ( V,\nabla,\Fil)_{f-1}.
\end{equation}
We defined an isomorphism $\varphi\colon C^{-1} \circ \overline{Gr} ( V,\nabla,\Fil) \to ( V,\nabla,\Fil)$ by
\begin{equation*} \footnotesize
\xymatrix@C=2mm@R=1cm{
C^{-1} \circ \overline{Gr} ( V,\nabla,\Fil) \ar[d]^{\varphi} \ar@{}[r]|{=}
& C^{-1} \circ \overline{Gr} ( V,\nabla,\Fil)_0 \ar[dr]^{\id} \ar@{}[r]|-{\oplus}
& C^{-1} \circ \overline{Gr} ( V,\nabla,\Fil)_1 \ar[dr]^(0.7){\id} \ar@{}[r]|-{\oplus}
& \cdots \ar[dr]^{\id} \ar@{}[r]|-{\oplus}
& C^{-1} \circ \overline{Gr} ( V,\nabla,\Fil)_{f-1} \ar[dlll]^{\varphi_0}\\
( V,\nabla,\Fil) \ar@{}[r]|{=}
&( V,\nabla,\Fil)_0 \ar@{}[r]|-{\oplus}
&( V,\nabla,\Fil)_1 \ar@{}[r]|-{\oplus}
&\cdots \ar@{}[r]|-{\oplus}
&( V,\nabla,\Fil)_{f-1} \\
}
\end{equation*}
Then tuple
\begin{equation} \label{eq_HDF2FFM}
(V,\nabla,\Fil,\varphi)
\end{equation}
forms a parabolic Fontaine-Faltings module.

\begin{remark} In order to construct a correspondence between periodic Higgs-de Rham flows and Fontaine-Faltings modules. We need overcome one obstacle. By shifting the flow $i$-times, we get an other $f$-periodic flow $(\Flow[i],\psi[i])$. From above construction, we can see that these two flows corresponding to isomorphic Fontaine-Faltings module. Hence the wanted correspondence is not injective.
In order to get an injective one, one needs to add endomorphism structures on such a Fontaine-Faltings module, such that different periodic flows corresponds to different Fontaine-Faltings modules with endomorphism structures.
\end{remark}

In the following, we construct natural $\bZ_{p^f}$-endomorphism structures
\[\iota_j \colon \bZ_{p^f}\rightarrow \End \Big((V,\nabla,\Fil,\varphi)\Big),\quad j\in \bZ.\]
which can be used to distinguish direct summands $(V_i,\nabla_i,\Fil_i)$ of the underlying de Rham bundle $(V,\nabla,\Fil)$.
\begin{lemma} \label{thm_ConstEndStructure} Suppose $\mathbb F_{p^f}\subseteq k$.
For any $j\in\bZ$, any $a\in\bZ_{p^f}$ and any local section $v_i\in V_i$, set
\[\iota_j(a)(v_i):=\sigma^{i+j}(a)\cdot v_i.\]
Then $\iota_j$ is an $\bZ_{p^f}$-endomorphism structure on $(V,\nabla,\Fil,\varphi)$.
\end{lemma}
\begin{proof}
Since $\nabla$ is $W(k)$-linear and $\Fil$ consists of sub-$W(k)$-modules, $\iota_j$ indeed gives an $\bZ_{p^f}$-endomorphism on $(V,\nabla,\Fil)$. Next, one only need to show $\iota_j$ preserves the Frobenius structure $\varphi$ in the Fontaine-Faltings module. In other words, we need to check the following diagram commutes for any $a\in\bZ_{p^f}$
\begin{equation*}
\xymatrix{
F^*\widetilde{V} \ar[r]^{\varphi} \ar[d]_{\id\otimes \iota_j(a)} & V \ar[d]^{\iota_j(a)}\\
F^*\widetilde{V} \ar[r]^{\varphi} & V\\
}
\end{equation*}
For any local section $v_{i\ell}\in\Fil^\ell V_i$, we have $\varphi(1\otimes[v_{i\ell}])\in V_{i+1}$. Thus
\[\iota_j(a)\circ\varphi(1\otimes[v_{i\ell}]) = \sigma^{i+1+j}(a)\cdot\varphi(1\otimes[v_{i\ell}]).\]
On the other hand, one has
\[\varphi\circ (\id\otimes \iota_j(a)) (1\otimes[v_{i\ell}]) = \varphi(1\otimes\sigma^{i+j}(a)\cdot [v_{i\ell}])=\sigma^{i+j+1}(a)\cdot \varphi(1\otimes [v_{i\ell}]),\]
where the last equality follows the $\sigma$-semilinearity of $\varphi$. Thus the Lemma follows.
\end{proof}

\begin{definition} Suppose $\bF_{p^f}\subseteq k$.
\begin{enumerate}
\item Let $(V,\nabla,\Fil,\iota)$ be a filtered parabolic de Rham bundle $V$ with an $\bF_{p^f}$-endomorphism structure $\iota$ over $(\bP^1_{W(k)},D_{W(k)})$. Then the filtration can be restricted on $V^{\iota=\sigma^i}$.

We call the sub parabolic de Rham bundle \[(V^{\iota=\id},\nabla\mid_{V^{\iota=\id}},\Fil_{V^{\iota=\id}}) =: (V,\nabla,\Fil)^{\iota=\id}\]
\emph{the $i$-th eigen component of $(V,\nabla,\Fil,\iota)$}. If $i=0$, then we call it \emph{the identity component of $(V,\nabla,\Fil,\iota)$.}
\item Let $(E,\theta,\iota)$ be a parabolic Higgs bundle $V$ with an $\bF_{p^f}$-endomorphism structure $\iota$ over $(\bP^1_{W(k)},D_{W(k)})$. Then the Higgs field can be restricted on $E^{\iota=\sigma^i}$.
We call the sub parabolic Higgs bundle \[(E^{\iota=\id},\theta\mid_{E^{\iota=\id}}) =: (E,\theta)^{\iota=\id}\]
\emph{the $i$-th eigen component of $(E,\theta,\iota)$}. If $i=0$, then we call it \emph{the identity component of $(E,\theta,\iota)$.}
\end{enumerate}

\end{definition}

By direct calculation, one has following result.
\begin{lemma} For any $j\in\bZ$, one has
\[\iota_{j+1} = \iota_j\circ\sigma \quad\text{and}\quad \iota_j=\iota_{j+f}.\]
For any $i=0,\cdots,f-1$, the direct summand $V_i$ is the identity component of $(V,\iota_j)$ if and only if $f\mid i+j$. In particular, by taking the identity components from different endomorphism structures, we can pick out different direct summands of the filtered de Rham bundle.
\end{lemma}

Taking grading of the underlying filtered de Rham bundle, one gets endomorphism structures on the graded Higgs bundle, still denoted by $\iota_j$ by abusing notion,
\[\iota_j\colon \bZ_{p^f} \rightarrow \End(E,\theta),\]
where $(E,\theta) = (E,\theta)_0 \oplus (E,\theta)_1 \oplus \cdots \oplus (E,\theta)_{f-1}$. By direct calculation, one has following result.
\begin{lemma} For any $j\in \bZ$, any $i\in\{0,1,\cdots,f-1\}$ and any local section $v_i\in E_i$
\[\iota_j(a)(v_i):=\sigma^{i+j-1}(a)\cdot v_i.\]
In particular, the Higgs bundle $(E_0,\theta_0)$ is the identity component of $(E,\theta,\iota_1)$.
\end{lemma}

For any integer $f$ such that $\bF_{p^f}\subseteq k$, denote by \emph{$\MFh(W(k))_{\bZ_{p^f}}$} the set of all isomorphic classes of Fontaine-Faltings module with an $\bZ_{p^f}$-endomorphism structure such that all eigen components\footnote{The condition $\bF_{p^f}\subseteq k$ ensure that we can take eigen components.} of the corresponding filtered de Rham bundles are contained $\MdRh(W(k))$ and all eigen components of the corresponding graded Higgs bundles are contained in $\High(W(k))$.


By the above construction of the parabolic version of Lan-Sheng-Zuo's equivalent functor, one gets following bijection, whose proof is the same as the original version of Lan-Sheng-Zuo.
\begin{proposition} Let $k'$ be a finite extension of $k$ containing $\bF_{p^f}$. Then one has a bijection
\[\PHDFhf(W(k')) \xrightarrow{1:1} \MFh(W(k'))_{\bZ_{p^f}}\]
sending an $f$ periodic flow $(\Flow,\psi)$ to $(V,\nabla,\Fil,\varphi,\iota_1)$, where $(V,\nabla,\Fil,\varphi)$ is given in \eqref{eq_HDF2FFM} and $\iota_1$ is given in \autoref{thm_ConstEndStructure}.
\end{proposition}

By base change from $k$ to $k'$, one gets the natural embedding
\[\PHDFhf(W(k)) \hookrightarrow \PHDFhf(W(k')).\]
\begin{corollary}
Let $k'$ be a finite extension of $k$ containing $\bF_{p^f}$. then the restriction of bijection induces an injection
\begin{equation}\label{eq_PHDFHfW2MFhf}
\PHDFhf(W(k)) \hookrightarrow \MFh(W(k'))_{\bZ_{p^f}}.
\end{equation}
\end{corollary}


\subsubsection{An equivalence relation on the set of isomorphic classes of Fontaine-Faltings modules in $\MFh$.} \label{subsec_HDF2FFM}

Let $k'$ be a finite field extension of $k$ containing $\bF_{p^f}$.

\begin{definition} \label{def_diffByConstantFFM}
Let $M$ and $M' \in \MFh(W(k'))_{\bZ_{p^f}}$. We call they are \emph{differed by a constant}, if there exists a constant Fontaine-Faltings module $M^\circ \in \MF_{[0,0]}^{\varphi}(W(k'))_{\bZ_{p^{f}}}$ of rank $1$ such that
\[M' = M \otimes M^\circ.\]
Clearly, differed by a constant is an equivalent relation on the set $\MFh(W(k'))_{\bZ_{p^f}}$. Denote by \emph{$[\MFh(W(k'))_{\bZ_{p^f}}]$} the set of all equivalent classes.
\end{definition}

\begin{remark} \label{rmk_FFotimesConstFF}
Suppose $M=(V,\nabla,\Fil,\varphi,\iota)$ and $M^\circ=(V^\circ,\varphi^\circ,\iota^\circ)$. Denote by $(V,\nabla,\Fil)_i$ the $i$-th eigen component of $(V,\nabla,\Fil,\iota)$ and denote by $V^\circ_i\simeq W(k')$ the $i$-th eigen component of $V^\circ$. Then
One the direct summand $(V,\nabla,\Fil)_i\otimes_{W(k')} V^\circ_j\simeq (V,\nabla,\Fil)_i$ of $(V,\nabla,\Fil)\otimes_{W(k)} V^\circ$, the action of $\iota$ and $\iota'$ are coincide if and only if $i=j$. Thus we can see that
the underlying filtered de Rham bundle of $(V,\nabla,\Fil)\otimes_{W(k)} V^\circ$ is
\[\bigoplus_{i=0}^{f-1} (V,\nabla,\Fil)_i\otimes_{W(k')} V^\circ_i \]
which is isomorphic to $(V,\nabla,\Fil)$ once we fixed a basis $e_i$ for each $V^\circ_i$, the map is given by $v_i\otimes e_i \mapsto v_i$.
We also decompose the Frobenius structure $\varphi$
\begin{equation*} \footnotesize
\xymatrix@C=2mm@R=1cm{
C^{-1} \circ \overline{Gr} ( V,\nabla,\Fil) \ar[d]^{\varphi} \ar@{}[r]|{=}
& C^{-1} \circ \overline{Gr} ( V,\nabla,\Fil)_0 \ar[dr]^{\varphi_0} \ar@{}[r]|-{\oplus}
& C^{-1} \circ \overline{Gr} ( V,\nabla,\Fil)_1 \ar[dr]^(0.7){\varphi_1} \ar@{}[r]|-{\oplus}
& \cdots \ar[dr]^{\varphi_{f-2}} \ar@{}[r]|-{\oplus}
& C^{-1} \circ \overline{Gr} ( V,\nabla,\Fil)_{f-1} \ar[dlll]^{\varphi_{f-1}}\\
( V,\nabla,\Fil) \ar@{}[r]|{=}
&( V,\nabla,\Fil)_0 \ar@{}[r]|-{\oplus}
&( V,\nabla,\Fil)_1 \ar@{}[r]|-{\oplus}
&\cdots \ar@{}[r]|-{\oplus}
&( V,\nabla,\Fil)_{f-1} \\
}
\end{equation*}
Suppose $\varphi_i^\circ(e_i) = a_ie_{i+1}$, then
we see that for any $v_i\ell\in \Fil^\ell V_i$
\[\varphi_{\rm tot}(1\otimes_{\Phi} [v_i]\otimes e_i) = \varphi_i(1\otimes_{\Phi} [v_i]) \otimes \varphi_i^\circ(e_i) = a_i \cdot \varphi(1\otimes_{\Phi} [v_i]) \otimes e_{i+1}.\]
If we identify $(V,\nabla,\Fil)_i\otimes_{W(k')} V^\circ_i$ with $(V,\nabla,\Fil)_i$ by sending $v_i\otimes e_i$ to $v_i$, then the Frobenius structure on $M'$ can be describe as
\begin{equation*} \footnotesize
\xymatrix@C=2mm@R=1cm{
C^{-1} \circ \overline{Gr} ( V,\nabla,\Fil) \ar[d]^{\varphi'} \ar@{}[r]|{=}
& C^{-1} \circ \overline{Gr} ( V,\nabla,\Fil)_0 \ar[dr]^{a_0\varphi_0} \ar@{}[r]|-{\oplus}
& C^{-1} \circ \overline{Gr} ( V,\nabla,\Fil)_1 \ar[dr]^(0.7){a_1\varphi_1} \ar@{}[r]|-{\oplus}
& \cdots \ar[dr]^{a_{f-2}\varphi_{f-2}} \ar@{}[r]|-{\oplus}
& C^{-1} \circ \overline{Gr} ( V,\nabla,\Fil)_{f-1} \ar[dlll]^{a_{f-1}\varphi_{f-1}}\\
( V,\nabla,\Fil) \ar@{}[r]|{=}
&( V,\nabla,\Fil)_0 \ar@{}[r]|-{\oplus}
&( V,\nabla,\Fil)_1 \ar@{}[r]|-{\oplus}
&\cdots \ar@{}[r]|-{\oplus}
&( V,\nabla,\Fil)_{f-1} \\
}
\end{equation*}
Moreover, if we choose the base suitable, we may even make $a_0=a_1=\cdots=a_{f-2}=1$. In this case, the only map need to change is $a_{f-1}$.
\end{remark}

\begin{lemma} \label{mthm_PHDFf2MFf}
The injection in \eqref{eq_PHDFHfW2MFhf} induces another one
\begin{equation} \label{eq_PHDFfW2MFfW}
[\PHDFhf(W(k))] \hookrightarrow [\MFh(W(k'))_{\bZ_{p^f}}].
\end{equation}
\end{lemma}

\begin{proof}
Let
\[(\Flow,\psi) = (\left\{
(\overline{V},\overline{\nabla},\overline{\Fil})_{-1},
(E,\theta)_{0},(V,\nabla,\Fil)_{0},(E,\theta)_{1},(V,\nabla,\Fil)_{1},\cdots\right\},\psi)\]
and
\[(\Flow',\psi') = (\left\{
(\overline{V},\overline{\nabla},\overline{\Fil})'_{-1},
(E,\theta)'_{0},(V,\nabla,\Fil)'_{0},(E,\theta)'_{1},(V,\nabla,\Fil)'_{1},\cdots\right\},\psi')\]
be two $f$-periodic flows. Let $M=(V,\nabla,\Fil,\varphi,\iota)$ be the associated Fontaine-Faltings module of $(\Flow,\psi)$. Then \begin{equation}\label{equ_dRtermsSum}
(V,\nabla,\Fil):= ( V,\nabla,\Fil)_0 \oplus ( V,\nabla,\Fil)_1 \oplus \cdots \oplus ( V,\nabla,\Fil)_{f-1}.
\end{equation}
and
\begin{equation*} \footnotesize
\xymatrix@C=2mm@R=1cm{
C^{-1} \circ \overline{Gr} ( V,\nabla,\Fil) \ar[d]^{\varphi} \ar@{}[r]|{=}
& C^{-1} \circ \overline{Gr} ( V,\nabla,\Fil)_0 \ar[dr]^{\id} \ar@{}[r]|-{\oplus}
& C^{-1} \circ \overline{Gr} ( V,\nabla,\Fil)_1 \ar[dr]^(0.7){\id} \ar@{}[r]|-{\oplus}
& \cdots \ar[dr]^{\id} \ar@{}[r]|-{\oplus}
& C^{-1} \circ \overline{Gr} ( V,\nabla,\Fil)_{f-1} \ar[dlll]^{\varphi_0}\\
( V,\nabla,\Fil) \ar@{}[r]|{=}
&( V,\nabla,\Fil)_0 \ar@{}[r]|-{\oplus}
&( V,\nabla,\Fil)_1 \ar@{}[r]|-{\oplus}
&\cdots \ar@{}[r]|-{\oplus}
&( V,\nabla,\Fil)_{f-1} \\
}
\end{equation*}
We also have similar diagram for $M'$. Then by remark \autoref{rmk_FFotimesConstFF}, $\varphi_0$ and $\varphi_0'$ are differed by a unit in $W(k')$. This means the original flows are differed by a constant.
\end{proof}

\subsubsection{representative element with cyclotomic determinant}
Let $k'$ be a finite field extension of $k$ containing $\bF_{p^f}$.
Denote by $k_2'$ the field extension of $k$ of degree $2$.

\begin{definition}
We say that a Fontaine-Faltings module \emph{has cyclotomic determinant} if its determinant is the cyclotomic Fontaine-Faltings module. Denote by $\MFh(W(k'))_{\bZ_{p^f}}^{\rm cy}$ the subset of $\MFh(W(k'))_{\bZ_{p^f}}$ consisting of elements with cyclotomic determinant and denote by $[\MFh(W(k'))_{\bZ_{p^f}}^{\rm cy}]$ the image of $\MFh(W(k'))_{\bZ_{p^f}}^{\rm cy}$ in $[\MFh(W(k'))_{\bZ_{p^f}}]$.
\end{definition}

\begin{proposition} \label{thm_MFfW2MFWcy}
Let $M \in \MFh(W(k'))_{\bZ_{p^f}}$. There exists a constant Fontaine-Faltings module $M^\circ\in\MF^\varphi_{[0,0]}(W(k'_2))_{\bZ_{p^f}}$ of rank $1$, such that $M\otimes M^\circ \in \MFh(W(k'_2))_{\bZ_{p^f}}^{\rm cy}$.
\end{proposition}

\begin{lemma}
Any object in $\MFh(W(k'))_{\bZ_{p^f}}$ has a constant determinant contained in $\MF^{\varphi}_{[1,1]}(W(k'))_{\bZ_{p^f}}$.
\end{lemma}
\begin{proof} Let $(V,\nabla,\Fil,\varphi,\tau)\in\MFh(W(k'))_{\bZ_{p^f}}$.
The associated Higgs-de Rham flow is of form
\begin{equation*} \tiny
\xymatrix@C=2mm{
&(\mL\oplus \mL^{-1},\nabla_0,\Fil) \ar[dr] && \cdots \ar[dr]&&
(\mL\oplus \mL^{-1},\nabla_{f-1},\Fil)\ar[dr] &
\\
(\mL\oplus \mL^{-1},\theta_0) \ar[ur] &&
(\mL\oplus \mL^{-1},\theta_1) \ar[ur] &&
(\mL\oplus \mL^{-1},\theta_{f-1}) \ar[ur] &&
(\mL\oplus \mL^{-1},\theta_f) \ar@/^16pt/[llllll]^{\simeq}_\psi \\
}
\end{equation*}
Taking determinant one gets
\begin{equation*} \tiny
\xymatrix@C=2mm{
&(\mO,\det(\nabla_0),\det(\Fil)) \ar[dr] && \cdots \ar[dr]&&
(\mO,\det(\nabla_{f-1}),\det(\Fil))\ar[dr] &
\\
(\mO,0) \ar[ur] &&
(\mO,0) \ar[ur] &&
(\mO,0) \ar[ur] &&
(\mO,0) \ar@/^16pt/[llllll]^{\simeq}_{\det\psi} \\
}
\end{equation*}
we note that the determinant of the Higgs field is trivial because the Higgs field is graded.

Write $\det(\nabla_i) = \rmd + \omega_i$. Due to the existence of Frobenius structure, one has (we denote $\omega_{f}:=\omega_0$)
\[\omega_{i+1} = F^*\omega_i.\]
Thus all $\omega_i=0$. This is because $p^\alpha\mid \omega$ implies $p^{\alpha+1}\mid F^*\omega$. In particular, the eigen component of the underlying de Rham bundles are all trivial. Thus it is constant.
\end{proof}

\begin{proof}[Proof of \autoref{thm_cyclDeterminant_FFM}]
By the lemma, we gets $\det(M)$ is constant and contained in $\MF^{\varphi}_{[1,1]}(W(k'))$. According \autoref{thm_sqroot_constant_FFM_level0}, there exists a constant $M^\circ\in\MF^{\varphi}_{[0,0]}(W(k'))$ such that
\[\det(M)\otimes M^\circ\otimes M^\circ = M_{cy}.\]
Since $M$ is of rank $2$, the determinant of $M\otimes M^\circ$ is cyclotomic.
\end{proof}

\begin{corollary} \label{thm_cyclDeterminant_FFM}
The base change from $W(k')$ to $W(k'_2)$ induces an injection
\[[\MFh(W(k'))_{\bZ_{p^f}}] \hookrightarrow [\MFh(W(k'_2))_{\bZ_{p^f}}^{\rm cy}].\]
\end{corollary}
\begin{proof}
The only thing we have to check is the injectivity. Suppose two Fontaine-Faltings module $M,M'\in \MFh(W(k'))_{\bZ_{p^f}}$ are differed by a constant in $\MFh(W(k'_2))_{\bZ_{p^f}}$. Then by \autoref{rmk_FFotimesConstFF}, we may identify the underlying filtered de Rham bundles with endomorphism structure. Then only the $f-1$-th eigen components of the Frobenius structures are differed by a unit $u\in W(k'_2)^\times$. But both Fontaine-Faltings module are contained in $\MFh(W(k'))_{\bZ_{p^f}}$, so the unit $u$ must contained in $W(k')$. In other word, they are contained in the same class in $[\MFh(W(k'))_{\bZ_{p^f}}]$.
\end{proof}

\newpage
\subsection{Frobenius action}
\subsubsection{The Frobenius action on $\High(\barFp)$} \label{sec_FrobActionHiggs}

Recall $k$ is a finite field with cardinality $q=p^h$ containing $k_0\ni \lambda$. By extension the coefficient, we may embedding $\High(k)$ into $\High(\bark)$.

Let $\Frob\colon \bP_{\bark}^1 \rightarrow \bP_{\bark}^1$ be the Frobenius endomorphism, i.e., the base change to $\bark$ of the morphism induced by the map $a \mapsto a^p$ on $\bP^1_{\bF_p}$. The pullback functor induces natural map
\[\Frob^*\colon\High(\bark) \rightarrow \HIG_{\Frob^{-1}(\lambda)}^{\mathrm{gr}{1\over2}}(\bark).\]
Denote $\Frob_{k_0}=\Frob^{h_0}$ and $\Frob_k=\Frob^{h}$. Since $\Frob^{h}(\lambda)=\lambda$, one gets a bijective endomapping
\[\Frob_k^*\colon\High(\bark) \rightarrow \High(\bark).\]
In our case, the mapping is easy to describe: if the zero of the Higgs field $\theta$ is $(\theta)_0=:a$, then the zero of the Higgs field $\Frob_k^*(\theta)$ is $\Frob_k^{-1}(a)$. In particular, one have following result.
\begin{lemma}
$\High(k) = \Big(\High(\bark)\Big)^{\Frob_k^*}$.
\end{lemma}
\begin{proof}
Since $(E,\theta)\in\High(k)$ if and only if $a:=(\theta)_0\in k$. This is also equivalent to $\Frob^{h}(a)=a$. Now the Lemma follows the description of the action of $\Frob$ on $\High(\bark)$.
\end{proof}



\newpage
\subsection{Lifting of parabolic Higgs-de Rham flows}
\subsubsection{Lifting the periodic parabolic Higgs bundles}

In this section, we lift those periodic Higgs bundles in $\High(S_1)$ to periodic ones in $\High(\overline{S})$ inductively, where $S=\Spec(W(\overline{k}))$.

Let $(\overline{E},\overline{\theta})_0$ be an $f$-periodic Higgs bundle in $\High(S_1)$ with the corresponding flow
\begin{equation*}
\xymatrix@C=2mm@R=5mm{
& {\scriptstyle (\overline V,\overline\nabla,\overline {E}^{1,0})_0} \ar[rd]|{\text{Gr}}
&
& {\scriptstyle (\overline V,\overline\nabla,\overline E^{1,0})_1} \ar@{.>}[rd]|{\text{Gr}}
& {\scriptstyle \quad \cdots} \ar@{}[d]|{\quad\cdots}
&
& {\scriptstyle \cdots} \ar@{}[d]|{\cdots}
& {\scriptstyle (\overline V,\overline\nabla,\overline E^{1,0})_{f-1}} \ar[rd]|{\text{Gr}}
&
\\
{\scriptstyle (\overline E,\overline\theta)_0} \ar[ru]|{\mC^{-1}}
&
& {\scriptstyle (\overline E,\overline\theta)_1} \ar[ru]|{\mC^{-1}}
&
& {\scriptstyle \quad \cdots}
& {\scriptstyle \cdots}
& {\scriptstyle \cdots} \ar[ru]|{\mC^{-1}}
&
& {\scriptstyle (\overline E,\overline\theta)_{f}},\ar@/^3pc/[llllllll]|{\simeq}_{\psi}
}
\end{equation*}

From now on we identify $(\overline E,\overline\theta)_{f}$ with $(\overline E,\overline\theta)_{0}$ via the isomorphism $\psi$.

\textbf{Lifting over $S_2$.} Choose a lifting $(E,\theta)_0$ of $(\overline{E},\overline{\theta})_0$ in $\High(S_2)$. By running the Higgs-de Rham flow over $S_2$, we gets
\begin{equation*}\tiny
\xymatrix@W=10mm@C=-3mm@R=5mm{
&& \cdots \ar@{}[d]|\cdots && \cdots \ar@{}[d]|\cdots \ar[dr]|{\Gr}
&& \cdots \ar@{}[d]|\cdots && \cdots \ar@{}[d]|\cdots \ar[dr]|{\Gr}
&& \cdots \ar@{}[d]|\cdots && \cdots \ar@{}[d]|\cdots \ar[dr]|{\Gr}
&& \cdots
\\
& (E,\theta)_{0} \ar[ur]^{\mC^{-1}_2} \ar@{..>}[dd]
&\cdots&\cdots& \cdots & (E,\theta)_{f} \ar[ur]^{\mC^{-1}_2} \ar@{..>}[dd] &\cdots&\cdots& \cdots & (E,\theta)_{2f} \ar[ur]^{\mC^{-1}_2} \ar@{..>}[dd] &\cdots&\cdots& \cdots & (E,\theta)_{3f} \ar[ur]^{\mC^{-1}_2} \ar@{..>}[dd]
\\
(\overline V,\overline\nabla,\overline E^{1,0})_{f-1} \ar[dr]|{\Gr}
&\\
&(\overline{E},\overline{\theta})_0
&\cdots&\cdots&\cdots&(\overline{E},\overline{\theta})_0
&\cdots&\cdots&\cdots&(\overline{E},\overline{\theta})_0
&\cdots&\cdots&\cdots&(\overline{E},\overline{\theta})_0 \\
}
\end{equation*}
\begin{remark}
In our case, the obstruction for lifting Hodge filtration vanish even the lifting is unique due to \autoref{thm_ClassfyR2PdE}. In particular, lifting flow is uniquely determined by the lifting $(E,\theta)_0$.
\end{remark}

Since the Higgs bundle in $\High(S)$ is uniquely determined by its zero, the lifting torsor space of $(\overline{E},\overline{\theta})_0$ is isomorphic to $\bA^1_k$ (non-canonically). In \cite{KYZ20D}, the operator $\Big(\Gr\circ C_2^{-1}\Big)^f$ induces a selfmap on this torsor space, which is of form $z\mapsto az^p+b$ if we choose an identification of the torsor space with the affine line over $k$. In particular, the solutions of the Artin-Schreier equation
\begin{equation} \label{equ:ArtinSchreier}
az^p+b = z
\end{equation}
correspond to $f$-periodic Higgs bundles in $\High(S_2)$ which lifts $(\overline E,\overline\theta)_{0}$.
Hence, if we extend the field $k$ a little bit, we can always find $f$-periodic Higgs bundles in $\High(S_2)$ which lifts $(\overline E,\overline\theta)_{0}$.

\begin{remark}
If $a=0$, then there are exact one periodic lifting of $(\overline E,\overline\theta)_{0}$ in $\High(S_2)$. In this case, we do not need extend the field, the lifting is already defined over $S_2$.

If $a\neq 0$, then there are exact $p$ periodic lifting of $(\overline E,\overline\theta)_{0}$ in $\High(S_2)$, once we enlarge the field $k$ a little bit properly.
\end{remark}

\textbf{Lifting over $S$.} Working the above lifting procedure inductively, we obtain $f$-periodic Higgs bundles in $\High(\overline{S})$, which lifts $(\overline E,\overline\theta)_{0}$.

\begin{remark}
The lifting of a periodic Higgs bundle over $\Fq$ to a periodic Higgs bundle over $\bZ_p^{ur}$ is in general not unique, as the solutions of the Artin-Schreier equation are not unique in general.
\end{remark}

\textbf{The associated parabolic Fontaine-Faltings modules.} Let $(E,\theta)_0\in \High(\overline{S})$ be $f$-periodic with the corresponding flow
\[\Big((V,\nabla,E^{1,0})_{-1},(E,\theta)_0,(V,\nabla,E^{1,0})_0,\cdots (E,\theta)_{f-1},(V,\nabla,E^{1,0})_{f-1},(E,\theta)_f,\cdots\Big)\]
Similarly, as in \autoref{subsec_HDF2FFM}, we gets a parabolic Fontaine-Faltings module over $(\bP_{\overline{S}}^1,D_{\overline{S}})/\overline{S}$.


\subsubsection{Lifting in supersingular case}
\begin{definition}
An element $\lambda$ in $\mO_S(S)=W(k)$ such that
\[\overline{\lambda}:=\lambda\pmod{p}\neq 0,1\in k\]
is called \emph{supersingular} if the elliptic curve
$C_{\overline\lambda}$ is supersingular.
\end{definition}


\begin{remark} By direct calculation, one can check that if $\lambda$ is supersingular, then the coefficient $a$ appeared in the Artin-Schreier equation \eqref{equ:ArtinSchreier} is trivial.
\end{remark}


\begin{theorem} \label{mthm_PHIGk2PHIGW}
Assume that $\lambda$ is supersingular. Any $f$-periodic parabolic Higgs bundle $(\overline E,\overline\theta)\in\PHighf(k)$ has a unique $f$-periodic lifting $(E,\theta) \in\PHighf(W(k))$. In other words, the modulo $p$ reduction induces a bijection
\[\PHighf(W(k)) \xrightarrow{1:1} \PHighf(k).\]
\end{theorem}

Recall the bijection $[\PHDFhf(W(k))] \xrightarrow{1:1} [\MFh(W(k))_{\bZ_{p^f}}]$ in \eqref{??}.
\begin{corollary} Assume that $\lambda$ is supersingular and suppose $\bF_{p^f}\subset k$. Then the operation of the Cartier functor on the direct sum of Higgs terms in the flow induces a natural bijective map
\[\PHighf(k) \simeq [\MFh(W(k))_{\bZ_{p^f}}]\]
\end{corollary}



\newpage

\section{\bf Overconvergent $F$-isocrystals} \label{sec_main_F_Isoc}

\subsection{$F$-crystals}

In this subsection, we also recall some basic definitions we need for this article, including those of convergent $F$-isocrystals, overconvergent $F$-isocrystals and convergent log-$F$-isocrystals from \cite[Definition 2.1, Definition 2.4 and Definition 7.1]{Ked22}.

Let $X$ be a proper smooth variety over $k$, $D$ be a normal crossing divisor in $X$ and $U = X-D$. We endow $X$ with the natural logarithmic structure induced by $D$, and simply write $(X,D)$ for the corresponding logarithmic scheme.

\subsubsection{(logarithmic) $F$-crystal}
Kato has defined (logarithmic) crystalline site $((X,D)/W)^{\log}_{\rm crys}$, and $\Crys((X,D)/W)$ the category of crystals in \emph{finite coherent $\mathcal{O}_{(X,D)/W}$-modules}. By functoriality of the crystalline topos, the absolute Frobenius $\Frob_{X}:X\rightarrow X$ gives a functor $\Frob_{X}^{*}\colon\Crys((X,D)/W) \rightarrow \Crys((X,D)/W)$. An \emph{(logarithmic) $F$-crystal} in finite, locally free modules on $U$ is a crystal $\mE$ in finite, locally free $\mathcal{O}_{(X,D)/W}$-modules together with an isogeny $F\colon \Frob_{X}^{*}\mE\rightarrow \mE$. The $\mathbb{Z}_{p}$-linear category of (logarithmic) $F$-crystals in finite, locally free modules is denoted as $\FCrys((X,D)/W)$.

\begin{theorem}[Kato] \label{thm_Kato_equivalent}
There is an equivalence between the following two categories:
\begin{enumerate}
\item[(a)] the category of crystals $\mE$ on $((X,D)/W)^{\log}_{\rm crys}$,
\item[(b)] the category of $\mO_{\mX}$-modules $V$ on $\mX$ with a quasi-nilpotent integrable logarithmic connection
\[\nabla\colon V\rightarrow V\otimes\Omega^1_{\mX/W}(\log\mD).\]
\end{enumerate}
\end{theorem}
\begin{remark} \label{rmk:26} We call the de Rham sheaf $(V,\nabla)$ associated to a crystal $\mE$ to be the \emph{realization of $\mE$ over $(\mX,\mD)$}. Kato's Theorem implies a logarithmic de Rham bundle is a realization of a logarithmic crystal if its connection is quasi-nilpotent. We sometimes simply call such a logarithmic de Rham sheaf $(V,\nabla)$ a logarithmic crystal over $(X,D)$.
\end{remark}

According \autoref{rmk:26}, we also write the logarithmic $F$-crystal as the triple $(V,\nabla,\mF)$.
\begin{corollary}
There is an equivalence between the following two categories
\begin{enumerate}
\item[$(a)$] the category of $F$-crystals $\mE$ on $((X,D)/W)^{\log}_{\rm crys}$,
\item[$(b)$] the category of triples $(V,\nabla,\Phi)$, where $V$ is vector bundle on $\mX$, $\nabla$ a quasi-nilpotent integrable logarithmic connection
\[\nabla\colon V\rightarrow V\otimes\Omega^1_{\mX/W}(\log\mD)\]
and $\Phi$ is an injection
\[\Phi\colon \varphi^*(V,\nabla)\hookrightarrow (V,\nabla).\]
\end{enumerate}
\end{corollary}

\begin{corollary}
Let $(V,\nabla,\Fil,\Phi)$ be a Fontaine-Faltings module. By forgetting the filtration, one gets an $F$-crystal over $(X,D)$.
\end{corollary}

\begin{remark}\label{rmk11}
For an $F$-crystal over $(X,D)$, consider its realization $(V,\nabla,\Phi)$ on the $p$-adic formal completion of $(\mX,\mD)$. The presence of the Frobenius structure forces the reductions modulo $\mathbb Z$ of the eigenvalues of the residue map would form a set stable under multiplication by $p$. In particular the eigenvalues are rational numbers. See \cite[7.2]{Ked22}.
\end{remark}

\subsection{$F$-isocrystals}
In this subsection, we also recall some basic definitions needed for this article, including those of convergent $F$-isocrystals, overconvergent $F$-isocrystals and convergent log-$F$-isocrystals from \cite[Definition 2.1, Definition 2.4 and Definition 7.1]{Ked22}.

Let $X$ be a proper smooth variety over $k$, $D$ be a normal crossing divisor in $X$ and $U = X-D$. We endow $X$ with the natural logarithmic structure induced by $D$, and simply write $(X,D)$ for the corresponding logarithmic scheme.


\subsubsection{overconvergent $F$-isocrystal}
Suppose there exists a lifting $\sigma\colon \mU\rightarrow\mU$ of the absolute Frobenius on
$U$. A \emph{convergent $F$-isocrystal} over $U$ is a de Rham bundle $\mE$ over the Raynaud generic fiber $\mU_K$ of the formal completion $\mU$ of $U$ along the special fiber $U$ together with an isomorphism $F\colon \sigma^*\mE\rightarrow\mE$ of de Rham bundles. Denote by \emph{$\FIsoc(U)$} the category of all convergent $F$-isocrystals over $U$. Up to canonical equivalence, this category does not depend on the choice of the lifting $\sigma$. In general, there may not exist a global lifting of the absolute Frobenius on $U$, but one can still define the category $\FIsoc(U)$ (see \cite[definition 2.1]{Ked22}). One way to do this is as follows: we can find local liftings of absolute Frobenius on $U$, define local categories by using these local liftings as above, and use the canonical equivalences between local categories to glue them into a global one.

A convergent $F$-isocrystal is called \emph{overconvergent} if it can be extended to a strict neighborhood of $\mU_K$ in $\mX_K$. Denote by \emph{$\FIsoc^\dagger (U)$} the category of all overconvergent $F$-isocrystal over $U$.

For each finite extension $L$ of $\bQ_p$ within $\overline{\bQ}_p$, let \emph{$\FIsoc^\dagger(U)_L$} denote the category of objects of $\FIsoc^\dagger(U)$ with a $\bQ_p$-linear action of $L$. Let \emph{$\FIsoc^\dagger(U)_{\overline\bQ_p}$} be the $2$-colimit of the category $\FIsoc^\dagger(U)_L$ over all finite extensions $L$ of $\bQ_p$ within $\overline\bQ_p$.

\subsubsection{characteristic polynomials of an overconvergent $F$-isocrystal}
Given an overconvergent $F$-isocrystal $\mE$ on $U$. For any closed point $x$ in $U$, the fiber $\mE_x$ of $\mE$ at $x$ carries an action of (geometric) Frobenius. We define the characteristic polynomial of $\mE$ at $x$ to be
\[P_x(\mE,t)=\det(1-Fr_x\cdot t\mid_{\mE_x}).\]

\subsubsection{convergent log-$F$-isocrystal}
A \emph{convergent log-$F$-isocrystal} is a logarithmic de Rham bundle over $\mX_K$ together with an isomorphism $F$ of logarithmic de Rham bundles similar as that in the definition of convergent $F$-isocrystal (see e.g. \cite[Definition 7.1]{Ked22}). For such objects, the residues of the underlying logarithmic isocrystal are automatically nilpotent. We denote by $\FIsoc^{\rm nilp}_{\log}(X,S)$ the category of all convergent log-$F$-isocrystals on the logarithmic pair $(X,S)$.

\begin{remark}
\begin{enumerate}
\item Under our assumption $X_K$ is proper, a convergent log-$F$-isocrystals can be algebraicalized to a vector bundle over $X_K$ together with an integral logarithmic connection and a parallel semilinear action.
\item To a logarithmic crystalline representation, we may attach an convergent log-$F$-isocrystal. For a logarithmic crystalline representation $\rho\colon \pi_1(U_K)\rightarrow \textrm{GL}_r(\mathbb{Z}_{p^f})$, according Faltings' definition of crystalline representation~\cite{Fal89}, there exists an attached logarithmic Fontaine-Faltings module $(V,\nabla,\Fil,\varphi,\iota)$\footnote{Faltings' original definition is for $\mathbb Z_p$-representations. It can be easily extended to $\mathbb Z_{p^f}$-representations by adding an endomorphism structures $\iota$ on the side of Fontaine-Faltings modules. More precisely, see \cite{LSZ19}.} Forgetting the filtration and tensoring $\mathbb Q_p$, one gets the attached convergent log-$F$-isocrystal $(V,\nabla,\varphi,\iota)_{\bQ_p}$.
\end{enumerate}
\end{remark}


\subsubsection{Trace of the Frobenius}
Let $(V,\nabla,\Fil,\varphi)$ be a logarithmic Fontaine-Faltings module over $(\mY,\mD_\mY)$.
For any closed point $x$ in $U_1$ with residue field $k'$, by the smoothness of $Y$, we can find a $\Spf(W)$-point $\widehat{x}$ in $\mU$ which lifts $x$. By restricting on $\widehat{x}$, we gets a Fontaine-Faltings module over this point, which is nothing just a finite generated free filtered $W(k')$-module $V_{\widehat x}$ together with a $\sigma$-semilinear isomorphism $F_{\widehat x}\colon \widetilde{V}_{\widehat x}\simeq V_{\widehat x}$, where $\widetilde{V}_{\widehat x} = \sum\limits_{\ell = a}^{b} \frac1{p^\ell}\Fil^\ell V_{\widehat x} \subset V_{\widehat x} \otimes \bQ_p$. By tensoring $\bQ_p$, one gets an $F$-isocrystal $(V_{\widehat{x}}\otimes \bQ_p,F_{\widehat{x}})$ over the finite field $k'$.

One can easily checks following result.
\begin{lemma} The $(V_{\widehat{x}}\otimes \bQ_p,F_{\widehat{x}})$ is isomorphic to the restriction of $\mE$ on $x$. In particular, the isomorphic class of $F$-isocrystal $(V_{\widehat{x}}\otimes \bQ_p,F_{\widehat{x}})$ does not depend on the choice of $\widehat{x}$.
\end{lemma}


\subsubsection{The dependence of the traces on the choices of the Frobenius structures}

\subsubsection{$F$-isocrystal over $k$ with coefficients}

Let $k$ be a finite field and Let $L$ be an algebraic extension of $\bQ_p$. Recall that the following are equivalent:
\begin{itemize}
\item an $F$-isocrystal over $k$ with coefficient in $L$ of rank $r$;
\item a free $W(k)[\frac1p]\otimes_{\bQ_p} L$-module of rank $r$ together with a $\sigma\otimes \id$-linear morphism
\[F\colon V\rightarrow V.\]
\item a $W(k)[\frac1p]$ vector space of rank $r[L:\bQ_p]$ endowed with a $\sigma$-semilinear isomorphism $F\colon V\rightarrow V$ and with an endomorphism structure
\[L\rightarrow \End(V,F).\]
\end{itemize}
In the following, we will always identify the three kinds of objects, and call them $F$-isocrystals over $k$ with coefficient $L$. Denote by $\FIsoc(k)_L$ the category of all $F$-isocrystals over $k$ with coefficient $L$.


\subsubsection{The $F$-isocrystal $\mE_{1/2}$.}

Since $p\geq3$, we may choose a square root $\sqrt{1-p}$ of $1-p$ in $\bQ_p$. Since $\bQ_{p^2}$ is an extension of $\bQ_p$ of degree $2$, we may find some $\zeta\in \bQ_{p^2}\setminus \bQ_p$ such that $\zeta^2 \in \bQ_{p}$. Thus $\sigma(\zeta)=-\zeta$, where $\sigma$ is the generator of the Galois group $\Gal(\bQ_{p^2}/\bQ_p)$, which is also the lifting of the absolute Frobenius map on $\bF_{p^2}$.

Let $V_{1/2}$ be a $\bQ_{p^2}\otimes_{\bQ_p}\bQ_{p^2}$-module of rank $1$ with basis $e$. Denote by $F_{1/2}$ a $\sigma\otimes\id$-linear endomorphism on $V$ given by
\[F_{1/2}(e) = (1\otimes 1 + \sqrt{1-p} \zeta\otimes \zeta^{-1}) e\]
Then
\[F_{1/2}^2(e) = (1\otimes 1 - \sqrt{1-p} \zeta\otimes \zeta^{-1})\cdot (1\otimes 1 + \sqrt{1-p} \zeta\otimes \zeta^{-1}) e = p.\]

According the equivalent relation, we get an $F$-isocrystal,denote by $\mE_{1/2}$, over $\bF_{p^2}$ with coefficient in $\bQ_{p^2}$.
Let $X$ be an varieties defined over $k$. Assume $k$ contains $\bF_{p^2}$. Then there is a structure morphism
\[f\colon X \rightarrow \Spec(\bF_{p^2}).\]
By pulling back $\mE_{1/2}$ along $f$ we get a constant overconvergent $F$-isocrystal of rank $1$ with coefficient in $\bQ_{p^2}$. By abusing notion, we still denote it by $\mE_{1/2}$.

\subsubsection{The cyclotomic $F$-isocrystal $\mathcal E_{cy}$}
\begin{definition}
Let $V$ be a $\Qp$-module of rank $1$ with basis $e$. Denote by $F$ the $\Qp$-linear endomorphism on $V$ by multiplying $p$. Then we get an $F$-isocrystal, denote by $\mathcal E_{cy}$, over $\Fp$.
\end{definition}

\begin{lemma}
$\mathcal E_{cy} = \mathcal E_{1/2}^{\otimes 2}$.
\end{lemma}


\subsubsection{The change of traces of Frobenius under twisting by $\mE_{1/2}$}

Let $k$ be a finite field with cardinality $p^h$. Let $L$ be an algebraic extension of $\bQ_{p}$.

Let $(V,F)$ be an $F$-isocrystal over $k$ with coefficient in $L$ of rank $r$, or equivalently, a free $W(k)[\frac1p]\otimes_{\bQ_p} L$-module of rank $r$ together with a $\sigma\otimes \id$-linear morphism
\[F\colon V\rightarrow V.\]
Then $F^h$ is a $W(k)[\frac1p]\otimes_{\bQ_p} L$-linear endomorphism on $V$. Denote by $P((V,F),t)$ the characteristic polynomial and by $\tr(V,F)$ the trace of $F^h$ acting on $V$.

\begin{lemma}
$P((V,F),t)\in L[t]$ and $tr(V,F)\in L$.
\end{lemma}
\begin{proof}
Let $e_1,\cdots,e_r$ be a system $W(k)[\frac1p]\otimes_{\bQ_p} L$-basis of $V$. Then $F$ can be represented as
\[F(e_1,\cdots,e_r) = (e_1,\cdots,e_r) A.\]
Thus
\[F^h (e_1,\cdots,e_r) = (e_1,\cdots,e_r) \underbrace{A\cdot A^{\sigma\otimes\id}\cdot A^{(\sigma\otimes\id)^2}\cdot \cdots \cdot A^{(\sigma\otimes\id)^{h-1}}}_{=: B}.\]
Since $B^{\sigma\otimes\id} = A^{-1} B A$, both $P((V,F),t)$ and $tr(V,F)$ are invariant under $\sigma\otimes\id$.
\end{proof}

Denote by $k'$ the field generated by $k$ and $\bF_{p^2}$ and denote by $L'$ the field generated by $L$ and $\bQ_{p^2}$. Then both
$R_1:=\bQ_{p^2}\otimes_{\bQ_p} \bQ_{p^2}$ and $R_2:=W(k)[\frac1p]\otimes_{\bQ_p} L$
can be viewed as a subring of $R:=W(k')[\frac1p]\otimes_{\bQ_p} L'$.
By extending the ring from $R_1$ and $R_2$ to $R$, we gets two objects in $\FIsoc(k')_{L'}$ from $(V,F)$ and $\mE_{1/2}$. We denote by
\[(V,F)\otimes \mE_{1/2}\]
their tensor product in the category $\FIsoc(k')_{L'}$.

\begin{lemma} Suppose $\bF_{p^2}\subseteq k$. Then $2\mid h$,
\[P((V,F)\otimes\mE_{1/2},t) = p^{rh/2}P((V,F),p^{-h/2}t) \quad \text{and} \quad \tr((V,F)\otimes\mE_{1/2}) = p^{h/2} \tr(V,F).\]
\end{lemma}

\begin{proof}
Clearly, the surjective $R$-module of $(V,F)\otimes \mE_{1/2}$ is
\[(V\otimes_{R_2}R)\otimes_R(V_{1/2}\otimes_{R_1}R)\]
which is free over $R$ of rank $r$ with generators
\[e_1':=(e_1\otimes1)\otimes (e\otimes1),\cdots,e'_r:=(e_r\otimes1)\otimes(e\otimes1).\] Denote $\eta:=1\otimes1+\sqrt{1-p}\zeta\otimes\zeta^{-1}$. Then
\[F(e'_1,\cdots e'_r) = (e'_1,\cdots,e'_r)\cdot A\eta.\]
The Lemma follows the following calculate:
\begin{equation*}
\begin{split}
F^h (e_1\otimes e,\cdots e_r\otimes e)
&= (e_1\otimes e,\cdots e_r\otimes e)\cdot A\eta\cdot (A\eta)^{\sigma\otimes\id} \cdot \cdots \cdot (A\eta)^{(\sigma\otimes\id)^{h-1}} \\
&= (e_1\otimes e,\cdots e_r\otimes e)\cdot p^{h/2}B\\
\end{split}
\end{equation*}
\end{proof}




\subsection{The convergence of parabolic Fontaine-Faltings modules.}

In this subsection, we construct the overconvergent $F$-isocrystals from parabolic Fontaine-Faltings modules.

\subsubsection{convergence of a logarithmic de Rham bundle over $(\mY_K,\mD_{\mY_K})$}
Recall that Kedlaya gave an equivalent functor \cite[6.4.1]{Ked07} from the category of convergent
logarithmic isocrystals\cite[6.1.7]{Ked07} to the category of convergent log de Rham bundles\cite[6.3.1]{Ked07}. So by restricting from the associated convergent logarithmic isocrystal, one gets an overconvergent isocrystal from a convergent logarithmic de Rham bundle. Back to our situation, we only need to show the convergence of the underlying logarithmic de Rham bundle of a logarithmic Fontaine-Faltings module. Before this, let us recall Kedlaya's definition of convergence.
\begin{definition}[{\cite[6.3.1]{Ked07}}]
A logarithmic de Rham bundle $(V,\nabla)$ over $(\mY_K,\mD_{\mY_K})$ is called \emph{convergent}, if there exists some strict neighborhood of $\mU_K$ in $\mY_K$, on which the restriction of $(V,\nabla)$ is overconvergent\footnote{See \cite[2.5.3 and 2.5.4]{Ked07}} along $\mZ_K$.
\end{definition}

\begin{remark}
According \cite[Proposition 2.5.6]{Ked07}, a logarithmic de Rham bundle over $(\mY_K,\mD_{\mY_K})$ is convergent if and only if for any $\eta\in[0,1)$, there exists a sufficient small strict neighborhood of $\mU_K$ in $\mY_K$, on which the restriction is $\eta$-convergent\footnote{See the explicit definition for $\eta$-convergent in \cite[Definition 2.4.2]{Ked07}}.
\end{remark}

\subsubsection{Generic fiber of a logarithmic de Rham bundle over $(\mY,\mD_\mY)$}

Let $(V,\nabla)$ be logarithmic de Rham bundle over $(\mY,\mD_\mY)$. By restriction on the Raynaud generic fiber, one gets a logarithmic de Rham bundle $(\mY_K,\mD_{\mY_K})$, which we will simply call \emph{the generic fiber of $(V,\nabla)$}, and denote by $(V_K,\nabla_K)$.

\begin{lemma}
\label{thm_MF_qnilp}
Let $(V,\nabla,\Fil,\varphi)$ be a parabolic Fontaine-Faltings module over $(\mY,\mZ)$. For any $\alpha \in \mathbb Q^n$, the generic fiber of the logarithmic de Rham bundle $(V_\alpha,\nabla_\alpha)$ is $\eta$-convergent for all $\eta\in[0,1)$.
\end{lemma}
\begin{proof}
By the definition, the filtration in any (parabolic) Fontaine-Faltings module has level contained in $[a,b]$with $b-a\leq p-2$. The grading structure in the corresponding graded Higgs bundle $(E,\theta)$ has level contained in $[0,p-2]$. in other words,
there exists graded decomposition $E=\oplus_{i=0}^{p-2} E_i$ such that
the Higgs field is a sum of maps $E_i\rightarrow E_{i+1}$ where $i$ run through $\{0,1,\cdots,p-3\}$. We consider the modulo $p$-reduction of the connection in the Fontaine-Faltings module, which comes from the modulo $p$ reduction of the graded Higgs bundle under the inverse Cartier functor. From the explicit construction of inverse Cartier functor\footnote{Note that the inverse Cartier functor $C_1^{-1}$ (the characteristic $p$ case) is introduced in the seminal work of Ogus-Vologodsky \cite{OgVo07}. See also \cite{LSZ19}.}, one has $\left(\nabla_{\partial}\right)^{p-1}\equiv 0\pmod{p}$. Thus the Lemma follows the definition of $\eta$-convergent in \cite[Definition 2.4.2]{Ked07} immediately.
\end{proof}

\begin{corollary} \label{thm_logFF2ConvLogdR} Let $(V,\nabla,\Fil,\varphi)$ be a parabolic Fontaine-Faltings module over $(\mY,\mZ)$. For any $\alpha\in\mathbb Q^n$, the generic fiber of the logarithmic de Rham bundle $(V_\alpha,\nabla_\alpha)$ is convergent.
\end{corollary}

\begin{proof}
The convergence follows \autoref{thm_MF_qnilp}, by Kedlaya's criterion \cite[2.5.6]{Ked07}.
\end{proof}
Together with Kedlaya's equivalent functor \cite[6.4.1]{Ked07}, we get functors from the category logarithmic Fontaine-Faltings modules $(\mY,\mD_\mY)$ to the category convergent logarithmic isocrystal over $(Y_1,D_1)$ indexed by $\alpha\in \mathbb Q^n$.
\begin{equation}\label{eq_logFF2logIsoc}
\left\{
\begin{array}{lll}
\text{parabolic Fontaine-Faltings}\\
\text{modules over $(\mY,\mD_{\mY})$}\\
\end{array}
\right\}
\xrightarrow{\quad\quad}
\left\{{\text{
convergent logarithmic
} \atop \text{
isocrystal over $(Y_1,D_{Y_1})$
}}\right\}
\end{equation}

Generally, we get a Frobenius structure on these convergent logarithmic isocrystal over $(Y_1,D_{Y_1})$.
\begin{proposition} \label{thm_paraFF2overconvFIsoc}
Let $(V,\nabla,\Fil,\varphi)$ be a parabolic Fontaine-Faltings module over $(\mY,\mD_\mY)$. Let $\{\mE_\alpha\}$ be the associated convergent logarithmic isocrystals over $(Y_1,D_1)$ given in \eqref{eq_logFF2logIsoc}. Then
\begin{enumerate}
\item[$(1)$] the de Rham bundles $F^*(\widetilde{V}_\alpha,\widetilde{\nabla}_\alpha)$ are convergent for all $\alpha\in\bQ^n$, and they have common restriction on the open subset $\mathcal Y_K^\circ$.
\item[$(2)$] The Frobenius structure in parabolic Fontaine-Faltings module induces an natural injective morphism of logarithmic de Rham bundles over $(\mY,\mD_\mY)$
\[\varphi\colon F^*(\widetilde{V}_0,\widetilde{\nabla}_0)\hookrightarrow (V_0,\nabla_0).\]
\item[$(3)$] After restricting $\mathcal E_\alpha$ onto the open subset $\mathcal Y_K^\circ$, one gets an overconvergent $F$-isocrystal over $(U_1,Y_1)$. In summary, we gets a functor
\begin{equation}\label{eq:logFF2logFIsoc}
\left\{
\begin{array}{lll}
\text{parabolic Fontaine-Faltings}\\
\text{modules over $(\mY,\mD_{\mY})$}\\
\end{array}
\right\}
\xrightarrow{\quad\quad}
\left\{{\text{
overconvergent
} \atop \text{
$F$-isocrystal over $(U_1,Y_1)$
}}\right\}
\end{equation}
\end{enumerate}

\end{proposition}
\begin{proof}
Clearly, (1) and (3) follows (2) directly. We show (2) as follows.

Since one always has injection $(V_0,\nabla_0,\Fil_0)\hookrightarrow (V,\nabla,\Fil)$ of filtered parabolic de Rham bundles, the front one endowed with trivial parabolic structure. After taking parabolic version of Faltings tilde functor and inverse Cartier functor, one gets an injective morphism between parabolic de Rham bundles
\[\mF^*(\widetilde{V}_0,\widetilde{\nabla}_0) \hookrightarrow \mF^*(\widetilde V,\widetilde\nabla).\]
Then composing with the Frobenius structure in the Fontaine-Faltings module, one gets the desired injective morphism.
\end{proof}

\begin{remark}
Due to the existence of the parabolic structure, the Frobenius map in (2) is not isomorphism in general. But if the parabolic structure is trivial(in other word, for a logarithmic Fontaine-Faltings module), we will indeed get a convergent logarithmic $F$-isocrystal over $(U_1,Y_1)$.
\end{remark}

We now have endomorphism structures involved.
\begin{corollary} \label{FFM2Fisoc} By forgetting the filtration, and then restricting on the Raynaud generic fiber, one gets the following functor
\begin{equation*}
\left\{
\begin{array}{lll}
\text{parabolic Fontaine-Faltings}\\
\text{modules over $(\mY,\mD_{\mY})/S$ with}\\
\text{$\bZ_{p^f}$-endomorphism structures}\\
\end{array}
\right\}
\xrightarrow{\quad\quad}
\left\{{\text{
overconvergent $F$-isocrystal over
} \atop \text{
$(Y_1,D_{Y_1})$ with coefficient in $\bQ_{p^f}$
}}\right\}
\end{equation*}
\end{corollary}


\subsection{Overconvergent $F$-isocrystals on the projective line}

\subsubsection{Overconvergent $F$-isocrystals with given exponents}

Denote by \emph{$\FIsoch(k)_{\bQ_{p^f}}$} the set of all rank-$2$ overconvergent $F$-isocrystal $\mE$ over $(\bP^1_{k},\{0,1,\lambda,\infty\})$ with coefficients in $\bQ_{p^f}$ such that the exponents along $0,1,\lambda$ are integers and the exponents along $\infty$ are half integers.

Let $k'$ be a field extension of $k$ containing $\bF_{p^f}$. For any $M\in \MFh(W(k'))_{\bZ_{p^f}}$, by \autoref{FFM2Fisoc}, we get an overconvergent $F$-isocrystal $\mE_M$ over $(\mathbb P^1_{k'},\{0,1,\lambda,\infty\})/k'$ endowed with an $\bQ_{p^f}$-endomorphism structure with the same exponents as $M$ (up to modulo $\bZ$). Thus $\mE_M\in \FIsoch(k')_{\bQ_{p^f}}$. This give us a
natural map
\begin{equation} \label{eq_MF2Fisoc}
\MFh(W(k'))_{\bZ_{p^f}} \rightarrow \FIsoch(k')_{\bQ_{p^f}}.
\end{equation}

\subsubsection{An equivalence relation on $\FIsoch(k')_{\bQ_{p^f}}$}

Let $k'$ be a field extension of $k$ containing $\bF_{p^f}$.

\begin{definition}
Let $\mE$ and $\mE'\in\FIsoch(k')_{\bQ_{p^f}}$. We call they are \emph{differed by a constant (over $k'$)}, if there exists an $F$-isocrystal $\mE^\circ$ over $k'$ with coefficient in $\bQ_{p^f}$ of rank $1$ such that
\[\mE' = \mE \otimes \mE^\circ.\]
Differed by a constant is an equivalent relation on the set
\[\FIsoch(k')_{\bQ_{p^f}}.\]
Denote by $[\FIsoch(k')_{\bQ_{p^f}}]$ the set of all equivalent classes.
\end{definition}

Denote by $\FIsoch(k')_{\bQ_{p^f}}^{\rm triv}$ the subset of $\FIsoch(k')_{\bQ_{p^f}}$ with trivial determinant, and denote by $[\FIsoch(k')_{\bQ_{p^f}}^{\rm triv}]$ the image of $\FIsoch(k')_{\bQ_{p^f}}^{\rm triv}$ in $[\FIsoch(k')_{\bQ_{p^f}}]$.

\begin{lemma} \label{mthm_Fisok2Fisokn}
Let $\mE$ and $\mE'\in\FIsoch(k')_{\bQ_{p^f}}$. Then for any $n\geq1$, they are differed by a constant over $k_n'$ if and only if they are differed by a constant over $k'$. Thus one has the natural injection
\[[\FIsoch(k')_{\bQ_{p^f}}] \hookrightarrow [\FIsoch(k'_n)_{\bQ_{p^f}}].\]
\end{lemma}

\begin{lemma} \label{mthm_FF2Isoc}
The map \eqref{eq_MF2Fisoc} induces an injection between the sets of equivalence classes
\[[\MFh(W(k'))_{\bZ_{p^f}}] \hookrightarrow [\FIsoch(k')_{\bQ_{p^f}}].\]
\end{lemma}

\begin{proof}
This follows the facts that the modulo $p$ reductions of the de Rham terms appeared in the Higgs-de Rham flow associating to any object in $\MFhf(W(k'))$ are all stable and the Hodge filtration is unique.
\end{proof}

\begin{lemma} \label{mthm_FF2Isoc_classes}
Assume $2\mid f$. Then the map in \autoref{mthm_FF2Isoc} (replacing $k'$ with $k'_2$) induces following injection
\[[\MFh(W(k_2'))_{\bZ_{p^f}}^{\rm cy}] \hookrightarrow [\FIsoch(k'_2)_{\bQ_{p^f}}^{\rm triv}].\]
\end{lemma}

\begin{proof}
For any $M\in \MFh(W(k'))_{\bZ_{p^f}}^{\rm cy}$, the $F$-isocrystal $\mE_M\otimes \mE_{1/2}^{-1}$ has trivial determinant.
\end{proof}

Together with natural mapping from periodic Higgs bundles to Fontaine-Faltings modules, we get the following result.
\begin{corollary} \label{thm_PHIG_k_to_F_Isoc_k}
Assume $\lambda\in W(k)$ is supersingular. Running Higgs-de Rham flow induces a natural injection
\begin{equation} \label{thm_Higgs2Fisoc}
\PHighf(k') \hookrightarrow [\FIsoch(k')_{\bQ_{p^f}}].
\end{equation}
\end{corollary}

\subsubsection{The Frobenius action on $\FIsochf(k)$}

\begin{proposition} \label{mthm_Higgs2FIsoc_Frob_Preserved}
Assume $\lambda\in W(k)$ is supersingular.
The injection in \eqref{thm_Higgs2Fisoc} is preserved by the actions of $\Frob_k$ on both sides.
\end{proposition}

\begin{proof}
Let $(E,\theta)\in \PHighf(k')$ with corresponding $F$-isocrystal $\mE\in \FIsoch(k')_{\bQ_{p^f}}$. Assume it initials the following periodic Higgs-de Rham flow
\[\Flow = \{(E,\theta)_0=(E,\theta),(V,\nabla,\Fil)_0,(E,\theta)_1,(V,\nabla,\Fil)_1,\cdots\}.\]
Since $\lambda$ is supersingular, it lifts uniquely (up to a constant) to an $f$-periodic flow in $\HDFh(W(k'))$.
\[\widehat\Flow = (\widehat E,\widehat\theta)_0,(\widehat V,\widehat\nabla,\widehat\Fil)_0,(\widehat E,\widehat\theta)_1,(\widehat V,\widehat\nabla,\widehat\Fil)_1,\cdots.\]
Then by forgetting the Frobenius structure in Fontaine-Faltings module, one gets the underlying filtered de Rham bundle
\[(\widehat V,\widehat\nabla,\widehat\Fil) = (\widehat V,\widehat\nabla,\widehat\Fil)_0\oplus (\widehat V,\widehat\nabla,\widehat\Fil)_1 \oplus \cdots \oplus (\widehat V,\widehat\nabla,\widehat\Fil)_{f-1}\]
with a $\bZ_{p^f}$-endomorphism structure $\iota$. We note that due to the existence of the Frobenius structure we have
\[\Gr((\widehat V,\widehat\nabla,\widehat\Fil)_{f-1})\simeq (\widehat E,\widehat\theta)_0.\]
Thus we can reconstruct $(E,\theta)$ from the $0$-th eigen component $(\widehat V,\widehat\nabla)_{f-1}$.

By the construction of $\mE$, its underlying de Rham bundle is just the generic fiber $(\widehat V_K,\widehat\nabla_K)$. We can also taking the $0$-th eigen component, which is just $(\widehat V_K,\widehat\nabla_K)_{f-1}$. Since the modulo $p$-reduction of $(\widehat V,\widehat\nabla)_{f-1}$ is stable, up to isomorphism $(\widehat V_K,\widehat\nabla_K)$ has a unique integral extension, which is just $(\widehat V,\widehat\nabla)_{f-1}$.

In summary, from $\mE$ we can reconstruct the Higgs bundle as follows:
\begin{itemize}
\item find the $0$-th eigen component of underlying de Rham bundle of $\mE$;
\item find an integral extension of the de Rham bundle in the first step.
\item take grading (the Hodge filtration is unique, due to \autoref{thm_ClassfyR2PdE}) and modulo $p$, one gets the original Higgs bundle $(E,\theta)$.
\end{itemize}
Now starting from $\Frob_k(\mE)$ and following steps as above, one then get the Higgs bundle $\Frob_k(E,\theta)$.
\end{proof}

\begin{remark}
Due to the existence of the Hodge filtration, there is no Frobenius action on the intermediate sets $[\PHDFhf(W(k_f))]$ and $[\MFh(W(k))_{\bZ_{p^f}}]$. If one forgets the Hodge filtrations in Fontaine-Faltings modules, then he will get some ``parabolic $F$-crystals'', on which there should exist an action of $\Frob_k$. And the natural maps between them should preserves the Frobenius structure.
\end{remark}

\newpage

\section{\bf $p$-to-$\ell$ companion} \label{sec_main_p_to_l_bijections}

\subsection{$\ell$-adic local systems}
\subsubsection{The character $\chi_{1/2}$}
Recall that the Galois group $G_{\bF_{p}}=\Gal(\overline{\bF}_{p}/\bF_{p})$ is isomorphic to $\widehat{\bZ}$ with a topological generator $\sigma$. Denote by
\[\chi_{1/2}\]
the $\bQ_\ell$-character of the subgroup $G_{\bF_{p^2}}=\Gal(\overline{\bF}_{p}/\bF_{p^2})$
given by
\[\chi_{1/2}(\sigma^{2}) = p.\]
Clearly $\chi_{1/2}^2=\bQ_\ell(1)$ is just the cyclotomic character.

\subsubsection{The change of characteristic polynomials under twisting by the character $\chi_{1/2}$}

Let $k$ be a finite field with cardinality $p^h$ containing $\bF_{p^2}$. The character $\chi_{1/2}$ can be restricted on the absolute Galois group $G_k$ of $k$. Let $\bL$ be a $\Qbar_\ell$-representation of the absolute Galois group $G_k$ of $k$.

Denote by $P(\bL,t)$ the characteristic polynomial and by $\tr(\bL)$ the trace of $\sigma^h$ acting on $\bL$. Then
\begin{lemma} let $\bL$ be a local system of rank $r$. Then
\[P(\bL\otimes \chi_{1/2},t) = p^{rh/2}P(\bL,p^{-h/2}t) \quad \text{and} \quad \tr(\bL\otimes\chi) = p^{h/2}\tr(\bL).\]
\end{lemma}

\newpage

\subsection{$\ell$-adic local systems over punctured projective line and Yu's formula}

\subsubsection{$\ell$-adic local systems over punctured projective line}

Denote by $\Loch(k)$ the set of isomorphic classes of rank $2$ tame $\ell$-adic local systems $\bL$ on the punctured projective line $\bP^1_{k} \setminus \{0,1,\lambda,\infty\}$ with following prescribed eigenvalues of the local monodromies:
\begin{itemize}
\item the local monodromies around $\{0,1,\lambda\}$ is unipotent;
\item the local monodromy around $\infty$ is quasi-unipotent and has double eigenvalue $-1$.
\end{itemize}

\subsubsection{An equivalence relation on $\Loch(k)$}

\begin{definition} Let $\bL$ and $\bL' \in \Loch(k)$. We call they are differed by a character, if there exists a character $\chi$ of the absolute Galois group $\Gal(\overline{k}/k)$ such that
\[\bL' = \bL \otimes \chi.\]
Denote by $[\Loch(k)]$ the set of all equivalent classes.
\end{definition}

\begin{lemma} All local systems in $\Loch(k)$ are geometrically irreducible.
\end{lemma}
\begin{proof} Suppose not. Then there exists some rank-1 sub local system $\bW$ of the geometric part of some $\bL$ in $\Loch(k)$. Then the local monodromy matrix of $\bW$ around $\{0,1,\lambda\}$ are all equal to 1, and around $\{\infty \}$ is $-1$. As the four generators $\{\gamma_0,\gamma_1,\gamma_\lambda,\gamma_\infty \}$ of the geometric fundamental group of $\bP^1 \setminus \{0,1,\infty,\lambda \}$ around the 4 punctures have one relation
\[\gamma_0 \cdot \gamma_1,\cdot \gamma_\lambda\cdot \gamma_\infty=e\]
we obtain\[1 \cdot 1 \cdot 1 \cdot (-1)=1,\]
which leads a contradiction.
\end{proof}

\begin{lemma} Let $\bL$ and $\bL' \in \Loch(k)$. Then $\bL$ and $\bL'$ are differed by a character if and only if they have isomorphic geometric parts (i.e., restrictions on $U_{\overline{k}}$).
\end{lemma}

\begin{proof} The ``only if'' part is trivial. Suppose there have the same geometric parts. Let $\rho$ and $\rho'$ be the representations of $\pi_1(U_k)$ associated to two equivalent local systems $\bL$ and $\bL'$. Then by assumption
\[\rho\mid_{\pi_1(U_{\overline k})} = \rho'\mid_{\pi_1(U_{\overline k})}.\]
Assume $\gamma \in \Gal(\overline{k}/k)$. Then for any two lifting $\widehat\gamma$ and $\widehat{\gamma}'$ of $\gamma$ in $\pi_1(\bP^1_k\setminus\{0,1,\lambda,\infty\})$, $\widehat\gamma^{-1}\cdot \widehat\gamma' \in \pi_1(U_{\overline{k}})$. So $\rho(\widehat\gamma^{-1}\cdot \widehat\gamma') = \rho'(\widehat\gamma^{-1}\cdot \widehat\gamma')$. This implies
\[\rho'(\widehat\gamma)\cdot \rho(\widehat{\gamma})^{-1} = \rho'(\widehat\gamma')\cdot \rho(\widehat{\gamma}')^{-1}\]
In other words, the value $\rho'(\widehat\gamma')\cdot \rho(\widehat{\gamma}')^{-1}$ does not depend on the choice of the lifting, denote it by $\chi(\gamma)$. We only need to show that $\chi$ is a character of $\Gal(\overline{k}/k)$.

Since $\pi_1(U_{\overline{k}})$ is a normal subgroup of $\pi_1(U_k)$, $\widehat\gamma^{-1}g\widehat\gamma \in \pi_1(U_{\overline{k}})$ for any $g\in \pi_1(U_{\overline{k}})$. So $\rho'(\widehat\gamma^{-1}g\widehat\gamma) = \rho(\widehat\gamma^{-1}g\widehat\gamma)$ and $\rho'(g)=\rho(g)$. Thus
\[\chi(\gamma) \rho(g) = \rho(g) \chi(\gamma).\]
By Schur's lemma,
\begin{equation}\label{equ:schur}
\chi(\gamma)\in \End(\rho\mid_{\pi_1(U_{\overline{k}})}) \cong \Qbar_\ell.
\end{equation}
Next, we only need to show $\chi$ is multiplicative. For any two elements $\gamma_1$ and $\gamma_2$ in $\Gal(\overline{k}/k)$, choose liftings $\widehat{\gamma}_1$ and $\widehat{\gamma}_2$ respectively. Then
\begin{equation*}
\begin{split}
\chi(\gamma_1\gamma_2) := & \rho'(\gamma_1\gamma_2)\cdot \rho(\gamma_1\gamma_2)^{-1} \\
= & \rho'(\gamma_1)\cdot\chi(\gamma_2)\cdot\rho(\gamma_1)^{-1} \\
\overset{\eqref{equ:schur}}{=}& \rho'(\gamma_1) \cdot\rho(\gamma_1)^{-1} \cdot \chi(\gamma_2)\\
=&\chi(\gamma_1)\cdot\chi(\gamma_2). \qedhere
\end{split}
\end{equation*}
\end{proof}

\begin{corollary} \label{mthm_LocSys_equivRelation}
One has an injection
\begin{equation*}
[\Loch(k)] \hookrightarrow \Loch(\overline{k})^{\Frob_k}.
\end{equation*}
\end{corollary}

\subsubsection{Yu's formula}
By Drinfeld and Deligne the set of $\Loch(\overline{k})^{\Frob_k}$ is finite. However, it is not clear that how the number depends on $q:=\#k$ precisely. Very recently Hongjie Yu \cite{Yu23} has solved Deligne's conjecture on counting $\ell$-adic local systems in terms of parabolic Higgs bundles. His general theorem applying to our special case turns out:
\begin{theorem}[Hongjie Yu{\cite{Yu23}}] \label{thm_Yu}
\[\#\High (k) = \# \Loch(\overline{k})^{\Frob_k}.\]
\end{theorem}
\begin{remark}
When $\lambda$ is supersingular, we will show this numeric Simpson correspondence in fact underlies a genuine Simpson correspondence, see \autoref{thm_genuineSimCorr}.
\end{remark}

\newpage

\subsection{Abe's theorem on Deligne's $p$-to-$\ell$ companion}

In this subsection, we choose a prime $\ell\not=p$ and fix an isomorphism $\phi\colon \overline {\bQ}_p\simeq \overline {\bQ}_\ell$. Our aim is to construct a natural injection
\[\High(k) \hookrightarrow \Loch(\overline{k})^{\Frob_k}.\]
The construction is diagrammatically sketched below:
\begin{equation*} \footnotesize
\xymatrix@R=1cm@C=1cm@M=4mm{
[\PHDFhf(W(k))]
\ar@{>->}[d]^{\autoref{mthm_PHDFf2MFf}}_{k\subseteq k' \atop \bF_{p^f}\subseteq k'} \ar[r]^{\autoref{mthm_PHDFf2PHIGf}}_{1:1}
& \PHighf(W(k))
\ar[r]^-{\lambda=\text{supersingular}}_-{1:1\atop \autoref{mthm_PHIGk2PHIGW}}
& \PHighf(k)
\ar[r]_{1:1 \atop \autoref{mthm_PHIGf2PHIG}}^{(\#k+1)!\mid f}
& \PHigh(k)
\ar[d]_{\lambda=\text{supersingular} \atop \autoref{mthm_HIG2PHIG}}^{1:1} \\
%----------------------------------------------
[\MFh(W(k'))_{\bZ_{p^f}}]
\ar@{>->}[rr]^{\autoref{mthm_FF2Isoc}}
\ar@{>->}[d]_{\autoref{thm_cyclDeterminant_FFM}}
\ar@{>->}[dr]
&& [\FIsoch(k')_{\bQ_{p^f}}]
\ar@{>->}[dd]^{\autoref{mthm_Fisok2Fisokn}}
& \High(k)
\ar@{>-->}[dd]^-{\exists}_-{\autoref{mthm_HIGk2Lock}} \\
%---------------------------------------------
[\MFh(W(k_2'))_{\bZ_{p^f}}^{\rm cy}]
\ar@{>->}[d]^{\autoref{mthm_FF2Isoc_classes}}
\ar@{>->}[r]
& [\MFh(W(k_2'))_{\bZ_{p^f}}]
\ar@{>->}[dr]^{\autoref{mthm_FF2Isoc}}
&\\
%---------------------------------------------
[\FIsoch(k'_2)_{\bQ_{p^f}}^{\mathrm{triv}}]
\ar@{>->}[rr] \ar@{>->}[d]_{p\text{-to-}\ell}^{\autoref{mthm_FIsoc2Loc}}
&
&[\FIsoch(k'_2)_{\bQ_{p^f}}]
& \Loch(\overline{k})^{\Frob_k}
\ar@{>->}[d]^{k\subset k'_2}\\
%-------------------------------------------
[\Loch(k'_2)]
\ar@{>->}[rrr]^{\autoref{mthm_LocSys_equivRelation}}
&
&
& \Loch(\overline{k})^{\Frob_{k'_2}} \\
}
\end{equation*}

\subsubsection{$p$-to-$\ell$ companion over projective line}

By applying $p$-to-$\ell$ companion, which was conjectured by Deligne and was proven by Abe in \cite{Abe18}, to an overconvergent $F$-isocrystal $\mE$ in $\FIsoch(k'_2)_{\bQ_{p^f}}^{\rm triv}$, then one gets a rank-2 $\ell$-adic irreducible local system $\bL_{\mE}$ on $(\bP_{k_f}^1\setminus \{0,1,\lambda,\infty\})$ with trivial determinant.

\begin{proposition} \label{mthm_FIsoc2Loc}
The local system $\bL_\mE$ is contained in $\Loch(k'_2)$. The $p$-to-$\ell$ companion induces us an injection
\[[\FIsoch(k'_2)_{\bQ_{p^f}}^{\rm triv}] \rightarrow [\Loch(k'_2)].\]
\end{proposition}

\begin{proof}
By \cite[ Corollary 2.5.4.]{Ked07}, the local system $\bL_\mE$ is docile along $0,1,\infty$, since $\mE$ has unipotent monodromy along $0,1,\infty$.

By tensoring the $F$-isocrystal and $\ell$-adic character associated to the rank-$1$ parabolic Fontaine-Faltings module $\mO(1/2(\infty)-1/2(0))$ respectively, one can shift the parabolic structure from $\infty$ to $0$. Thus by direct calculation, we get the eigenvalue of the monodromy is $-1$ and the exponents of the residue is $1/2$.

Next, we need to show the injectivity. Suppose two $F$-isocrystals $\mE,\mE'\in \FIsoch(k'_2)_{\bQ_{p^f}}^{\rm triv}$ have equivalent $\ell$-adic companions. In other word, $\bL_\mE$ and $\bL_{\mE'}$ are differed by a character of $\Gal(\bark/k'_2)$. Hence there exists a finite extension $k_2''$ of $k_2'$ such that the base change of $\bL_\mE$ and $\bL_{\mE'}$ from $k_2'$ to $k_2''$ are coincide with each other. Now by the bijection of $p$-to-$\ell$ companion, one gets
\[\mE \simeq \mE' \in \FIsoch(k''_2)_{\bQ_{p^f}}^{\rm triv}.\]
In general, one cannot descent this isomorphism to an isomorphism in $\FIsoch(k''_2)_{\bQ_{p^f}}^{\rm triv}$, this is because the Frobenius structure is non-linear. But they underlying overconvergent isocrystals are isomorphic to each other, as these overconvergent isocrystals are irreducible. Thus they Frobenius structure is differed by a constant.
\end{proof}

\begin{proposition} \label{mthm_HIGk2Lock}
Suppose $\lambda$ is supersingular, by composing the morphisms in diagram ahead this section, we construct an injective map
\[\High(k) \hookrightarrow \Loch(\overline{k})^{\Frob_{k'_2}}.\]
More finely, the image of this map is contained in $\Loch(\overline{k})^{\Frob_{k}}$, and one get an injection
\[\High(k) \hookrightarrow \Loch(\overline{k})^{\Frob_{k}}.\]
\end{proposition}
\begin{proof} Let $(E,\theta)\in \High(k)$ and let $\mE$ be the associated $F$-isocrystal in $\FIsoch(k'_2)^{\rm triv}_{\bQ_{p^f}}$. By \autoref{mthm_Higgs2FIsoc_Frob_Preserved}, $\mE$ is invariant under the action of $\Frob_{k}$. Since the $p$-to-$\ell$ companion is preserved by $\Frob_k$, the corresponding $\ell$-adic local system is also invariant under the action of $\Frob_k$.
\end{proof}



By Yu's formula \autoref{thm_Yu} for numeric Simpson correspondence, we gets a genuine Simpson correspondence.
\begin{corollary} \label{thm_genuineSimCorr}
Assume that $\lambda$ is supersingular. Then the injection
\[\High(k) \rightarrow \Loch(\overline{k})^{\Frob_{k}}\]
is actual a bijection.
\end{corollary}

\begin{corollary}
The trace field of any local system $\bL\in \Loch(\overline{k})^{\Frob_{k}}$ are unramified above $p$.
\end{corollary}

\newpage

\section{\bf Constructing family of abelian varieties over $\Fq$ via Langlands correspondence and lifting Hodge filtration of relative differential forms to characteristic zero} \label{sec_main_family_over_k}

\subsection{Drinfeld theorem on Langlands correspondence over function field of characteristic $p$}

\subsubsection{Drinfeld theorem}

A key ingredient in the proof of our main results is the following \autoref{thm_Drinfeld_GL2}, which is a byproduct of Drinfeld's first work on the Langlands correspondence for $\mathrm{GL}_2$ \cite{Dri77}. We first record a setup.
\begin{setup}\label{setup:curve_finite_field}
Let $p$ be a prime number and let $q=p^a$. Let $C/\Fq$ be a smooth, affine, geometrically irreducible curve with smooth compactification $\bar{C}$. Let $Z:=\bar{C}\setminus C$ be the reduced complementary divisor.
\end{setup}

\begin{theorem}\label{thm_Drinfeld_GL2}(Drinfeld) Notation as in \autoref{setup:curve_finite_field} and let $\bL$ be a rank 2 irreducible $\Qlbar$ sheaf on $C$ with determinant $\Qlbar(1)$. Suppose $\bL$ has infinite local monodromy around some point at $\infty\in Z$. Then $\bL$ comes from a family of abelian varieties in the following sense: let $\EK$ be the field generated by the Frobenius traces of $\bL$ and suppose $[\EK:\bQ]=h$. Then there exists an abelian scheme
\[
\pi_C\colon A_{C}\rightarrow C
\]
of dimension $h$ and an isomorphism $\EK\cong \End_{C}(A_C)\otimes\bQ$, realizing $A_C$ as a $\GL_2(\EK)$-type abelian scheme, such that $\bL$ occurs as a summand of $R^1(\pi_C)_*\Qlbar$. Moreover, $A_{C}\rightarrow C$ is totally degenerate around $\infty$.
\end{theorem}

\subsubsection{Application of Drinfeld's theorem in our case}

Given a local system $\bL\in \Loch(k_2')$ which is fixed by the Frobenius $\Frob_k$ and has cyclotomic determinant. Then the restriction of $\bL$ to the geometric fundamental group is irreducible with infinite local monodromy at least on one puncture. Denote by $\EK$ the trace field of $\bL$. By applying Drinfeld's \autoref{thm_Drinfeld_GL2} to $\bL$, there exists an abelian scheme of $\GL_2(\EK)$-type
\[\pi \colon A \to U_{k_2'}\]
over the punctured projective line at $\{0,1,\infty,\lambda\}$ and with $\bL$ being an eigen summand of the associated local system.

\begin{lemma}
Let $\bV\coloneqq R^1_\et \pi_* \Qbar_{\ell}= \bigoplus_{i=1}^g \bL_i$ be the eigen decomposition with $\bL_1=\bL$. Then all $\bL_i$'s are contained in $\Loch(k_2')$ and are fixed by the Frobenius $\Frob_k$ with cyclotomic character.
\end{lemma}
\begin{proof}
This is because all of them are conjugate to $\bL$.
\end{proof}

\subsubsection{Eigen decomposition of the crystalline cohomology} \label{sec_subsub_family_to_rank2_cystal}
Denote by $\bD(\pi)=R^1_\crys \pi_* \mO_{A,crys}$ the Dieudonn\'e crystal over $\bP^1_{k_2'}\setminus\{0,1,\infty,\lambda\}$ attached to $\pi$, on which the ring $\mO_\EK$ naturally acts on $\bD(\pi)$. Forgetting the Verschiebung structure in the Dieudonn\'e crystal and tensoring $\bQ$, then one get a convergent $F$-isocrystal over $U_{k_2'}$, which is overconvergent by \cite[3.17]{Tri08}. One can decomposes it via the action of $\mO_\EK$ after extending the coefficient to $\bZ_{p^f}$. In particular, we have decomposition
\[\bD(\pi) = \bigoplus_{i=1}^{g}\mE_i.\]

Since $\{\mE_i\}$ and $\{\bL_i\}$ are all coming from the same family, they are all companion to each other under the $p$-to-$\ell$ companion. Thus by the discuss of the bijections in \autoref{sec_main_p_to_l_bijections}, we have following result.

\begin{lemma} \label{thm_p_to_l_from_family}
All eigen components $\mE_i$ come from Fontaine-Faltings modules in $\MFh(W(k_2'))^{\rm cy}_{\bZ_{p^f}}$, in the following sense, for each $i$ there exists an Fontaine-Faltings module $M_i$ such that its associated $F$-isocrystal $\mE_{M_i}$ coincides with $\mE_i$ over the $U_{k_2'}$.
\end{lemma}

Forgetting the Verschiebung structure and taking the realization of $\bD(\pi)$ over $U_{W(k_2')}=\bP^1_{W(k_2')}\setminus\{0,1,\infty,\lambda\}$, one gets a de Rham bundle with a Frobenius structure
\[(V,\nabla,\Phi)_U= \Big(\bD(\pi)\Big)(U_{W(k_2')}).\]
According \autoref{thm_p_to_l_from_family}, there exists Fontaine-Faltings module $(V,\nabla,\Fil,\Phi)^{\rm FF}$ such that
\[(V,\nabla,\Phi)_U \otimes \bQ \cong (V,\nabla,\Phi)^{\rm FF} \mid_{U} \otimes \bQ.\]
After extending the coefficient from $\bZ_p$ to $\bZ_{p^f}$ on both sides, since all conjugation of $\EK$ are contained in $\bZ_{p^f}$, one gets $\EK$-eigen components $(V,\nabla,\Phi)_{i,U}$ and $(V,\nabla,\Phi)^{\rm FF}_{i}$, which are endowed with the natural $\bZ_{p^f}$-endomorphism structures $\iota_{i,U}$ and $\iota_i^{\rm FF}$. By choosing suitable order, we may assume
\[(V,\nabla,\Phi,\iota)_{i,U}\otimes \bQ_p \cong (V,\nabla,\Phi,\iota)^{\rm FF}_{i}\mid_{U} \otimes \bQ_p\]
the left and right sides corresponding $\mE_i$ and $M_i$ respectively. In other words, $(V,\nabla,\Phi)_{i,U}$ can be extended to boundary parabolically after tensoring $\bQ$.

By multiplying suitable power of $p$, we may assume under the above isomorphism, one has
\[(V,\nabla,\Phi,\iota)_{i,U} \subseteq (V,\nabla,\Phi,\iota)^{\rm FF}_{i}\mid_{U}.\]

\subsubsection{Verschiebung and $p$-isogeny}

By extending the coefficient, one gets a Verschiebung on $(V,\nabla,\Phi)\otimes \bQ_p$. By restricting onto the new lattice, one gets a Verschiebung structure $\mV$ on $(V,\nabla,\Fil,\Phi)^{FF}$. By adding back the Verschiebung structures on both sides, we gets an isogeny between two Dieudonn\'e crystals. By \cite[Lemma 2.13]{KrPa22}, one has the following result:
\begin{lemma}
There exists an isogenous abelian scheme over $U$, which we just call again $f: A\to U_{k_2'}$ such its $F$-crystal is equal to that of the Fontaine-Faltings module.
\end{lemma}

\subsubsection{Hodge filtration and its lifting}

By taking relative differential $1$-forms attached to $f$ one gets the Hodge filtration on $(V,\nabla)^{FF}\otimes \bF_{q^{2f}}$ given by
\[E^{'1,0}\coloneqq R^0f'_*\Omega^1 _{A'/\bP^1} (\log \Delta) \subset (V,\nabla)^{FF}\otimes \bF_{q^{2f}}=R^1_{dR}f_*(\Omega^\bullet_{A/\bP^1}(\log \Delta),d),\]
which is a rank-$g$ sub bundle.

\begin{proposition}
The Hodge filtration is coincide with that coming from the family. In other words, $\Fil$ is a filtration lifts the Hodge filtration relative differential $1$-forms attached to $f$.
\end{proposition}

\begin{proof}
Since the relative Frobenius $\Phi$ on the Fontaine-Faltings module satisfies the strong $p$-divisible condition with respect to the filtration $\Fil$, the Hodge filtration $E^{1,0}$ coincides with the modulo $p$ reduction of the filtration on the Fontaine-Faltings module.
\end{proof}

\subsubsection{Summary}
Sum up what we have done in this subsection:
\begin{theorem} \label{thm_main_loc_to_family_char_p}
\label{thm_construction_family_mod_p}
Suppose $\lambda\in W(k)$ is supersingular. Consider $(\bP^1,\{0,1,\lambda,\infty\})$ the projective line with four marked points. Then for a given a local system $\bL\in \Loch(k_2')^{|Frob_x}$, then there exists an abelian scheme
\[f: A \to U_{k_2'}\]
of $\GL_2(\EK)$-type such that
such that
\begin{enumerate}[1).]
\item all eigen sheaves $\bL_i$'s are contained in $\Loch(k_2')^{\Frob_x}$ and $\bL_1=\bL$;
\item the Dieudonn\'e crystal attached to $f$ underlies a parabolic Fontaine-Faltings module $(V,\nabla,\Fil,\Phi)^{FF}$;
\item The Hodge filtration $\Fil$ on the Fontaine-Faltings module (mod $p$) coincides with the Hodge filtration $E^{1,0}$ of relative differential 1-forms of $f$. Consequently, $E^{'1,0}$ extends to a
sub bundle on the $\mO_\EK$-log Dieudonn\'e crystal attached to $f$.
\end{enumerate}
\end{theorem}

\newpage

\section{Obstruction to lifting a family of abelian varieties} \label{sec_main_compare_obstructions}

In this section, we prove that the Kodaira-Spencer map sends the obstruction of lift the classifying map to the obstruction of lift the Hodge filtration. We firstly fix some notations.
\begin{itemize}
\item[$\bullet$]$k$: a perfect field of characteristic $p > 0$.
\begin{itemize}
\item $W := W(k)$,
\item $K := {\rm Frac} W$.
\end{itemize}
\item[$\bullet$]$X$: a (proper) smooth curve over $W$.
\begin{itemize}
\item $X_k$: the special fiber of $X$;
\item $X_n$: the modulo $p^n$ reduction of $X$;
\item $X_K$: the generic fiber of $X$;
\item $\mathcal X$: the $p$-adic formal completion of $X$ along the special fiber $X_k$;
\item $\mathcal X_K$: the rigid analytic space associated to $\mathcal X$.
\end{itemize}
\item[$\bullet$] $D\subset X$: a relative divisor, flat over $W$.
\item[$\bullet$] $U=X\setminus D$: the complement of $D$ in $X$.
\begin{itemize}
\item $U_k$, $U_n$, $U_K$, $\mathcal U$, $\mathcal U_K$ defined similar as those for $X$.
\item $\mathcal D_K$: the complement of $\mathcal U_K$ in $\mathcal X_K$ (finite union of disks);
\end{itemize}
\item [$\bullet$] $\mathcal A_{g,N}$: the module space of principal polarized abelian varieties of dimensional $g$ with full $N$-level for $N\geq 3$.
\begin{itemize}
\item $\pi_{univ}\colon\mathcal E_{univ}\rightarrow \mathcal A_{g,N}$: the universal family of abelian varieties.
\item $\overline{\mathcal A}_{g,N}\supset \mathcal A_{g,N}$: the compactification of $\mathcal A_{g,N}$.
\item $\overline \pi_{univ}\colon \overline{\mathcal E}_{univ}\rightarrow \overline{\mathcal A}_{g,N}$: the family of generalized abelian varieties with full $N$-level over $X(N)$.
\end{itemize}
\item[$\bullet$] $\mathbb{D}(\mathcal E)$: the (logarithmic) Dieudonn\'e $F$-crystal associated to a (semistable) family $\mathcal E$ over $X_k$.
\begin{itemize}
\item $\mathbb D(\mathcal E)(X_n)$ the realization of $\mathbb{D}(\mathcal E)$ on $X_n$, which is a vector bundle over $X_n$ together with a connection and a Frobenius endomorphism.
\item $\mathbb D(\mathcal E)(\mathcal X) = \varprojlim\limits_n\mathbb D(\mathcal E)(X_n)$ the realization of $\mathbb{D}(\mathcal E)$ on $\mathcal X$.
\end{itemize}
\end{itemize}

\subsection{Obstruction of lifting a morphism.}
Let $f\colon (X,M_X)\rightarrow (S,M_S)$ be a morphism between schemes with fine logarithmic structures. Denote by $\Omega_{X/S}^1(\log(M_X/M_S))$ the sheaf of logarithmic differentials; write this simply by $\omega^1_{X/S}$ if there is no risk of confusion about the logarithmic structures. The dual, a.k.a. the sheaf of logarithmic vector fields, is denoted by $\Theta_{X/S}(\log(M_X/M_S))$ or $\Theta_{X/S}^{\log}$.

We will periodically refer to the following type of commutative diagram of fine logarithmic schemes:
\begin{equation}
\label{diag:smooth}
\xymatrix{(T_0,M_{T_0}) \ar[r]^{g_0} \ar[d]^{\iota} & (X,M_X) \ar[d]^{f} \\
(T,M_T) \ar[r]^{t} 
& (S,M_S) \\}
\end{equation}
where $\iota$ is an exactly closed immersion and $T_0$ is defined in $T$ by a quasi-coherent sheaf of ideals $I$ with $I^2=0$. As usual, because $I^2=0$, $I$ is naturally a quasi-coherent sheaf of modules on $T_0$.

\begin{definition}\cite[3.3 on p. 201]{Kat89} A morphism $f\colon (X,M_X)\rightarrow (S,M_S)$ between fine logarithmic schemes is called \emph{smooth} if
\begin{enumerate}
\item $f$ is locally of finite presentation and
\item For any commutative diagram as in \eqref{diag:smooth}, locally on $T$ there exists a lift $g\colon(T,M_T)\rightarrow (X,M_X)$ of $g_0$ such that $g\circ \iota=g_0$ and $f\circ g =t$.
\end{enumerate}
\end{definition}
Condition (2) is the logarithmic version of formal smoothness for schemes.

We now follow \cite[Proposition 3.9 on p. 203]{Kat89} to understand the space of lifts. Suppose $g$ and $g'$ are two liftings of $g_0$. We define an element $\alpha_{g,g'}$ in $\mathrm{Hom}(g_0^*\omega^1_{X/S},I)$ which satisfies
\begin{itemize}
\item $\alpha_{g,g'}(\mathrm{d}a) = g^*(a)-g'^*(a)$ for $a\in \mathcal O_X$ and
\item $\alpha_{g,g'}(\mathrm{d}\log a) = u(a) -1$ for $a \in M_X$,
\end{itemize}
where $u(a)$ is the unique local section of $\ker(\mathcal O^*_T \rightarrow \mathcal O^*_{T_0}) \subset M_T$ such that $g^*(a)=g'^*(a)*u(a)$. By the arguments of \cite[p. 203]{Kat89}, $g=g'$ if and only if $\alpha_{g,g'} =0$.

In general, there is an obstruction to lift the map $g_0$ globally; this obstruction can be described using the $\alpha$ just defined. Choose local liftings $g_i$. On the overlap open set the lifting $g_i$ differs with $g_j$ by $\alpha_{ij} = \alpha_{g_i,g_j}$.
\begin{lemma}(Kato) \label{obs:map} Let $f\colon (X,M_X)\rightarrow (S,M_S)$ be a smooth morphism between schemes with fine logarithmic structures. For any commutative diagram as in (\eqref{diag:smooth})
\begin{enumerate}
\item[i.] The $\alpha(g_0):=(\alpha_{ij})$ is a well-defined element in
$$H^1(T_0,\mathcal Hom(g_0^* \omega^1_{X/S},I))=H^1(T_0,g_0^* \Theta^{\log}_{X/S}\otimes I)$$
which does not depend on the choice of local liftings. It is the obstruction to lift $g_0$ globally, i.e. $\alpha=0$ if and only if there exists an $(S,M_S)$-morphism $g:(T,M_T)\rightarrow (X,M_X)$ lifting $g_0$;
\item[ii.] if $\alpha(g_0)=0$, the set of lifts $g$ of $g_0$ is an affine space under $H^0(T_0,\mathcal Hom(g_0^* \omega^1_{X/S},I))=\mathrm{Hom}(g^*_0\omega_{X/S}^1,I)$.
\end{enumerate}

\end{lemma}

\subsection{Obstruction of lifting a sub-bundle}
We next consider the obstruction to lifting a sub-bundle. In this paragraph, the logarithmic structures play no role. Let $\iota:T_0\rightarrow T$ be a square zero thickening with ideal sheaf $I$. Let $V$ be a vector bundle over $T$ together with an symmetric isomorphism
\[\tau\colon V\rightarrow V^t\]
in the sence that $\tau^t=\tau$ where $V^t$ is the dual vector bundle of $V$. Then $\tau$ can be view as an element in $\mathrm{Sym}^2(V^t)$
\[\tau\in\mathrm{Sym}^2(V^t).\]
Let $\overline{L}$ be a vector sub-bundle of $\overline{V}=V\otimes_{\mathcal O_T} \mathcal O_{T_0}$ such that
\[\tau(\overline{L}) = (\overline{V}/\overline{L})^t.\]
This is equivalent to say that
\[\tau \pmod{I} \in \ker\left( \mathrm{Sym}^2(\overline{V}^t) \rightarrow \mathrm{Sym}^2(\overline{L}^t) \right)\]
\begin{lemma}
Zariski locally on $T$ there exist liftings $L$ of $\overline{L}$ such that $\tau(L)=(V/L)^t$. In other words,
\[\tau \in \ker\left( \mathrm{Sym}^2(V^t) \rightarrow \mathrm{Sym}^2(L^t) \right)\]
\end{lemma}
\begin{proof}
Locally choose a basis $e_1,\cdots,e_{2g}$ of $L$ such that $\overline{L}$ is generated by $e_1,\cdots,e_g \pmod{I}$. Denote by $f_1,\cdots,f_{2g}$ the dual basis of $e_1,\cdots,e_{2g}$. Then locally $\tau$ can be represented as
\[\tau = \sum_{i=1}^{2g} a_{ij} \cdot f_i\otimes f_j \in \mathrm{Sym}^2(V^t)\]
with $a_{ij}=a_{ji}$ for each pair $(i)$. Then $\tau(\overline{L}) = (\overline{V}/\overline{L})^t$ means that the coefficients matrix of $\tau$ under the basis $e_1,\cdots,e_{2g}$ has following form
\[A=\left(\begin{array}{cccccc}
a_{1,1} & \cdots & a_{1,g} & a_{1,g+1} & \cdots & a_{1,2g} \\
\vdots & & \vdots & \vdots & & \vdots \\
a_{g,1} & \cdots & a_{g,g} & a_{g,g+1} & \cdots & a_{g,2g} \\
a_{g+1,1} & \cdots & a_{g+1,g} & a_{g+1,g+1} & \cdots & a_{g+1,2g} \\
\vdots & & \vdots & \vdots & & \vdots \\
a_{2g,1} & \cdots & a_{2g,g} & a_{2g,g+1} & \cdots & a_{2g,2g} \\
\end{array}\right) = \left(\begin{array}{cc}
A_{11} & A_{12} \\ A_{21} & A_{22}\\
\end{array}\right)\]
with $A_{11}=0\pmod{I}$ and $A_{12}=A_{21}^T$ invertible.
Now it is easy to find a matrix $Q\in I^{g\times g}$ such that
\[\left(\begin{array}{cc}1&Q^T\\&1\\\end{array}\right) \left(\begin{array}{cc}
A_{11} & A_{12} \\ A_{21} & A_{22}\\
\end{array}\right) \left(\begin{array}{cc}1&\\Q&1\\\end{array}\right)\]
has form $\left(\begin{array}{cc} 0&B_{12}\\B_{21}&A_{22}\\ \end{array}\right)$. This means that the coefficients matrix of $\tau$ under the basis $(e_1,\cdots,e_{2g})\left(\begin{array}{cc}1&\\Q&1\\\end{array}\right)$ has form $\left(\begin{array}{cc} 0&B_{12}\\B_{21}&A_{22}\\ \end{array}\right)$. Denote $L$ the subsheaf generated $e_1,\cdots,e_g$. Then $\tau(L)=(V/L)^t$, since $B_{12}=B_{21}^T$ is invertible.
\end{proof}
In general, there is an obstruction to get a global lifting $L$ of $\overline{L}$ such that
\[\tau(L)=(V/L)^t.\]
We choose local liftings $L_i$ of $\overline{L}$ with $\tau(L_i)=(V/L_i)^t$. Then one has following commutative diagram
\begin{equation*}
\xymatrix{
L_i \ar[r] \ar[d]^{\tau}_{\simeq} & V \ar[r] \ar[d]^{\tau}_{\simeq} & V/L_j \ar[d]^{\tau}_{\simeq}\\
(V/L_i)^t \ar[r] & V^t \ar[r] & L_j^t \\
}
\end{equation*}
Consider the composition $\alpha_{ij}\colon L_i \hookrightarrow V \xrightarrow{\tau} V^t \twoheadrightarrow L_j^t$ over the overlap open subset; this morphism is zero modulo $I$. Thus, it factors thought a map $\beta_{ij}: \overline{L_i} \rightarrow \overline{L_j}^t \otimes I$.
\begin{equation}
\label{eq_map_beta}
\xymatrix{L_i \ar@{->>}[d] \ar@{^(->}[r] & V \ar[r]^{\tau}_{\simeq} & V^t \ar@{->>}[r] & L_j^t\\
\overline{L_i} \ar[rrr]^{\beta_{ij}} &&& \overline{L_j}^t \otimes_{\mathcal O_{T_0}} I \ar@{^(->}[u]\\}
\end{equation}
Since $\tau$ is self dual, one has $\alpha_{ij}^t=\alpha_{ji}$. Thus $\beta_{ij}\in \mathrm{Sym}^2(\overline{L}^t)\otimes_{\mathcal O_{T_0}}I$.
From the definition, it is easy to see that $\beta_{ij}=0$ if and only if $L_i=L_j$. One concludes the following result.
\begin{lemma}
\label{obs:fil}
Let $i:T_0\rightarrow T$ be a square-zero thickening with ideal sheaf $I$. Let $V$ be a vector bundle over $T$ with a symmetric isomorphism
\[\tau\colon V\rightarrow V^t.\]
Let $\overline{L}$ be a sub bundle of $\overline{V}=V\otimes_{\mathcal O_T} \mathcal O_{T_0}$ such that $\tau(\overline{L})=(\overline{V}/\overline{L})^t$.
\begin{enumerate}
\item [i).]
Then $\beta(\overline{L}):=(\beta_{ij})$ (see. Diagram~\eqref{eq_map_beta}) is a well-defined element in $H^1(T_0,\mathrm{Sym}^2(\overline{L}^t)\otimes I))$,which does not depend on the choice of local liftings. It is the obstruction to lift $\overline{L}$ globally, that is, $\beta=0$ if and only if there exists a lift $L\subset V$ of $\overline{L}$ such that $\tau(L)=(V/L)^t$.
\item [ii).]
If $\beta(\overline{L})=0$, the set of isomorphism classes of liftings of $\overline{L}$ is an affine space under $H^0(T_0,\mathrm{Sym}^2(\overline{L}^t)\otimes I))$.
\end{enumerate}
\end{lemma}

\subsection{Identifying obstruction groups via Higgs field}
Let $k$ be a perfect field with characteristic $p>0$ and let $W:=W(k)$ be the ring of $p$-typical Witt vectors. Let $X/\Spec(W)$ be a smooth $W$-scheme. (We will soon add the assumption that $X/\Spec(W)$ is proper, but for now we only require smoothness.) Given a relative normal crossing divisor $D$ on $X$, we set
\[M_D := \{g \in \mathcal O_X \mid g \text{is invertible outside} D\} \subset \mathcal O_X.\]
Then $M_D$ is a fine logarithmic structure on $X$. Let $(X_r,M_{D_r})$ denote the reduction modulo $p^r$. Then for any $s\geq r$, $(X_s,M_{D_s})$ is an object of the (logarithmic crystalline) site $((X_r,M_{D_r})/W)_{crys}^{log}$.

\subsubsection{The classifying mapping and families of abelian varieties}

Let $\overline{\varphi}_r: X_r \rightarrow {\mathcal A_{g,N}}$ be a morphism between schemes. The morphism $\varphi_r$ induces a semistable family of abelian varieties with full $N$-level over $X_r$
\[\mathcal E_{X_r}:=\varphi_r^*(\overline{\mathcal E}_{univ}).\]

\subsubsection{Dieudonn\'e crystal associated to $\mathcal E_{X_r}$ and its realizations}
Denote by $\pi_{{X_k}}\colon \mathcal E_{{X_k}} \rightarrow {X_k}$ the reduction modulo $p$. We let $\mathbb D(\mathcal E_{{X_k}})$ denote the attached logarithmic Dieudonn\'e crystal on $(({X_k},M_{D_{{X_k}}})/W)^{\log}_{crys}$. We denote by $(V_{\mathcal E_{{X_k}}/X_n},\nabla_{\mathcal E_{{X_k}}/X_n})$ the realization of $\mathbb D(\mathcal E_{{X_k}})$ on $(X_n,M_{D_{X_n}})$,furnished by \cite[Theorem 6.2(b) on p. 218]{Kat89}. Take the inverse limit
\[\varprojlim\limits_n (V_{\mathcal E_{{X_k}}/X_n},\nabla_{\mathcal E_{{X_k}}/X_n}) =: (V_{\mathcal E_{{X_k}}/\mathcal X},\nabla_{\mathcal E_{{X_k}}/\mathcal X}),\]
which is a vector bundle over formal scheme $\mathcal X$ with a connection.
We emphasize that $(V_{\mathcal E_{{X_k}}/\mathcal X},\nabla_{\mathcal E_{{X_k}}/\mathcal X})$ and $(V_{\mathcal E_{{X_k}}/X_n},\nabla_{\mathcal E_{{X_k}}/X_n})$ only depend on $\pi_{{X_k}}$.

\begin{remark} The polarization on $\mathcal E_{X_k}$ induces isomorphism
\[\tau\colon (V_{\mathcal E_{{X_k}}/\mathcal X},\nabla_{\mathcal E_{{X_k}}/\mathcal X}) \rightarrow (V_{\mathcal E_{{X_k}}/\mathcal X},\nabla_{\mathcal E_{{X_k}}/\mathcal X})^t.\]
\end{remark}

\begin{remark}
There are several ways of constructing the crystal $\mathbb D(\mathcal E_{{X_k}})$. For instance, one may take relative logarithmic crystalline cohomology of $\pi_{{X_k}}$. Alternatively, if ${U_k}\subset {X_k}$ is the subset over which $\pi_{{X_k}}$ is a smooth morphism of schemes, set $G_{{U_k}}:=\mathcal E_{{U_k}}[p^{\infty}]$. Applying the contravariant Dieudonn\'e functor to $G_{{U_k}}$, we obtain a Dieudonn\'e crystal on ${U_k}$. As $\pi_{\mathcal E_{{X_k}}}$ is semistable, We note that this Dieudonn\'e crystal has logarithmic poles along $D_{{X_k}}={X_k}\backslash {U_k}$.
\end{remark}

\begin{lemma}
\label{independence}
Assume that $\pi'_{X_k}\colon \mathcal E'_{{X_k}}\rightarrow {X_k}$
is another semistable family of abelian variety, which coincides with $\pi_{{X_k}}$ on the open set ${U_k}={X_k}\setminus {D_k}$. Then one has an isomorphism
\[\mathbb D(\mathcal E_{{X_k}}) \simeq \mathbb D(\mathcal E'_{{X_k}}).\]
\end{lemma}

\begin{proof}
Since $\pi'_{X_k}$ and $\pi_{X_k}$ are coincide over $U_k$, one has an isomorphism $\mathbb D(\mathcal E_{{X_k}})\mid_{U_k} \simeq \mathbb D(\mathcal E'_{{X_k}})\mid_{U_k}$ of convergent $F$-isocrystals over $U_k$. Recall \cite[Theorem 5.2.1]{Ked07} and \cite[Theorem 6.4.5]{Ked07}, the composition functor defined by restriction
\[F\textrm{-Isoc}_{\log}(U_k,X_k) \rightarrow F\textrm{-Isoc}^\dagger(U_k) \rightarrow F\textrm{-Isoc}(U_k)\]
is fully faithful from the category of convergent log-$F$-isocrystals over $(U_k,X_k)$ to the category of the convergent $F$-isocrystals over $U_k$. Thus there exists an isomorphism $\mathbb D(\mathcal E_{{X_k}}) \simeq \mathbb D(\mathcal E'_{{X_k}})$ extending $\mathbb D(\mathcal E_{{X_k}})\mid_{U_k} \simeq \mathbb D(\mathcal E'_{{X_k}})\mid_{U_k}$.
\end{proof}

\subsubsection{The Hodge filtration}
By instead taking relative \emph{logarithmic de Rham cohomology} of $\pi_{X_r}$, we obtain a Griffiths-tranverse filtration on $(V_{\mathcal E_{{X_k}}/X_r},\nabla_{\mathcal E_{{X_k}}/X_r})$,which we denote by
\begin{equation}
\label{equ:Fil}
\mathrm{Fil}_{\mathcal E_{X_r}/X_r} \subset (V_{\mathcal E_{{X_k}}/X_r},\nabla_{\mathcal E_{{X_k}}/X_r}).
\end{equation}
\begin{remark}
\label{rmk_HodgeFil_polarization}
This filtration is known as the Hodge bundle and satisfies
\[\tau(\mathrm{Fil}_{\mathcal E_{X_r}/X_r}) \cong \left(V_{\mathcal E_{X_r}/X_r}/\mathrm{Fil}_{\mathcal E_{X_r}/X_r}\right)^t.\]
\end{remark}

\subsubsection{The associated graded Higgs bundle and Kodaira-Spencer map.}
Taking the associated graded Higgs bundle, one gets
\[(E_{\mathcal E_{X_r}},\theta_{\mathcal E_{X_r}})
= \mathrm{Gr}_{\mathrm{Fil}_{\mathcal E_{X_r}/X_r}}
(V_{\mathcal E_{{X_k}}/X_r},\nabla_{\mathcal E_{{X_k}}/X_r})
= (E^{1,0}_{\mathcal E_{X_r}/X_r} \oplus E^{0,1}_{\mathcal E_{X_r}/X_r},\theta_{\mathcal E_{X_r}}),\]
where $E^{1,0}_{\mathcal E_{X_r}/X_r} = R^0\pi_{X_r,*} \omega^1_{\mathcal E_{X_r}/X_r}$, $E^{0,1}_{\mathcal E_{X_r}/X_r} = R^1\pi_{X_r,*} \mathcal O_{\mathcal E_{X_r}}$ and $\theta_{\mathcal E_{X_r}}$ is the graded Higgs field
\[\theta_{\mathcal E_{X_r}}\colon
E^{1,0}_{\mathcal E_{X_r}/X_r}
\rightarrow
E^{0,1}_{\mathcal E_{X_r}/X_r}
\otimes \omega^1_{X_r/W_r}.\]
We rewrite the Higgs field in the form
\[\theta_{\mathcal E_{X_r}}\colon \Theta_{X_r/W}^{\log}\rightarrow \mathcal Hom(E^{1,0}_{\mathcal E_{X_r}/X_r},E^{0,1}_{\mathcal E_{X_r}/X_r}),\]
which is also known as Kodaira-Spencer map.
Due to the existence of principal polarization, by \autoref{rmk_HodgeFil_polarization}, the Higgs field $\theta_{\mathcal E_{X_r}}$ facts through, still denoted by $\theta_{\mathcal E_{X_r}}$,
\[\theta_{\mathcal E_{X_r}} \colon \Theta_{X_r/W}^{\log}\rightarrow \mathrm{Sym}^2 \left(E^{1,0}_{\mathcal E_{X_r}/X_r}\right)^t\]

\subsubsection{Dieudonn\'e crystal, Filtered de Rham bundle and Higgs bundle associated to the universal family}
Similarly starting from the universal family abelian varieties over the moduli space $\mathcal A_{g,N}$, one gets Dieudonn\'e crystal $\mathbb D(\overline{\mathcal E}_{univ})$, and filtered logarithmic de Rham bundle over $\overline{\mathcal A_{g,N}}$
\[\mathrm{Fil}_{\mathcal E_{\mathcal A_{g,N}}} \subset (V_{\mathcal E_{\mathcal A_{g,N}}},\nabla_{\mathcal E_{\mathcal A_{g,N}}})\]
And the Kodaira-Spence map
\begin{equation}
\label{theta:universal}
\theta_{\mathcal E_{{\mathcal A_{g,N}}}}\colon
\Theta^{\log}_{{\mathcal A_{g,N}}/W} \longrightarrow \mathrm{Sym}^2 \left( E^{1,0}_{\overline{\mathcal E}_{univ}/{\overline{\mathcal A}_{g,N}}}\right)^t.
\end{equation}

\begin{lemma} [Faltings-Chai]
$\theta_{\mathcal E_{{\mathcal A_{g,N}}}}$ is an isomorphism.
\end{lemma}
\begin{remark}
Recall the associated Higgs bundle sends a local (logarithmic) vector field $\partial$ to
\begin{equation*}
\theta_{\mathcal E_{{\mathcal A_{g,N}}}}(\partial) = \partial \circ \theta_{\mathcal E_{{\mathcal A_{g,N}}}}\colon
\xymatrix{E^{1,0}_{\mathcal E_{{\mathcal A_{g,N}}}/{\mathcal A_{g,N}}} \ar[r]^(0.4){\theta_{\mathcal E_{{\mathcal A_{g,N}}}}} & E^{0,1}_{\mathcal E_{{\mathcal A_{g,N}}}/{\mathcal A_{g,N}}} \otimes \omega^1_{{\mathcal A_{g,N}}} \ar[r]^(0.6)\partial & E^{0,1}_{\mathcal E_{{\mathcal A_{g,N}}}/{\mathcal A_{g,N}}}.}
\end{equation*}
\end{remark}

\begin{remark}
Since the family $\mathcal E_{X_r}$ is the pull back of the universal family via the classifying mapping $\overline{\varphi}_r$, all Dieudonn\'e crystals and Filtered de Rham bundles and Higgs bundles are the pullbacks of those associated to the universal family via the classifying mapping.
\end{remark}

\subsubsection{Identifying the obstruction groups}
By \autoref{obs:map}, the obstruction to lift $\overline{\varphi}_r$ is located in $H^1(X_k,\varphi_k^* \Theta^{\log}_{{\mathcal A_{g,N}}/W})$.
If the obstruction vanishes, then all liftings form an $H^0(X_k,\varphi_k^* \Theta^{\log}_{{\mathcal A_{g,N}}/W})$-torsor.
By \autoref{obs:fil}, the obstruction to lift $\mathrm{Fil}_k$ is located in
\[H^1\left(X_k,\mathrm{Sym}^2 \left(E^{1,0}_{\mathcal E_{X_1}/X_1}\right)^t\right) = H^1\left(X_k,\varphi^*_k\mathrm{Sym}^2 \left(E^{1,0}_{\overline{\mathcal E}_{univ}/{\overline{\mathcal A}_{g,N}}}\right)^t\right).\]
If the obstruction vanishes, then all liftings form a homogeneous space over the group
\[H^0\left(X_k,\mathrm{Sym}^2 \left(E^{1,0}_{\mathcal E_{X_1}/X_1}\right)^t\right) = H^0\left(X_k,\varphi^*_k\mathrm{Sym}^2 \left(E^{1,0}_{\overline{\mathcal E}_{univ}/{\overline{\mathcal A}_{g,N}}}\right)^t\right).\]
By \eqref{theta:universal}, one has an isomorphism between the obstruction groups
\begin{equation}
\label{equ_ObsGroup1}
\varphi_k^*\theta_{\mathcal E_{{\mathcal A_{g,N}}}}\colon H^1(X_k,\varphi_k^* \Theta^{\log}_{{\mathcal A_{g,N}}/W}) \overset\sim\longrightarrow H^1\left(X_k,\varphi_k^* \mathrm{Sym}^2 \left(E^{1,0}_{\overline{\mathcal E}_{univ}/{\overline{\mathcal A}_{g,N}}}\right)^t\right)
\end{equation}
and an isomorphism between the torsor groups
\begin{equation}
\label{equ_ObsGroup2}
\varphi_k^*\theta_{\mathcal E_{{\mathcal A_{g,N}}}}\colon H^0(X_k,\varphi_k^* \Theta^{\log}_{{\mathcal A_{g,N}}/W}) \overset\sim\longrightarrow H^0\left(X_k,\varphi_k^* \mathrm{Sym}^2 \left(E^{1,0}_{\overline{\mathcal E}_{univ}/{\overline{\mathcal A}_{g,N}}}\right)^t\right).
\end{equation}

\subsection{Comparing the obstructions}

In this subsection, we show that to give a lift $\overline{\varphi}_{r+1}\colon X_{r+1}\rightarrow {\overline{\mathcal A}_{g,N}}$ of $\overline{\varphi}_r$ is equivalent to give a lift of the Hodge filtration onto the realization of $\mathbb D(\mathcal E_{{X_k}})$ on $X_{r+1}$. The main result is
\begin{theorem}
\label{thm_main_comparing_obstruction}
\begin{enumerate}
\item The obstruction of lifting $\overline{\varphi}_r$ maps to the obstruction of lifting the filtration in \eqref{equ:Fil} under the map $\varphi_k^*\theta_{\mathcal E_{{\mathcal A_{g,N}}}}$ in \eqref{equ_ObsGroup1}.
\item Suppose the obstructions vanish. For any lifting $\overline{\varphi}_{r+1}$ of $\overline{\varphi}_r$, one gets $\overline{\varphi}_{r+1}^*(\mathrm{Fil}_{\mathcal E_{univ}/\mathcal A_{g,N}})$ a lifting of the filtration $\mathrm{Fil}_{\mathcal E_{X_r}/X_r}$.
\end{enumerate}
\end{theorem}

Since the obstructions are defined as the differences of local liftings, to show \autoref{thm_main_comparing_obstruction}, one only need to show the following result.
\begin{lemma} Let $U_{g,N}$ be an open subvariety of $\overline{\mathcal A}_{g,N}$. Denote by $U_r:=\varphi_r(U_{g,N})$, which is an open subscheme of $X_r$. Denote by $U_{r+1}$ the open subscheme of $X_{r+1}$, which has the same underlying topological space as $U_{r}$. By shrinking the open subset $U_{g,N}$, we assume there exists a local lifting $\varphi_{r+1}$ of $\overline{\varphi}_r$ over $U_{r+1}$. Denote by $\mathrm{Fil}_{r+1}$ the pullback of $\mathrm{Fil}_{\mathcal E_{univ}/\mathcal A_{g,N}}$ along $\varphi_{r+1}$ which is a lifting of the filtration $\mathrm{Fil}_{\mathcal E_{X_r}/X_r}$ over $U_{r+1}$. Then the following diagram communicates
\begin{equation*}
\xymatrix@R=2cm@C=4cm{
\left\{{\text{all local lifting of} \atop \text{$\overline{\varphi}_{r}$ over $U_{r+1}$}}\right\}
\ar[r]^{\varphi'_{r+1}\mapsto \varphi'^*_{r+1} \left(\mathrm{Fil}_{\mathcal E_{univ}/\mathcal A_{g,N}}\right)}
\ar[d]_{\varphi'_{r+1}\mapsto \varphi'_{r+1}-\varphi_{r+1}}
&
\left\{{\text{all local lifting of} \atop \text{$\mathrm{Fil}_{\mathcal E_{X_r}/X_r}$ over $U_{r+1}$}}\right\}
\ar[d]^{\mathrm{Fil}'_{r+1} \mapsto \mathrm{Fil}'_{r+1} - \mathrm{Fil}_{r+1}}
\\
\left(\varphi_k^* \Theta^{\log}_{{\mathcal A_{g,N}}/W}\right)(U_1)
\ar[r]_{\varphi_k^*\theta_{\mathcal E_{{\mathcal A_{g,N}}}}}
&
\varphi_k^* \mathrm{Sym}^2 \left(E^{1,0}_{\overline{\mathcal E}_{univ}/{\overline{\mathcal A}_{g,N}}}\right)^t(U_1)
\\
}
\end{equation*}
\end{lemma}

\begin{proof}
Let $\varphi_{r+1}$ and $\varphi'_{r+1}$ be two lifting of $\overline{\varphi}_r$ over $U_{r+1}$. Denote by
\[\varphi'_{r+1}-\varphi_{r+1} := \alpha \in \mathrm{Hom}(\varphi_{1}^*\omega_{{\mathcal A_{g,N}}}^1(U_{g,N}),\mathcal O_{X_1}(U_{1})) = \left(\varphi_1^*\Theta^{\log}_{{\mathcal A_{g,N}}/W}\right) (U_{1})\]
defined by the following formula (take $t_1,\cdots,t_d$ such that $\omega_{\mathcal A_{g,N}}(U_{g,N}))$ has basis $\{\mathrm{d}\log(t_i)\}_{1\leq i\leq d}$
\[\alpha(\mathrm{d}\log t_i)= \frac{\varphi'^*_{r+1}(t_i)/\varphi_{r+1}^*(t_i)-1}{p^r} \pmod{p}.\]
Denote by $\varphi_1$ the restriction of $\varphi_r$ on $X_1$. Then
\begin{equation}
\label{alpha:explicite}
\alpha = \sum_{i=1}^{d} \frac{\varphi'^*_{r+1}(t_i)/\varphi_{r+1}^*(t_i)-1}{p^r}\cdot \varphi_1^*\left(t_i\frac{\partial}{\partial t_i}\right)
\end{equation}
Now consider $\mathbb D(\mathcal E)$ the logarithmic crystal associated to the semi-stable family $\overline{\mathcal E}_{univ}/{\overline{\mathcal A}_{g,N}}$.

Denote $\mathcal E_{U_{r+1}}:= \varphi_{r+1}^* \mathcal E_{{\mathcal A_{g,N}}}$ and $\mathcal E'_{U_{r+1}}:= \varphi'^*_{r+1} \mathcal E_{{\mathcal A_{g,N}}}$, then $\mathcal E_{U_{r+1}}\mid_{X_1} = \mathcal E'_{U_{r+1}}\mid_{X_1} =: \mathcal E_{U_1}$ and one has a natural isomorphisms
\[\pi: \varphi_{r+1}^*(V_{\mathcal E_{{\mathcal A_{g,N}}}},\nabla_{\mathcal E_{{\mathcal A_{g,N}}}}) \overset{\sim}\longrightarrow \mathbb D(\mathcal E_{U_1})(U_{r+1},D_{U_{r+1}}),\]
\[\pi': \varphi'^*_{r+1}(V_{\mathcal E_{{\mathcal A_{g,N}}}},\nabla_{\mathcal E_{{\mathcal A_{g,N}}}}) \overset{\sim}\longrightarrow \mathbb D(\mathcal E_{U_1})(U_{r+1},D_{U_{r+1}}).\]
Thus one gets an isomorphism of de Rham bundles
\[\pi'\circ\pi^{-1}:\varphi'^*_{r+1}(V_{\mathcal E_{{\mathcal A_{g,N}}}},\nabla_{\mathcal E_{{\mathcal A_{g,N}}}}) \overset{\sim}\longrightarrow \varphi_{r+1}^*(V_{\mathcal E_{{\mathcal A_{g,N}}}},\nabla_{\mathcal E_{{\mathcal A_{g,N}}}}).\]
Let $(Z,M_Z)$ be the PD-envelope of the diagonal morphism
\[({\mathcal A_{g,N}},M_{D_{{\mathcal A_{g,N}}}}) \rightarrow ({\mathcal A_{g,N}},M_{D_{{\mathcal A_{g,N}}}})\times_W ({\mathcal A_{g,N}},M_{D_{{\mathcal A_{g,N}}}}).\]
and $(Z_1,M_{Z_1})$ the first infinitesimal neighborhood of $({\mathcal A_{g,N}},M_{D_{{\mathcal A_{g,N}}}})$ in $(Z,M_Z)$.

Since $\varphi_{r+1}$ and $\varphi'_{r+1}$ are equal after reduction modulo $p^r$, by the universal property of the first infinitesimal neighborhood the morphism $(\varphi_{r+1},\varphi'_{r+1})$ from $U_{r+1}$ to ${\mathcal A_{g,N}} \times_W {\mathcal A_{g,N}}$ factors through $(Z_1,M_{Z_1})$.
\begin{equation*}
\xymatrix{U_{r+1} \ar[rrr]^(0.4){(\varphi_{r+1},\varphi'_{r+1})} \ar[dr]^{\delta} & & & ({\mathcal A_{g,N}},M_{D_{{\mathcal A_{g,N}}}})\times_W ({\mathcal A_{g,N}},M_{D_{{\mathcal A_{g,N}}}})\\
& (Z_1,M_{Z_1}) \ar[r] & (Z,M_Z) \ar[ur] & \\}
\end{equation*}
Let $p_1,p_2\colon (Z_1,M_{Z_1}) \rightarrow ({\mathcal A_{g,N}},M_{D_{{\mathcal A_{g,N}}}})$ be the first and the second projections, respectively. According to \cite[6.7]{Kat89}, one has an isomorphism $\eta: p_2^* V\simeq p_1^* V$ given by
\begin{equation} \label{eq_Taylor_Conn}
\eta(p_2^*(v)) = p_1^*(v) + \sum_{i=1}^{d} p_1^*\left(\nabla\left(t_i\frac{\partial}{\partial t_i}\right)(v)\right)\cdot \left(\frac{p_2^*(t_i)}{p_1^*(t_i)}-1\right),
\end{equation}
where $v$ is any local section of $V$. Pulling back the isomorphism $\eta$ back onto $U_{r+1}$ via $\delta$ one gets an isomorphism $\delta^*(\eta)$ which is just equal $\pi'\circ\pi^{-1}$ by the definition of pull back of a crystal.

Recall the pull back filtration $\mathrm{Fil}^1_{U_{r+1}}V_{\mathcal E_{U_{r+1}}} := \varphi_{r+1}^*\left(\mathrm{Fil}^1_{\mathcal E_{{\mathcal A_{g,N}}}}V_{\mathcal E_{{\mathcal A_{g,N}}}}\right)$ on $\varphi_{r+1}^*(V_{\mathcal E_{{\mathcal A_{g,N}}}},\nabla_{\mathcal E_{{\mathcal A_{g,N}}}})$. Consider following commutative diagram
\begin{equation*}
\xymatrix@C=1cm{
\varphi'^*_{r+1}\left(\mathrm{Fil}^1_{\mathcal E_{{\mathcal A_{g,N}}}}V_{\mathcal E_{{\mathcal A_{g,N}}}}\right)
\ar@{^(->}[r]
\ar@{..>}@/^25pt/[rrr]
&
\varphi'^*_{r+1}V_{\mathcal E_{{\mathcal A_{g,N}}}} \ar[r]^\simeq_{\pi'\circ\pi^{-1}}
\ar[d]_{\tau}^{\simeq}
&
\varphi_{r+1}^*V_{\mathcal E_{{\mathcal A_{g,N}}}} \ar[d]^{\tau}_{\simeq} \ar@{->>}[r]
&
\varphi_{r+1}^*\left(V_{\mathcal E_{{\mathcal A_{g,N}}}}/\mathrm{Fil}^1_{\mathcal E_{{\mathcal A_{g,N}}}}V_{\mathcal E_{{\mathcal A_{g,N}}}}\right) \ar[d]^{\tau}_{\simeq}
\\
&
\varphi'^*_{r+1}V^t_{\mathcal E_{{\mathcal A_{g,N}}}} \ar[r]^\simeq_{\pi'\circ\pi^{-1}}
&
\varphi_{r+1}^*V^t_{\mathcal E_{{\mathcal A_{g,N}}}},\ar@{->>}[r]
&\varphi_{r+1}^*\left(\mathrm{Fil}^1_{\mathcal E_{{\mathcal A_{g,N}}}}V_{\mathcal E_{{\mathcal A_{g,N}}}}\right)^t.
\\
}
\end{equation*}
Since $\theta_{\mathcal E_{{\mathcal A_{g,N}}}}$ is the associated graded of $\nabla_{\mathcal E_{{\mathcal A_{g,N}}}}$, by \eqref{eq_Taylor_Conn}, the above dotted arrow is given by
\begin{equation} \label{eq_Taylor_Higgs}
\varphi'^*_{r+1}(v) \mapsto \sum_{i=1}^{d}\varphi_{r+1}^*\left(\theta\left(t_i\frac{\partial}{\partial t_i}\right)(v)\right) \cdot \left(\frac{\varphi^*_{r+1}(t_i)}{\varphi^*_{r+1}(t_i)}-1\right).
\end{equation}
Dividing by $p^r$ and considering the reduction modulo $p$ of the dotted arrow, one gets a morphism of sheaves over $U_{1}$
\[\mathrm{Fil}^1_{U_{1}}V_{\mathcal E_{U_{1}}}
\rightarrow
\frac{V_{\mathcal E_{U_{1}}}}
{\mathrm{Fil}^1_{U_{1}}V_{\mathcal E_{U_{1}}}},\]
By \eqref{eq_map_beta}, the difference between the two filtrations
\[\beta := \mathrm{Fil}'_{r+1} - \mathrm{Fil}_{r+1}\]
is defined as the composition
\[\beta\colon
\mathrm{Fil}^1_{U_{1}}V_{\mathcal E_{U_{1}}}
\rightarrow
\frac{V_{\mathcal E_{U_{1}}}}
{\mathrm{Fil}^1_{U_{1}}V_{\mathcal E_{U_{1}}}} \rightarrow \left(\mathrm{Fil}^1_{U_{1}}V_{\mathcal E_{U_{1}}}\right)^t.\]
By \eqref{eq_Taylor_Higgs}, the composition $\beta$ can be computed explicitly
\begin{equation} \label{beta_ij:explicite}
\beta = \sum_{i=1}^d \frac{\varphi_{r+1}^*(t)/\varphi^*_{r+1}(t)-1}{p^r} \cdot \varphi_1^*\left( \theta_{\mathcal E_{{\mathcal A_{g,N}}}}(\partial/\partial t)\right) = \varphi_k^*\theta_{\mathcal E_{{\mathcal A_{g,N}}}}(\alpha).
\end{equation}
Thus, the lemma follows.
\end{proof}

\section{\bf Lifting abelian scheme from characteristic $p$ to characteristic zero by Grothendieck-Messing-Kato logarithmic deformation theorem} \label{sec_main_lifting}

In this section, we assume $p>3$. Let $f\colon A\rightarrow U_{k_2'}$ be an abelian scheme constructed in \autoref{thm_construction_family_mod_p}. In this section, we will extend it to a family of abelian varieties over the complex projective line. By base change and Zarhin's trick we may get a classifying mapping of the family. Then deforming the family is equivalent to deforming the classifying mapping. By enlarging the base field $k$, we simply assume the family $f$ is over $U_{k}$
\[f\colon A\rightarrow U_k.\]

\subsection{classifying mapping} \label{sec_sub_classifying_mapping}
In order to get a classifying map from the base into a fine moduli space of principal polarized abelian varieties, we need to add a level structure and a principal polarization to the family.

\subsubsection{level structure} \label{sec_subsub_level}

Firstly, by base change, we add a level structure.
\begin{lemma} \label{thm_add_level}
By enlarging $k$, there exists a finite covering between two projective smooth curve over $k$
\[\pi_k\colon C_k \rightarrow \bP_k^1\]
which is \'etale over $U_k$ such that the pullback family of $f$
\[f_{\pi_k}\colon A_{\pi_k} \rightarrow \pi_k^{-1}(U_k)\]
has full $3$-level structure.
\end{lemma}
\begin{proof}
Let $k(t)$ be the function field of the projective line and $A_\eta$ is the generic fiber of $f$. Then by adding the coordinates of all torsion points of $A_\eta$ of order $3$ to $k(t)$, one gets a separable finite field extension of $k(t)$. In particular, one gets a curve $C_{\kappa}$ over some finite extension $\kappa$ of $k$ and a finite morphism $\pi_k\colon C_\kappa\rightarrow \mathbb P^1_k$ such that $\pi$ is \'etale over $U_k$. By enlarging the field $k$, we may assume $\kappa=k$ and $\pi_k$ is a $k$-morphism between two proper smooth curves over $k$. The curve $C_k$ satisfies our requirement clearly.
\end{proof}

By the smoothness, we find and fix a lifting of $\pi_k$ over $W(k)$
\[\pi\colon C\rightarrow \bP^1_{W(k)}.\]
Denote by $D\subset C$ the pullback divisor of $\{0,1,\lambda,\infty\}$ under $\pi$. Then
\[\pi_k^{-1}(U_k) = C_k\setminus D_k,\]
where $D_k = \pi_k^{-1}(\{0,1,\lambda,\infty\})$.

\subsubsection{Zarhin's trick} \label{sec_subsub_Zarhin}

Let $f_{\pi_k}\colon A_{\pi_k} \rightarrow C_k$ be the pullback family given as in \autoref{thm_add_level}. By Zarhin trick, the fiber product
\[ f^{(4,4)}_{\pi_k} : A^{4,4}_{\pi_k}:=(A_{\pi_k} \times A_{\pi_k}^t)^4 \to C_k\setminus D_k.\]
carries a principle polarization
\[\iota: A^{(4,4)}_{\pi_k} \xrightarrow{\simeq} \left(A_{\pi_k}^{(4,4)}\right)^t.\]

\subsubsection{Faltings-Chai's compactification}
By Faltings-Chai Theorem \cite{FaCh90}, there exists a fine arithmetic moduli space $\mathcal A_{8g,3}$ of principle polarized abelian varieties with level-$3$ structure, which is smooth over $\mathbb Z[e^{{2i \pi \over 3}},1/3]$. The moduli space carries
an universal abelian scheme
\[\pi_{univ} \colon \mE_{univ} \rightarrow \mA_{8g,3}.\]

Further more, there exists a smooth Toroidal compactification $\overline{\mA_{8g,3}} \supset \mA_{8g,3}$ over $\mathbb Z[e^{{2i \pi \over 3}},1/3]$ and a smooth compactification of the universal abelian scheme
\[\overline{\pi}^{uni}: \overline{\mE}_{univ} \to \overline{\mA}_{8g,3}\]
such that $\mathcal A \setminus \mathcal A^0=:\Delta$ is a relative normal crossing divisor over $\overline{\mA}_{8g,3}\setminus \mA_{8g,3}=:\infty$.

\subsubsection{Classifying mapping}
Recall the notation $\overline{\mA}_{8g,3}$, which is the compactification of the moduli space $\mA_{8g,3}$ of principal polarized abelian varieties of dimension $8g$ with full $3$-level. There is a universal family $\mE_{univ}$ of abelian varieties over $\mA_{8g,3}$ which can be extended to a family $\overline{\mE}_{univ}$ of generalized abelian varieties with full $N$-level. By the universal property of the moduli space, one gets a classifying mapping.
\begin{proposition} 
\label{thm_classifying_mapping_k}
There exists a unique morphism $\overline{\varphi}_k\colon C_k \rightarrow \overline{\mA}_{8g,3}$ such that
\[\overline{\varphi}_k(\mE_{univ})\mid_{C_k\setminus D_k} = A^{(4,4)}_{\pi_k}.\]
\end{proposition}
\begin{proof}
By the universal property of $\mA_{8g,3}$, there exists $\varphi_k\colon C_k\setminus D_k \rightarrow A_{8g,3}$ such that
\[\overline{\varphi}_k(\mE_{univ})\mid_{C_k\setminus D_k} = A^{(4,4)}_{\pi_k}.\]
. Since $\overline{\mA}_{8g,3}$ is projective and regular, the mapping $\varphi_k$ can be extended uniquely.
\end{proof}
\begin{remark}
To lift the family $A^{(4,4)}_{\pi_k}$ is equivalent to lift the classifying map $\overline{\varphi}_k$.
\end{remark}

\subsection{Polarization on the log Dieudonn\'e module}
\label{sec_sub_polarization_and_Fil}


\subsubsection{log Dieudonn\'e module of $f_{\pi_k}$}
Let
$(V,\nabla,\Phi,\mathcal V)_{f_{\pi_k}}$ denote the realization of the logarithmic Dieudonn\'e module of $f_{\pi_k}$ over $(\mC,\mD)$,the $p$-adic formal completion of $(C,D)$. By the $\GL_2$-action, its decomposing as form
\[(V,\nabla,\Phi,\mathcal V)_{f_{\pi_k}}=\bigoplus_{i=1}^g(V,\nabla,\Phi,\mathcal V)_{f_{\pi_k},i}.\]
By \autoref{thm_construction_family_mod_p}, the triple $(V,\nabla,\Phi,\mathcal V)_{f_{\pi_k} i}$ underlies a log Fontaine-Faltings module
$\pi_{par}^*(V,\nabla,F, \Phi)^{FF}_i$,which is the parabolic pullback of a parabolic Fontaine-Faltings module. Hence, it carries the Hodge filtration
\[ F_{f_{\pi_k}} = \bigoplus_{i=1}^g \pi_{par}^* F_i \subset V_{f_{\pi_k}}.\]
where $ \pi^*_{par} F_i=: \mathcal L^{1,0}_i$ is a positive line bundle on $\mC$ over $W(k)$, and $V_{f_{\pi_k} i}/\mathcal L^{1,0}_i=:\mathcal L^{0,1}_i$ is a negative line bundle
with $\mathcal L^{0,1}_i=\left(\mathcal L^{1,0}_i\right)^{-1}$.

\subsubsection{log Dieudonn\'e module of $f^t_{\pi_k}$}
Similarly, we find the realization of the logarithmic Dieudonn\'e module attached to $f_{\pi_k}^t$
$$ (V,\nabla,\Phi,\mathcal V)_{f^t_{\pi_k}}=(V,\nabla,\Phi,\mathcal V)^\vee_{f_{\pi_k}}
=\bigoplus_{i=1}^g(V,\nabla,\Phi,\mathcal V)^\vee_{f_{\pi_k} i}$$
which carries the Hodge filtration
\[ F_{f^t_{\pi_k}} = (V_{f_{\pi_k}}^{\oplus 4}/ F_{f_{\pi_k}})^\vee= \bigoplus_{i=1}^g \mL_i^{0,1\vee} \subset V_{f^t_{\pi_k}} = V^\vee_{f_{\pi_k}} .\]

\subsubsection{log Dieudonn\'e module of $f^{(4,4)}_{\pi_k}$ and $(f^{(4,4)}_{\pi_k})^t$}
Putting everything together, we find the realizations of the logarithmic Dieudonn\'e module attached to $f^{(4,4)}_{\pi_k}$ and $(f^{4,4}_{\pi_k})^t$
\[(V,\nabla,\Phi,\mathcal V)_{f^{(4,4)}_{\pi_k}} = (V,\nabla,\Phi,\mathcal V)_{f_{\pi_k}i}^{\oplus 4} \oplus (V,\nabla,\Phi,\mathcal V)_{f_{\pi_k}i}^{\vee \oplus 4}\]
\[(V,\nabla,\Phi,\mathcal V)_{(f^{(4,4)}_{\pi_k})^t} = (V,\nabla,\Phi,\mathcal V)_{f_{\pi_k}i}^{\vee \oplus 4} \oplus (V,\nabla,\Phi,\mathcal V)_{f_{\pi_k}i}^{\oplus 4}.\]
and Hodge filtrations on the realizations
\[F_{f^{(4,4)}_{\pi_k}} = F_{f_{\pi_k}}^{\oplus 4} \oplus F_{f^t_{\pi_k}}^{\oplus 4} \subset (V,\nabla,\Phi,\mathcal V)_{f^{(4,4)}_{\pi_k}}\]
\[F_{(f^{(4,4)}_{\pi_k})^t} =F_{f^t_{\pi_k}}^{\oplus 4} \oplus F_{f_{\pi_k}}^{\oplus 4} \subset (V,\nabla,\Phi,\mathcal V)_{(f^{(4,4)}_{\pi_k})^t}.\]

Consequently, $F_{f^{(4,4)}_{\pi_k}}$ is a positive vector bundle and $V_{f^{(4,4)}_{\pi_k}}/F_{f^{(4,4)}_{\pi_k}}\cong \left(F_{f^{(4,4)}_{\pi_k}}\right)^\vee$ is a negative vector bundle.

\subsubsection{The isomorphism induced by the principal polarization}
The principal polarization $\iota$ induces an isomorphism, by abusing notion we still denote it by $\iota$
\[\iota: (V,\nabla,\Phi,\mathcal V)_{f^{(4,4)}_{\pi_k}} \rightarrow (V,\nabla,\Phi,\mathcal V)_{(f^{4,4}_{\pi_k})^t}\cong \left((V,\nabla,\Phi,\mathcal V)_{f^{(4,4)}_{\pi_k}}\right)^\vee.\]
\begin{proposition} \label{thm_polarization_and_filtration}
$\iota(F_{f^{(4,4)}_{\pi_k}}) = (V_{f^{(4,4)}_{\pi_k}}/F_{f^{4,4}_{\pi_k}})^\vee$.
\end{proposition}

\begin{proof}
First we consider modulo $p$. Since the abelian scheme $f^{(4,4)}_{\pi_k}$ is a principal polarized abelian scheme
and $ F_{f^{(4,4)}_{\pi_k}}$ modulo $p$ is the Hodge filtration of the relative differential forms on $f^{(4,4)}_{\pi_k}$ and
$V_{f^{(4,4)}_{\pi_k}}/F_{f^{(4,4)}_{\pi_k}})^\vee$ is the Hodge filtration of the relative differential forms on $(f^{4,4}_{\pi_k})^t$.
Hence the isomorphism $\iota$ between the filtered Dieudonn\'e modules modulo $p$ is nothing but the isomorphism $\iota$ between the filtered de Rham bundles. Hence
\[\iota\left(F_{f^{(4,4)}_{\pi_k}}\right) \pmod{p} = (V_{f^{(4,4)}_{\pi_k}}/F_{f^{4,4}_{\pi_k}})^\vee\pmod{p}.\]
We will prove this lemma by contradiction. Suppose $\iota(F_{f^{(4,4)}_{\pi_k}})\neq (V_{f^{(4,4)}_{\pi_k}}/F_{f^{4,4}_{\pi_k}})^\vee$. Then there exists some $n\geq1$ such that
\begin{equation} \label{eq_mod_pn}
\iota\left(F_{f^{(4,4)}_{\pi_k}}\right) \pmod{p^n} = (V_{f^{(4,4)}_{\pi_k}}/F_{f^{4,4}_{\pi_k}})^\vee\pmod{p^n}
\end{equation}
and
\begin{equation} \label{eq_mod_pn+1}
\iota\left(F_{f^{(4,4)}_{\pi_k}}\right) \pmod{p^{n+1}} = (V_{f^{(4,4)}_{\pi_k}}/F_{f^{4,4}_{\pi_k}})^\vee\pmod{p^{n+1}}.
\end{equation}
Consider the composition
\[F_{f^{(4,4)}_{\pi_k}} \xrightarrow{\quad \iota \quad} (V_{f^{(4,4)}_{\pi_k}})^\vee \twoheadrightarrow (F_{f^{4,4}_{\pi_k}})^\vee\pmod{p^{n+1}},\]
which is zero modulo $p^n$ by \eqref{eq_mod_pn} and nonzero modulo $p^{n+1}$ by \eqref{eq_mod_pn+1}. Thus by dividing $p^n$ and reduction modulo $p$, the composition induces a non-zero morphism
\[F_{f^{(4,4)}_{\pi_k}} \pmod{p} \rightarrow (F_{f^{(4,4)}_{\pi_k}})^\vee \pmod{p}.\]
But, this contradicts to the fact that $F_{f^{(4,4)}_{\pi_k}} \pmod{p}$ is positive.
\end{proof}

\subsection{Grothendieck-Messing-Kato logarithmic deformation theorem}

\begin{theorem}[Grothendieck-Messing-Kato logarithmic deformation theorem] \label{thm_lifting_family_to_W}
Let $(Y,D)$ be a smooth curve over $W(k)$ together with a relative normal crossing divisor $D$. Let
\[\psi_1: Y_k \rightarrow \overline{\mA}_{8g,3}\]
be a morphism such that $\psi_1(Y_k\setminus D_k)\subset \mA_{8g,3}$. Assume that
\begin{enumerate}
\item the the pulled back Hodge bundle $E^{1,0}_{\psi_1}:=\psi^*E^{1,0}_{\overline{\mA}_{8g,6}}$ is positive, i.e. any quotient bundle of $E^{1,0}_{\psi_1}$ has positive degree with respect to an ample divisor $H$ on $Y$ and $E^{0,1}_{\psi_1}$ is negative.
\item $E^{1,0}_{\psi_1}$ has a lifting as a sub vector bundle $F\subset (V,\nabla,\Phi,\mathcal V)_{\psi_1}$ over $W(k)$ and compatible with the polarization, i.e.
\[\iota(F)=(V_{\psi_1}/F)^\vee.\]
\end{enumerate}
Then $\psi_1$ lifts to a log map $\psi$ over $W(k)$ and such that
the sub bundles $E^{1,0}_{\psi}$ and $F$ in $(V,\nabla,\Phi,\mathcal V)_{\psi_1}$ coincide with each other.
\end{theorem}
\begin{proof}
Take a collection of local liftings
$\{\psi_{2 \beta}\}_\beta$ over $W_2(k) $ of $\psi_1$, which induces a collection of local liftings
$\{E^{1,0}_{\psi_{2 \beta}}\}_\beta$ over $W_2(k)$ of $E^{1,0}_{\psi_1}$.
Since by assumption $E^{1,0}_{\psi_1}$ has a global lifting $F\otimes W_2(k)$,
the obstruction cocycle defined by $\{E^{1,0}_{\psi_{2 \beta}}\}_\beta$ vanishes in $H^1(C\otimes k,\mathrm{Sym}^2 E^{0,1}_{\psi_1}). $ Hence, by \autoref{thm_main_comparing_obstruction} the obstruction cocycle defined by $\{\psi_{2 \beta}\}_\beta$ vanishes in $H^1(C_k,\psi^*_1 \Theta^{\log}_{{\mathcal A_{g,N}}/W} )$ and one obtains a global lifting $\psi_2$ over $W_2(k)$ of $\psi_1$.

We show now two sub bundles $E^{1,0}_{\psi_2}$ and $F\otimes W_2(k)$ in $(V,\nabla,\Phi,\mathcal V)_{\psi_1}\otimes W_2(k)$ coincide with each other.
Take the quotient bundle
$$0\to F\otimes W_2(k)\to V_{\psi_1}\otimes W_2(k)\to Q\otimes W_2(k).\to 0$$
and the projection
$$\alpha: E^{1,0}_{\psi_2}\hookrightarrow V_{\psi_1}\otimes W_2(k)\to Q\otimes W_2(k).$$
Since $\alpha=0$ (mod $p$). We obtain the map
$${\alpha\over p}: E^{1,0}_{\psi_1}=E^{1,0}_{\psi_2}\otimes k\to Q\otimes k =E^{1,0\vee}_{\psi_1}.$$
By the assumption $E^{1,0}_{\psi_1}$ is positive and $E^{1,0\vee}_{\psi}$ is negative, which implies that
${\alpha\over p}=0$. Hence $\alpha=0$ and $E^{1,0}_{\psi_2}=F\otimes W_2(k)$.

By repeating the above procedure inductively we finish the proof.
\end{proof}

\begin{corollary}
The classifying mapping $\overline{\varphi}_k\colon C_k \rightarrow \mA_{8g,3}$ can be lifted to a mapping
\[\overline{\varphi}\colon C \rightarrow \mA_{8g,3}.\]
\end{corollary}
\begin{proof}
Back to our situation, we have a principal polarized abelian scheme
\[f^{(4,4)}_{\pi_k}: A^{(4,4)}_{\pi_k} \to C_k\] semistable bad reduction on $D$ and carries a level-3 structure, whose Dieudonn\'e module $(V,\nabla,\Phi,\mathcal V)_{f^{(4,4)}_{\pi_k}}$
carries a Hodge filtration
\[V_{f^{(4,4)}_{\pi_k}} \supset F_{f^{(4,4)}_{\pi_k}}\]
lifting the Hodge filtration $E^{1,0}_{f^{(4,4)}_{\pi_k}}$ of the relative differential forms attached to $f_{\pi_k}^{(4,4)}$. The sub bundle
$F_{f^{(4,4)}_{\pi_k}}$ is positive and compatible with the principal polarization
\[\iota(F_{f^{(4,4)}_{\pi_k}}) = (V_{f^{(4,4)}_{\pi_k}})/ F_{f^{(4,4)}_{\pi_k}})^\vee.\]
Since the pullback of the completion of the universal family $$\psi_1^*f^{uni}: \psi_1^*\mathcal A\to C_k$$
is also a semistable model of the smooth part of $f^{(4,4)}_{\pi_k},$ we have an isomorphism between the log Dieudonn\'e modules
$$(V,\nabla,\Phi,\mathcal V)_{f^{(4,4)}_{\pi_k}}\simeq (V,\nabla,\Phi,\mathcal V)_{\psi_1}$$
as the canonical extension of Dieudonn\'e module on the smooth part. This isomorphism
induces an isomorphism between Hodge bundles over the close fiber
$$( (V,\nabla,\Phi,\mathcal V)_{f^{(4,4)}_{\pi_k}}\otimes k \supset E^{1,0}_{f^{(4,4)}_{\pi_k}})\simeq ( (V,\nabla,\Phi,\mathcal V)_{\psi_1}\otimes k\supset E^{1,0}_{\psi_1}).$$
Thus the log map
\[\overline{\varphi}_k: C_k\to \overline{\mA}_{8g,6}\]
satisfies the conditions required in \autoref{thm_lifting_family_to_W}, hence it lifts to a
\[\overline{\varphi}: \mC \to \overline{\mA}_{8g,6}\]
such that the log Fontaine-Faltings module
attached to the pulled back abelian scheme $\psi^*f^{uni}: \psi^* \mathcal A\to C$ has the form
\[ (V,\nabla,E^{1,0},\Phi)_\psi\simeq \bigoplus_{i=1}^{g}(\pi^*(V,\nabla,F,\Phi)^{FF \oplus 4}_{i}\oplus
\pi^*(V,\nabla,F,\Phi)^{FF \vee \oplus 4}_{i})\]
where $(V,\nabla,F,\Phi)^{FF}_{i}\in \MFh(W(k))_{\mathbb Z_{p^f}}$. Since $C$ is projective over $W$,the $\overline{\varphi}$ is algebraic.
\end{proof}

Summerizing what we have done above
\begin{theorem}
Let $\lambda\in W(k)$ be supersingular. Let $(E,\theta)\in \PHighf(W(k))$. Then after enlarging the field $k$, there exists a finite cover $\pi: (C,D)\to(\mathbb P^1_{W(k)},\{0,1,\lambda,\infty\})$ \'etale on $\mathbb P^1_{W(k)}-\{0,1,\lambda,\infty\}$ and a family of abelian varieties
\[f^{(4,4)}_{\pi}: A^{(4,4)}_{\pi}\to C\]
with semistable bad reduction on $D$ such that $\pi^*(E,\theta)$ is realized by $f^{(4,4)}_{\pi}$. That is, the Higgs bundle attached to $f^{(4,4)}_{\pi}$ is of form
\[(E,\theta)_{f^{(4,4)}_{\pi}} =\bigoplus_{i=1}^{8g}\pi^*(E,\theta)_i\]
with all $(E,\theta)_i\in\High(W(k))$
and $(E,\theta)\simeq (E,\theta)_1$.
\end{theorem}

\subsection{Lifting the abelian scheme over complex numbers}

Let $L$ be a number field and take an algebraic number $\lambda \in L\setminus\{0,1\}$. Then for almost all place $\frakp$, $L$ is unramified at $\frakp$, the number $\lambda$ is $\frakp$-adic integral, and $\lambda\not\equiv 0,1 \pmod{\frakp}$.
We choose and fix such a place $\frakp$ of $L$.

Assume $\lambda$ is supersingular, in the sense that the elliptic curve defined by the Weierstrass equation \[y^2=x(x-1)(x-\lambda)\]
is supersingular over the residue field $k_\frakp$.

\begin{theorem} \label{thm_family_from_W_to_number_field}
For any any Higgs bundle $(E,\theta)\in \PHigh(W(k_\frakp))$. After enlarging the field $L$,there exists a finite covering $\pi\colon C\rightarrow \bP^1_L$ defined over $L$ and a family of abelian variety over $C$ such that $\pi^*(E,\theta)$ is a direct summand of the Higgs bundle attached to this family.
\end{theorem}

\begin{proof}
Applying \autoref{thm_lifting_family_to_W} to any Higgs bundle $(E,\theta)\in \PHigh(W(k_\frakp))$, there exist
\begin{itemize}
\item a finite extension $k$ of $k_\frakp$,
\item a curve $C$ defined over $W(k)$,
\item a finite covering mapping
\[\pi: (C,D)\to (\mathbb P^1_{W(k)},\{0,1,\lambda,\infty\}))\]
which is \'etale outsider $\{0,1,\lambda,\infty\}$,and
\item an abelian scheme
\[f^{(4,4)}_{\pi}: A^{(4,4)}_{\pi}\to C\]
with semistable bad reduction over $D$
\end{itemize}
such that the Higgs bundle attached to $f^{(4,4)}_{\pi}$ has the form
\[(E,\theta)_{f^{(4,4)}_{\pi}}=\bigoplus_{i=1}^{8g}\pi^*(E,\theta)_i\]
where all $(E,\theta)_i\in \PHigh(W(k))$ with $(E,\theta)_1=(E,\theta)$ and $\pi^*$ is the parabolic pullback.

As the singular fibers of the abelian scheme $f^{(4,4)}_{\pi}$ over $D$ are maximal degenerated, as in \cite[Section 4]{KYZ22}, it is rigid. Hence, the abelian scheme is defined over some number field $L'$. In other words, there exists a finite field extension $L'$ such that the curve $C$ is defined over $L'$ and the abelian scheme
\[f^{(4,4)}_{\pi}: A^{(4,4)}_{\pi}\to C\]
is also defined over $L'$. In particular, all $p$-adic sub Higgs bundles $\pi^*(E,\theta )_i$ in the above decomposition are in fact algebraic sub Higgs bundles of the Higgs bundle $(E,\theta)_{f^{(4,4)}_{\pi}}/L$ attached to $f^{(4,4)}_{\pi}/L$.
\end{proof}

\subsection{Descending the abelian scheme $f^{(4,4)}_{\pi}$ over $C$ to $\mathbb P^1_L$.}

Applying Weil restriction along $\pi\colon C\rightarrow \mathbb P^1_L$ to the family in \autoref{thm_family_from_W_to_number_field}, one obtains an abelian scheme
\[h\colon B\to \mathbb P^1_L\]
with bad reduction on $\{0,1,\lambda,\infty\}$ and such that $(E,\theta)$ is a direct summand of the Higgs bundle $(E,\theta)_{h}$ attached to the abelian scheme $h$.

We take then the simple factor, say $h'$ of $h$ such that $(E,\theta )$ is contained in the Higgs bundle $(E,\theta)_{h'}$.
In the following, we show that the family $h'$ is of $\text{GL}_2$-type.

\begin{lemma} \label{thm_sub_arith_loc_sys} Let $\mathbb V_{h'_0}$ denote the Betti local system attached to the smooth fiber space of $h'$. Then there exists a number field $\EK$ such that $\mathbb V_{h'_0}\otimes \mO_\EK$ contains a rank-2 sub local system $\mathbb W$.
\end{lemma}
\begin{proof}
Consider the moduli space $\text{Grass}( 2,\mathbb V_{h_0'})$ rank-2 sub local systems in $\mathbb V_{h'_0}$. Then it is defined over $\mathbb Z$. As $(E,\theta)$ is a sub Higgs bundle of parabolic degree zero in
$(E,\theta)_{h'}$, by Simpson correspondence we obtain a rank-2 complex sub local system
\[\mathbb W_{(E,\theta)}\subset \mathbb V_{h'_0}\otimes \mathbb C.\]
Hence, there exists a number field $\EK$ such that $\mathbb W_{(E,\theta)}$ is defined by $\mO_{\EK}$. In particular, we find a rank-2 sub local system $\mathbb W$ in $\mathbb V_{h'_0}\otimes \mO_{\EK}$.
\end{proof}

By applying Simpson theorem \autoref{thm_Simpson} we show the following result.
\begin{theorem} \label{thm_family_GL2_component}
The abelian scheme $h'$ is of $\text{GL}_2$-type and $(E,\theta)$ is isomorphic to an eigen-sheaf of the Higgs bundle attached to $h'$.
\end{theorem}
\begin{proof}
Consider the rank-2 sub local system $\mathbb W\subset \mathbb V_{h'_0}\otimes \mathcal O$ constructed in \autoref{thm_sub_arith_loc_sys}. Then the Higgs bundle corresponding to $\mathbb W$ is tautologically a graded sub Higgs bundle $(E,\theta)_{\mathbb W}\subset(E,\theta)_{h'}$. Further more, the Higgs bundles corresponding all Galois conjugates $\mathbb W^\sigma$ are graded sub Higgs bundles of $(E,\theta)_{h'}$. By Simpson's theorem, we find an abelian scheme over $\mathbb P^1$ of $\text{GL}_2$-type, such that $\mathbb W$ is an eigen-sheaf attached to this abelian scheme. By the construction this abelian scheme is a sub abelian scheme of $h'$. As $h'$ is already simple. We show $h'$ is of $\text{GL}_2$-type. Since $(E,\theta)$ is stable, it is isomorphic to an eigen sheaf attached to $h'$.
\end{proof}

\section{Isomonodromy Deformations of eigen local systems attached to abelian schemes of $\text{GL}_2$-type over $\mathbb P^1-\{0,\,1,\,\lambda,\,\infty\}$ over $\mathbb C$}

In \autoref{thm_family_GL2_component}, for a given supersingular $\lambda_0$, we found an abelian scheme $h_{\lambda_0}': B'_{\lambda_0}\to \mathbb P^1$ of $\text{GL}_2(\EK)$-type and with bad reduction on $\{0,\,1,\,\lambda_0,\,\infty \} $ of type $(1/2)_\infty$, with the eigen sheave decomposition of the filtered parabolic de Rham bundle
\[(V,\nabla,E^{1,0},\Phi)_{h'_{\lambda_0}}=\bigoplus_{i=1}^{g}(V,\nabla,E^{1,0},\Phi)_{h'_{\lambda_0} i},\]
where $E^{1,0}_{h'\,i}\simeq \mL$ and $V_{h'\,i}/ E^{1,0}_{h'\,i}\simeq \mL^{-1}$ for all $i=1,\cdots,g$. In this subsection we study the deformation of $h'_{\lambda_0}$ with $\lambda $ moving in $M_{0,4}$.

\begin{lemma} \label{thm_formal_deform_over_M04}
Let $\hat {U}_{\lambda_0}\subset M_{0,4}$ be a formal neighborhood of $\lambda_0$. Denote by \[S_{4,0}|_{\hat U_{\lambda_0}}\to \hat U_{\lambda_0}\] the pullback of the universal family. Then
\begin{enumerate}
\item by the isomonodromy deformation of the local system $\mathbb V_{h'_{\lambda_0}}$ (or the realization of the crystal attached to $h'_{\lambda_0}$) there is a natural lifting of the de Rham bundle on $S_{0,4}|_{\hat U_{\lambda_0}}$ endowed with an $\EK$-action
\[(V,\nabla,\Phi)_{ S_{0,4}|_{\hat U_{\lambda_0}}}=\bigoplus_{i=1}^{g}(V,\nabla,\Phi)_{ S_{0,4}|_{\hat U_{\lambda_0}} i}.\]
\item The obstruction for lifting the Hodge filtration
\[E^{1,0}_{h'_{\lambda_0}} =\bigoplus_{i=1}^{g} E^{1,0}_{{h'_{\lambda_0} i}}\subset (V,\nabla, \Phi)_{h'_{\lambda_0}}\]
to a sub bundle of $ (V,\nabla,\Phi)_{ S_{0,4}|_{\hat U_{\lambda_0}}}$ vanishes.
\end{enumerate}
\end{lemma}

\begin{proof}
Since the obstruction for lifting the Hodge filtration
\[E^{1,0}_{h'_{\lambda_0}} =\bigoplus_{i=1}^{g} E^{1,0}_{{h'_{\lambda_0} i}}\subset (V,\nabla, \Phi)_{h'_{\lambda_0}}\]
to a sub bundle of $ (V,\nabla,\Phi)_{ S_{0,4}|_{\hat U_{\lambda_0}}}$
lies in
$$ \bigoplus_{i=1}^{g}H^1(\mathbb P^1,E^{1,0\vee} _{h_{\lambda_0} i} \otimes (V_{h'\,i}/ E^{1,0}_{{h'_0}i})) =
\bigoplus_{i=1}^{g}H^1(\mathbb P^1,\mathcal O(-1) )=0, $$
we obtain a lifting of $E^{1,0}_{h'_{\lambda_0}}$
\[ (V,\nabla,E^{1,0},\Phi)_{S_{0,4}|_{\hat U_{\lambda_0}}}=\bigoplus_{i=1}^{g}(V,\nabla,E^{1,0},\,\Phi)_{ S_{0,4}|_{\hat U_{\lambda_0}}i}\]
and with an $\EK$-multiplication.
\end{proof}

Similar to we have done in \autoref{sec_main_lifting}, the lifting of the Hodge filtration shall lead a lifting some classifying mapping. As in \autoref{thm_add_level}, we add the full $3$-level structure to $h'_{\lambda_0}$, then one gets a finite covering mapping
\[\pi_{\lambda_0}: (C,D)_{\lambda_0}\to (\bP^1,\{0,\,1,\,\lambda_0,\,\infty\})\]
ramified only at the punctures. Now, we vary $\lambda$ in $M_{0,4}$, the covering can be extended locally around $\lambda=\lambda_0$. Thus after passing through a finite \'etale covering $\widetilde{M_{0,4}}$ of $M_{0,4}$, we extend the covering globally and get a finite cover $\pi$ as in the following Cartier diagram
\begin{equation*}
\xymatrix{
(C,D)_{\lambda_0} \ar[r] \ar[d]^{\pi_{\lambda_0}} & C_{\widetilde{S}_{0,4}} \ar[d]^{\pi} \\
(\bP^1,\{0,\,1,\,\lambda_0,\,\infty\}) \ar[r] \ar[d] & \widetilde{S}_{0,4} \ar[d] \ar[r] & S_{0,4} \ar[d]\\
\{\lambda_0\} \ar@{^(->}[r] & \widetilde{M}_{0,4} \ar[r]^{\text{finite}} & M_{0,4}\\
}
\end{equation*}

Since the base change family $h'_{\pi_{\lambda_0}}: B'_{\pi_{\lambda_0}}\to C_{\lambda_0}$ has the full $3$-level structure, which induces a log period mapping into a smooth compactification of the fine Hilbert modular variety defined by the multiplication field $\EK$
\[\psi_{h'_{\pi_{\lambda_0}}}: (C,D)_{\lambda_0}\to (\overline{\mathcal H}_{\EK},\infty).\]
Together with Grothendieck-Messing-Kato log deformation theorem, the \autoref{thm_formal_deform_over_M04} implies that there is a lifting of the period mapping over $\hat U_{\lambda_0}$
\[\psi_{\pi_{\hat U_{\lambda_0}}}: C_{\widetilde S_{0,4}}|_{\hat U_{\lambda_0}}\to (\overline{\mathcal H}_{\EK},\infty).\]

\begin{lemma}
The lifting $\psi_{\pi_{\hat U_{\lambda_0}}}$ over the formal neighborhood $\hat U_{\lambda_0}$ extends to $\widetilde{M}_{0,4}$. In other words, there exists
\[\psi \colon \widetilde S_{0,4} \to (\overline{\mathcal H}_{\EK},\infty)\]
such that $\psi_{\pi_{\hat U_{\lambda_0}}} = \psi\mid_{ C_{\widetilde S_{0,4}}|_{\hat U_{\lambda_0}}}$.
\end{lemma}
\begin{proof}
For a positive integer $d$, we consider the moduli functor
\begin{equation*}
\widetilde{M}_{0,4}\mathrm{-Sch} \longrightarrow \mathrm{Set}
\end{equation*}
defined by
\[T \mapsto \{\psi\colon C_{\widetilde{S}_{0,4}}\times_{\widetilde{M}_{0,4}} T \rightarrow (\overline{\mathcal H_{\EK}},\infty) \mid \deg\psi\leq d\}\]
Then the functor is represented by finite type $M_{0,4}$-scheme $\mathcal M_{\pi,d}$. Let $\beta$ denote the structure morphism
\[\beta: \mathcal M_{\pi,d}\to \widetilde{M}_{0,4}.\]

Let $d_0:=\deg \psi_{h'_{\pi_{\lambda_0}}},$ then the existence of the lifting $\psi_{\pi_{\hat U_{\lambda_0}}}$ implies
\[\hat U_{\lambda_0}\subset \beta(\mathcal M_{\pi,d_0})\subseteq \widetilde{M}_{0,4}.\]
Hence the constructible
subset $\beta(\mathcal M_{\pi,d_0})$ contains a non-empty Zariski open set $U\subseteq \widetilde{M}_{0,4}$. By means of the isomonodromy deformation, there exists a finite covering
\[\widetilde{\widetilde{M}}_{0,4}\rightarrow \widetilde{M}_{0,4}\]
which is \'etale over $U$ (denote by $\widetilde{U}=\psi^{-1}(U)$)
such that there exists a mapping
\[\psi_C \colon C_{\widetilde{S}_{0,4}} \times_{\widetilde{M}_{0,4}} \widetilde{U} \rightarrow (\overline{\mathcal H_{\EK}},\infty)\]
which extends $\psi_{\pi_{\hat U_{\lambda_0}}}$. By pullback the universal family of abelian varieties along $\psi_C$, one gets a family of abelian varieties \[h'_{\widetilde U}: B'_{\widetilde U}\to C_{\widetilde{S}_{0,4}} \times_{\widetilde{M}_{0,4}} \widetilde{U}.\]

Since the local system associated to this family has trivial local monodromy around the fibers $C_{\widetilde{S}_{0,4}} \times_{\widetilde{M}_{0,4}} \{\lambda\}$ for each $\lambda \in \widetilde{\widetilde{M}}_{0,4}\setminus \widetilde{U}$, we can extend the abelian scheme across those fibers and get a family
\[h'_{\widetilde{\widetilde{M}}_{0,4}}: B'_{\widetilde{\widetilde{M}}_{0,4}} \to C_{\widetilde{S}_{0,4}} \times_{\widetilde{M}_{0,4}} \widetilde{\widetilde{M}}_{0,4},\]
where $\widetilde{\widetilde M}_{0,4}\to \widetilde{M}_{0,4}$ is a finite \'etale covering. By taking Weil restriction, we descend
this abelian scheme back to $\widetilde{S}_{0,4}$ and get a mapping $\psi$.
\end{proof}

Summarizing what we have done above we prove the main theorem of this paper
\begin{theorem} \label{thm_Higgs_k_to_family_M04} \label{thm_main_Higg_mod_p_to_largest_family}
Let $L$ be a number field and let $\lambda_0 \in M_{0,4}(L)$. Assume $\frakp$ is a finite place such that $\lambda_0$ is an $\frakp$-adic integer such that $\lambda_0$ is supersingular at $\frakp$. Then any
$(\overline E,\overline\theta)\in \mathrm{HIG}_{\lambda_0}^{{\rm gr}{1\over2}}(\barFp)$ lifts uniquely to a motivic Higgs bundle $(E,\theta)$ in $\mathrm{HIG}_{\lambda_0}^{{\rm gr}{1\over2}}(\overline{\mathbb Q})$. More precisely, there exists a finite \'etale covering change $\widetilde M_{0,4}\to M_{0,4}$ (depending on $(\overline{E},\overline{\theta})$) and an abelian scheme
\[f\colon A\to \widetilde S_{0,4}\]
of $\text{GL}_2$-type,
with bad reduction on the four punctures and such that
$(E,\theta)$ is an eigen sheaf of the Higgs bundle attached to the abelian scheme $f_{\lambda_0}$.
\end{theorem}

Since there are infinitely many Higgs bundles $(\overline E,\overline\theta)\in M^{1/2}_{Hig\,\overline\lambda_0}(\overline{\mathbb F}_p)$ in \autoref{thm_main_Higg_mod_p_to_largest_family}, one gets following result.
\begin{corollary}
There exist infinitely many abelian schemes of the form given in \autoref{thm_main_Higg_mod_p_to_largest_family}
\end{corollary}

The main result of Lin-Sheng-Wang says that if $(E,\theta)$ is an eigen Higgs bundle of the Higgs bundle attached an abelian scheme $f: A\to \mathbb P^1$ with bad reduction on $\{0,\,1,\,\lambda,\,\infty\}$ then the zero of the Higgs field $(\theta)_0$ is a torsion point w.r.t. the elliptic curve $C_\lambda$. We show now the converse direction
\begin{theorem} \label{thm_main_motivic_torsion_description}
Given a 4-punctured complex projective line $(\mathbb P^1,\{0,\,1,\,\lambda,\,\infty\})$ and a Higgs bundle
$(E,\theta)\in \High(\bC)$. Assume the zero of the Higgs field is a torsion point with respect to the elliptic curve $C_\lambda\colon y^2=x(x-1)(x-\lambda)$. Then $(E,\theta)$ is motivic, and there exists a family of abelian varieties $f: A\to \mathbb P^1$ of $\GL_2$-type such that $(E,\theta)$ is an eigen Higgs bundle attached to $f$.
\end{theorem}
\begin{proof}
Let $g\colon C\to \bP^1$ be the Legendre family. Then one may identify the smooth locus of $g$ with $M_{0,4}$, the moduli space of projective line with $4$-punctures, which sends $\lambda$ to the projective line with punctures at $\{0,1,\lambda,\infty\}$. For any $\lambda\neq 0,1,\infty$, the fiber of $g$ at $\lambda$ is just the elliptic curve given by the double cover $\pi_\lambda\colon C_\lambda\to \bP^1$ ramified on $\{0,1,\lambda,\infty\}$



Assume $(E,\theta)$ is motivic. Then the modulo $\frakp$ reduction of $(E,\theta)$ is periodic for almost all places $\frakp$. According \autoref{thm_LSW}, the modulo $\frakp$ reduction of $(\theta)_0$ is torsion. By a theorem of Pink \cite{Pin04}, it itself is torsion. Conversely, assume $(\theta)_0$ is a torsion point with order $m$, in the following we show $(E,\theta)$ is motivic.

We choose a number field $K$ and an integer $\lambda_0\in \mO_K$ such that $C_{\lambda_0}$ is an elliptic curve with complex multiplication. Choose a sufficient large place $\frakp$ such that the reduction of $C_{\lambda_0}$ at $\frakp$ is supersingular and $\frakp\nmid m$. Let $\Sigma_m\subset \widetilde{C}$ be the $m$-torsion (multiple) section, $T_m=\pi(\Sigma_m)\subset \widetilde{S}_{0,4}$.
Then $T_m$ is \'etale over $\widetilde{M}_{0,4}$. Let $T'_m$ be the irreducible component of $T_m$ containing $(\theta)_0$.
\begin{equation*}
\xymatrix@C=2mm{
C_{\lambda_0} \ar@{^(->}[r] \ar[d] & \widetilde{C}\ar[r] \ar[d] & C \ar[d] \\
\bP^1\setminus\{0,1,\lambda_0,\infty\} \ar[d] \ar@{^(->}[r] & \widetilde{S}_{0,4}\ar[r] \ar[d]& S_{0,4}\ar[d]\\
\{\lambda_0\} \ar@{^(->}[r] & \widetilde{M}_{0,4} \ar[r] & M_{0,4}
}
\qquad
\xymatrix{
& C_\lambda \ar@{^(->}[r] \ar[d] & \widetilde{C} \ar@{}[r]|\supset \ar[d] & \Sigma_m \\
(\theta)_0 \ar@{}[r]|-\in & \bP^1\setminus\{0,1,\lambda,\infty\} \ar[d] \ar@{^(->}[r] & \widetilde{S}_{0,4} \ar[d] \ar@{}[r]|\supset & T'_m \\
& \{\lambda\} \ar@{^(->}[r] & \widetilde{M}_{0,4}\\
}
\end{equation*}
Choose a Higgs bundle $(E_0,\theta_0)$ in $\mathrm{HIG}_{\lambda_0}^{{\rm gr}{1\over2}}(\bC)$ with zero located in the intersection set $T'_m\cap \bP^1\setminus\{0,1,\lambda_0,\infty\} \subset \widetilde{S}_{0,4}$. Then the zero of this Higgs field $\theta_0$ is torsion of order $m$. Since the modulo $\frakp$ reduction of the Higgs bundle $(E_0,\theta_0)$ is also torsion and of order $m$ with $\frakp\nmid m$. By \autoref{thm_LSW}, the reduction $(E_0,\theta_0)\pmod{\frakp}$ is periodic. According \autoref{thm_Higgs_k_to_family_M04}, there exists a family of abelian varieties $f\colon A\to \widetilde{S}_{0,4}$ of $\GL_2$-type such that $(E_0,\theta_0)\pmod{\frakp}$ is an eigen Higgs bundle attached to $f_{\lambda}$. In other words, there is an eigen component $(E_0,\theta_0)^{per}$ of the Higgs bundle attached to the family $f$ such that \[(E_0,\theta_0)^{per}\mid_{(\bP^1,\{0,1,\lambda_0,\infty\})} \pmod{\frakp} = (E_0,\theta_0) \pmod{\frakp}.\]
Since $(E_0,\theta_0)^{per}$ comes from families of abelian varieties, it is motivic. Thus the zero of the Higgs field is an algebraic section consisting of torsion points.

We claim that the torsion points in the section are all of order $m$. By the constancy of the order, we only need to show the order $(E_0,\theta_0)^{per}\mid_{(\bP^1,\{0,1,\lambda_0,\infty\})}=(E_0,\theta_0)$. Since modulo $\frakp$ mapping is injective for torsion points of order coprime to $p$, we only need to show the order of $\theta_0^{per}$ is coprime to $p$. This follows the fact that $C_{\lambda_0}$ is endowed with complex multiplication. Because the number field generated by an $p$-torsion point is unramified over $\Qp$, and the field generated by the zero of $\theta_0^{per}\mid_{(\bP^1,\{0,1,\lambda_0,\infty\})}$ is unramified.

Thus torsion section $(\theta^{per}_0)_0$ passes through $(\theta_0)_0$. By the choice of $(E_0,\theta_0)$, it also passes through $(\theta)_0$. Hence $(E,\theta)$ is motivic.
\end{proof}



\appendix
\section{Arithmetic Simpson Correspondence and $GL_2$-Motivic Local Systems over $\mathbb P^1\setminus\{0,1,\lambda,\infty\}$}

This appendix is a record of conference report given in the conference\\
\begin{center}
{\tiny \blue https://irma.math.unistra.fr/\~{}lfu/Activities/Sino-French\%20AG\%20Conference.html}
\end{center}
by the second author.



\autoref{conj:SYZ} predicts that 

\begin{itemize}
		\item there exists $26$ (classes of) families of elliptic curves $f:\mathcal Y \rightarrow \mathbb P^1$ with bad reduction over $D=\{0,1,\lambda,\infty\}$ and the set of zeros $\{(\tau)_0\}$ of the Kodaira-Spence maps equals  to $C_\lambda^{\mathrm{tor}}/\{\pm1\}$ of orders $1,2,3,4,6$. 
		\item in general, there exists families of $g$-dimensional abelian varieties $f:\mathcal Y \rightarrow \mathbb P^1$ endowed with real multiplication $L$, with bad reduction over $D$,  the set of zeros $\{(\tau)_0\}$ of the Kodaira-Spence maps equals to $C_\lambda^{\mathrm{tor}}/\{\pm1\}$ of orders $d$ and such that $[\mathbb Q(\zeta_d)^+:\mathbb Q]=g$. 
	\end{itemize}  


\newlength\savedwidth
\newcommand\whline{\noalign{\global\savedwidth\arrayrulewidth
		\global\arrayrulewidth 3pt}
	\hline
	\noalign{\global\arrayrulewidth\savedwidth}}
\newlength\savewidth
\newcommand\shline{\noalign{\global\savewidth\arrayrulewidth
		\global\arrayrulewidth 1.5pt}
	\hline
	\noalign{\global\arrayrulewidth\savewidth}}
 
	{\scriptsize 
	\begin{center}
		\begin{tabular}{|c|c|c|c|c|c|}
			\hline
			$g$  & $d=$ order of $(\tau)_0$  & $L=\mathbb Q(\zeta_d)^+$ &  number (of classes) of families \\ \shline
			$1$ & 1,\,2,\,3,\,4,\,6  & $\mathbb Q$   & $26$ \\ \shline
			\multirow{4}{*}{2} & 5 & $\mathbb Q(\sqrt{5})$ & 6 \\ \cline{2-4}
			&   8                &$\mathbb Q(\sqrt{2})$& 12\\ \cline{2-4}
			&    10               &$\mathbb Q(\sqrt{10})$& 18 \\ \cline{2-4}
			&      12             &$\mathbb Q(\sqrt{3})$& 24 \\ \shline
			\multirow{4}{*}{3} & 7 & $\mathbb Q(\zeta_{7}+\zeta_{7}^{-1})$ & 8 \\ \cline{2-4}
			& 9 & $\mathbb Q(\zeta_{9}+\zeta_{9}^{-1})$   & 12 \\ \cline{2-4}
			& 14 &  $\mathbb Q(\zeta_{14}+\zeta_{14}^{-1})$ & 24\\ \cline{2-4}
			& 18 &  $\mathbb Q(\zeta_{18}+\zeta_{18}^{-1})$ & 36 \\ \shline   
			\multirow{5}{*}{4} & 15  & $\mathbb Q(\zeta_{15}+\zeta_{15}^{-1})$ & 24
			\\ \cline{2-4}
			&   16               & $\mathbb Q(\zeta_{16}+\zeta_{16}^{-1})$ &       24            \\ \cline{2-4}
			&  20                & $\mathbb Q(\zeta_{20}+\zeta_{20}^{-1})$ &     36              \\ \cline{2-4}
			&  24                & $\mathbb Q(\zeta_{24}+\zeta_{24}^{-1})$ &     48              \\ \cline{2-4}
			&   30               & $\mathbb Q(\zeta_{30}+\zeta_{30}^{-1})$ &     72              \\ \hline
			$\vdots$&  $\vdots$ & $\vdots$ & $\vdots$  \\ \hline 
		\end{tabular} 
	\end{center}
}


J. Lu, X. Lv and J. Yang found that there indeed exist $26$ (classes of) families of elliptic curves, which are list as in the following table  

	\renewcommand\arraystretch{2}
	\begin{adjustwidth}{-6mm}{0mm} 
		\tiny
		\begin{tabular}{|c|l|c|l|}
			\cline{1-3}
			{order of $(\tau)_0$} &  elliptic curves $\mathcal Y/\mathbb P^1$ with bad reductions over  $\{0,1,\infty,\lambda\}$ & $a$ \\ \shline
			{1} & $y^2=x(x-t+\lambda)(x-t+\lambda t)$ & $-$ \\ \shline
			\multirow{3}{*}{2} & $y^2=(x-1)(x-\lambda)(x-t)$ & $-$ \\ \cline{2-3}
			& $y^2=x(x-\lambda)(x-t)$ & $-$ \\ \cline{2-3}
			& $y^2=x(x-1)(x-t)$ & $-$ \\ \shline
			{3} & $y^2=x^3+\frac{(a-3)^2t-4a}{4(a-1)}x^2-\frac{a-3}{2}tx+\frac{a-1}{4}t$ & $\lambda(a+1)(a-3)^3+16a^3=0$ \\ \shline
			\multirow{3}{*}{4} & $y^2=x^3+4(t-a)x^2+(t-1)(t-\lambda)x $ & $a^2-\lambda=0$ \\ \cline{2-3}
			& $y^2=x^3+4(t-a)x^2+t(t-\lambda)x$ & $a^2-2a+\lambda=0$ \\ \cline{2-3}
			& $y^2=x^3+4(t-a)x^2+(t^2-t)x$ & $a^2-2\lambda a+\lambda=0$ \\ \shline
			\multirow{3}{*}{6} & $y^2=(1-t)x^3+\frac{(a-3)^2-4a(1-t)}{4(a-1)}x^2-\frac{a-3}{2}x+\frac{a-1}{4}$ & $(a+1)(a-3)^3+16(1-\lambda) a^3=0$ \\ \cline{2-3}
			& $y^2=(\lambda-t)x^3+\frac{(a-3)^2\lambda-4a(\lambda-t)}{4(a-1)}x^2-\frac{a-3}{2}\lambda x+\frac{a-1}{4}\lambda$ & $\lambda(a+1)(a-3)^3+16(\lambda-1)a^3=0$ \\ \cline{2-3}
			& $y^2=tx^3+\frac{(a-3)^2-4at}{4(a-1)}x^2-\frac{a-3}{2}x+\frac{a-1}{4}$ & $(a+1)(a-3)^3+16\lambda a^3=0$ \\ \cline{1-3}
		\end{tabular} 
	\end{adjustwidth}



\def\fai{\varphi_{\lambda,{p}}}


The self map $\fai=\mathrm{Gr}\circ \mathcal C_{1,2}^{-1}$ on $\mathcal M_{\mathbb{F}_q}\simeq \mathbb P^1_{\mathbb{F}_q}$ induced by Higgs-de Rham flow
has the following explicit form 
\begin{equation*}
\fai(z)=\frac{z^p}{\lambda^{p-1}}\cdot\left( \frac{f_\lambda(z^p)}{g_\lambda(z^p)}\right)^2,
\end{equation*}
where {\tiny
\[ \makebox[-5mm]{} f_\lambda(z^p)=\det\left(\begin{array}{ccccc}  \frac{\lambda^p(1-z^p)-(\lambda^p-z^p)\lambda^{2}}{2}&
\frac{\lambda^p(1-z^p)-(\lambda^p-z^p)\lambda^{3}}{3}  &\cdots&
\frac{\lambda^p(1-z^p)-(\lambda^p-z^p)\lambda^{(p+1)/2}}{(p+1)/2} \\ \frac{\lambda^p(1-z^p)-(\lambda^p-z^p)\lambda^{3}}{3} &
\frac{\lambda^p(1-z^p)-(\lambda^p-z^p)\lambda^{4}}{4} &\cdots&
\frac{\lambda^p(1-z^p)-(\lambda^p-z^p)\lambda^{(p+3)/2}}{(p+3)/2} \\   \vdots&\vdots&\ddots&\vdots\\      \frac{\lambda^p(1-z^p)-(\lambda^p-z^p)\lambda^{(p+1)/2}}{(p+1)/2} &
\frac{\lambda^p(1-z^p)-(\lambda^p-z^p)\lambda^{(p+3)/2}}{(p+3)/2} &\cdots&
\frac{\lambda^p(1-z^p)-(\lambda^p-z^p)\lambda^{p-1}}{p-1} \\  
\end{array} \right)\] }
and 
{\tiny \[\makebox[-5mm]{} g_\lambda(z^p)=\det\left(\begin{array}{ccccc}  \frac{\lambda^p(1-z^p)-(\lambda^p-z^p)\lambda^1}{1}&
\frac{\lambda^p(1-z^p)-(\lambda^p-z^p)\lambda^{2}}{2}  &\cdots&
\frac{\lambda^p(1-z^p)-(\lambda^p-z^p)\lambda^{(p-1)/2}}{(p-1)/2} \\ \frac{\lambda^p(1-z^p)-(\lambda^p-z^p)\lambda^{2}}{2} &
\frac{\lambda^p(1-z^p)-(\lambda^p-z^p)\lambda^{3}}{3} &\cdots&
\frac{\lambda^p(1-z^p)-(\lambda^p-z^p)\lambda^{(p+1)/2}}{(p+1)/2} \\   \vdots&\vdots&\ddots&\vdots\\      \frac{\lambda^p(1-z^p)-(\lambda^p-z^p)\lambda^{(p-1)/2}}{(p-1)/2} &
\frac{\lambda^p(1-z^p)-(\lambda^p-z^p)\lambda^{(p+1)/2}}{(p+1)/2} &\cdots&
\frac{\lambda^p(1-z^p)-(\lambda^p-z^p)\lambda^{p-2}}{p-2} \\  
\end{array} \right).\]} 











\begin{equation*}
\begin{split}
&\varphi_{\lambda,3}(z)= z^3 \left(\frac{z^3+\lambda(\lambda+1)}{(\lambda+1)z^3+\lambda^2}\right)^2;\\
& \varphi_{-1,3}(z)=z^{3^2};\\
&\makebox[16cm]{}\\
& \varphi_{\lambda,5}(z)= z^5\left(\frac{z^{10}-\lambda(\lambda+1)(\lambda^2-\lambda+1)z^5+\lambda^4(\lambda^2-\lambda+1)}{(\lambda^2-\lambda+1)z^{10}-\lambda^2(\lambda+1)(\lambda^2-\lambda+1)z^5+\lambda^6 }\right)^2;\\
&\varphi_{\lambda,5}(z)=z^{5^2} \text{ if and only if  $\lambda$ is a $6$-th primitive root of unit}; \\
\end{split}
\end{equation*}  




For  $k=\bF_{3^4}$ and  $\lambda\in k\setminus\{0,1\}$, the map $\varphi_{\lambda,3}$ is a self $k$-morphism on $\mathbb P^1_k$
\[\varphi_{\lambda,3}:k\cup \{\infty\}\rightarrow k\cup \{\infty\}.\] 
For  $\alpha=\sqrt{1+\sqrt{-1}}$ as  a  generator of $k=\bF_{3^4}$ over $\bF_3$, every elements in $k$ can be uniquely expressed in form $a_3\alpha^3+a_2\alpha^2+a_1\alpha+a_0$, where $a_3,a_2,a_1,a_0\in\{0,1,2\}$. \\[.2cm]
We use the integer $3^3a_3+3^2a_2+3a_1+a_0\in [0,80]$ stand for the element $a_3\alpha^3+a_2\alpha^2+a_1\alpha+a_0\in k$
§\[\varphi_{\lambda,3}:\{0,1,2,\cdots,80,\infty\}\rightarrow \{0,1,2,\cdots,80,\infty\}.\]§





\begin{center}
	\begin{tikzpicture}
	[L1Node/.style={circle,draw=black!50, very thick, minimum size=7mm}]
	\node[L1Node] (n1) at (-2, 1){$21$}; 
	\draw [thick,->](-1.62,0.8) -- (-0.4,0.2);
	\node[L1Node] (n1) at (-2.2, 0){$43$};
	\draw [thick,->](-1.8,0) -- (-0.45,0);
	\node[L1Node] (n1) at (-2, -1){$54$};
	\draw [thick,->](-1.62,-0.8) -- (-0.4,-0.2); 
	\node[L1Node] (n1) at (0, 0){$27$}; 
	\draw [thick,->](0.4,0) -- (1.6,0);
	\node[L1Node] (n1) at (2, 0){$~6\,$};  
	\draw [thick,->](2.3,0.3) .. controls (4,2) and (4,-2) .. (2.35,-0.35);   
	\end{tikzpicture}	
\end{center} 
  \[\rho: \pi_1\left(\mathbb P_{W(\bF_{3^4})[1/3]}^1\setminus\left\{0,1,\infty,2\sqrt{1+\sqrt{-1}}\right\}\right)\longrightarrow \mathrm{GL}_2(\bF_3).\]



\begin{center}
	\begin{tikzpicture}
	[L1Node/.style={circle,draw=black!50, very thick, minimum size=7mm}]
	\node[L1Node] (n1) at (0, 1){$47$}; 
	\draw [thick,->](0.38,0.8) -- (1.6,0.2);
	\node[L1Node] (n1) at (0, -1){$60$};
	\draw [thick,->](0.38,-0.8) -- (1.6,-0.2);
	\node[L1Node] (n1) at (2, 0){$31$};
	\draw [thick,->](2.3,0.3) .. controls (3,1) and (4,1) .. (4.65,0.35); 
	\node[L1Node] (n1) at (7, 1){$35$}; 
	\draw [thick,->](6.62,0.8) -- (5.4,0.2);
	\node[L1Node] (n1) at (7, -1){$57$};
	\draw [thick,->](6.62,-0.8) -- (5.4,-0.2);
	\node[L1Node] (n1) at (5, 0){$15$};
	\draw [thick,->](4.7,-0.3) .. controls (4,-1) and (3,-1) .. (2.35,-0.35); 
	\end{tikzpicture}
\end{center}
\[\rho:       \pi_1\left(\mathbb P_{W(\bF_{3^4})[1/3]}^1\setminus\left\{0,1,\infty,\sqrt{-1}\right\}\right)\longrightarrow \mathrm{GL}_2(\bF_{3^2});\] 











\begin{center}
	\begin{tikzpicture}
	[L1Node/.style={circle,draw=black!50, very thick, minimum size=7mm}]
	\node[L1Node] (n1) at (-1,-2.4){$21$}; 
	\draw [thick,->](-0.6,-2.4) -- (0.5,-2.4);
	\node[L1Node] (n1) at (1,-2.4){$64$}; 
	\draw [thick,->](1.3,-2.1) -- (2.05,-1.35);
	\node[L1Node] (n1) at (2.4,-1){$48$}; 
	\draw [thick,->](2.4,-0.6) -- (2.4,0.5);
	\node[L1Node] (n1) at (2.4,1){$53$}; 
	\draw [thick,->](2.1,1.3) -- (1.35,2.05);
	\node[L1Node] (n1) at (1,2.4){$24$}; 
	\draw [thick,->](0.6,2.4) -- (-0.5,2.4);
	\node[L1Node] (n1) at (-1,2.4){$37$}; 
	\draw [thick,->](-1.3,2.1) -- (-2.05,1.35);
	\node[L1Node] (n1) at (-2.4,1){$78$}; 
	\draw [thick,->](-2.4,0.6) -- (-2.4,-0.5);
	\node[L1Node] (n1) at (-2.4,-1){$77$}; 
	\draw [thick,->](-2.1,-1.3) -- (-1.35,-2.05);
	\end{tikzpicture}
\end{center}
\[ \rho: \pi_1\left(\mathbb P_{W(\bF_{3^8})[1/3]}^1\setminus\left\{0,1,\infty,\sqrt{-1}\right\}\right)\longrightarrow \mathrm{GL}_2(\bF_{3^8}).\]




\def\hzero{0}\def\hone{1.2}\def\htwo{2.4}\def\hthree{3.6}\def\hfour{4.8}\def\hfive{5.4}
\def\hsix{5.76}\def\hseven{6}\def\height{6.18}\def\hend{7.3}\def\hinf{7.5}
\def\wzero{0}\def\wone{2}\def\wtwo{4}\def\wthree{6}\def\wfour{8}\def\wfive{9}\def\wsix{9.6}
\def\wseven{10}\def\weight{10.3}\def\wend{12.3}\def\winf{12.5} 
\def\rang{-60}

\begin{adjustwidth}{-8mm}{0mm}
\begin{center}
	\begin{tikzpicture}
	\filldraw[black] (\wzero,\hzero) circle (2pt) node[below]{};
	\draw[black, thick,->]  (\wzero,\hzero) -- (\wzero,\hend) node[black,below right]{$\mathrm{GL(\hat{\mathbb Z})}$};
	
	\filldraw[blue] (\wzero,\hone) circle (2pt) node[black,left] {$\mathrm{GL}(\mathbb Q_2)$};
	\draw[blue, thick]  (\wzero,\hone) -- (\wend,\hone);
	
	\filldraw[blue] (\wzero,\htwo) circle (2pt) node[black,left] {$\mathrm{GL}(\mathbb Q_3)$};
	\draw[blue, thick]  (\wzero,\htwo) -- (\wend,\htwo);
	
	\filldraw[blue] (\wzero,\hthree) circle (2pt) node[black,left] {$\mathrm{GL}(\mathbb Q_5)$};
	\draw[blue, thick]  (\wzero,\hthree) -- (\wend,\hthree);
	
	\filldraw[blue] (\wzero,\hfour) circle (2pt) node[black,left] {$\mathrm{GL}(\mathbb Q_p)$};
	\draw[blue, thick]  (\wzero,\hfour) -- (\wend,\hfour);
	
	\filldraw[blue] (\wzero,\hfive) circle (1.5pt) node[black,left] {$\mathrm{GL}(\mathbb Q_\ell)$};
	\draw[blue, thick]  (\wzero,\hfive) -- (\wend,\hfive);
	\node[blue] at (3.5,5.5){$\ell$-adic representation(Fontaine-Mazur)};	
	\filldraw[blue] (\wzero,\hsix) circle (1.4pt);
	\filldraw[blue] (\wzero,\hseven) circle (1.2pt);
	\filldraw[blue] (\wzero,\height) circle (1pt);
	
	\filldraw[blue] (\wzero,\hinf) circle (2pt) node[black,left] {$\mathrm{GL}(\mathbb C)$};
	\draw[blue, thick]  (\wzero,\hinf) -- (\winf,\hinf);
	\draw[black, thick,->] (\wzero,\hzero) -- (\wend,\hzero) node[above left]{$\pi^{\mathrm{et}}_1(X/\mathbb Q,*)$};
	
	\begin{rotate}{-90}{ }\end{rotate}
	
	\filldraw[green] (\wone,\hzero) circle (2pt) node[black,below] {\tiny $\pi^{\mathrm{et}}_1(X/\mathbb{Q}_2,*)$ };
	\draw[green, thick]  (\wone,0) -- (\wone,\hend);
	
	\filldraw[green] (\wtwo,\hzero) circle (2pt) node[black,below] {\tiny $\pi^{\mathrm{et}}_1(X/\mathbb{Q}_3,*)$ };
	\draw[green, thick]  (\wtwo,\hzero) -- (\wtwo,\hend);
	
	\filldraw[green] (\wthree,\hzero) circle (2pt) node[black,below] {\tiny $\pi^{\mathrm{et}}_1(X/\mathbb{Q}_5,*)$ };
	
	\draw[green, thick]  (\wthree,\hzero) -- (\wthree,\hend);
	
	\filldraw[green] (\wfour,\hzero) circle (2pt) node[black,below] {\tiny $\pi^{\mathrm{et}}_1(X/\mathbb{Q}_p,*)$ };
	\draw[green, thick]  (\wfour,\hzero) -- (\wfour,\hend);
	
	\filldraw[green] (\wfive,\hzero) circle (1.5pt);
	\draw[green, thick]  (\wfive,\hzero) -- (\wfive,\hend);
	\filldraw[green] (\wsix,\hzero) circle (1.4pt);
	\filldraw[green] (\wseven,\hzero) circle (1.2pt);
	\filldraw[green] (\weight,\hzero) circle (1pt);
	
	\filldraw[green] (\winf,\hzero) circle (2pt) node[black,below] {\tiny $\pi^{\mathrm{top}}_1(X/\mathbb{C},*)$};
	\draw[green, thick]  (\winf,\hzero) -- (\winf,\hinf);
	\draw[red, thick] (\wzero,\hzero) -- (13,7.8); 
	\filldraw[black] (\wone,\hone) circle (2pt) node[below right] {Crystalline};
	\filldraw[black] (\wone,\htwo) circle (1.5pt);
	\filldraw[black] (\wone,\hthree) circle (1.5pt);
	\filldraw[black] (\wone,\hfour) circle (1.5pt); 
	\filldraw[black] (\wone,\hfive) circle (1.5pt);
	\filldraw[black] (\wone,\hsix) circle (1.4pt);
	\filldraw[black] (\wone,\hseven) circle (1.2pt); 
	\filldraw[black] (\wone,\height) circle (1pt); 
	
	\filldraw[black] (\wtwo,\hone) circle (1.5pt);
	\filldraw[black] (\wtwo,\htwo) circle (2pt) node[below right] {Crystalline};
	\filldraw[black] (\wtwo,\hthree) circle (1.5pt);
	\filldraw[black] (\wtwo,\hfour) circle (1.5pt); 
	\filldraw[black] (\wtwo,\hfive) circle (1.5pt);
	\filldraw[black] (\wtwo,\hsix) circle (1.4pt);
	\filldraw[black] (\wtwo,\hseven) circle (1.2pt); 
	\filldraw[black] (\wtwo,\height) circle (1pt); 
	
	\filldraw[black] (\wthree,\hone) circle (1.5pt);
	\filldraw[black] (\wthree,\htwo) circle (1.5pt);
	\filldraw[black] (\wthree,\hthree) circle (2pt) node[below right] {Crystalline};
	\filldraw[black] (\wthree,\hfour) circle (1.5pt); 
	\filldraw[black] (\wthree,\hfive) circle (1.5pt);
	\filldraw[black] (\wthree,\hsix) circle (1.4pt);
	\filldraw[black] (\wthree,\hseven) circle (1.2pt); 
	\filldraw[black] (\wthree,\height) circle (1pt); 
	
	\filldraw[black] (\wfour,\hone) circle (1.5pt);
	\filldraw[black] (\wfour,\htwo) circle (1.5pt);
	\filldraw[black] (\wfour,\hthree) circle (1.5pt);
	\filldraw[black] (\wfour,\hfour) circle (2pt) node[below right] {Crystalline}; 
	\filldraw[black] (\wfour,\hfive) circle (1.5pt);
	\filldraw[black] (\wfour,\hsix) circle (1.4pt);
	\filldraw[black] (\wfour,\hseven) circle (1.2pt); 
	\filldraw[black] (\wfour,\height) circle (1pt); 
	
	\filldraw[black] (\wfive,\hone) circle (1.5pt);
	\filldraw[black] (\wfive,\htwo) circle (1.5pt);
	\filldraw[black] (\wfive,\hthree) circle (1.5pt);
	\filldraw[black] (\wfive,\hfour) circle (1.5pt); 
	\filldraw[black] (\wfive,\hfive) circle (1.5pt);
	
	\filldraw[black] (\wsix,\hone) circle (1.4pt);
	\filldraw[black] (\wsix,\htwo) circle (1.4pt);
	\filldraw[black] (\wsix,\hthree) circle (1.4pt);
	\filldraw[black] (\wsix,\hfour) circle (1.4pt); 
	\filldraw[black] (\wsix,\hsix) circle (1.4pt);
	
	\filldraw[black] (\wseven,\hone) circle (1.2pt);
	\filldraw[black] (\wseven,\htwo) circle (1.2pt);
	\filldraw[black] (\wseven,\hthree) circle (1.2pt);
	\filldraw[black] (\wseven,\hfour) circle (1.2pt); 
	\filldraw[black] (\wseven,\hseven) circle (1.2pt);
	
	\filldraw[black] (\weight,\hone) circle (1pt);
	\filldraw[black] (\weight,\htwo) circle (1pt);
	\filldraw[black] (\weight,\hthree) circle (1pt);
	\filldraw[black] (\weight,\hfour) circle (1pt);  
	\filldraw[black] (\weight,\height) circle (1pt);  
	\filldraw[black] (\winf,\hinf) circle (2pt) node[below right] {Hodge Type};
	\node[red] at (1,0.8){\begin{rotate}{31}{Arithmetic Fontaine-Faltings module}\end{rotate}};
	\draw[green, thick,->] (9.4,3)--(\wfive,3);
	\node[green] at (12.3,3){$\ell$ to $\ell'$(or $p$) companions(Deligne)}; 
	\end{tikzpicture}
\end{center}
\end{adjustwidth}

















\begin{thebibliography}{{Ked}07}

\bibitem[Abe18]{Abe18}
Tomoyuki Abe.
\newblock Langlands correspondence for isocrystals and the existence of
  crystalline companions for curves.
\newblock {\em J. Amer. Math. Soc.}, 31(4):921--1057, 2018.

\bibitem[Bea82]{Bea82}
Arnaud Beauville.
\newblock Les familles stables de courbes elliptiques sur {${\bf P}^{1}$}
  admettant quatre fibres singuli\`eres.
\newblock {\em C. R. Acad. Sci. Paris S\'{e}r. I Math.}, 294(19):657--660,
  1982.

\bibitem[CS08]{KeSi08}
Kevin Corlette and Carlos Simpson.
\newblock On the classification of rank-two representations of quasiprojective
  fundamental groups.
\newblock {\em Compos. Math.}, 144(5):1271--1331, 2008.

\bibitem[Dri77]{Dri77}
V.~G. Drinfeld.
\newblock Elliptic modules. {II}.
\newblock {\em Mat. Sb. (N.S.)}, 102(144)(2):182--194, 325, 1977.

\bibitem[EG18]{EsGr18}
H\'{e}l\`ene Esnault and Michael Groechenig.
\newblock Cohomologically rigid local systems and integrality.
\newblock {\em Selecta Math. (N.S.)}, 24(5):4279--4292, 2018.

\bibitem[EG20]{EsGr20}
H\'{e}l\`ene Esnault and Michael Groechenig.
\newblock Rigid connections and {$F$}-isocrystals.
\newblock {\em Acta Math.}, 225(1):103--158, 2020.

\bibitem[Fal89]{Fal89}
Gerd Faltings.
\newblock Crystalline cohomology and {$p$}-adic {G}alois-representations.
\newblock In {\em Algebraic analysis, geometry, and number theory ({B}altimore,
  {MD}, 1988)}, pages 25--80. Johns Hopkins Univ. Press, Baltimore, MD, 1989.

\bibitem[FC90]{FaCh90}
Gerd Faltings and Ching-Li Chai.
\newblock {\em Degeneration of abelian varieties}, volume~22 of {\em Ergebnisse
  der Mathematik und ihrer Grenzgebiete (3) [Results in Mathematics and Related
  Areas (3)]}.
\newblock Springer-Verlag, Berlin, 1990.
\newblock With an appendix by David Mumford.

\bibitem[FO22]{FoOu}
Jean-Marc Fontaine and Yi~Ouyang.
\newblock {\em Theory of p-adic Galois Representations}.
\newblock 2022.

\bibitem[IS07]{IySi07}
Jaya N.~N. Iyer and Carlos~T. Simpson.
\newblock A relation between the parabolic {C}hern characters of the de {R}ham
  bundles.
\newblock {\em Math. Ann.}, 338(2):347--383, 2007.

\bibitem[Kat89]{Kat89}
Kazuya Kato.
\newblock Logarithmic structures of {F}ontaine-{I}llusie.
\newblock In {\em Algebraic analysis, geometry, and number theory ({B}altimore,
  {MD}, 1988)}, pages 191--224. Johns Hopkins Univ. Press, Baltimore, MD, 1989.

\bibitem[{Ked}07]{Ked07}
Kiran~S. {Kedlaya}.
\newblock Semistable reduction for overconvergent \(f\)-isocrystals. i:
  Unipotence and logarithmic extensions.
\newblock {\em Compos. Math.}, 143(5):1164--1212, 2007.

\bibitem[Ked22]{Ked22}
Kiran~S. Kedlaya.
\newblock Notes on isocrystals.
\newblock {\em J. Number Theory}, 237:353--394, 2022.

\bibitem[KP22]{KrPa22}
Raju Krishnamoorthy and Ambrus P{\'a}l.
\newblock {Rank 2 local systems and Abelian varieties II}.
\newblock {\em Compositio Mathematica}, 158(4):868--892, 2022.

\bibitem[KS20]{KrSh20}
Raju Krishnamoorthy and Mao Sheng.
\newblock Periodic de rham bundles over curves, 2020.
\newblock arXiv:2011.03268.

\bibitem[KYZ20]{KYZ20D}
Raju Krishnamoorthy, Jinbang Yang, and Kang Zuo.
\newblock Deformation theory of periodic higgs-de rham flows, 2020.
\newblock arXiv:2005.00579.

\bibitem[KYZ22]{KYZ22}
Raju Krishnamoorthy, Jinbang Yang, and Kang Zuo.
\newblock Constructing abelian varieties from rank 2 galois representations,
  2022.
\newblock arXiv:2208.01999.

\bibitem[LS18]{LaSi18}
Adrian Langer and Carlos Simpson.
\newblock Rank 3 rigid representations of projective fundamental groups.
\newblock {\em Compos. Math.}, 154(7):1534--1570, 2018.

\bibitem[LSW22]{LSW22}
Xiaojin Lin, Mao Sheng, and Jianping Wang.
\newblock A torsion property of the zero of kodaira-spencer over $\mathbb{P}^1$
  removing four points, 2022.
\newblock arXiv:2212.02038.

\bibitem[LSZ19]{LSZ19}
Guitang Lan, Mao Sheng, and Kang Zuo.
\newblock Semistable {H}iggs bundles, periodic {H}iggs bundles and
  representations of algebraic fundamental groups.
\newblock {\em J. Eur. Math. Soc. (JEMS)}, 21(10):3053--3112, 2019.

\bibitem[LW10]{LiWa10}
Chang-Shou Lin and Chin-Lung Wang.
\newblock Elliptic functions, {G}reen functions and the mean field equations on
  tori.
\newblock {\em Ann. of Math. (2)}, 172(2):911--954, 2010.

\bibitem[OV07]{OgVo07}
A.~Ogus and V.~Vologodsky.
\newblock Nonabelian {H}odge theory in characteristic {$p$}.
\newblock {\em Publ. Math. Inst. Hautes \'Etudes Sci.}, (106):1--138, 2007.

\bibitem[Pin04]{Pin04}
Richard Pink.
\newblock On the order of the reduction of a point on an abelian variety.
\newblock {\em Math. Ann.}, 330(2):275--291, 2004.

\bibitem[Sim92]{Sim92}
Carlos~T. Simpson.
\newblock Higgs bundles and local systems.
\newblock {\em Inst. Hautes \'{E}tudes Sci. Publ. Math.}, (75):5--95, 1992.

\bibitem[{Sta}17]{stacks-project}
The {Stacks Project Authors}.
\newblock exit{Stacks Project}.
\newblock \url{http://stacks.math.columbia.edu}, 2017.

\bibitem[SYZ22]{SYZ22}
Ruiran Sun, Jinbang Yang, and Kang Zuo.
\newblock Projective crystalline representations of \'{e}tale fundamental
  groups and twisted periodic {H}iggs--de {R}ham flow.
\newblock {\em J. Eur. Math. Soc. (JEMS)}, 24(6):1991--2076, 2022.

\bibitem[SZ12]{ShZu12}
Mao Sheng and Kang Zuo.
\newblock Periodic higgs subbundles in positive and mixed characteristic, 2012.
\newblock arXiv:1206.4865.

\bibitem[Tri08]{Tri08}
Fabien Trihan.
\newblock A note on semistable {Barsotti}-{Tate} groups.
\newblock {\em J. Math. Sci., Tokyo}, 15(3):411--425, 2008.

\bibitem[Yu23]{Yu23}
Hongjie Yu.
\newblock Rank 2 $\ell$-adic local systems and higgs bundles over a curve,
  2023.
\newblock arXiv:2301.13157.

\bibitem[YZ23]{YaZu23a}
Jinbang Yang and Kang Zuo.
\newblock Constructing algebraic solutions of painleve vi equation from
  $p$-adic hodge theory and langlands correspondence, 2023.
\newblock arXiv:2301.10054.

\end{thebibliography}















\end{document}
