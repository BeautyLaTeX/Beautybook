\documentclass[zihao=-4,fontset=none]{Beautybook-CN}
\coverstyle={ % 封面键值列表
    cover-choose=cn, % cn (需新增项\entitle{#}); en ; enfig ; birkar
}
\mathstyle={ % 数学字体键值列表
    math-font=mtpro2, %plain (默认数学字体); mtpro2
}
\RequirePackage{stys/settings-CN}
\usepackage{zhlipsum}
\begin{document}
\thispagestyle{empty}
\entitle{Notes for Differential Geometry} % cncover专用
\title{Reviewing Material}
\subtitle{复代数几何学笔记}
\edition{First Edition}
\bookseries{Preperation For The Final Test}
\author{Shilong.Lu}
\pressname{Springer}
\presslogo{inner_pics/Springer-logo.png}
\coverimage{inner_pics/coverimage.jpg} % cncover专用
%\coverimage{inner_pics/ivy-ge998908f8_1280.jpg}
\makecover % 封面生成命令

% 主题色调
\definecolor{bg}{HTML}{e0e0e0}
\definecolor{fg}{HTML}{203A43}
\colorlet{outermarginbgcolor}{bg}
\colorlet{outermarginfgcolor}{fg}
\colorlet{framegolden}{fg}
\colorlet{framegray}{黛绿!5}
\colorlet{headlinecolor}{靛蓝}
\chapoddimage{inner_pics/songodd.png}
\chapevenimage{inner_pics/songeven.png}
% set the contents of the outer margin on even and odd pages for scrheadings, plain and scth
\oddoutermargin{\cabin\leftmark} % Odd 奇数页
\makeatletter
\evenoutermargin{\cabin\@title} % Even 偶数页
\setlength\footheight{25pt}
\ofoot{} % 页脚信息
\makeatother
%
\thispagestyle{empty}
\begin{titlepage}
\thispagestyle{empty}
    \begin{center}
    
    %% Print the bookseries
    {\makeatletter
    \ifdefvoid{\@bookseries}{}{\bigskip\normalfont\fontsize{20}{20}\selectfont\@bookseries}
    \makeatother}

        \bigskip
        \bigskip

    %% Print the title
    {\makeatletter
    \fontsize{35}{35}\rmfamily\bfseries\selectfont\@title
    \makeatother}
    
    \bigskip
    \bigskip
    
    \bigskip
    \bigskip
    \bigskip

    %% Print the name of the author
    {\makeatletter
    \fontsize{25}{25}\rmfamily\selectfont\@author
    \makeatother}
    
    \bigskip
    \bigskip

    
    \vfill
    
    
    \bigskip
    \bigskip
    
    {\makeatletter
    \fontsize{25}{25}\rmfamily\selectfont\@pressname
    \makeatother}
    \end{center}
    
    \end{titlepage}
    \let\cleardoublepage\clearpage

    \thispagestyle{empty}
    \begin{center}
        {\fontsize{20}{20}\rmfamily\selectfont    内\hspace{1em}容\hspace{1em}简\hspace{1em}介}\\ 
        \bigskip

        本书是复分析学的结课期考复习资料总结,主要包括了考试的证明题型以及各类的识记知识点,如黎曼映射定理、广义Schwarz引理等等。本书由本人期末写成, 仅用于复习.


        本书是复分析学的结课期考复习资料总结,主要包括了考试的证明题型以及各类的识记知识点,如黎曼映射定理、广义Schwarz引理等等。本书由本人期末写成, 仅用于复习.
        \bigskip
        
        This book is a summary of the final examination review materials of complex analysis, mainly including the proof question types of the exam and various knowledge points, such as Riemann mapping theorem, generalized Schwarz lemma and so on. This book was written by me at the end of the semester and is for review only.

        \bigskip

        {\fontsize{10}{10}\rmfamily\bfseries\selectfont 图书在版编目(CIP)数据}\\[-2ex]
        \tikz\draw[line width=1pt,black] (0,0)-- (.99\linewidth,0);\\[-.8ex]
        微分几何学笔记/(中)陆世龙著 --广东; 家里蹲出版社,2023.01 \\ 
        (复习系列丛书)\\ 
        {\bfseries ISBN: } 978-7-03-002970-6\\ 
        I.微……\circled{1} 陆……\\ 
        中国版本图书馆CIP数据核字(2023)第00022893号\\[-2ex]
        \tikz\draw[line width=1pt,black] (0,0)-- (.99\linewidth,0);\\[-.8ex]
        {\itshape 责任编辑: 陆\hspace{1em}世\hspace{1em}龙 / 责任校对: 陆\hspace{1em}世\hspace{1em}龙\\ 
        责任印刷: 陆\hspace{1em}世\hspace{1em}龙}
        \vfill

        {\makeatletter
    \fontsize{10}{10}\rmfamily\selectfont\@pressname 
    \makeatother}出版\\ 
    {\footnotesize 广东省茂名市信宜市290号}\\ 
    \href{https://www.lushilong.com}{陆世龙}\\ 
    {\fontsize{10}{10}\rmfamily\selectfont 广东省印刷有限公司} 印刷\\ 
    家里蹲出版社发行\hspace{2em} 各地新华书店经销\\ 
    \bigskip
    *\\
    \bigskip
    {\footnotesize\begin{tabular}{ll} 2023年02月第一版 & 开本: $850\times1168\quad \quad1/32$\\ 
    2023年05月第二次印刷 & 印张: $18 5/8$\\ 
    \multicolumn{2}{c}{字数:493 000}\\
    \end{tabular}}\\ 
    {\fontsize{10}{10}\rmfamily\bfseries\selectfont 定价: 89.00元}\\ 
    {\footnotesize (如有印刷问题,我社负责调换)}


    \end{center}





 % 内封面页
%
\frontmatter % 前置材料
\pagenumbering{Roman}
\thispagestyle{empty}
\addcontentsline{toc}{chapter}{前言}
\chapter*{前言}
怀着复杂的心情写下了这本不算是笔记的笔记,大差不离就是抄写本吧! 但无论如何, 这是我自己写的一些学习感悟以及重要内容抄录,作为人生中第一本自己写的书,还是很激动的.


\hfill 
\begin{tabular}{lr}
    &----- 陆世龙\\ 
&2023年 01月 11日
\end{tabular}

\begin{center}
    \vfill
    \thepage
\end{center}
\let\cleardoublepage\clearpage % 前言
\thispagestyle{empty}
\tableofcontents\let\cleardoublepage\clearpage % 目录

\mainmatter % 正文
\pagenumbering{arabic}
\partimage{inner_pics/part.png}
\partabstract{\hspace{2em}本书系统地论述了微分几何的基本知识. 作者用前3章,  以及第6章共计4章的篇幅介绍了流形、多重线性函数、向量场、外微分、李群和活动标架等基本知识和工具.  基于上述基础知识,  论述了微分几何的核心问题,  即联络、黎曼几何、以及曲面论. 第7章是当前十分活跃的研究领域——复流形. 陈省身先生是此研究领域的大家,  此章包含有作者独到、深刻的见解和简捷、有效的方法. 第8章的Finsler几何是本书第2版新增加的一章,  它是陈省身先生近年来一直倡导的研究课题,  其中Chern联络具有突出的性质,  它使得黎曼几何成为Finsler几何的特殊情形. 最后两个附录,  介绍了大范围曲线论和曲面论,  以及微分几何与理论物理关系的论述,  为这两个活跃的前沿领域提出了不少进一步的研究课题.

\hspace*{2em} 本书的作者之一是已故数学家陈省身先生\cite{Huybrechts2010Complex},  他开创并领导着整体微分几何、纤维丛微分几何、“陈省身示性类”等领域的研究,  他是第一个获得世界数学界最高荣誉“沃尔夫奖”的华人,  被称为“当今最伟大的数学家”,  被国际数学界尊为“微分几何之父”. }
\part{微分几何讲义一陈省身}

\chapter{测试章 Test chapter}
\section{测试节本书系统地论述了微分几何的基本知识. 作者用前3章,  以及第6章共计4章的篇幅介绍了流形、多重线性函数、向量场、外微分、李群和活动标架等基本知识和工具. }

$x^2+y^2=z^2$\\
{\largetitlestyle 测试一下字体命令} | {\kaishu 测试一下字体命令} | {\songti 测试一下字体命令} | {\fangsong 测试一下字体命令}\\
{\xingkai 测试一下字体命令} | {\dbs 测试一下字体命令} | {\lishu 测试一下字体命令}
\zhlipsum
\newpage
test
\zhlipsum{1-9}
\newpage
test 
\zhlipsum
\zhlipsum

\appendix % 附录章节
%


{ % 限制空页面样式命令作用范围
\normalem
\printbibliography[
heading=bibintoc,
title={参考文献}
]
\printindex
\thispagestyle{empty}}
\bottomimage{inner_pics/ivy-ge998908f8_1280.jpg}
\ISBNcode{\EANisbn[ISBN=978-80-7340-097-2]} %
\summary{本书是复分析学的结课期考复习资料总结,主要包括了考试的证明题型以及各类的识记知识点,如黎曼映射定理、广义Schwarz引理等等.本书由本人期末写成, 仅用于复习. 本书是复分析学的结课期考复习资料总结,主要包括了考试的证明题型以及各类的识记知识点,如黎曼映射定理、广义Schwarz引理等等.本书由本人期末写成, 仅用于复习. 本书是复分析学的结课期考复习资料总结,主要包括了考试的证明题型以及各类的识记知识点,如黎曼映射定理、广义Schwarz引理等等.本书由本人期末写成, 仅用于复习.}
\makebottomcover
\end{document} 