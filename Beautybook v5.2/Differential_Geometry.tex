\documentclass[twoside]{Beautybook/Beautybook}
\overfullrule=1pt
%******************************************************************************%
%%=================================================%%
%%-------------------------- 彩虹书边垂直位置调节 --------------------------------%%
%%=================================================%%
\setlength{\outermarginwidth}{2cm}
\setlength\baryshift{-2.5\outermarginwidth} %彩虹书边垂直位置调节
\setlength\chapterNameheight{2.6\outermarginwidth} %章节名称高度调节 #\outermarginwidth
\setlength\hdheight{50pt} %页眉高度
\setlength\TitleWidth{2cm}
\setlength\ChapterBackGroundHeight{4\outermarginwidth}
\setlength\CoverLogoyshift{\footheight}
\FootName{ 陆~世~龙}
%%=================================================%%
%%-------------------------- 彩虹书边垂直位置调节 --------------------------------%%
%%=================================================%%
%******************************************************************************%
%\setlist{itemsep=-2pt} % Reducing white space in lists slightly
%%====================== 颜色设置 =========================%%
\colorlet{outermarginbgcolor}{cyan4!30}
\colorlet{outermarginfgcolor}{cyan4}
\colorlet{ocre}{outermarginbgcolor} %Part TOC background color
\addtokomafont{outermargin}{\color{cyan4}}
%\colorlet{outermarginfontcolor}{brown6}
%%====================== 颜色设置 =========================%%
\renewcommand{\deg}{\si{\degree}\xspace} % Use \deg easily, everywhere
\DeclareMathOperator{\Aut}{Aut}
\DeclareMathOperator\Char{Char}
\DeclareMathOperator\diag{diag}
\DeclareMathOperator\Arg{Arg}%辐角
\DeclareMathOperator\im{Im}
\DeclareMathOperator{\End}{End}
\newcommand{\F}{\mathcal{F}}
\newcommand{\R}{\mathbb{R}}
\newcommand{\Diff}[2][]{\frac{\partial #1}{\partial #2}}
\newcommand{\Dif}[2]{\frac{\dd #1}{\dd #2}}
%poisoning库用于绘制和定位nodes

%%%%%%%%%%%%%%%%%%%%%%%%%%%%%
%%%%% Begin of document %%%%%
%%%%%%%%%%%%%%%%%%%%%%%%%%%%%
\makeindex[intoc,title=关键词索引,columnseprule]%columnsep=2pt
% \newcommand\Figure[4]{%
%     \begin{center}
%     \captionsetup{type=figure} %消除 Package caption Warning: The option `hypcap=true' will be ignored for this
% %(caption)                particular \caption on input line XX.警告信息
%         \includegraphics[#1]{#2}
%         \captionof{figure}{#3}\label{#4}
%     \end{center}
% }%用法 : 用于tcolorbox中插入图片
% \newcommand\Tikz[3]{%
%     \begin{center}
%     \captionsetup{type=figure} %消除 Package caption Warning: The option `hypcap=true' will be ignored for this(caption)                particular \caption on input line XX.警告信息
%         #3 %最后一个参数为Tikz画图代码窗口
%         \captionof{figure}{#1}\label{#2} %第一与第二个参数是Tikz画图的名称与标签
%     \end{center}
% } %用法 : 用于tcolorbox中插入Tikz绘图
% \graphicspath{{figures/}{Beautybook/images/}} %图片相对路径
% %%===================================================%%
% %                                                          公式标注                                                               %%
% %%===================================================%%
% %%标记内容的命令 (作用是创建标记点的标签名称与要编辑内容间的链接,以便于在Tikz中引用该标签时相当于引用要标记内容所处位置)
% % Commands for Highlighting text -- non tikz method
% \newcommand{\highlight}[2]{\colorbox{#1!17}{$\displaystyle #2$}}
% %\newcommand{\highlight}[2]{\colorbox{#1!17}{$#2$}}
% \newcommand{\highlightdark}[2]{\colorbox{#1!47}{$\displaystyle #2$}}
% % Commands for Highlighting text -- non tikz method
% \renewcommand{\highlight}[2]{\colorbox{#1!17}{#2}}
% \renewcommand{\highlightdark}[2]{\colorbox{#1!47}{#2}}
% %%%综合标记命令为一条
% \newcommand\Tikzmark[3]{
%     \tikzmarknode{#1}{\highlight{#2}{$#3$}} %% #1-- Tikzmarknode (即标记点的标签名称设置) ; #2 -- 标记颜色 ; #3 -- 要标记的内容
% }
% %%===================================================%%
% %                                                          公式标注                                                               %%
% %%===================================================%%
\AtBeginDocument{\RenewCommandCopy\qty\SI}
\begin{document}
%******************************************************************************%
%%=================================================%%
%%----------------------------------------封面------------------------------------------%%
%%=================================================%%
%% Defining the main parameters
\title{微分几何笔记}
\subtitle{微分几何与代数几何}
\author{陆世龙}
\press{广西民族大学自然科学出版社}
\editor{陆世龙主编}
\version{第四版}
\motto{路就在我们的脚下,如果都不愿意去动身出发,那么谁也无法挽救自己!记住,懒惰是所有人的公敌,只有克服懒惰,才有可能成功!\hfill --陆世龙言}
\coverequation{Stokes公式 : $\iint\limits_{S}\kuohao{\Diff[R]{y}-\Diff[Q]{z}}\dd y\dd z+\kuohao{\Diff[P]{z}-\Diff[R]{x}}\dd z\dd x+\kuohao{\Diff[Q]{x}-\Diff[P]{y}}\dd x\dd y=\oint_{L}P\dd x+Q\dd y+R\dd z$.}
\coverimage{figures/draw-3d-intersecting-surfaces.pdf} % Aspect ratio of 2:3 (portrait) recommended
\covertikz{figures/Nine-pointCircle.pdf}
\coverlogo{Beautybook/images/Guangximinzu.png}
\colorlet{coverfgcolor}{purple5}
\colorlet{coverbgcolor}{purple5!30}
\makecover
%******************************************************************************%
%%=================================================%%
%%----------------------------------------封面------------------------------------------%%
%%=================================================%%


%% Roman page numbering
\frontmatter %前言
%% Roman page numbering
\setcounter{page}{1}
\pagenumbering{Roman}
\begin{titlepage}

\begin{center}

%% Print the title
{\makeatletter
\largetitlestyle\fontsize{45}{45}\rmfamily\selectfont\@title
\makeatother}

%% Print the subtitle
{\makeatletter
\ifdefvoid{\@subtitle}{}{\bigskip\bfseries\fontsize{20}{20}\selectfont\@subtitle}
\makeatother}

\bigskip
\bigskip

by

\bigskip
\bigskip

%% Print the name of the author
{\makeatletter
\largetitlestyle\fontsize{25}{25}\rmfamily\selectfont\@author
\makeatother}

\bigskip
\bigskip

%% Print table with names and student numbers
\setlength\extrarowheight{2pt}
\begin{tabular}{lc}
    学生姓名 & 学号 \\\midrule
    陆世龙 & 201713030506 \\
\end{tabular}

\vfill

%% Print some more information at the bottom
\begin{tabular}{ll}
    导师: & 卢卫君 \\
    %Teaching Assistant: & I. Surname \\
    答辩日期: &  2022.04.10\\
    科目: & {\bf 微分几何与代数几何}
\end{tabular}

\bigskip
\bigskip

%% Add a source and description for the cover and optional attribution for the template
%\begin{tabular}{p{15mm}p{10cm}}
%    Cover: & Canadarm 2 Robotic Arm Grapples SpaceX Dragon by NASA under CC BY-NC 2.0 (Modified) \\
%    % Feel free to remove the following attribution, it is not required - still appreciated :-)
%    Style: & TU Delft Report Style, with modifications by Daan Zwaneveld
%\end{tabular}

\end{center}

%% Insert the TU Delft logo at the bottom of the page
\begin{tikzpicture}[remember picture, overlay]
    \node[above=10mm] at (current page.south) {%
        \includegraphics{Beautybook/images/logo-black}
    };
\end{tikzpicture}

\end{titlepage}
 %标题页
\renewcommand{\contentsname}{\bf 目\quad 录}
\renewcommand\listfigurename{\bf 插\ 图\ 目\ 录}
\renewcommand\listtablename{\bf 表\ 格\ 目\ 录}
\tableofcontents\let\cleardoublepage\clearpage
\listoffigures\let\cleardoublepage\clearpage
\listoftables\let\cleardoublepage\clearpage
\newpage %防止页码错误

%% Arabic page numbering
\mainmatter % 正文
\setcounter{page}{1}
\pagenumbering{arabic}
%\colorlet{outermarginbgcolor}{orange!30}
%\colorlet{outermarginfgcolor}{orange}
%\addtokomafont{outermargin}{\color{orange}}
\evenoutermargin{微分几何笔记} %偶数页页边内容
\partimage{figures/draw-3d-intersecting-surfaces.pdf}
\partabstract{微分流形是代数几何与微分几何领域的必学知识} % Part简介
\part{流形上的微积分}
\setlength\ChapterBackGroundHeight{4.1\outermarginwidth}
    \chapter{向量函数}
\begin{introduction}
\item 向量函数
\item 微分流形
\end{introduction}
\section{基本概念}
\begin{definition}[向量函数][def:向量函数]
    设$D$是一个集合,若给定一映射$\mathbf{r}$将$D$中的每一元素都映射为$\mathbb{R}^3$中的一个位置向量,则称$\mathbf{r}$为一个向量函数.

    特别地,当$D$为一开区间$(a,b)$时,向量函数$\mathbf{r}$为一元向量函数,记为$\mathbf{r}=\mathbf{r}(t),t\in D$.

    当$D$为一开区域$(a,b)\times (c,d)$时,向量函数$\mathbf{r}$称为二元向量函数,记为$\mathbf{r}=\mathbf{r}(u,v),u\in (a,b),v\in (c,d)$.
\end{definition}
\begin{definition}[$m$维流形][def:流形]
  设$M$是Hausdorff Space. 若对任意的$x\in M$,均有在$M$中包含$x$的一个邻域$U$, 使得$U$\textbf{同胚(homeomorphic)}\index{同胚映射}于$m$维Euclidian Space $\R^m$的的一个开集. 则称$M$为一个$m$维\textbf{流形(也称为拓扑流形)}\index{流形}\cite{陈省身,陈维桓2001.10}
\end{definition}
\subsection{流形上的双全纯映射(即同胚映射)}
\indent\textbf{流形$M$的两个非空开集之间同样可以建立同胚映射}.

设$\kuohao{U,\varphi_U}$和$\kuohao{V,\varphi_V}$是流形$M$的两个\textbf{坐标卡}\index{坐标卡},如果$U\bigcap V\neq \emptyset$,则$\varphi_U \kuohao{U\bigcap V}$和$\varphi_V \kuohao{U\bigcap V}$是$\R^m$上的两个非空开集,且映射:
\begin{equation}\label{eq:流形上的同胚}
  \varphi_V\circ \varphi_U^{-1} \big|_{\varphi_U \kuohao{U\bigcap V}}\colon \varphi_U \kuohao{U\bigcap V}\to \varphi_V \kuohao{U\bigcap V}.
\end{equation}
建立起了这两个开集间的同胚,其逆映射就是$\varphi_U\circ \varphi_V^{-1}\big|_{\varphi_V \kuohao{U\bigcap V}}$.

\paragraph{光滑函数}
设$f$是定义在$m$维光滑流形$M$上的实函数.若$p\in M,(U,\varphi_U)$是包含$p$点的容许坐标卡,那么$f\circ \varphi_U^{-1}$是定义在欧式空间$\R^m$的开集$\varphi_U (U)$上的实函数.如果$f\circ \varphi_U^{-1}$在点$\varphi_U (p)\in \R^m$是$C^\infty$的\footnote{即$f\circ \varphi_U^{-1}$在点$\varphi_U (p)$的一个邻域内有任意多次连续偏导数},则称函数$f$在点$p\in M$是$C^\infty$的.


{\bf 函数$f$在点$p$的可微性与包含$p$的容许坐标卡的选取是无关的}.实际上,若还有另一个包含$p$点的容许坐标卡$(V,\varphi_V)$,则$U\bigcap V\neq \emptyset$,且
\[f\circ\varphi_V^{-1}=\kuohao{f\circ\varphi_U^{-1}}\circ\kuohao{\varphi_U\circ\varphi_V^{-1}}\]
从而因$\varphi_U\circ\varphi_V^{-1}$是光滑的,故$f\circ\varphi_V^{-1}$与$f\circ\varphi_U^{-1}$在相应点都是可微的.

如果实函数$f$在$M$上处处是$C^\infty$的,则称$f$是$M$上的 $C^\infty$--函数,或称为$f$是 $M$上的\textbf{光滑函数}. $M$上光滑函数全集为 $C^\infty (M)$.光滑函数是光滑流形之间的光滑映射的重要特例.

\begin{definition}[][def:光滑映射]
设 $f\colon M\to N$是光滑流形 $M$到 $N$的一个连续映射,且 $\dim M=m,\dim N=n$.若在一点 $p\in M$,存在点 $p$的容许坐标卡 $(U,\varphi_U)$和点 $f(p)$的容许坐标卡 $(V,\psi_V)$,使得映射
\[
  \psi_V \circ f\circ \varphi_U^{-1}\colon \varphi_U (U)\to \psi_V (V)
\]
在点 $\varphi_U (p)$上是 $C^\infty$的,则称映射 $f$在点 $p$是 $C^\infty$的.若映射 $f$在 $M$的每一点 $p$都是 $C^\infty$的,则称 $f$是从 $M$到 $N$的\textbf{光滑映射}.
\end{definition}
\section{形式与函数芽}
\subsection{微分形式}\index{微分形式}
\begin{figure}[htbp]
    \centering
    \begin{tikzpicture}[>=stealth,spy using overlays= {rectangle, magnification=5, connect spies}] %spy using outlines= {circle, magnification=6, connect spies} %圆形放大镜
        \draw[gray!30,very thin] (-5,-1) grid (5,9);
        \draw[->,thin] (-5,0) -- (5,0) node[right] {$x$};
        \draw[->,thin] (0,-1) -- (0,9) node[left] {$y$};
        \foreach \x in {-5,...,5}{
            \node[below,gray,font=\small] at (\x,-1) {$\x$};
        }
        \foreach \y in {-1,...,9}{
            \node[left,gray,font=\small] at (-5,\y) {$\y$};
        }
        \draw[thick,blue,domain=-5:2.2,smooth] plot (\x,{e^\x}) node[right] {$f(x)=e^x$};
        \draw[thick,black,domain=-2:5] plot (\x,{\x+1}) node[above] {$f(x)=x+1$};
        \coordinate (intersection) at (0,1);
        \node[red,left] at (intersection) {$(0,1)$};
        \node[black] at (1,e) {$(1,e)$};
        \draw[very thick,cyan,->] (2,-0.5) -- (0.1,0.95);
        \draw[shift={(0,-2)},thick,teal,domain=0:5] plot (\x,{\x+1}) node[above] {$f_1(x)=x-1$};
        \draw[rotate=30,shift={(0,-0.64)},magenta,domain=-0.95:6.4,thick] plot (\x,{\x+1}) node[above] {$g_1(x)$};
        \shade[ball color=gray] (intersection) circle [radius=2pt];%(2pt)也行
        \coordinate (spypoint) at (0,1);% The point to be magnified
        \coordinate (magnifyglass) at (-3,5);% The point where to see
        \spy [gray!50, size=2.5cm] on (spypoint) in node[fill=white] at (magnifyglass);
    \end{tikzpicture}
    \label{fig:微分形式解说}
    \caption{微分形式解说}
\end{figure}
在$(0,1)$附近,两函数靠得很近,而在$(0,1)$该点上,两函数的变化完全一样,或者说\textbf{二者在该点的局部具有相似性}!而在实数的整体上,$f$和$g$是完全不同的.但是
这里的函数$f$和$g$有着同样的微分形式,因为局部放大图中看到,它们在$(0,1)$的附近几乎无法区分.

形象地说,假如在$(0,1)$上放置一个人,不管他踩在哪一条曲线上,他都感觉是站在$45^\circ$的斜坡上.而如果向下平移函数$g$,如图中青色直线所示,依然不会改变$g$在点$(0,1)$的斜率,感觉一模一样,所以微分形式不变.
不过,如果我们旋转函数$g$的图像,如图中粉色直线所示,则在点$(0,1)$的坡度就变陡峭了,这时斜率发生了变化,我们放置在该点处的人可能就站不稳了,那么新函数$g_1(x)$在$(0,1)$点的微分形式发生了变化.其根本原因在于局部的导数值发生了变化.
至此,我们对微分形式有了一个大致的感觉: 与\textbf{导数}相关.接下来说明如何定义微分形式.
\begin{center}
    \begin{tikzcd}
    \text{实数域上的一元光滑函数}\ar[->,>=stealth,d]\\
    \text{一元光滑函数芽}\ar[->,>=stealth,d]\\
    \text{一元函数的$1$--形式}
    \end{tikzcd}
\end{center}
现在只看微分形式中的$1$--形式,分三步理解.首先,光滑实值函数在任意一点处都有无穷阶连续导函数.在这里,我们只考虑定义域为全体实数的函数. 如一次函数,二次函数,指数函数等等,都是定义在
全体实数上的一元光滑函数.不过光滑函数依然是一个整体的概念.微分形式是局部的观点,因此我们要想办法看一点的附近.遵循着这种局部化的思想,就有了" 芽"
这一充满了局部风格的概念.

\textbf{芽的思想,本质上是根据函数在一点附近的局部表现,对这些函数进行分类,即所谓的等价关系}
\begin{definition}[芽][def:芽]
    芽是定义在拓扑空间上函数集合的一种等价关系.
    定义在拓扑空间上的两个函数$f$和$g$,在点$x$处属于同一支芽,当且仅当存在一个开集$S$,使得$S$包含点$x$,且在$S$上$f$和$g$的函数值处处相等.
\end{definition}
这里的 拓扑空间可以选择实数集,开集就是开区间之并,而点$x$就是定义函数芽和微分形式的地方.

\subsection{向量空间与对偶空间}
\renewcommand{\introductionname}{小~节~提~要}
\begin{introduction}
    \item 引子:光滑运动
    \item 向量与方向导数
    \item 向量与$1$--形式
    \item 向量空间与对偶空间的形式化定义
    \item 对偶空间的对偶
\end{introduction}
\renewcommand{\introductionname}{章~节~摘~要}
在光滑曲线所处的三维空间定义一光滑的多元(此处为三元)函数 $f$,这里$f$是三元实值函数,而曲线函数就是将某个实数区间$I$里的数\textbf{映射为}曲线上的点. 而如此复合后的函数就是一个一元实值函数 $g$.它们之间的关系如下交换图所示:
\begin{center}
    \begin{tikzcd}
    I\ar[->,>=stealth,r,"\gamma"] \ar[->,>=stealth,rd,"g=f\circ \gamma"swap]& \mathbb{R}^3 \ar[->,>=stealth,d,"f"]\\
    & \mathbb{R}
\end{tikzcd}
\end{center}
现设
\[\begin{split}
    \gamma(t)&=(x(t),y(t),z(t))\\
    \Dif{(f\circ \gamma)}{t}&=\Diff[f]{x}\Dif{x}{t}+\Diff[f]{y}\Dif{y}{t}+\Diff[f]{z}\Dif{z}{t}\\
&=(\Diff[f]{x},\Diff[f]{y},\Diff[f]{z})\cdot \begin{pmatrix}
    \Dif{x}{t}\\ \Dif{y}{t}\\ \Dif{z}{t}
\end{pmatrix}
\end{split}\]
,并引入表达式
\begin{equation}
    \label{eq:change}
        \begin{array}{c}
                \dd x^i \cdot \vec{e}_{x_j}=\delta_j^i=
                \begin{cases}
                    0,&i\neq j,\\
                    1,&i=j.
                \end{cases}\hfill\text{($f$沿$x^i$轴对$\vec{e}_{x_j}$方向的导数)}\\
                \dd f =\Diff[f]{x}\dd \vec{x}+\Diff[f]{y}\dd \vec{y}+\Diff[f]{z}\dd \vec{z}.\hfill\text{($f$的$1$--形式坐标表示)}
            \end{array}
\end{equation}
,此复合函数的导数是关于坐标分量的表达式,叫沿光滑曲线的方向导数.将其带入运算得
\[\Dif{(f\circ \gamma)}{t}=\dd f\cdot\left(\Dif{x}{t}\vec{e}_x+\Dif{y}{t}\vec{e}_y+\Dif{z}{t}\vec{e}_z\right)\]
$1$--形式定义为光滑函数芽的等价类,与坐标系无关.从而由于$1$--形式与方向导数均独立于任何人为的坐标系,故切矢的定义也与坐标系无关.
其中切矢即$\gamma(t)$,$1$--形式指的是$\dd f$.

\section{向量空间与对偶空间}
\begin{definition}[向量空间][def:向量空间]
    域$F$上的向量空间$V$是一个{\color{magenta}\textbf{集合}},在其上定义了两种运算:
    \begin{enumerate}
        \item 向量加法: $V\times V\to V$,把$V$中的两个元素$u$和$v$映射到$V$中另一个元素,记作$u+v$;
        \item 标量乘法: $F\times V\to V$,把$F$中的一个元素$a$和$V$中的一个元素$u$变为$V$中的另一个元素,记作$a\cdot u$.
    \end{enumerate}
    并且向量空间还需在上述两种运算基础上满足以下性质:
    \begin{enumerate}
        \item 向量加法
        \begin{description}
            \item[结合律:] $u+(v+w)=(u+v)+w$,
            \item[交换律:] $u+v=v+u$,
            \item[向量加法单位元:] 在$V$中存在一个叫做"零向量"的元素 ,记作$\bm{0}$,使得对于$V$中的任意的向量$u$,都有$u+0=u$,
            \item[向量加法逆元:] 对$V$中的任意向量$u$,都存在$v\in V$,使得$u+v=\bm{0}$,并称向量$v$为向量$u$在$V$中的逆元,
        \end{description}
        \item 标量乘法
        \begin{enumerate}
            \item 标量乘法对向量加法满足分配律: $a\cdot (v+w)=a\cdot v+a\cdot w$,
            \item 标量乘法对域的加法满足分配律: $(a+b)\cdot u=a\cdot u+b\cdot u$,
            \item 标量乘法对标量域的乘法相容: $(ab)u=a(bu)$,
            \item 标量乘法有单位元: 域$F$的乘法单位元"$1$" 满足: 对任意的$v\in V$,都有$1\cdot v=v$.
        \end{enumerate}
    \end{enumerate}
\end{definition}
上述向量空间的定义采用了"\textbf{协变构造}" (即无需人为干预的,不是生拉硬拽的,自然的构造方式)的定义方式,从而摆脱了坐标系的束缚.

由此,我们想到,如果之前的$1$--形式能构成一个向量空间,则就是一个协变构造.基于此思想,接下来我们证明$1$--形式确实可以构造出一个向量空间.

\begin{proposition}[][prop:1.2.1]
  $\mathbb{R}^n$中一点处的全体光滑函数芽构成一个向量空间$\mathcal{F}_p$.
\end{proposition}
现用$[f]$表示在同一光滑函数芽里的全体光滑函数,即$f$所代表的函数芽.
\begin{remark}
  若规定此处代表函数$f$只能是线性函数,则这些线性函数对加法与数乘封闭,从而构成一个向量子空间.
\end{remark}
\begin{corollary}[$1$--形式全体构成一个向量空间][cor:1.2.1]
  $\mathbb{R}^n$中任一点$p$处的全体$1$--形式构成一个向量空间.
\end{corollary}
\begin{remark}
  $1$--形式是光滑函数芽的其中一种类别.
\end{remark}
\begin{definition}[对偶空间][def:对偶空间]
  设$V$为域$F$上的向量空间.定义其对偶空间为 $V^*$.
  \[V^*=\{f(u),u\in V\mid \text{$f\colon V\to F$是{\bf\color{magenta} 线性映射}}\}\]
  即$V^*$是从$V$到$F$的所有\textbf{线性函数}的集合,其中$F$称为向量空间$V$的{\bf\color{magenta} 标量域}. $V^*$本身是域$F$上的向量空间,且
  满足域$F$的加法和标量乘法(以域$F$为标量域).

  定义$V$上一点$p$处的切矢空间为$1$--形式向量空间的对偶空间.
\end{definition}
\begin{remark}
  切矢的对偶是$1$--形式,即对偶的对偶又回到了原向量空间.
\end{remark}
\begin{proposition}[][prop:1.2.2]
  \textbf{有限维}向量空间$V$与$V^{**}$自然同构.
\end{proposition}
\begin{proof}
  首先,经分析知,证明两空间同构的要点在于: 存在两空间之间的一一映射保持运算.故设$\eta\colon V\to V^{**}$,
  有
\begin{figure}[h]
  \vspace{\baselineskip}
  \begin{equation}
    \label{eq:homeomophic}
    \left(\tikzmarknode{a}{\highlight{purple}{\color{black}$\eta$}}(\tikzmarknode{u}{\highlight{blue}{\color{black}$u$}})\right)(\tikzmarknode{p}{\highlight{magenta}{\color{black}$\phi$}})=\tikzmarknode{ph}{\highlight{magenta}{\color{black}$\phi$}}(u).
\end{equation}
\vspace*{0.5\baselineskip}
\begin{tikzpicture}[overlay,remember picture,>=stealth,nodes={align=left,inner ysep=1pt},<-]
        % For "mu"
        \path (a.north) ++ (0,2em) node[anchor=south east,color=purple!67] (scalep){\textbf{同构映射}};
        \draw [color=purple!57](a.north) |- ([xshift=-0.3ex,color=red]scalep.south west);
        % For "b"
        \path (u.south) ++ (0,-1.5em) node[anchor=north west,color=blue!67] (mean){\textbf{向量}};
        \draw [color=blue!57](u.south) |- ([xshift=-0.3ex,color=blue]mean.south east);
        % Ts to Td
        \path (p.north) ++ (-1.5em,2em) node[anchor=south west,color=DeepPink!67] (scalep){\textbf{对偶线性函数}};
        \draw[<->,color=magenta!57] (p.north) -- ++(0,0.67)  -| node[] {} (ph.north);
    \end{tikzpicture}
  \caption{同构映射}\label{fig:同构映射}
\end{figure}
$\eta$把$V$中向量$u$映射到$V^{**}$里的一个向量$\phi(u)$,即对偶的对偶.\textbf{映射作用于对偶总是等价于对偶作用于原向量}.
这就形成了从$V$到$V^{**}$的一个{\color{purple}\textbf{正规嵌入}}$\lambda$.
以下证明该正规嵌入$\lambda$是单射,又由于$V$与$V^{**}$均是有限维的,这可以保证该嵌入是一个满射,从而该嵌入是双射,且其保持了$V$中的运算,故而是一个同构映射.

单射性. 设$u,v\in V$,有
\begin{equation*}
\begin{split}
  \eta(u)&=\eta(v)\\
    &\Rightarrow \forall\phi\in V^{*},\phi(u)=\phi(v)\\
    &\Rightarrow\phi\in V^{*},\phi(u-v)=0\\
    &\xLongrightarrow[]{\text{非零向量的线性函数值不恒等于$0$}} u=v.
\end{split}
\end{equation*}
从而$\eta$是单射.

满射性. 由于$V$与$V^{**}$的基底间是一一对应的,即$V^{**}$中的每一个向量$\eta(u)$,都存在$u\in V$映射到它,故而$\eta$是满射.
\begin{equation*}
  \begin{split}
    u&=\sum u^i \vec{e}_i\\
    \eta(u)&=\sum u^i \vec{e}_{i**}.
  \end{split}
\end{equation*}
\end{proof}
\begin{remark}
  $\mathbb{R}^n$中的切矢空间与$1$--形式空间互为对偶.
\end{remark}
\section{基底与ZFC公理系统}
\renewcommand{\introductionname}{小~节~提~要}
\begin{introduction}
  \item 向量空间的例子
  \item 向量空间的基底
  \item 协变构造与逆变构造
  \item 基底的存在性与ZFC公理系统
\end{introduction}
\renewcommand{\introductionname}{章~节~摘~要}
标量乘法是函数与函数间的一个映射,而向量的加法则是两函数作用效果的叠加,故向量空间中的向量都是函数.

\begin{example}[复数域$\mathbb{C}$][exam:C]
  复数域$\mathbb{C}$是一个向量空间,且其与$\mathbb{R}^2$同构.
\end{example}
\subsection{基底}
\begin{definition}[向量空间的基底][def:向量空间的基底]
  向量空间$V$的基底$\mathscr{B}$为一向量的集合,满足:
  \begin{enumerate}
    \item 若$\mathscr{B}$中有限个向量的线性组合为零向量,则组合中各系数都必为标量域中的零元素.
    \item $V$中任一个向量都可以表示成$\mathscr{B}$中\textbf{有限个}向量的线性组合.
  \end{enumerate}
\end{definition}
\begin{proposition}[][prop:1.3.1]
  有限维向量空间的维数固定,即基底中的向量个数为定值.
\end{proposition}
\begin{proposition}[][prop:1.3.2]
\textbf{有限维}向量空间与其对偶空间的维数相同且二者同构. $V\cong V^{*}$.
\end{proposition}
\begin{proof}
  首先设$\psi$是从$V$到$V^*$的一个映射,则对任意的$u\in V$,令
  $\vec{e}_j$表示向量空间的基底,$\vec{e}^i$表示根据$e_j$构造出来的对偶空间的基底,则有
  \begin{equation}
\label{eq:jidi}
\vec{e}^i (\vec{e}_j)=\begin{cases}
  1,&i=j;\\
  0,&i\neq j.
\end{cases}
  \end{equation}
从而将该映射表示为基底的线性组合后有
  \begin{equation*}
    \begin{split}
      \psi(u)&\xLongrightarrow[]{u=\sum_i \vec{e}_i},\\
      &\implies \sum_i a_i \psi(\vec{e}_i),\\
      &\implies \sum_i a_i \psi(\vec{e}_i) \left[\vec{e}^i (\vec{e}_i)\right],\\
      &\implies \left(\sum_j \psi (\vec{e}_j)\vec{e}^j\right)\left(\sum_i a_i \vec{e}_i\right).
    \end{split}
  \end{equation*}
故而它们的基底一一对应,映射$\psi$为一个同构映射.如下式所示
\begin{center}
\begin{tikzcd}
    \psi(u)=\sum_i a_i \vec{e}^i \ar[d,<->,"\psi"]\\
    u=\sum_i a_i \vec{e}_i
\end{tikzcd}
\end{center}

\end{proof}
\begin{remark}
  这种证明方法的缺点是,如若没有指定空间的基底的话,这个同构映射便无法定义出来.对于这种不是自然的构造方式,一般称之为 "逆变构造".
\end{remark}
\subsection{ZFC公理系统}
\begin{proposition}[][prop:1.3.3]
如果承认ZFC公理系统,那么任一向量空间都存在基底.
\end{proposition}
\begin{axiom}[(ZFC $9^{\mathrm{th}}$) : 选择公理][axim:ZFC9]
任何集合上都可以定义选择函数$f$,对任意的非空子集$S$满足$f(S)$是$S$里的元素.
\end{axiom}
接下来,都默认承认选择公理成立,并以此为依据证明向量空间基底的存在性.
\begin{axiom}[(ZFC $8^{\mathrm{th}}$) : 幂集公理][axiom:ZFC8]
定义集合$V$的幂集为
\begin{equation}
  \label{def: 幂集}
  \mathscr{S}(V)=\{X\mid X\supset V\}.
\end{equation}
\end{axiom}
接下来用反证法证明.
\begin{proof}
  证明分两步,第一步先承认一点,如下所述的无尽序列中的元素不构成集合,从而由反证法推导出矛盾,紧接着再证明这一点即得证.

  第一步,在如下所述的无尽序列中的元素不构成集合的前提下进行分析,推导出矛盾点.

  假设向量空间$V$不存在基底,这意味着我们可以在空间$V$中找到一组线性无关的向量$\{v_1,v_2,\cdots,v_n\}\in \mathscr{S}(V)$.该线性无关向量组的存在性由
  \begin{axiom}[(ZFC $3^{\mathrm{th}}$) : 分类公理][axiom:ZFC3]
      集合的子集可由命题进行描述.而命题作用于幂集 $\mathscr{S}(V)$.
  \end{axiom}
  保证.故任意选择$V$中的$n$个线性无关向量组,总能找到第$n+1$个向量与这$n$个向量线性无关.根据ZFC第九公理,这$n+1$个向量组成的向量组也是线性无关的,且新的向量组属于幂集$\mathscr{S}(V)$.

  当我们无限次地重复以上步骤,我们可以得到一个无尽序列,它由一系列迭代生成的线性无关向量组所构成.事实上,该序列中的元素无法构成集合!亦即$\mathscr{S}(V)$的子集不是集合!而这显然是矛盾的.
  因此任意的向量空间都存在基底.

  第二步,证明上述前提的命题为真.

    证明的关键在于,这样反复添加向量所构成的无尽序列全体不构成集合.下面解释原因.
    \begin{definition}[VonNeumann序数][def:Von Neumann序数]
      小数一定是大数的元素,即$A<B$,当且仅当$A\in B$.具体构造方法是,在每个序数$A$的下一个序数$A+1$都定义为:
      \[A+1=\{A,\{A\}\}\]
    \end{definition}
    现根据添加向量的先后顺序,把这些向量子集与Von Neumann序数对应起来.
    \begin{equation}
      \begin{split}
        &\{v_1,v_2,\cdots,v_n\} \longleftrightarrow \{\emptyset\},\\
        &\{v_1,v_2,\cdots,v_{n+1}\} \longleftrightarrow \{\emptyset,\{\emptyset\}\},\\
        &\{v_1,v_2,\cdots,v_{n+2}\} \longleftrightarrow \{\{\emptyset,\{\emptyset\}\},\{\emptyset,\{\emptyset\}\}\}\\
        &\cdots
      \end{split}
    \end{equation}
    \begin{proposition}[][prop:1.4.1]
    Von Neumann 序数全体不构成集合.
    \end{proposition}
    \begin{proof}
      假设Von Neumann 序数全体构成一个集合$S$,则可以推出Burali-Forti悖论.构造$S+1=\{S,\{S\}\}$,显然,$S$是$S+1$的元素,故而$S<S+1$.
      但与此同时,$S+1$也是一个序数,所以应该属于$S$,即$S+1\in S$,从而$S+1<S$,矛盾.
    \end{proof}
    因而由于Von Neumann序数全体不构成集合,那么由ZFC第六公理--替代公理,与其一一对应的无尽序列作为其原像也不可能构成集合.
\end{proof}
ZFC的剩余几条公理.
\begin{axiom}[(ZFC $1^{\mathrm{th}}$) : 外延公理][axiom:ZFC1]
两集合相等当且仅当它们的元素一一相等.
\end{axiom}
\begin{remark}
  这是比较集合的一般方法.
\end{remark}
\begin{axiom}[(ZFC $2^{\mathrm{th}}$) : 正规公理][axiom:ZFC2]
  任一非空集合中,都有一个元素与本集合不交,故任一集合不可能是自身的元素.
\end{axiom}

\begin{axiom}[(ZFC $4^{\mathrm{th}}$) : 配对公理][axiom:ZFC4]
  两个元素组成的集合是唯一的.
\end{axiom}
\begin{axiom}[(ZFC $5^{\mathrm{th}}$) : 并集公理][axiom:ZFC5]
  集合内如果有某些元素也是集合,则它们的并集是唯一的.
\end{axiom}
\begin{axiom}[(ZFC $7^{\mathrm{th}}$) : 无穷公理][axiom:ZFC7]
强行存在一个无限集$I$,$\exists I\left(\phi\in I\bigwedge\forall x\in I\left(\left(x\bigcup \{x\}\right)\in I\right)\right)$
\end{axiom}
\section{张量积与缩并}
\subsection{张量分析初步}
\begin{typicalbox}
  强行存在一个无限集$I$,$\exists I\left(\phi\in I\bigwedge\forall x\in I\left(\left(x\bigcup \{x\}\right)\in I\right)\right)$
\end{typicalbox}



\evenoutermargin{复几何 Complex Geometry} %偶数页页边内容
\setlength\ChapterBackGroundHeight{4\outermarginwidth}
    \chapter{复几何 Complex Geometry}
\section{局部理论}
\subsection{多变量的全纯函数}
本节主要介绍一些多复变的基本结论.
首先回忆一些单复变中的结论. 设 $U \in \mathbb{C}$ 为开集,函数 $f: U \rightarrow \mathbb{C}$ 称为全纯的, 若对于任意 $z_{0} \in U$ ,存在半径 $\varepsilon>0$ 的球 $B_{\varepsilon}\left(z_{0}\right) \subset U$ ,使 $f$ 在 $B_{\varepsilon}\left(z_{0}\right)$ 上可展开为收敛的幂级数,即
\begin{equation}
  f(z)=\sum_{n=0}^{\infty} a_{n}\left(z-z_{0}\right)^{n}, \forall z_{0} \in B_{\varepsilon}\left(z_{0}\right) .
\end{equation}
全纯的等价条件之一是Cauchy-Riemann方程. 设 $z=x+i y $,$ f$ 可视作两变量的复值函数,即 $f(x, y)=u(x, y)+i v(x, y)$. 则 $f$ 全纯当且仅当 $u$ 和 $v$ 是连续可微的,且
\begin{equation}\label{C.-R. Equation}
  \frac{\partial u}{\partial x}=\frac{\partial v}{\partial y}, \frac{\partial u}{\partial y}=-\frac{\partial v}{\partial x}
\end{equation}
定义微分算子
\begin{equation}\label{eq:1.3}
  \frac{\partial}{\partial z}:=\frac{1}{2}\left(\frac{\partial}{\partial x}-i \frac{\partial}{\partial y}\right), \frac{\partial}{\partial \bar{z}}:=\frac{1}{2}\left(\frac{\partial}{\partial x}+i \frac{\partial}{\partial y}\right)
\end{equation}
此时Cauchy-Riemann方程可写为
\[
  \frac{\partial f}{\partial \overline{z}}=0.
\]
全纯的另一个等价条件为Cauchy积分公式.一个函数$f\colon U\mapsto \mathbb{C}$为全纯的当且仅当它连续可微,且$\forall B_\varepsilon (z_0)$\newline $\subset U$,有下列公式
\begin{equation}\label{}
  f(z_0)=\frac{1}{2\pi i}\int_{\partial B_\varepsilon(x_0)} \frac{f(z)}{z-z_0}\dd z.
\end{equation}
此公式同样适用于函数$f\colon \overline{B_\varepsilon (z_0)}\to \mathbb{C}$,若它连续且在内部全纯.


下面是几个单复变中的重要结论.
\begin{theorem}[极大值原理][thm:极大值原理]
  设$U\in \mathbb{C}$为连通开集,$f\colon U\to \mathbb{C}$全纯且非常值,则$|f|$在$U$中没有局部最大值.

  若$U$有界且$f$可连续延拓到其边界上,则$|f|$的最大值在$\partial U$上取得.
\end{theorem}
\begin{theorem}[唯一性定理][thm:唯一性定理]
  设 $U \in \mathbb{C}$ 为连通开集, $f, g: U \rightarrow \mathbb{C}$ 为两个全纯函数, 且对一个开子集 $V \subset U$ 中 的所有 $z$ 有 $f(z)=g(z)$ ,则 $f \equiv g$.
\end{theorem}
\begin{theorem}[Riemman扩张定理][thm:Riemann 扩张定理]
  设 $f: B_{\varepsilon}\left(z_{0}\right) \backslash\left\{z_{0}\right\} \rightarrow \mathbb{C}$ 为有界全纯函数. 则 $f$ 可被唯一延拓为全纯函数 $f: B_{\varepsilon}\left(z_{0}\right) \rightarrow \mathbb{C}$
\end{theorem}
\begin{theorem}[Riemann映射定理][thm:Riemann映射定理]
  设 $U \subset \mathbb{C}$ 为单连通开真子集,则 $U$ 双全纯同胚于单位球 $B_{1}(0)$.


  也可以表述为:\;
  假设$\Omega$是适当(非整个复平面也非空集)的单连通区域.如果$z_0\in \Omega$,对于从该区域到单位开圆盘的共形映射
\[
  F\colon \Omega \to \mathbb{D}
\]
满足$F(z_0)=0,F\pr (z_0)>0$,那么该映射存在且唯一.
\end{theorem}
可以看出,这个定理说明了任何适当单连通区域与单位开圆盘是共形等价的.进一步,\textbf{任何适当单连通区域都是共形等价的.}


回顾一下拓扑的知识,我们知道适当单连通区域之间是同胚的,那么黎曼映射定理告诉我们,它们还在共形的意义下是等价的.


从分析的角度,共形映射是非常好的映射.这个定理告诉我们,如果一个函数的性质不好,但至少在局部我们是可以将其共形映射到其他好的地方.


从几何的角度,黎曼映射定理也告诉我们,保角映射在局部是成立的.
\subsection{证明前置准备}
需要用到极大模原理、Schwarz引理
\begin{lemma}[Schwarz引理][lem:Schwarz引理]
设 $f$ 为单位圆盘 $\mathbb{D}=\{z|| z \mid<1\}$ 到自身的解析映射,即 $f: \mathbb{D} \rightarrow \mathbb{D}$ ,并且满足 $f(0)=0$ ,那么我们有
\begin{enumerate}
  \item 对任意的 $z \in \mathbb{D}$ ,有 $|f(z)| \leq|z|$ .
  \item  若某一个 $z_{0} \neq 0$ ,我们有 $\left|f\left(z_{0}\right)\right|=\left|z_{0}\right|$ ,那么 $f$ 是一个旋转.
  \item  $\left|f^{\prime}(0)\right| \leq 1$ ,若有一个点 $z_{0}$ 使得等号成立,那么 $f$ 是一个旋转.
\end{enumerate}
\end{lemma}
现在,我们来介绍证明存在唯一性的思路.

\textbf{唯一性}


对于唯一性,我们可以假设存在满足条件的两个函数 $f,g$,并利用施瓦茨引理证明它们相等.因此唯一性非常好证.

\textbf{存在性}


存在性的证明非常长,因此我们分为两部分.


首先,我们将会引进一些数学工具,分别是正规,内闭一致有界,等度连续的概念,并且使用它们来证明\textbf{蒙泰尔定理}.随后,我们便开始着手于黎曼映射定理的证明步骤.
下面先来证明\textit{存在性}.
\begin{proof}
\textbf{正规族(normal family)}


让 $\Omega$ 是 $\mathbb{C}$ 上的开集,称全纯函数族 $\mathcal{F}$ 是正规的,如果 $\mathcal{F}$ 中任意序列都有内闭一致收敛子列.


可以发现,这其实是实数中有界数列必有收敛子列的函数版本,但为了证明这个定理,我们需要引 进其他的数学工具.在引入之前,我们先考察为什么对于函数族我们不能直接用.实际上这与一元 实函数列的收敛是类似的,逐点收敛不一定有一致收敛,因此我们要先引入一致这个很好的东西.

\textbf{内闭一致有界 (uniformly bounded on compact subsets)}


称函数族 $\mathcal{F}$ 是一致有界的,如果对于 $\Omega$ 中的任意紧集 $K \subset \Omega$ ,存在 $B>0$ ,使得 $|f(z)| \leqslant B \quad (\forall z \in K , \forall f \in \mathcal{F})$

\textbf{等度连续 (equicontinuous)}


称函数族 $\mathcal{F}$ 在紧集 $\Omega$ 上是等度连续的,如果 $\forall \varepsilon>0$ ,存在 $\delta>0$ ,使得当 $z, \omega \in K$ ,且$|z-w|<\delta$时,有
\[
  |f(z)-f(w)|<\varepsilon\; (\forall f\in \mathcal{F}).
\]

现在我们回过头来看看数学分析里面.内闭一致有界实际上是函数列一致有界在紧集上的版本,这个在数学分析中我们非常熟悉,一致有界保证了函数列整体能够被同一个上界所控制,而不是会随着函数的变化而变化.


一致连续说明了一个函数的平坦性,而\textbf{等度连续是比一致连续更强的概念},因为它对于整一个函数族中的函数都要成立.


那么现在我们要研究的函数是一个全纯函数列$\mathcal{F}$ ,那么它有什么良好的性质呢?我们有以下定理来说明它们之间的关系.
\begin{theorem}[][thm:穷竭紧集列]
  假设 $\mathcal{F}$ 是开集$\Omega$ 上全纯函数族,并且在$\Omega$ 的紧子集上一致有界,那么我们有
\begin{enumerate}
  \item $\mathcal{F}$在 $\Omega$ 的任何紧子集等度连续(内闭等度连续).
  \item (Montel's theorem) $\mathcal{F}$是一个正规族. 在叙述证明之前,我们需要引入一个辅助概念.
\end{enumerate}


穷竭(exhaustion)紧集列


在一个开集$\Omega$中,我们称一个紧集列$\{K_l \} |_{l=1}^\infty$是穷竭的当它满足
\begin{enumerate}
  \item 对任意的$l=1,2,\cdots, \quad K_l\subset K_{l+1}$;
  \item  任何紧集$K\in \Omega$ 包含在某一个 $K_l$中.特别的,我们有$\Omega=\bigcup_{l=1}^{\infty}K_l$.
\end{enumerate}
\end{theorem}
下面先证明\thmref{thm:穷竭紧集列}的存在性:

我们将$\Omega$分为两种情况.若$\Omega$有界,那么让$K_l=\{z\mid \abs{z-w}\geqslant 1/l,\forall w\in \partial\Omega\}$ ($\partial\Omega$为$\Omega$的边界); 若无界,则取在前面的集合加上$|z|\leqslant l$的条件即可.存在性得证.


我们可以发现,穷竭紧集列可以逼近整个开集.并且任意一个紧集都包含在该紧集列中的一个元素,因此我们可以用它来证明\textbf{内闭收敛性}.

\begin{proof}[\thmref{thm:穷竭紧集列}的证明]
  我们先给出第一部分的证明.记$D_r (z_0)=\{z\mid \abs{z-z_0}<r\}$,让$K$是$\Omega$的一个紧子集,选定足够小的$r>0$,使得$D_{3r}(z)\subset \Omega$(由于圆盘为开集,这是必定取得到的).让$z,w\in K$,且$\abs{z-w}<r$,让$\gamma$为$D_{2r}(w)$的边界.因此通过\textbf{Cauchy积分公式}我们有
  \[f(z)-f(w)=\frac{1}{2\pi i}\int_{\gamma}f(\xi)\zhkuohao{\frac{1}{\xi-z}-\frac{1}{\xi-w}}\dd \xi,\]
  随后我们要对其进行上界控制.由于$\xi\in \gamma,\abs{z-w}<r$,我们有
  \[\abs{\frac{1}{\xi-z}-\frac{1}{\xi-w}}=\frac{\abs{z-w}}{\abs{\xi-z}\abs{\xi-w}}\leqslant \frac{\abs{z-w}}{r^2}.\]
  因此假设$B=\sup_{f\in \mathcal{F}}f(z)$,其中$z\in D_{2r}(w)$我们有
  \[\abs{f(z)-f(w)}\leqslant \frac{1}{2\pi}\frac{\abs{z-w}}{r^2}B\int_\gamma \dd \xi=\frac{1}{2\pi}\frac{4\pi r}{r^2}B\abs{z-w}=\frac{2}{r}\abs{z-w},\]
  由于$\frac{1}{r}$的有界性,我们上述不等式写为
  \[\abs{f(z)-f(w)}\leqslant C\abs{z-w}, \]
  这样,这个等式对于任意的$z,w\in K,\abs{z-w}<r$和$f,\in \mathcal{F}$都成立,这就说明了$\mathcal{F}$的内闭等度连续性.

  随后我们证明第二部分.为了证明第二部分,我们要使用实数的\textbf{有界数列必有收敛子列}的性质和\textbf{对角线论证法}.

  让$\{f_n\}_{n=1}^\infty$是$\F$在紧集$K\subset \Omega$上的一个函数序列.随后我们选择在$\Omega$上稠密的一个点列$\{w_j\}_{j=1}^\infty$(可数稠密点列是可以取到的,比如稠密有理点列,原因会在后面揭晓).由于$\{f_n\}$是一致有界的,那么它一定存在一个子列$\{f_{n,1}\}=\{f_{1,1},f_{2,1},f_{3,1},\cdots\}$,使得$f_{n,1}(w_1)$点态收敛(有界数列必有收敛子列).

  随后我们可以仿照证明二元函数有界数列必有收敛子列的思路,取子列的子列,即在函数列$\{f_{n,1}\}$中取子列$\{f_{n,2}\}=\{f_{1,2},f_{2,2},f_{3,2}\cdots\}$,使得$f_{n,2}(w_2)$点态收敛.我们将其写为表格,即
  \begin{equation}
    \begin{pmatrix}
      f_{1,1}&f_{2,1}&\cdots &f_{1,n}&\cdots\\
f_{2,1}&f_{2,2}&\cdots &f_{2,n}&\cdots\\
\vdots &\vdots &\ddots &\vdots &\cdots\\
f_{n,1}&f_{n,2}&\cdots &f_{n,n}&\cdots\\
\vdots &\vdots &\ddots &\vdots &\cdots\\
    \end{pmatrix}
  \end{equation}
  这样,我们只要取$g_n=f_{n,n}$,并考虑其组成的对角线函数列$\{g_{n.n}\}$.显然的$g_n (w_j)$是点态收敛的.下面我们要用$g_n$的等度连续性,说明它的\textbf{一致连续性},给定一个$\varepsilon>0$,我们取等度连续性中的$\delta$,且由于$K$的紧性与$\{w_j\}_{j=1}^\infty$的稠密性,我们可以从可数覆盖集$\{D_\delta (w_j)\}$中取出一个有限覆盖,记为$\{d_\delta (w_1),\cdots,D_\delta (w_J)\}$,在其中我们都有点态收敛性.当$N$足够大,且$n,m>N$时,有
  \[\abs{g_m (w_j)- g_n (w_j)}<\varepsilon,\; \forall j=1,2\cdots,J.\]
  这里我们队员点态收敛,取收敛的点为稠密集,可以使用有限覆盖定理,保证两个不同函数差值能够被控制,且其中的$\delta$是与所选的点无关的(不是稠密集则没有这个性质).这样我们就可以使用经典的三角不等式对其进行控制了.若$z\in K$,那么对于某一个$j,z\in D_\delta (w_j)$,这样我们有
  \[\abs{g_n (z)-g_m (z)}\leqslant \abs{g_n (z)-g_n (w_j)}+\abs{g_n (w_j)-g_m (w_j)}+\abs{g_m (w_j)-g_m (z)}<3\varepsilon.\]
  这样我们就完成了对$g_n$一致收敛性的证明.\textbf{我们现在证明了一致收敛有界函数咧在一个紧集上必有一致收敛子列},下面就是用穷竭紧集列来证明内闭一致收敛了,技巧也是取子列的子列和对角线论证.取$\Omega$的一个穷竭紧集列$K_1\subset K_2 \subset \cdots \subset K_l\subset \cdots$,假定$\{g_{n,1}\}$是$\{f_n\}$在$\{g_{n.n}\}$在任意紧集$K_l$上都一致收敛.而由于$\Omega$中的每一个紧集都包含在某一个$K_l$中,我们就证完内闭一致收敛性了.

  在继续我们的证明前,我们先来考察这两个结论.对于第一个结论,这时解析函数列独有的特性,一个实函数的例子为 $f_n (x)=\sin (nx),x\in (0,1)$,该函数列一致有界,但不等度连续,并且在紧集上没有收敛子列.

对于第二个结论,不是解析函数列独有的.只要函数族$\F$是內闭一致有界且等度连续即可,解析函数的条件是可以去掉的(在证明的时候我们没有用到解析性!),对满足条件的任意的函数族我们称为 Arzela-Ascoli theorem .

这样我们来证明一下一个有用的推论.
\begin{corollary}[][cor:1]
  若 $\Omega$是$\mathcal{C}$ 上的连通开集,且$\{f_n\}$是单叶解析函数的函数列,在$\Omega$上内闭一致收敛.那么$f$是单射或者是一个常值映射.
\end{corollary}
\begin{proof}[\corref{cor:1}的证明]
  若$f$不是单射,那么存在$z_1\neq z_2$,使得$f(z_1)=f(z_2)$,定义微分算子\[g_n (z)=f_n (z)-f_n (z_1),\]
  它内闭一致收敛于$g(z)=f(z)-f(z_1)$.又若$f$不是常值函数,那么$z_2$是一个孤立零点,因此
  \[
    \frac{1}{2\pi i}\int_{\gamma} \frac{g\pr (\xi)}{g(\xi)}\dd \xi =1,
  \]
  这里$\gamma$是以$z_2$为圆心的一个小圆(由孤立零点,总可以取到使得只有该孤立零点的小圆),同时,在$\gamma$上
  \[
    \frac{1}{g_n}\to \frac{1}{g},
  \]
  且该收敛为一致收敛.这样我们会有一致收敛
  \[
    \frac{1}{2\pi i}\int_{\gamma} \frac{g_n\pr (\xi)}{g_n
    (\xi)}\dd \xi \to \frac{1}{2\pi i}\int_{\gamma} \frac{g\pr (\xi)}{g(\xi)}\dd \xi
  \]
  但是$g_n$在$\gamma$上是没有零点的($f(z)$的单射性),有
  \[
    \frac{1}{2\pi i}\int_{\gamma} \frac{g_n\pr (\xi)}{g_n (\xi)}\dd \xi=0\neq 1.
  \]
  这就产生了矛盾.
\end{proof}
\end{proof}
\textbf{Riemann映射定理的证明}
\textbf{唯一性证明}


设$f_1(z)$与$f_2 (z)$都是满足定理要求的函数,那么对于函数
\[
  F(z)=f_1 (f_2^{-1}(z))
\]
它把单位圆盘映射到单位圆盘,并且$F(0)=0$,由Schwarz引理,就有
\[
  \abs{F(z)}\leqslant \abs{z}
\]
那么将$z$用$f^{-1}(z)$替换,就有
\[
  \abs{f_1 (z)}\leqslant \abs{f_2 (z)},
\]
同理也有$\abs{f_2 (z)}\leqslant \abs{f_1 (z)}$,那么
\[
  \abs{f_1 (z)}= \abs{f_2 (z)}.
\]
则由于$f_1 (z_0)=f_2 (z_0)=0,f_1\pr (z_0)>0$和$f_2\pr (z_0)>0$,那么
\[
  \frac{f_1 (z)}{f_2 (z)}
\]
在单位圆盘内解析.而$\abs{\frac{f_1 (z)}{f_2 (z)}}=1$,那么两者之间仅相差一个旋转,设有某一个常数$\theta\in \mathbb{R}$,有
\[
  f_1 (z)=e^{i\theta}f_2 (z).
\]
但是两者在$z_0$的导数都是大于零的,因此$e^{i\theta}=1$,那么我们就证明了两者相等,即唯一性得证.

下面叙述存在性的证明.
\textbf{存在性证明}

\textbf{第一步,证明任意适当的单连通区域$\Omega$可以共形等价到包含原点的单位圆盘的子集}.\newline
由于单连通区域是适当的,因此我们能找到一个复数$\alpha\not\in \Omega$,这样对于函数
\[
  f(z)=\ln (z-\alpha)
\]
上的道路是不会绕过奇点的,因此是全纯的.这样,我们有$e^{f(z)}=z-\alpha$,若$e^{f(z_1)}=e^{f(z_2)}$,那么有$z_1=z_2$,因此$f(z)$是一个单射.由于$\Omega$不绕过奇点,并且是一个单射,因此对于$\Omega$中的一个点$z$任意变动,不会绕过远点一圈再回来,这样我们取一个点$w\in \Omega$,有
\[
  f(z)\neq f(w)+2\pi i,\; \forall z\in \Omega,
\]
下面使用反证法.如若不然,有一点$f(z_0)=f(w)+2\pi i$,两边取指数,就有$e^{f(z_0)}=e^{f(w)}$,则就有$z_0=w$ ,那么$2\pi i=0$,矛盾.同时,我们证明$f(w)$ 与$f(w)+2\pi i$严格分离(即两者的闭包之交为空集).如若不然,我们可以在$\Omega$ 取一点列 $\{z_n\}$,使得$f(z_n)\to f(w)+2\pi i$ ,但由于连续性,我们有 $z_n \to w$,但这又说明了 $f(z_n)\to f(w)$,矛盾.由严格分离性,我们考虑映射
\[
  F(z)=\frac{1}{f(z)-\kuohao{f(w)+2\pi}}
\]
在$\Omega$上是有界的.并且由于$f(z)$为单射, $F(z)$也为单射.这样我们对其做一个放缩
\[
  F_1 (z)=\frac{1}{1+\sup F(z)}F(z)
\]
便将$\Omega$映射到半径小于$1$的圆盘内,我们再做一个平移,即可使其包含原点.(使其包含原点是为了后面证明更方便)

\textbf{第二步,由共形等价,我们可以直接设 $\Omega$为$\mathbb{D}$ 上包含原点的开集.考虑上面的单叶解析函数族,并证明$\sup \abs{f\pr (0)}$是可以达到的.}

即我们考虑
\[
  \F =\{f\mid \Omega \to \mathbb{D},f(0)=0,f\text{单叶解析}\}
\]
首先,由于恒等映射在$\F$中,因此集合非空.同时,由于 $f$ 将单位圆盘的子集映射到单位圆盘,因此 $\F$是一致有界的.

这样,利用柯西不等式,我们估计$\abs{f\pr (0)}$,我们有

\[
  \abs{f\pr (0)}\leqslant \frac{\max_{z\in C}\abs{f(z)}}{\rho}
\]
其中 $C$为取定的半径为 $\rho$的开圆盘$D_\rho$ .这样,由于$\Omega$包含原点,我们可以取到一个$\overline{\rho}$(做闭包),使得该不等式对所有的$z$ 都成立.因此对于任意的 $f\in \F$,$\abs{f\pr (0)}$是一致有界的.这样我们令
\[
  s=\sup_{f\in \F}\abs{f\pr (0)},
\]
我们可以在$\F$中选取函数列$\{f_n\}$,其$\abs{f\pr (n)}\to s,n\to\infty$. 那样由蒙泰尔定理,可以取一个子列,使得其在$\Omega$上的任意紧集内闭一致收敛到$s$.并且,恒等映射在 $\F$中, 那么$s\geqslant 1$.我们假设$f$不是常值函数(总能取到,比如恒等映射),由前面的\corref{cor:1}, $f$就是单射.


又由于连续性,对$z\in\Omega$ ,有 $\abs{f(z)}\leqslant 1$.又由最大模原理,就有 $\abs{f(z)}<1$.那么由连续性,$f(0)=0$ ,这样$f\in\F$,且有 $\abs{f\pr (0)}=s$ .


\textbf{第三步,我们论证上面得到的$f$是一个从$\Omega$到 $\mathbb{D}$ 的共形映射.即$f$就是我们要找的共形映射.}


我们已经知道所得到的 $f$ 是单叶解析的,下面要说明它是一个满射,即$f$ 映满了$\mathbb{D}$ .我们使用反证法,若有一个 $\alpha\in\mathbb{D}$ 使得$f(z)\neq \alpha$,那么我们考虑单位圆盘的自同构

\[\psi_\alpha (z)=\frac{\alpha-z}{1-\overline{\alpha} z}\]

因此该自同构将 $\alpha$ 映射到$0$ (也将 $0$映射到 $\alpha$ ),那么区域$\Omega$经过两个映射的复合$U=(\psi_\alpha \circ f)(\Omega)$ 后,我们发现区域$D$ 是不包含原点的(因为$f(z)$不会等于 $\alpha$ ).


这样我们论述了一件事情,那就是我们将包含原点的$\Omega$ 可以共形等价到一个不包含原点的 $U$中,且$U$ 还是单位圆盘$\mathbb{D}$的子集.这样是有好处的,因为我们下面就要使用平方根函数,去将区域$U$ 向外扩张,并由此与 $\abs{f\pr (0)}$上确界性质相悖.


考虑一个新函数
\[g(z)=\sqrt{z}\; (z\in U),\]
那么由于$U$是不包含原点的,选定一个解析分支,$g(z)$是一个解析函数.这样我们就可以考虑
\[F=\psi_{g(\alpha)}\circ g\circ \psi_\alpha \circ f\]
我们下面要证明$F\in \F$.可以看出 $F(0)$就是解析的,且
\[F\colon 0\mapsto 0\mapsto \alpha \mapsto \sqrt{\alpha}\mapsto 0\]
因此 $F(0)=0$ .并且由于其中的函数都不会将$\Omega$ 中的元素映射到单位圆盘外,那么 $F(z)$就将 $\Omega$ 映射到$\mathbb{D}$.我们还需证明它是一个单射,这是显然的,因为 $f,\psi_{\alpha},\psi_{g(\alpha)}$ 都是单射的,我们只需考察平方根函数.但这也明显是对的,这由$U$ 不包含原点可以推出(注意 $\Omega$是单连通,那么 $U$也是单连通).

那么我们现在写出 $f$的表达式,我们有
\[f=\psi_{\alpha}^{-1}\circ g^{-1}\circ \psi_{g(\alpha)}^{-1}\circ F\]
但要注意
\[g^{-1}(w)=w^2\]
不是一个单射.现在我们考察$f\pr (0)$与 $F\pr (0)$.为了看的更方便,我们写出映射
\[0\stackrel{F}{\mapsto}0\stackrel{\psi_{g(\alpha)}^{-1}}{\mapsto}g(\alpha)\stackrel{g^{-1}}{\mapsto}\alpha\stackrel{\psi_\alpha^{-1}}{\mapsto}0,\]
那么我们发现,记 $\Psi=\psi_\alpha^{-1}\circ g^{-1}\circ \psi_{g(\alpha)}^{-1}$那么
\[\Psi\colon 0\mapsto 0\]
且由于$g^{-1}$为平方函数,那么 $\Psi$将单位圆盘映射到单位圆盘.那么由施瓦茨引理,我们就有$\abs{\Psi\pr (0)}\leqslant 1$并且,由于经过了平方函数,该映射不可能是一个旋转,因此等号不成立,我们就有
\[\abs{\Psi\pr (0)}<1,\]
那么我们就有
\[f\pr (0)=\Psi\pr (0)F\pr (0),\]
这样使用两边取模,就有
\[\abs{f\pr (0)}<\abs{F\pr (0)}.\]
这与 $\abs{f\pr (0)}$上确界的性质相悖,证毕.
\end{proof}
\renewcommand{\Remark}{$\bccrayon$ 总~结}
\begin{remark}[]
实际上,证明的主要思路就是,将任意单连通区域压缩到单位圆盘内,然后再使其慢慢扩充到整个单位圆盘.具体的扩充操作上面已经给出,不包含原点的单连通区域可用平方根函数,包含原点的单连通区域就将其共形映射到另外一个不包含原点的单连通区域,扩张后再利用共形映射的逆映射,即可扩充.


在证明的时候,多次使用共形等价的性质.我们从拓扑的观点来看,解析同胚(共形等价)实际上是比同胚更强的性质,因为它还加上解析的条件.
\end{remark}
\renewcommand{\Remark}{$\bcbook$ 注}
\begin{theorem}[Liouville定理][thm:Liouville定理]
  有界全纯函数 $f: \mathbb{C} \rightarrow \mathbb{C}$ 是常值. 因此 $\mathbb{C}$ 与 $B_{\varepsilon}(0)$ 不双全纯同胚.
\end{theorem}
\begin{theorem}[留数定理][thm:留数定理]
  设 $f: B_{\varepsilon}(0) \backslash\{0\} \rightarrow \mathbb{C}$ 为全纯函数, 则 $f$ 可被展开为Laurant级数 $f(z)=\sum_{n=-\infty}^{\infty} a_{n} z^{n}$. 其中系数 $a_{-1}=(1 / 2 \pi i) \int_{|z|=\varepsilon / 2} f(z) d z$.
\end{theorem}


将全纯的概念扩充至多个变量. 首先定义\textbf{多重圆盘}:  $B_{\varepsilon}(w)=\left\{z|| z_{i}-w_{i} \mid<\varepsilon_{i}\right\}$ ,其中 $\varepsilon:=\left(\varepsilon_{1}, \ldots, \varepsilon_{n}\right)$.
\begin{definition}[][def:1.1.1]
设 $U \subset \mathbb{C}^{n}$ 为开集, $f: U \rightarrow \mathbb{C}$ 为连续可微函数. 称 $f$ 为全纯的, 若Cauchy-Riemann方程\eqref{C.-R. Equation}对所有坐标$z_i=x_i+iy_i$成立,即
\begin{equation}\label{C.-R.}
  \frac{\partial u}{\partial x_i}=\frac{\partial v}{\partial y_i},\frac{\partial u}{\partial y_i}=\frac{\partial v}{\partial x_i},i=1,\cdots,n.
\end{equation}
定义微分算子
\[
  \frac{\partial}{\partial z_i}:=\frac{1}{2}\kuohao{\Diff{x_i}-i\Diff{y_i}},\Diff{\overline{z}_i}:=\frac{1}{2}\kuohao{\Diff{x_i}+i\Diff{y_i}}.
\]
此时\eqref{C.-R.}可被写为
\[
  \Diff[f]{\overline{z}_i}=0,\forall i=1,\cdots,n.
\]
\end{definition}
下面讨论多变量函数的Cauchy积分公式.

    \chapter{泛函分析}
\section{赋范空间}
\begin{theorem}[线性空间][thm:fufankongjian]
	学习
\end{theorem}

%\begin{tikzpicture}[options]
%	\begin{scope}[latex-latex]
%		\filldraw[fill=, draw=] ;
%	\end{scope}
%\end{tikzpicture}













%\clearpage
%% change some settings
%\colorlet{outermarginbgcolor}{orange!30}
%\colorlet{outermarginfgcolor}{orange}
%\addtokomafont{outermargin}{\color{blue!50!green}}
%\evenoutermargin{}
%\partabstract{} % Part简介

\partabstract{}
\part*{附录}
%%---------------------------------------  APPENDIX  START  -------------------------------------%%
%% Letters for chapters
    \appendix
%\clearpage
%\colorlet{outermarginbgcolor}{teal!30}
%\colorlet{outermarginfgcolor}{teal}
%\addtokomafont{outermargin}{\color{teal!50!red}}
%\evenoutermargin{}
%\partabstract{} % Part简介

%%----------------------------------------  APPENDIX  END  ---------------------------------------%%


%%---------------------------------------  BIBLIOGRAPHY  START  -------------------------------------%%
%% Prevent urls running into margins in bibliography
\setcounter{biburlnumpenalty}{7000}
\setcounter{biburllcpenalty}{7000}
\setcounter{biburlucpenalty}{7000}

%% Add bibliography
\printbibliography[heading=bibintoc,title= 参考文献]

%%-----------------------------------------  BIBLIOGRAPHY  END  ---------------------------------------%%


%%------------------------------------------  INDEX  START  ----------------------------------------%%
    \printindex
%%-------------------------------------------  INDEX  END  ------------------------------------------%%
\end{document}
