\newcommand*{\x}{\mathbf{x}}
\newcommand*{\y}{\mathbf{y}}
\newcommand*{\z}{\mathbf{z}}
\chapter{Differentiable Manifolds}
\section{Exercises 0}
\begin{proposition}[][Proposition A.17 (Properties of the Subspace Topology).]
    Let $X$ be a topological space and let $S$ be a subspace of $X$.\cite{Huybrechts2010Complex}
    \begin{enumerate}[label=(\alph*)]
        \item {\color{靛蓝}\textsc{Characteristic Property}: If $Y$ is a topological space, a map $F: Y \rightarrow S$ is continuous if and only if the composition  $\iota_S \circ F: Y \rightarrow X$ is continuous, where $\iota_S: S \hookrightarrow X$ is the inclusion map (the restriction of the identity map of $X$ to $S$ ).}
        \item The subspace topology is the unique topology on $S$ for which the characteristic property holds.
        \item A subset $K \subseteq S$ is closed in $S$ if and only if there exists a closed subset $L \subseteq X$ such that $K=L \cap S$.
        \item The inclusion map $\iota_S: S \hookrightarrow X$ is a topological embedding.
        \item If $Y$ is a topological space and $F: X \rightarrow Y$ is continuous, then $\left.F\right|_S: S \rightarrow Y$ (the restriction of $F$ to $S$ ) is continuous.
        \item If $\mathscr{B}$ is a basis for the topology of $X$, then $\mathscr{B}_S=\{B \cap S: B \in \mathscr{B}\}$ is a basis for the subspace topology on $S$.
        \item If $X$ is Hausdorff, then so is $S$.
        \item If $X$ is first-countable, then so is $S$.
        \item If $X$ is second-countable, then so is $S$.
    \end{enumerate}
\end{proposition}
\begin{proposition}[][Proposition A.23 (Properties of the Product Topology).]
    Suppose $X_1, \ldots, X_k$ are topological spaces, and let $X_1 \times \cdots \times X_k$ be their product space.
    \begin{enumerate}[label=(\alph*)]
    \item {\color{靛蓝}\textsc{Characteristic Property}: If $B$ is a topological space, a map $F: B \rightarrow$ $X_1 \times \cdots \times X_k$ is continuous if and only if each of its component functions $F_i=$ $\pi_i \circ F: B \rightarrow X_i$ is continuous.}
    \item The product topology is the unique topology on $X_1 \times \cdots \times X_k$ for which the characteristic property holds.
    \item Each projection map $\pi_i: X_1 \times \cdots \times X_k \rightarrow X_i$ is continuous.
    \item Given any continuous maps $F_i: X_i \rightarrow Y_i$ for $i=1, \ldots, k$, the product map $F_1 \times \cdots \times F_k: X_1 \times \cdots \times X_k \rightarrow Y_1 \times \cdots \times Y_k$ is continuous, where
    \begin{align*}
    F_1 \times \cdots \times F_k\left(x_1, \ldots, x_k\right)=\left(F_1\left(x_1\right), \ldots, F_k\left(x_k\right)\right) .
    \end{align*}
    \item If $S_i$ is a subspace of $X_i$ for $i=1, \ldots, n$, the product topology and the subspace topology on $S_1 \times \cdots \times S_n \subseteq X_1 \times \cdots \times X_n$ coincide.
    \item For any $i \in\{1, \ldots, k\}$ and any choices of points $a_j \in X_j$ for $j \neq i$, the map $x \mapsto\left(a_1, \ldots, a_{i-1}, x, a_{i+1}, \ldots, a_k\right)$ is a topological embedding of $X_i$ into the product space $X_1 \times \cdots \times X_k$.
    \item If $\mathscr{B}_i$ is a basis for the topology of $X_i$ for $i=1, \ldots, k$, then the collection
    \begin{align*}
    \mathscr{B}=\left\{B_1 \times \cdots \times B_k: B_i \in \mathscr{B}_i\right\}
    \end{align*}
    is a basis for the topology of $X_1 \times \cdots \times X_k$.
    \item Every finite product of Hausdorff spaces is Hausdorff.
    \item Every finite product of first-countable spaces is first-countable.
    \item Every finite product of second-countable spaces is second-countable.
\end{enumerate}
\end{proposition}
\begin{example}[][Example 1.8 (Product Manifolds).][Exam:1.8]
    Suppose $M_1, \ldots, M_k$ are topological manifolds of dimensions $n_1, \ldots, n_k$, respectively. The product space $M_1 \times \cdots \times M_k$ is shown to be a topological manifold of dimension $n_1+\cdots+n_k$ as follows. \textbf{\itshape It is Hausdorff and second-countable by Propositions A.17 (g), (i) and A.23 (h), (j), so only the locally Euclidean property needs to be checked.} Given any point $\left(p_1, \ldots, p_k\right) \in$ $M_1 \times \cdots \times M_k$, we can choose a coordinate chart $\left(U_i, \varphi_i\right)$ for each $M_i$ with $p_i \in U_i$. {\bf\itshape The product map
\begin{align*}
\varphi_1 \times \cdots \times \varphi_k: U_1 \times \cdots \times U_k \rightarrow \mathbb{R}^{n_1+\cdots+n_k}
\end{align*}
is a homeomorphism onto its image, which is a product open subset of $\mathbb{R}^{n_1+\cdots+n_k}$.} Thus, $M_1 \times \cdots \times M_k$ is a topological manifold of dimension $n_1+\cdots+n_k$, with charts of the form $\left(U_1 \times \cdots \times U_k, \varphi_1 \times \cdots \times \varphi_k\right)$.

\end{example}
\begin{example}[][Example 1.9 (Tori).]
    For a positive integer $n$, the $\boldsymbol{n}$-torus (plural: tori) is the product space $\mathbb{T}^n=\mathbb{S}^1 \times \cdots \times \mathbb{S}^1$. By the discussion above, it is a topological $n$-manifold. (The 2-torus is usually called simply the torus.)
\end{example}
\begin{example}[][Example 1.34 (Smooth Product Manifolds).][Exam:1.34]
    If $M_1, \ldots, M_k$ are smooth manifolds of dimensions $n_1, \ldots, n_k$, respectively, we showed in \autoref{Exam:1.8} that the product space $M_1 \times \cdots \times M_k$ is a topological manifold of dimension $n_1+\cdots+n_k$, with charts of the form $\left(U_1 \times \cdots \times U_k, \varphi_1 \times \cdots \times \varphi_k\right)$. {\bf\itshape Any two such charts are smoothly compatible because, as is easily verified,
    \begin{align*}
    \left(\psi_1 \times \cdots \times \psi_k\right) \circ\left(\varphi_1 \times \cdots \times \varphi_k\right)^{-1}=\left(\psi_1 \circ \varphi_1^{-1}\right) \times \cdots \times\left(\psi_k \circ \varphi_k^{-1}\right),
    \end{align*}
    which is a smooth map.} This defines a natural smooth manifold structure on the product, called the product smooth manifold structure. For example, this yields a smooth manifold structure on the $n$-torus $\mathbb{T}^n=\mathbb{S}^1 \times \cdots \times \mathbb{S}^1$.
\end{example}
\begin{Exercise}[][Product Manifold]
    Let $M$ and $N$ be differentiable manifolds and let $\{(U_\alpha,\x_\alpha)\}, \{(V_\alpha,\y_\alpha)\}$ be differentiable structure on $M$ and $N$, respectively. Consider the cartesian product $M\times N$ and the mappings $\z_{\alpha\beta}(p,q)=(\x_\alpha (p),\y_\beta (q)), p\in U_\alpha, q\in V_\beta$.
    \begin{enumerate}[label=(\bf\alph*)]
        \item Prove that $\{(U_{\alpha\beta},\z_{\alpha\beta})\}$ is a differentiable structure on $M\times N$ in which the projections $\pi_1\colon M\times N \to M$  and $\pi_2\colon M\times N \to N$ are differentiable. With this differentiable structure $M\times N$ is called the product manifold of $M$ with $N$.
        \item Show that the product manifold $S^1\times \cdots\times S^1$ of $n$ circles $S^1$, where $S^1\subset \bR^2$ has the usual differentiable structure , is diffeomorphic to the $n$ torus $T^n$ of \textsf{Example 4.9 (a)}.
    \end{enumerate}
\end{Exercise}
\begin{proof}
For (a), \begin{enumerate}
    \item (Open Covering) As 
    \begin{align*}
        \z_{\alpha\beta} \colon & U_\alpha\times V_\beta \to \x_\alpha (U_\alpha)\times \y_{\beta}(V_\beta)\subset M\times N\\ 
        &(p,q) \mapsto (\x_\alpha(p),\y_\beta(q))
    \end{align*}
        is a homeomorphism, there is a open covering of $M\times N$ , i.e. 
        \[\bigcup_{\alpha,\beta}\z_{\alpha\beta}(U_\alpha,V_\beta)=\bigcup_\alpha \x_\alpha (U_\alpha)\times \bigcup_\beta \y_\beta (V_\beta)=M\times N.\]
    \item (Atlas Compatibility) When $\z_{\alpha\beta}(U_\alpha,V_\beta)\bigcap \z_{\gamma\delta}(U_\gamma,V_\delta)\neq \emptyset$, one has 
    \[\z_{\gamma\delta}^{-1}\circ \z_{\alpha\beta} (p,q)=\z_{\gamma\delta}^{-1}(\x_\alpha(p),\y_\beta(q))=(\x_\gamma^{-1}\circ \x_\alpha(p),\y_\delta^{-1}\circ \y_\beta(q))\]
    is differentiable, i.e. any two charts of atlas are compatible.
\end{enumerate}
Thus, by definition, with this differentiable structure, $M\times N$ is a differentiable manifold.

For (b), define a mapping
\begin{align*}
\begin{aligned}
& \varphi: \mathbb{R}^n / G \rightarrow \overbrace{\mathbb{S}^1 \times \ldots \times \mathbb{S}^1}^{\mathrm{n} \text { copies }} \\
& \left(x_1, \ldots, x_n\right) \mapsto\left(e^{2 \pi i \frac{x_1}{m_1}}, \ldots, e^{2 \pi i \frac{x_n}{m_n}}\right)
\end{aligned}
\end{align*}
It is obviously that $\varphi$ is an immersion and submersion from its Jacobi matrix:
\begin{align*}
J(\varphi)=\left(\begin{array}{ccc}
\frac{2 \pi i}{m_1} e^{2 \pi i \frac{x_1}{m_1}} & \\
& \ddots & \\
& & \frac{2 \pi i}{m_n} e^{2 \pi i \frac{x_n}{m_n}}
\end{array}\right)
\end{align*}
So $\varphi$ is diffeomorphism.
\end{proof}

\begin{Exercise}
    Prove that the tangent bundle of a differentiable manifold $M$ is orientable (even though $M$ may be not).
\end{Exercise}
\begin{proof}
    Choose a parametrization $\left\{\left( U_\alpha, X_\alpha\right)\right\}$ of M. it induce a parametrization of TM:
\begin{align*}
\begin{aligned}
 Y_\alpha : & U_\alpha \times\mathbb{R}^n \rightarrow \pi^{-1}\left(X_\alpha\left(U_\alpha\right)\right)\subset \textrm{TM} \\
& (x, t) \mapsto\left(X_\alpha(x), d X_\alpha(t)\right)
\end{aligned}
\end{align*}

The corresponding coodinate transformation is
\begin{align*}
Y_\beta^{-1} Y_\alpha(x, t)=\left(X_\beta^{-1} X_\alpha(x), d X_\beta^{-1} d X_\alpha(t)\right)=\left(X_\beta^{-1} X_\alpha(x), d\left(X_\beta^{-1} X_\alpha\right)(t)\right)
\end{align*}
To calculate its Jacobi matrix, we get
\begin{align*}
J\left(Y_\beta^{-1} Y_\alpha\right)=\left[\begin{array}{cc}
J & O \\
* & J
\end{array}\right], \quad J=J\left(X_\beta^{-1} X_\alpha\right)
\end{align*}
And its determination
\begin{align*}
\operatorname{det} J\left(Y_\beta^{-1} Y_\alpha\right)=(\operatorname{det} J)^2>0
\end{align*}
So, TM is an orientable manifold.
\end{proof}

\begin{Exercise}
    Prove that:
    \begin{enumerate}[label=(\alph*)]
        \item A regular surface $S \subset \mathbb{R}^3$ is an orientable manifold if and only if there exists a differentiable mapping of $N: S \rightarrow \mathbb{R}^3$ with $N_p \perp T_p S,\left|N(p)\right|=1, \forall p \in S$,
        \item the M\"{o}bius band (Example 4.9 (b)) is non--orientable.
    \end{enumerate}
\end{Exercise}

\begin{proof}
    \begin{enumerate}[label=(\alph*)]
        \item $(\Rightarrow)$ Choose a parametrization $\left\{\left(X_\alpha, U_\alpha\right)\right\}$ of $S$. For any points $p \in X_\alpha\left(U_\alpha\right) \cap X_\beta\left(U_\beta\right) \subset S$, which $p=X_\alpha\left(x_1, x_2\right)=X_\beta\left(y_1, y_2\right)$. Since $S$ is an orientable, thus
$\operatorname{det} d\left(X_\beta^{-1} \circ X_\alpha\right)>0, \quad \forall \alpha, \beta$
Define a mapping
\begin{align*}
N(p)=N\left(X_\alpha\left(x_1, x_2\right)\right):=\frac{\frac{\partial}{\partial x_1} \wedge \frac{\partial}{\partial x_2}}{\left|\frac{\partial}{\partial x_1} \wedge \frac{\partial}{\partial x_2}\right|}, \frac{\partial}{\partial x_j} \in T_p S=\mathbb{R}^2 \subset \mathbb{R}^3
\end{align*}
Then
\begin{align*}
N(p)&=N\left(X_\beta\left(y_1, y_2\right)\right)\\
&=\frac{\frac{\partial}{\partial y_1} \wedge \frac{\partial}{\partial y_2}}{\left|\frac{\partial}{\partial y_1} \wedge \frac{\partial}{\partial y_2}\right|}=\frac{\operatorname{det}\left(d\left(X_\beta^{-1} \circ X_\alpha\right)\right)}{\left|\operatorname{det}\left(d\left(X_\beta^{-1} \circ X_\alpha\right)\right)\right|} \cdot \frac{\frac{\partial}{\partial x_1} \wedge \frac{\partial}{\partial x_2}}{\left|\frac{\partial}{\partial x_1} \wedge \frac{\partial}{\partial x_2}\right|}=\frac{\frac{\partial}{\partial x_1} \wedge \frac{\partial}{\partial x_2}}{\left|\frac{\partial}{\partial x_1} \wedge \frac{\partial}{\partial x_2}\right|}
\end{align*}
$N(p)$ is well defined.
$(\Leftarrow)$ Suppose that $N(p)$ is a differentiable mapping as we known.
        \item We say that $\mathrm{M}$ is orientable if and only if there exists an atlas $A=\left\{\left(U_\alpha, \phi_\alpha\right)\right\}$ such that $\operatorname{det}\left(J\left(\phi_\alpha \circ \phi_\beta^{-1}\right)\right)>0$, if it is defined.

        Assume that the M\"obius strip (band) is orientable. Then we would be able to define a map: $x \rightarrow \boldsymbol{n}_{\boldsymbol{x}}$ that sends $x$ to a unit vector normal to the surface in such a way that the map is continuous. Since $M$ is two-dimensional and embedded in 3-space, this map is determined by the value at a single point (because you have two choices, one in each direction from the surface). Now observe that if you follow a loop around the strip, the value of $\boldsymbol{n}_{\boldsymbol{x}}$ changes sign when you return to $\boldsymbol{x}$ from the other side.

        Reference: \href{https://mathinsight.org/moebius_strip_not_orientable}{The M\"obius strip (band) is non-orientable.}
    \end{enumerate}
    
\end{proof}
4.
\begin{proof}
    For simplicity, assume (without loss of generality) that in the following, by "chart" I mean ``connected chart". 
    
    By definition, a manifold $M$ is orientable iff you can find a covering of $M$ by coherently oriented charts. Meaning that the determinant of the derivative of the transition functions between overlapping charts (i.e. $\det(J)$) are all positive. Next, note that given two charts $(U,f), (V,g)$ of $M$ of opposite orientation (i.e. the transition function is of negative determinant), the charts $(U,f^*),(V,g)$ are of the same orientation, where $f^*$ is defined by changing the sign of the first coordinate of $f$. Call $(U,f^*)$ the ``modified" chart $(U,f)$. 
    
    Thirdly, note that a manifold $M$ is orientable iff given any covering of $M$ by charts, it is always possible to modify them (in the above sense) to obtain a coherently oriented covering of $M$. 
    
    $(\Longleftarrow)$ Clearly.
    
    $(\Longrightarrow)$ Let $\{(U_i,f_i)\}$ be any covering of $M$ by charts. Take a coherently oriented covering of $M$ by charts $\{(V_j,g_j)\}$. For each $U_i$, chose a $ V_j $ that intersects it. Are $U_i$ and $V_j$ of the same orientation? If so, do nothing. If not, modify $(U_i,f_i)$ to $(U_i,f^*_i)$. Note that for two charts to have the same orientation is an equivalence relation on the set of all charts of a manifold. Clearly then, this process makes $\{(U_i,f_i)\}$ into a coherently oriented covering.  
    
    Now let $M$ be $2$ dimensional. Let $X=([0,1] x ]-1,1[)/(0,t)~(1,-t)$ be the M\"obius band, and $f:X\longrightarrow M$ be an embedding. Consider $\{(U_i,\phi_i)\}$ a covering of $f(X)$ by charts lying entirely in $f(X)$ ($f(X)$ is open in $M$). Since the ``pullback charts" $(f^{-1}(U_i), \phi_i \circ f)$ make up a covering of the nonorientable manifold $X$, it is impossible to modify the covering $\{(U_i,\phi_i)\}$ to make it coherently oriented. Then just extend $\{(U_i,\phi_i)\}$ to a covering of the whole of $M$. This covering either cannot be modified to be coherently oriented, so $M$ is nonorientable. %QED

Reference: \href{https://www.physicsforums.com/threads/how-to-understand-that-rp2-is-non-orientable.420456/}{$\bR P^n$ is non-orientable.}

\end{proof}
5.
\begin{proof}
    (a) $\tilde{\varphi}: P^2 \rightarrow \mathbb{R}^4$ is a immersion.
Pf: Since
\begin{align*}
d \tilde{\varphi}_{[p]}=d \varphi_p=J_p(F)=\left(\begin{array}{rrrr}
2 x & y & z & 0 \\
-2 y & x & 0 & z \\
0 & 0 & x & y
\end{array}\right)^T
\end{align*}
Let $p=(0,0,1)$ For symmetry of sphere $\mathbb{S}^2$.
\begin{align*}
\Rightarrow J_p(F)=\left(\begin{array}{llll}
0 & 0 & 1 & 0 \\
0 & 0 & 0 & 1 \\
0 & 0 & 0 & 0
\end{array}\right)^T
\end{align*}
$\operatorname{rank} d F_p=\operatorname{dim} T_p \mathbb{S}^2=2$, So $\tilde{\varphi}$ is a immersion.

(b) $\tilde{\varphi}$ is injective; together with (a) and the compactness of $P^2$, this implies that $\tilde{\varphi}$ is an embedding.
Pf: If $\tilde{\varphi}([p])=\tilde{\varphi}([q])$, where $p=(x, y, z), q=\left(x^{\prime}, y^{\prime}, z^{\prime}\right)$, it means
\begin{align*}
\begin{cases}x^2+y^2+z^2=x^{\prime 2}+y^{\prime 2}+z^{\prime 2}=1 \\ x^2-y^2=x^2-y^{\prime 2} \\ x y=x^{\prime} y^{\prime} & (* 1) \\ x z=x^{\prime} z^{\prime} & (* 2) \\ y z=y^{\prime} z^{\prime} & (* 3)\end{cases}
\end{align*}
From $(* 1) \times(* 2)$ we obtained
\begin{align*}
x^2 y z=x^2 y^{\prime} z^{\prime}
\end{align*}
If $y z \neq 0$, by $(* 3)$ we get
\begin{align*}
x= \pm x^{\prime}
\end{align*}
By the continuity of the equations, when $y \rightarrow 0$ or $z \rightarrow 0$, it still holds.
Similarly, we can get $y= \pm y^{\prime}, z= \pm z^{\prime}$. Thus, $p=-q \in[p]$.
\end{proof}

6.
\begin{Exercise}
    Show that the mapping $G\colon \bR^2\to\bR^4$ given by 
    \[G(x,y)=((r\cos y+a)\cos x,(r\cos y+a)\sin x, r\sin y\cos \frac{x}{2},r\sin x\sin \frac{x}{2}), \]
    with $(x,y)\in\bR^2$ induces an embedding of the klein bottle into $\bR^4$.
\end{Exercise}

\begin{proof}
% To show that the mapping $G\colon \bR^2\to\bR^4$ given by 
% \[G(x,y)=((r\cos y+a)\cos x,(r\cos y+a)\sin x, r\sin y\cos \frac{x}{2},r\sin x\sin \frac{x}{2}), (x,y)\in\bR^2\]
% induces an embedding of the Klein bottle into $\bR^4$, we need to show that $G$ is an immersion (i.e. its derivative is injective at every point), and that $G$ is a homeomorphism onto its image.
% First, we compute the derivative of $G$:
% $$
% DG(x,y) = \begin{pmatrix}
% -(r\cos y +a) \sin x & (r\cos y+a)\cos x & -\frac{1}{2}r\sin y \sin \frac{x}{2} & \frac{1}{2}r\cos y\cos \frac{x}{2}\\
% (-r\sin y +a) \cos x & (-r\sin y+a)\sin x &r\cos y\cos \frac{x}{2} & r\cos y\sin \frac{x}{2}\\
% \cos y \cos x & \cos y \sin x &\sin y \cos \frac{x}{2} & \sin y \sin \frac{x}{2}\\
% \cos x & \sin x & 0 & 0
% \end{pmatrix}.
% $$
% The derivative $DG(x,y)$ has rank 4 unless $r=0\text{ or }1$ and $a=0$ (when $r=0$, the mapping degenerates to a point, while when $r=1$ and $a=0$, the mapping corresponds to the usual parametrization of the torus). So $G$ is non-degenerate.

% To show that $G$ is a homeomorphism onto its image, we need to show that $G$ is continuous, injective (so that we can define its inverse), and that the inverse is also continuous. Continuity of $G$ follows from the continuity of its component functions and elementary properties of the trigonometric functions.

% To show that $G$ is injective, we need to show that distinct points in the domain of $G$ are mapped to distinct points in the image of $G$. This follows from the fact that the horizontal and vertical components of $G(x,y)$ depend only on $x$ and $y$ respectively, while the other two components involve both $x$ and $y$ in a nontrivial way.

% To show that the inverse of $G$ is continuous, we need to show that the preimage of any open set in the image of $G$ is open in the domain of $G$. This can be done using the inverse function theorem, which guarantees that $G$ is a local diffeomorphism at every point where $DG$ is invertible.
% Therefore, we have shown that $G$ induces an embedding of the Klein bottle into $\bR^4$.

To show that $G$ induces an embedding of the Klein bottle into $\bR^4$, we need to show that $G$ is an immersion and a homeomorphism onto its image.

First, we will show that $G$ is an immersion. To do this, we need to show that $DG(x,y)$ is injective for all $(x,y) \in \bR^2$. We have already computed the matrix $DG(x,y)$ in the previous answer, so we only need to check that its determinant is nonzero. The derivative of $G$ at $(x,y)\in \bR^2$ is the $4\times 2$ matrix 
$$
DG(x,y) = \begin{pmatrix}
-(r\cos y +a) \sin x & (r\cos y+a)\cos x &a-r\sin y \sin \frac{x}{2} & \frac{1}{2}r\cos x\cos \frac{x}{2}\\
-(r\cos y +a) \cos x &a-(r\cos y+a)\sin x & 0 & \frac{1}{2}r\cos x\sin \frac{x}{2}\\
0 & -r\cos y \cos \frac{x}{2} & -\frac{1}{2}r\sin y\sin \frac{x}{2} & r\sin y\cos \frac{x}{2}\\
0 &-r\sin y\sin \frac{x}{2} & \frac{1}{2}r\sin x\cos \frac{x}{2} & \frac{1}{2}r\sin x\sin \frac{x}{2}
\end{pmatrix}.
$$

Then the determinant of $\det DG(x,y)$ is 
$$
\det DG(x,y) = -\frac{1}{2}r^2\sin^2 x\sin y \neq 0,
$$
\textit{\textbf{ so $DG(x,y)$ is always invertible (i.e. $G$ is non-degenerate). Therefore,  $G$ is an immersion}}.

Next, we will \itbf{ show that $G$ is a homeomorphism onto its image}. To do this, we need to show that \itbf{ $G$ is bijective, continuous, and has a continuous inverse}.

To show that $G$ is bijective, we need to show that distinct points in the domain of $G$ are mapped to distinct points in the image of $G$, and that every point in the image of $G$ is the image of some point in the domain of $G$. 

The first part follows from the fact that the horizontal and vertical components of $G(x,y)$ depend only on $x$ and $y$, respectively, while the other two components involve both $x$ and $y$ in a nontrivial way. Specifically, if $(x1,y1) \neq (x2,y2)$, then there must be at least one component of $G(x1,y1)$ that is different from the corresponding component of $G(x2,y2)$.

The second part follows from the fact that every point on the Klein bottle can be parametrized by $(x,y) \in [0,2\pi) \times [0,2\pi)$, which is precisely the domain of $G$. Therefore, $G$ is bijective.
To show that $G$ is continuous, we need to show that each component of $G$ is a continuous function of $(x,y)$. We can see that this is the case, since each component is a sum and product of trigonometric functions, which are all continuous.

To show that $G^{-1}$ is continuous, we need to show that $G^{-1}(U)$ is open in $\bR^2$ for every open set $U$ in the image of $G$. To do this, we will use the inverse function theorem, which states that if $\det DG(x,y) \neq 0$ for all $(x,y) \in U$, then $G$ is a local diffeomorphism on $U$ and $G^{-1}$ is continuous on $G(U)$. By our earlier computation, we know that $\det DG(x,y) \neq 0$ for all $(x,y) \in \bR^2$, so we don't need to worry about singular points.

Therefore, we have shown that $G$ induces an embedding of the Klein bottle into $\bR^4$.
\end{proof}
7.
\begin{proof}
    
\end{proof}
8. 
\begin{proof}
\begin{figure}[htbp]
    \centering
    \includegraphics[width=.5\linewidth]{figures/ex8.jpg}
    \caption{exercise 8}
    \label{fig:ex8}
\end{figure}
    Pf: Let $\varphi: U \subset M_1 \rightarrow \varphi(U) \subset V_\alpha \cap V_\beta \subset M_2, V_\alpha, V_\beta$ are charts of $M_2$. Since $\varphi$ is a local diffeomorphisma, $\left\{\varphi^{-1}\left(U \cap X_\alpha\left(V_\alpha\right)\right), X_\alpha \circ \varphi\right\}$ is a parametrization of $M_1$. Since $M_2$ is orientable, i.e. $\operatorname{det}\left(d\left(X_\beta^{-1} \circ X_\alpha\right)\right)>0$. Therefore, for $p \in\left(U_1 \cap X_\alpha\left(V_\alpha\right)\right) \cap\left(U_2 \cap X_\beta\left(V_\beta\right)\right) \neq \varnothing$
\begin{align*}
\operatorname{det}\left(d\left(\left(X_\beta \circ \varphi\right)^{-1} \circ X_\alpha \circ \varphi\right)\right)=\operatorname{det}\left(d\left(\varphi^{-1} X_\beta^{-1} \circ X_\alpha \circ \varphi\right)\right)=\operatorname{det}\left(d\left(X_\beta^{-1} \circ X_\alpha\right)\right)>0 .
\end{align*}
\end{proof}


\begin{Exercise}[][Exer 9.]
    Let $G \times M \rightarrow M$ be a properly discontinuous action of a group $G$ on a differentiable manifold $M$.

a) Prove that the manifold $M / G$ (Example 4.8) is oriented if and only if there exists an orientation of $M$ that is preserved by all the diffeomorphisms of $G$.

b) Use a) to show that the projective plane $P^2(\mathbb{R})$, the Klein bottle and the Mobius band are non-orientable.

c) Prove that $P^2(\mathbb{R})$ is orientable if and only if $n$ is odd.
\end{Exercise}
\begin{proof}
 a) if part: Let $\left(U_\alpha, x_\alpha\right)$ be an orientation of $\mathrm{M}$ that is preserved by all the diffeomorphisms of $G$, i.e.
\begin{align*}
W=U_\beta \cap g\left(U_\alpha\right) \neq \varnothing \Rightarrow \operatorname{det}\left(x_\beta^{-1} \circ g \circ x_\alpha\right)>0
\end{align*}
We claim that $\left(\pi\left(U_\alpha\right), \pi \circ x_\alpha\right)$ is an orientation of $M / G$. Indeed,
\begin{align*}
\pi\left(U_\alpha\right) \cap \pi\left(U_\beta\right) \neq \varnothing \Rightarrow \operatorname{det}\left(\left(\pi \circ x_\beta\right)^{-1} \circ\left(\pi \circ x_\alpha\right)\right)=\operatorname{det}\left(x_\beta^{-1} \circ g \circ x_\alpha\right)>0
\end{align*}
for some $g \in G$.
Only if part: We know the atlas of $M / G$ is induced from $M$, hence the conclusion follows from the reverse of the "if part".

b) Let $G=\{I d, A\}$ where $A$ is the antipodal map. Recall that
\begin{align*}
    \text{Projective $2-$ space $P^2(\mathbb{R})=S^2 / G$, where $S^2$} =2-\operatorname{dim} \text{sphere}\\
\text{Klein bottle $K=\mathbb{T}^2 / G$, where $\mathbb{T}^2$}=2-\operatorname{dim} \text{torus}\\
\text{Mobius band $M=C / G$, where $C$}=2-\operatorname{dim} \text{cylinder}
\end{align*}


Clearly, $S^2, \mathbb{T}^2, C$ are orientable $2-\operatorname{dim}$ manifols, but $A$ reverse the orientation of $\mathbb{R}^3$, hence $S^2, \mathbb{T}^2, C$. The conclusion follows from a).

c) We've the following equivalence:
\begin{align*}
\begin{aligned}
& P^n(\mathbb{R}) \text { is orientable } \Leftrightarrow A \text { preserves the orientation of } S^n(\text { by } a) \text { ) } \\
& \Leftrightarrow A \text { preserves the orientation of } \mathbb{R}^{n+1} \\
& \text { (The orientation is induced from } \mathbb{R}^{n+1} \text { ) } \\
& \Leftrightarrow(n+1) \text { is even } \\
& \Leftrightarrow n \text { is odd } \\
&
\end{aligned}
\end{align*}
\end{proof}
10.
\begin{proof}
    
\end{proof}

11. $\left(\mathbb{R}, \mathbf{x}_1\right), \mathbf{x}_1: x \mapsto x,\left(\mathbb{R}, \mathbf{x}_2\right), \mathbf{x}_2: x \mapsto x^3$.

    (a) $i d:\left(\mathbb{R}, \mathbf{x}_1\right) \rightarrow\left(\mathbb{R}, \mathbf{x}_2\right)$ is not a diffeomorphism.

    (b) $f:\left(\mathbb{R}, \mathbf{x}_1\right) \rightarrow\left(\mathbb{R}, \mathbf{x}_2\right)$ is a diffeomorphism, where $f(x)=x^3$.
\begin{proof}
(a) It is not differentiable at $x=0$ for $\mathbf{x}_2^{-1} \circ i d: x \mapsto \sqrt[3]{x}$.

(b) $\mathbf{x}_2^{-1} \circ f \circ \mathbf{x}_1(x)=x$. Obviously, it is a diffeomorphism.
\end{proof}





























